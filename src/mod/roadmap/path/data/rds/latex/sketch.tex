\documentclass[12pt, a4paper]{article}
\usepackage{amsmath}
\usepackage{amssymb}

\makeatletter
\newcommand{\superimpose}[2]{{
  \ooalign{
    \hfil$\m@th#1\@firstoftwo#2$\hfil\cr
    \hfil$\m@th#1\@secondoftwo#2$\hfil\cr
  }
}}
\makeatother

\newcommand{\stack}[2]{\mathbin{\mathpalette\superimpose{{#1}{#2}}}}
\newcommand{\minuszero}{\stack{0}{-}}

\title{
    Towards a Probabilistic Career Roadmap
}
\author{Cao Bittencourt}
\date{\today}

\begin{document}

\maketitle

% Abstract
\begin{abstract}
  \noindent
  Assemblying a non-deterministic (i.e. probabilistic) model for career progression is no trivial task. This document attempts to summarize approaches to solve this problem.
  % \noindent
  % \textbf{Keywords:} Matching algorithms; Similarity coefficients; Career choice; Career development; Vocational choice; Occupational Information Network; O*NET.
\end{abstract}

\section{Definitions}
Let \newline \center
\begin{tabular}{cc}
  Notation                  & Definition                                                                             \\
  $\Lambda = [1 .. n]$      & Set of all $n \in \mathbb{N}$ occupations                                              \\
  $k \in \Lambda$           & The starting occupation                                                                \\
  $q \in \Lambda$           & The goal occupation                                                                    \\
  $\bar{x}_q \geq 0$        & The mean expected work experience for occupation $q$                                   \\
  $\bar{\tau}_q \geq 0$     & The mean expected years of education for occupation $q$                                \\
  $\ddot{x}_{kq} \geq 0$    & The equivalent work experience person $k$ has of occupation $q$                        \\
  $\ddot{\tau}_{kq} \geq 0$ & The equivalent years of education person $k$ has of occupation $q$                     \\
  $a_{1} \minuszero a_{2}$  & The minus-or-zero operator, equivalent to $[a_{1} - a_{2} > 0] \times (a_{1} - a_{2})$ \\
\end{tabular}

\section{Approach 1: }
\begin{gather}
  (\bar{x}_q \minuszero \bar{x}_k)
  +
  (\bar{\tau}_q \minuszero \bar{\tau}_k)
\end{gather}
\end{document}
