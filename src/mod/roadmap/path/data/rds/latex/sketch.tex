\documentclass[12pt, a4paper]{article}
\usepackage{amsmath}
\usepackage{amssymb}

\makeatletter
\newcommand{\superimpose}[2]{{
  \ooalign{
    \hfil$\m@th#1\@firstoftwo#2$\hfil\cr
    \hfil$\m@th#1\@secondoftwo#2$\hfil\cr
  }
}}
\makeatother

\newcommand{\stack}[2]{\mathbin{\mathpalette\superimpose{{#1}{#2}}}}
\newcommand{\minuszero}{\stack{0}{-}}
\newcommand{\Eta}{\text{H}}

\title{
    Towards a Probabilistic Career Roadmap
}
\author{Cao Bittencourt}
\date{\today}

\begin{document}

\maketitle

% Abstract
\begin{abstract}
  \noindent
  Assemblying a non-deterministic (i.e. probabilistic) model for career progression is no trivial task. This document attempts to summarize approaches to solve this problem.
  % \noindent
  % \textbf{Keywords:} Matching algorithms; Similarity coefficients; Career choice; Career development; Vocational choice; Occupational Information Network; O*NET.
\end{abstract}

\section{Definitions}
Let \newline \newline
\begin{tabular}{ll}
  Notation                  & Definition                                                                        \\
  $\Lambda := [1 .. n]$     & Set of all $n \in \mathbb{N}$ occupations                                         \\
  $k \in \Lambda$           & The starting occupation                                                           \\
  $q \in \Lambda$           & The goal occupation                                                               \\
  $\bar{x}_q \geq 0$        & The mean expected work experience for occupation $q$                              \\
  $\bar{\tau}_q \geq 0$     & The mean expected years of education for occupation $q$                           \\
  $\ddot{x}_{kq} \geq 0$    & The equivalent work experience person $k$ has of occupation $q$                   \\
  $\ddot{\tau}_{kq} \geq 0$ & The equivalent years of education person $k$ has of occupation $q$                \\
  $a_{1} \minuszero a_{2}$  & The minus-or-zero operator, equivalent to $[(a_{1} - a_{2}) > 0] (a_{1} - a_{2})$ \\
\end{tabular}

\section{Approach 1: Dijkstra's Algorithm}
Assume each career $q \in \Lambda$ has a set of $v_q \in p_q \in [1 .. w_q]$
unique job posts, and that each job post has its own minimum requirements in
terms of related work experience and education, as well as a probability, given
their rarity:

\begin{gather}
  x_{v_q} \geq 0
  ,
  \tau_{v_q} \geq 0
  ,
  P(v_q) \geq 0
  ,
  \sum_{v_q=1}^{p_q} P(v_q) = 1
  , \\
  \
  \forall
  \
  v_q \in [1..p_q] \land p_q \in [1..w_q]
  .
\end{gather}

Furthermore, for each job post in $k$ and $q$, we can estimate the baseline
cost of a career transition as
\begin{align}
  \Eta_{v_k}^{v_q}
   & :=
  \frac{
    (x_{v_q} \minuszero \ddot{x}_{v_k}^{q})
    +
    (\tau_{v_q} \minuszero \ddot{\tau}_{v_k}^{q})
  }{
    [\ddot{s}_{kq} \geq 0.5] s_{kq}
  }
  \\
   & :=
  \frac{
    (x_{v_q} \minuszero x_{v_k} [\ddot{s}_{kq} \geq 0.5] s_{kq})
    +
    (\tau_{v_q} \minuszero \tau_{v_k} [\ddot{s}_{kq} \geq 0.5] s_{kq})
  }{
    [\ddot{s}_{kq} \geq 0.5] s_{kq}
  }
  \\
   & =
  \left(
  \frac{x_{v_q}}{[\ddot{s}_{kq} \geq 0.5] s_{kq}}
  \minuszero
  x_{v_k}
  \right)
  +
  \left(
  \frac{\tau_{v_q}}{[\ddot{s}_{kq} \geq 0.5] s_{kq}}
  \minuszero
  \tau_{v_k}
  \right)
  \in
  [0, \infty)
  ,
\end{align}
such that the minimum cost, in years, to go from a job post (or vertex) $v_k$
in occupation $k$'s grid of job posts to another such job post (or vertex) $v_q$
is proportional to the time to close any experience and education gaps after accounting for the similarity of both occupations (thus, allowing for $k$ to ``recycle'' some of their experience and education). Note, as well, that if occupations $k$ and $q$ are not sufficiently similar (i.e. $[\ddot{s}_{kq} \geq 0.5] s_{kq} = 0$), the cost of a career progression goes to infinity.

It is also important to note the definition above [does not account for] the
probability of person $k$, currently at vertex $v_k$, actually finding job
$v_q$; or, in other words, we're not assigning any cost to the probability of
$v_q$ itself, which is, of course, unrealistic. For in the labor market, some
opportunities are evidently easier to come by than others. [And this is not
    even taking competition into consideration.] Therefore, we should, at least add
a term
\begin{gather}
  \Eta(P(v_q))
  :=
  \Eta_{v_q}
  \geq 0
  \therefore
  \\
  \Eta_{v_k}^{v_q}
  :=
  \left(
  \frac{x_{v_q}}{[\ddot{s}_{kq} \geq 0.5] s_{kq}}
  \minuszero
  x_{v_k}
  \right)
  +
  \left(
  \frac{\tau_{v_q}}{[\ddot{s}_{kq} \geq 0.5] s_{kq}}
  \minuszero
  \tau_{v_k}
  \right)
  +
  \Eta_{v_q}
  ,
\end{gather}
that we might call the ``inherent'' cost of $v_q$, which is the time to ``stumble upon'' such a job opportunity while actively searching in the labor market (i.e. it is the cost of waiting, or the cost of rarity).

\end{document}
