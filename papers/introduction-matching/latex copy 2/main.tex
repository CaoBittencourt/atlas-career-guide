% \documentclass[12pt]{article}
\documentclass{article}
\usepackage{xr}
\usepackage{physics}
\usepackage{float}
\usepackage{tabularray}
\UseTblrLibrary{booktabs}
\usepackage[
    backend=biber,
    style=apa
]{biblatex}
\usepackage[toc]{appendix}

% Equations
% % equivalence function
\newcommand{\eq}{\text{eq}}

% skill set interchangeability
\newcommand{\esset}{\textit{\ss}}

% skill set interchangeability vector
\newcommand{\essetvec}{\textbf{\textit{\ss}}}

% skill set interchangeability matrix
\newcommand{\Esset}{\textbf{\ss}}

% attribute equivalence
% \newcommand{\aeq}{\textit{ä}}
\newcommand{\aeq}{{\"a}}

% linear-logistic trigonometrically-scaled equivalence estimator
\newcommand{\EquivalenceEstimator}{
% \begin{gather}
%     \label{eqn:equivalence_estimator}
%     \eq(x,M) = 
%         x\left\{
%             1 + M (1 - x) \exp[-
%                 \tan\bigg(
%                     \frac{\pi}{2}
%                     \cos
%                         \bigg(
%                             \frac{\pi}{2} 
%                             % M (1 - M)
%                             x (1 - M)
%                         \bigg)
%                 \bigg)
%                 (x - M)
%             ]
%         \right\} ^ {
%             -\frac{M}{x}
%     },\\
%     x, M \in [0,1].
% \end{gather}
% 
\begin{gather}
    \label{eqn:equivalence_estimator}
    \eq(x,M) = 
        x\left\{
            1 + M (1 - x) \exp[-
                b(x - M)
            ]
        \right\} ^ {
            -\frac{M}{x}
    },\\
    b = 
    \tan\bigg[
        \frac{\pi}{2}
        \cos^{M(1 - M)}
        % \cos^{x(1 - M)}
            \bigg(
                \frac{\pi}{2} 
                % M (1 - M)
                x (1 - M)
            \bigg)
    \bigg],\\
    x, M \in [0,1].
\end{gather}
% 
% \begin{gather}
%     \label{eqn:equivalence_estimator}
%     \eq(x,M) = 
%         x\left\{
%             1 + M (1 - x) \exp[-
%                 \tan\bigg(
%                     \frac{\pi}{2}
%                         \cos^{x(1 - M)}
%                         \bigg(
%                             \frac{\pi}{2} 
%                             x (1 - M)
%                         \bigg)
%                     \bigg)
%                 (x - M)
%             ]
%         \right\} ^ {
%             -\frac{M}{x}
%     },\\
%     x, M \in [0,1].
% \end{gather}
}

% skill set interchangeability
\newcommand{\InterchangeabilityFunction}{
    \begin{gather}
        \esset_{k,q} = 
        \esset(s(\boldsymbol{a_k}, \boldsymbol{a_q}), M) = 
        \eq(s(\boldsymbol{a_k}, \boldsymbol{a_q}), M)
    \end{gather}
}

% interchangeability matrix
\newcommand{\InterchangeabilityMatrix}{
    \begin{gather}
        \Esset = 
        \begin{bmatrix}
            \textit{\ss}_{1,1} & \hdots & \textit{\ss}_{n,1}\\
            \vdots & \ddots & \vdots\\
            \textit{\ss}_{1,n} & \hdots & \textit{\ss}_{n,n}\\
        \end{bmatrix} = 
        \begin{bmatrix}
            1 & \hdots & \textit{\ss}_{k,1} & \hdots & \textit{\ss}_{n,1}\\
            \vdots & \ddots & \vdots & \vdots & \vdots\\
            \textit{\ss}_{1,k} & \hdots & 1 & \hdots & \textit{\ss}_{n,k}\\
            \vdots & \vdots & \vdots & \ddots & \vdots\\
            \textit{\ss}_{1,n} & \hdots & \textit{\ss}_{k,n} & \hdots & 1\\
        \end{bmatrix}\\
    \end{gather}
}

% binary interchangeability (hireability)
\newcommand{\SignalFunction}{
    \begin{gather}
        h_{k,q} = 
        h(\esset_{k,q}) = 
        \begin{cases}
            1, \text{if } \esset_{k,q} \geq 0.5.\\
            0, \text{otherwise.}
        \end{cases}
    \end{gather}
}

% attribute equivalence matrix
\newcommand{\AttributeEquivalence}{
    \begin{gather}
        \aeq_{i}^{k} = 
        \aeq(\boldsymbol{a_k}, M) = 
        \eq\left(
            \frac{
                a_{i}^{k}
            }{
                \max{a_{j}^{k}}
            }
            , M
        \right)
    \end{gather}
}
% equivalence function
\newcommand{\eq}{\text{eq}}

% skill set interchangeability
\newcommand{\esset}{\textit{\ss}}

% skill set interchangeability vector
\newcommand{\essetvec}{\textbf{\textit{\ss}}}

% skill set interchangeability matrix
\newcommand{\Esset}{\textbf{\ss}}

% attribute equivalence
% \newcommand{\aeq}{\textit{ä}}
\newcommand{\aeq}{{\"a}}

% linear-logistic trigonometrically-scaled equivalence estimator
\newcommand{\EquivalenceEstimator}{
% \begin{gather}
%     \label{eqn:equivalence_estimator}
%     \eq(x,M) = 
%         x\left\{
%             1 + M (1 - x) \exp[-
%                 \tan\bigg(
%                     \frac{\pi}{2}
%                     \cos
%                         \bigg(
%                             \frac{\pi}{2} 
%                             % M (1 - M)
%                             x (1 - M)
%                         \bigg)
%                 \bigg)
%                 (x - M)
%             ]
%         \right\} ^ {
%             -\frac{M}{x}
%     },\\
%     x, M \in [0,1].
% \end{gather}
% 
\begin{gather}
    \label{eqn:equivalence_estimator}
    \eq(x,M) = 
        x\left\{
            1 + M (1 - x) \exp[-
                b(x - M)
            ]
        \right\} ^ {
            -\frac{M}{x}
    },\\
    b = 
    \tan\bigg[
        \frac{\pi}{2}
        \cos^{M(1 - M)}
        % \cos^{x(1 - M)}
            \bigg(
                \frac{\pi}{2} 
                % M (1 - M)
                x (1 - M)
            \bigg)
    \bigg],\\
    x, M \in [0,1].
\end{gather}
% 
% \begin{gather}
%     \label{eqn:equivalence_estimator}
%     \eq(x,M) = 
%         x\left\{
%             1 + M (1 - x) \exp[-
%                 \tan\bigg(
%                     \frac{\pi}{2}
%                         \cos^{x(1 - M)}
%                         \bigg(
%                             \frac{\pi}{2} 
%                             x (1 - M)
%                         \bigg)
%                     \bigg)
%                 (x - M)
%             ]
%         \right\} ^ {
%             -\frac{M}{x}
%     },\\
%     x, M \in [0,1].
% \end{gather}
}

% skill set interchangeability
\newcommand{\InterchangeabilityFunction}{
    \begin{gather}
        \esset_{k,q} = 
        \esset(s(\boldsymbol{a_k}, \boldsymbol{a_q}), M) = 
        \eq(s(\boldsymbol{a_k}, \boldsymbol{a_q}), M)
    \end{gather}
}

% interchangeability matrix
\newcommand{\InterchangeabilityMatrix}{
    \begin{gather}
        \Esset = 
        \begin{bmatrix}
            \textit{\ss}_{1,1} & \hdots & \textit{\ss}_{n,1}\\
            \vdots & \ddots & \vdots\\
            \textit{\ss}_{1,n} & \hdots & \textit{\ss}_{n,n}\\
        \end{bmatrix} = 
        \begin{bmatrix}
            1 & \hdots & \textit{\ss}_{k,1} & \hdots & \textit{\ss}_{n,1}\\
            \vdots & \ddots & \vdots & \vdots & \vdots\\
            \textit{\ss}_{1,k} & \hdots & 1 & \hdots & \textit{\ss}_{n,k}\\
            \vdots & \vdots & \vdots & \ddots & \vdots\\
            \textit{\ss}_{1,n} & \hdots & \textit{\ss}_{k,n} & \hdots & 1\\
        \end{bmatrix}\\
    \end{gather}
}

% binary interchangeability (hireability)
\newcommand{\SignalFunction}{
    \begin{gather}
        h_{k,q} = 
        h(\esset_{k,q}) = 
        \begin{cases}
            1, \text{if } \esset_{k,q} \geq 0.5.\\
            0, \text{otherwise.}
        \end{cases}
    \end{gather}
}

% attribute equivalence matrix
\newcommand{\AttributeEquivalence}{
    \begin{gather}
        \aeq_{i}^{k} = 
        \aeq(\boldsymbol{a_k}, M) = 
        \eq\left(
            \frac{
                a_{i}^{k}
            }{
                \max{a_{j}^{k}}
            }
            , M
        \right)
    \end{gather}
}

% Tables
% % General Occupational Statistics
\newcommand{\GeneralOccupationalStatistics}{
    \begin{longtblr}[
        caption = {General Occupational Statistics},
        label = {tab:onet_careers},
        note{1} = {Number of SOC occupations plus variants.},
        note{2} = {Total employment levels in the United States in 2022.},
        note{3} = {Employment-weighted mean wages in the United States in 2022.},
        note{4} = {Total cluster wages as a percentage of total wages in the United States in 2022.},
        remark{Source} = {\cites{bls2022data}{onet2023data}.}
    ]{
        colspec = {X[0.5, l, m]X[0, c, m]X[0.25, c, m]X[0.25, c, m]X[0, c, m]},
        width = \linewidth,
        rowhead = 1,
        rowfoot = 0
    }
        \toprule
            Cluster & N\TblrNote{1} & Employment\TblrNote{2} & Wage\TblrNote{3} & Market Share\TblrNote{4} \\ 
        \midrule
            Business Management \& Administration &  65 & 26,639,371 & \$59,924.00 & 21.05\% \\ 
            Health Science &  99 & 16,856,404 & \$62,753.00 & 13.95\% \\ 
            Marketing &  26 & 13,961,830 & \$45,512.00 & 8.38\% \\ 
            Manufacturing & 138 & 13,644,785 & \$44,169.00 & 7.95\% \\ 
            Hospitality &  42 & 17,416,290 & \$28,965.00 & 6.65\% \\ 
            Transportation, Distribution \& Logistics &  62 & 11,018,200 & \$45,511.00 & 6.61\% \\ 
            Architecture \& Construction &  86 & 9,005,327 & \$53,142.00 & 6.31\% \\ 
            Education \& Training &  64 & 7,266,480 & \$62,621.00 & 6.00\% \\ 
            Finance &  22 & 5,255,514 & \$70,325.00 & 4.87\% \\ 
            Law, Public Safety, Corrections \& Security &  35 & 4,790,620 & \$61,985.00 & 3.92\% \\ 
            Information Technology &  22 & 3,588,065 & \$82,061.00 & 3.88\% \\ 
            Human Services &  38 & 7,823,620 & \$36,809.00 & 3.80\% \\ 
            Science, Technology, Engineering \& Mathematics &  70 & 2,207,815 & \$100,683.00 & 2.93\% \\ 
            Agriculture, Food \& Natural Resources &  40 & 2,140,244 & \$51,134.00 & 1.44\% \\ 
            Arts, Audio/Video Technology \& Communications &  40 & 1,692,270 & \$57,836.00 & 1.29\% \\ 
            Government \& Public Administration &  24 & 1,166,588 & \$62,636.00 & 0.96\% \\
        \bottomrule
    \end{longtblr}
}

% Summary of Occupations
\newcommand{\SummaryOccupations}{
    \begin{table}[H]
        \centering
            \begin{talltblr}[
            caption = {Summary of Sample Occupations},
            label = {tab:summary_occupations},
            note{1} = {``STEM" stands for ``Science, Technology, Engineering, and Mathematics".},
            note{2} = {Total employment levels in the United States in 2022.},
            note{3} = {Median wages in the United States in 2022.},
            remark{Source} = {\cites{bls2022data}{onet2023data}.}
            ]{
            colspec = {ccccc},
            % colspec = {X[c, m]Q[c, m]X[c, m]X[c, m]X[c, m]},
            width = \linewidth
            }
                \toprule
                    SOC & Occupation & Cluster & Employment\TblrNote{2} & Wage\TblrNote{3} \\ 
                \midrule
                    19-2012 & Physicists & STEM\TblrNote{1} & 18,840 & \$152,430.00 \\
                    17-2141 & Mechanical Engineers & STEM\TblrNote{1} & 277,560 & \$95,300.00 \\ 
                    13-2041 & Credit Analysts & Finance & 71,960 & \$77,440.00 \\ 
                    35-9021 & Dishwashers & Hospitality & 431,840 & \$28,130.00 \\ 
                \bottomrule
            \end{talltblr}
    \end{table}
} 

% Mechanical Engineers
\newcommand{\MatchesEngineers}{
    \begin{longtblr}[
        caption = {Best Career Matches -- Mechanical Engineers},
        label = {tab:similarity_engineers},
        remark{Note} = {Similarity scores estimated with Euclidean matching and linear weights.},
        remark{Source} = {Author's calculation based on \cite{onet2023data}.}
    ]{
        colspec = {X[l, m]X[0, c, m]},
        width = \linewidth,
        rowhead = 1,
        rowfoot = 0
    }
        \toprule
            Comparison Occupation & Similarity \\ 
        \midrule
            Mechanical Engineers & 1.00 \\ 
            Aerospace Engineers & 0.90 \\ 
            Marine Engineers and Naval Architects & 0.88 \\ 
            Nuclear Engineers & 0.88 \\ 
            Mechatronics Engineers & 0.87 \\ 
            Microsystems Engineers & 0.86 \\ 
            Photonics Engineers & 0.86 \\ 
            Fuel Cell Engineers & 0.86 \\ 
            Automotive Engineers & 0.86 \\ 
            Materials Engineers & 0.86 \\ 
            Mining and Geological Engineers, Including Mining Safety Engineers & 0.86 \\
        \bottomrule
    \end{longtblr}
}

% Physicists
\newcommand{\MatchesPhysicists}{
    \begin{longtblr}[
        caption = {Best Career Matches -- Physicists},
        label = {tab:similarity_physicists},
        remark{Note} = {Similarity scores estimated with Euclidean matching and linear weights.},
        remark{Source} = {Author's calculation based on \cite{onet2023data}.}
    ]{
        colspec = {X[l, m]X[0, c, m]},
        width = \linewidth,
        rowhead = 1,
        rowfoot = 0
    }
        \toprule
            Comparison Occupation & Similarity \\ 
        \midrule
            Physicists & 1.00 \\ 
            Astronomers & 0.85 \\ 
            Mathematicians & 0.80 \\ 
            Physics Teachers, Postsecondary & 0.80 \\ 
            Materials Scientists & 0.79 \\ 
            Engineering Teachers, Postsecondary & 0.79 \\ 
            Nanosystems Engineers & 0.78 \\ 
            Aerospace Engineers & 0.77 \\ 
            Biochemists and Biophysicists & 0.77 \\ 
            Nuclear Engineers & 0.77 \\ 
            Computer and Information Research Scientists & 0.77 \\ 
        \bottomrule
    \end{longtblr}
}

% Credit Analysts
\newcommand{\MatchesCreditAnalysts}{
    \begin{longtblr}[
        caption = {Best Career Matches -- Credit Analysts},
        label = {tab:similarity_credit_analysts},
        remark{Note} = {Similarity scores estimated with Euclidean matching and linear weights.},
        remark{Source} = {Author's calculation based on \cite{onet2023data}.}
    ]{
        colspec = {X[l, m]X[0, c, m]},
        width = \linewidth,
        rowhead = 1,
        rowfoot = 0
    }
        \toprule
            Comparison Occupation & Similarity \\ 
        \midrule
            Credit Analysts & 1.00 \\ 
            Accountants and Auditors & 0.87 \\ 
            Loan Interviewers and Clerks & 0.86 \\ 
            Budget Analysts & 0.85 \\ 
            Securities, Commodities, and Financial Services Sales Agents & 0.85 \\ 
            Financial Examiners & 0.85 \\ 
            Business Intelligence Analysts & 0.85 \\ 
            Tax Preparers & 0.85 \\ 
            Insurance Underwriters & 0.84 \\ 
            Bookkeeping, Accounting, and Auditing Clerks & 0.84 \\ 
            Tax Examiners and Collectors, and Revenue Agents & 0.83 \\ 
        \bottomrule
    \end{longtblr}
}

% Dishwashers
\newcommand{\MatchesDishwashers}{
    \begin{longtblr}[
        caption = {Best Career Matches -- Dishwashers},
        label = {tab:similarity_dishwashers},
        remark{Note} = {Similarity scores estimated with Euclidean matching and linear weights.},
        remark{Source} = {Author's calculation based on \cite{onet2023data}.}
    ]{
        colspec = {X[l, m]X[0, c, m]},
        width = \linewidth,
        rowhead = 1,
        rowfoot = 0
    }
        \toprule
            Comparison Occupation & Similarity \\ 
        \midrule
            Dishwashers & 1.00 \\ 
            Janitors and Cleaners, Except Maids and Housekeeping Cleaners & 0.88 \\ 
            Maids and Housekeeping Cleaners & 0.88 \\ 
            Pressers, Textile, Garment, and Related Materials & 0.87 \\ 
            Models & 0.87 \\ 
            Postal Service Mail Sorters, Processors, and Processing Machine Operators & 0.87 \\ 
            Sewing Machine Operators & 0.86 \\ 
            Laundry and Dry-Cleaning Workers & 0.86 \\ 
            Orderlies & 0.86 \\ 
            Cutters and Trimmers, Hand & 0.86 \\ 
            Shoe and Leather Workers and Repairers & 0.86 \\
        \bottomrule
    \end{longtblr}
}

% Similarity matrix
\newcommand{\SimilarityMatrix}{
    \begin{table}[H]
        \centering
        \begin{talltblr}[
            caption = {Similarity Matrix},
            label = {tab:similarity_matrix},
            remark{Note 1} = {Similarity scores estimated with Euclidean matching and linear weights.},
            remark{Note 2} = {Comparison occupations on the horizontal axis.},
            remark{Source} = {Author's calculation based on \cite{onet2023data}.}
        ]{
            colspec = {X[c, m]X[c, m]X[c, m]X[c, m]X[c, m]},
            width = \linewidth
        }
            \toprule
                & Mechanical Engineers & Physicists & Credit Analysts & Dishwashers \\ 
            \midrule
                Mechanical Engineers & 1.00 & 0.75 & 0.54 & 0.38 \\ 
                Physicists & 0.80 & 1.00 & 0.57 & 0.34 \\ 
                Credit Analysts & 0.75 & 0.66 & 1.00 & 0.54 \\ 
                Dishwashers & 0.62 & 0.53 & 0.66 & 1.00 \\ 
            \bottomrule
        \end{talltblr}
    \end{table}
}

% Detailed skill sets
\newcommand{\DetailedSkillSets}{
    \begin{longtblr}[
        caption = {Detailed Skill Sets},
        label = {tab:detailed_skill_sets},
        remark{Source} = {\cite{onet2023data}.}
    ]{
        % colspec = {X[1.5, c, m]X[1, c, m]X[1, c, m]X[1, c, m]X[1, c, m]},
        colspec = {X[0.4, l, m]X[0.2, c, m]X[0.2, c, m]X[0.2, c, m]X[0.2, c, m]},
        width = \linewidth,
        rowhead = 1,
        rowfoot = 0
    }
        \toprule
            Competency & Mechanical Engineers & Physicists & Credit Analysts & Dishwashers \\
        \midrule
            Engineering and Technology & 87 & 86 & 13 & 0 \\ 
            Industrial Design & 84 & 45 & 0 & 0 \\ 
            Applied Mathematics & 79 & 94 & 68 & 0 \\ 
            Mechanical & 78 & 28 & 0 & 12 \\ 
            Physics & 73 & 97 & 0 & 0 \\ 
            Oral Comprehension & 70 & 84 & 59 & 32 \\ 
            Reading Comprehension & 70 & 84 & 57 & 30 \\ 
            Oral Expression & 70 & 82 & 59 & 30 \\ 
            Deductive Reasoning & 70 & 79 & 64 & 29 \\ 
            Pure Mathematics & 70 & 79 & 59 & 14 \\ 
            Mathematical Reasoning & 68 & 84 & 59 & 16 \\ 
            Written Comprehension & 68 & 79 & 59 & 29 \\ 
            Complex Problem-Solving & 68 & 70 & 43 & 30 \\ 
            Computers and Electronics & 67 & 83 & 43 & 0 \\ 
            Natural Science & 66 & 82 & 5 & 0 \\ 
            Operations Analysis & 66 & 52 & 34 & 0 \\ 
            Active Learning & 63 & 79 & 52 & 29 \\ 
            Information Ordering & 63 & 71 & 52 & 30 \\ 
            Number Facility & 61 & 79 & 63 & 16 \\ 
            Critical Thinking & 61 & 70 & 57 & 30 \\ 
            Monitoring & 61 & 57 & 43 & 30 \\ 
            Written Expression & 59 & 80 & 57 & 23 \\ 
            Inductive Reasoning & 59 & 79 & 59 & 29 \\ 
            Category Flexibility & 59 & 73 & 50 & 29 \\ 
            Problem Sensitivity & 59 & 70 & 57 & 29 \\ 
            Judgment and Decision & 59 & 64 & 54 & 30 \\ 
            Originality & 57 & 75 & 37 & 25 \\ 
            Writing & 57 & 71 & 55 & 25 \\ 
            Native Language & 57 & 70 & 60 & 38 \\ 
            Active Listening & 57 & 68 & 57 & 32 \\ 
            Near Vision & 57 & 57 & 61 & 34 \\ 
            Administration and Management & 57 & 54 & 42 & 37 \\ 
            Quality Control Analysis & 57 & 46 & 11 & 29 \\ 
            Chemistry & 56 & 61 & 0 & 21 \\ 
            Fluency of Ideas & 55 & 73 & 41 & 27 \\ 
            Speaking & 55 & 68 & 57 & 30 \\ 
            Visualization & 55 & 61 & 25 & 29 \\ 
            Systems Analysis & 55 & 57 & 43 & 16 \\ 
            Systems Evaluation & 55 & 55 & 45 & 16 \\ 
            Speech Recognition & 54 & 57 & 50 & 30 \\ 
            Technology Design & 54 & 48 & 0 & 0 \\ 
            Operations Monitoring & 54 & 34 & 14 & 30 \\ 
            Instructing & 52 & 66 & 34 & 20 \\ 
            Time Management & 52 & 52 & 43 & 30 \\ 
            Troubleshooting & 52 & 29 & 0 & 29 \\ 
            Education and Training & 51 & 72 & 29 & 41 \\ 
            Administrative & 51 & 40 & 59 & 0 \\ 
            Learning Strategies & 50 & 71 & 36 & 18 \\ 
            Selective Attention & 50 & 57 & 41 & 30 \\ 
            Persuasion & 50 & 50 & 36 & 27 \\ 
            Coordination & 50 & 48 & 36 & 30 \\ 
            Production and Processing & 49 & 34 & 29 & 0 \\ 
            Speech Clarity & 48 & 71 & 45 & 27 \\ 
            Flexibility of Closure & 48 & 55 & 45 & 29 \\ 
            Perceptual Speed & 48 & 46 & 45 & 30 \\ 
            Customer and Personal Service & 48 & 37 & 39 & 38 \\ 
            Far Vision & 46 & 52 & 30 & 32 \\ 
            Management of Personnel Resources & 45 & 45 & 27 & 29 \\ 
            Speed of Closure & 43 & 55 & 41 & 29 \\ 
            Negotiation & 43 & 39 & 39 & 29 \\ 
            Social Perceptiveness & 41 & 45 & 41 & 29 \\ 
            Service Orientation & 41 & 43 & 43 & 29 \\ 
            Visual Color Discrimination & 41 & 41 & 21 & 29 \\ 
            Management of Material Resources & 41 & 32 & 13 & 13 \\ 
            Multitasking & 39 & 41 & 29 & 29 \\ 
            Management of Financial Resources & 39 & 30 & 27 & 11 \\ 
            Hearing Sensitivity & 39 & 30 & 16 & 29 \\ 
            Memorization & 37 & 48 & 34 & 21 \\ 
            Auditory Attention & 37 & 32 & 21 & 29 \\ 
            Telecommunications & 37 & 23 & 7 & 0 \\ 
            Equipment Selection & 37 & 21 & 0 & 29 \\ 
            Installation & 37 & 0 & 0 & 0 \\ 
            Building and Construction & 36 & 15 & 16 & 0 \\ 
            Personnel and Human Resources & 35 & 38 & 23 & 0 \\ 
            Public Safety and Security & 35 & 26 & 15 & 26 \\ 
            Programming & 34 & 55 & 14 & 0 \\ 
            Depth Perception & 34 & 30 & 5 & 29 \\ 
            Operation and Control & 34 & 9 & 0 & 30 \\ 
            Sales and Marketing & 33 & 20 & 29 & 0 \\ 
            Finger Dexterity & 32 & 29 & 25 & 34 \\ 
            Law and Government & 32 & 28 & 54 & 28 \\ 
            Reaction Time & 30 & 0 & 0 & 29 \\ 
            Transportation & 28 & 12 & 0 & 33 \\ 
            Communications and Media & 25 & 46 & 23 & 19 \\ 
            Economics and Accounting & 25 & 26 & 74 & 0 \\ 
            Arm-Hand Steadiness & 25 & 0 & 11 & 41 \\ 
            Repairing & 25 & 0 & 0 & 29 \\ 
            Equipment Maintenance & 23 & 0 & 0 & 29 \\ 
            Medicine and Dentistry & 23 & 0 & 0 & 0 \\ 
            Geography & 21 & 15 & 25 & 0 \\ 
            Biology & 20 & 16 & 0 & 0 \\ 
            Trunk Strength & 14 & 14 & 14 & 43 \\ 
            Manual Dexterity & 14 & 0 & 9 & 52 \\ 
            Wrist-Finger Speed & 9 & 0 & 0 & 21 \\ 
            Rate Control & 5 & 0 & 0 & 32 \\ 
            Spatial Orientation & 5 & 0 & 0 & 14 \\ 
            Psychology & 0 & 28 & 10 & 0 \\ 
            Therapy and Counseling & 0 & 21 & 0 & 0 \\ 
            Sociology and Anthropology & 0 & 0 & 16 & 0 \\ 
            History and Archeology & 0 & 0 & 13 & 0 \\ 
            Control Precision & 0 & 0 & 7 & 32 \\ 
            Extent Flexibility & 0 & 0 & 0 & 43 \\ 
            Static Strength & 0 & 0 & 0 & 39 \\ 
            Multilimb Coordination & 0 & 0 & 0 & 37 \\ 
            Stamina & 0 & 0 & 0 & 32 \\ 
            Gross Body Coordination & 0 & 0 & 0 & 30 \\ 
            Speed of Limb Movement & 0 & 0 & 0 & 30 \\ 
            Response Orientation & 0 & 0 & 0 & 29 \\ 
            Dynamic Strength & 0 & 0 & 0 & 25 \\ 
            Gross Body Equilibrium & 0 & 0 & 0 & 21 \\ 
            Glare Sensitivity & 0 & 0 & 0 & 21 \\ 
            Sound Localization & 0 & 0 & 0 & 14 \\ 
            Night Vision & 0 & 0 & 0 & 14 \\ 
            Peripheral Vision & 0 & 0 & 0 & 14 \\ 
            Explosive Strength & 0 & 0 & 0 & 11 \\ 
            Dynamic Flexibility & 0 & 0 & 0 & 0 \\ 
            Foreign Language & 0 & 0 & 0 & 0 \\ 
            Philosophy and Theology & 0 & 0 & 0 & 0 \\ 
            Fine Arts & 0 & 0 & 0 & 0 \\ 
            Food Production & 0 & 0 & 0 & 0 \\
        \bottomrule
    \end{longtblr}
}
% General Occupational Statistics
\newcommand{\GeneralOccupationalStatistics}{
    \begin{longtblr}[
        caption = {General Occupational Statistics},
        label = {tab:onet_careers},
        note{1} = {Number of SOC occupations plus variants.},
        note{2} = {Total employment levels in the United States in 2022.},
        note{3} = {Employment-weighted mean wages in the United States in 2022.},
        note{4} = {Total cluster wages as a percentage of total wages in the United States in 2022.},
        remark{Source} = {\cites{bls2022data}{onet2023data}.}
    ]{
        colspec = {X[0.5, l, m]X[0, c, m]X[0.25, c, m]X[0.25, c, m]X[0, c, m]},
        width = \linewidth,
        rowhead = 1,
        rowfoot = 0
    }
        \toprule
            Cluster & N\TblrNote{1} & Employment\TblrNote{2} & Wage\TblrNote{3} & Market Share\TblrNote{4} \\ 
        \midrule
            Business Management \& Administration &  65 & 26,639,371 & \$59,924.00 & 21.05\% \\ 
            Health Science &  99 & 16,856,404 & \$62,753.00 & 13.95\% \\ 
            Marketing &  26 & 13,961,830 & \$45,512.00 & 8.38\% \\ 
            Manufacturing & 138 & 13,644,785 & \$44,169.00 & 7.95\% \\ 
            Hospitality &  42 & 17,416,290 & \$28,965.00 & 6.65\% \\ 
            Transportation, Distribution \& Logistics &  62 & 11,018,200 & \$45,511.00 & 6.61\% \\ 
            Architecture \& Construction &  86 & 9,005,327 & \$53,142.00 & 6.31\% \\ 
            Education \& Training &  64 & 7,266,480 & \$62,621.00 & 6.00\% \\ 
            Finance &  22 & 5,255,514 & \$70,325.00 & 4.87\% \\ 
            Law, Public Safety, Corrections \& Security &  35 & 4,790,620 & \$61,985.00 & 3.92\% \\ 
            Information Technology &  22 & 3,588,065 & \$82,061.00 & 3.88\% \\ 
            Human Services &  38 & 7,823,620 & \$36,809.00 & 3.80\% \\ 
            Science, Technology, Engineering \& Mathematics &  70 & 2,207,815 & \$100,683.00 & 2.93\% \\ 
            Agriculture, Food \& Natural Resources &  40 & 2,140,244 & \$51,134.00 & 1.44\% \\ 
            Arts, Audio/Video Technology \& Communications &  40 & 1,692,270 & \$57,836.00 & 1.29\% \\ 
            Government \& Public Administration &  24 & 1,166,588 & \$62,636.00 & 0.96\% \\
        \bottomrule
    \end{longtblr}
}

% Summary of Occupations
\newcommand{\SummaryOccupations}{
    \begin{table}[H]
        \centering
            \begin{talltblr}[
            caption = {Summary of Sample Occupations},
            label = {tab:summary_occupations},
            note{1} = {``STEM" stands for ``Science, Technology, Engineering, and Mathematics".},
            note{2} = {Total employment levels in the United States in 2022.},
            note{3} = {Median wages in the United States in 2022.},
            remark{Source} = {\cites{bls2022data}{onet2023data}.}
            ]{
            colspec = {ccccc},
            % colspec = {X[c, m]Q[c, m]X[c, m]X[c, m]X[c, m]},
            width = \linewidth
            }
                \toprule
                    SOC & Occupation & Cluster & Employment\TblrNote{2} & Wage\TblrNote{3} \\ 
                \midrule
                    19-2012 & Physicists & STEM\TblrNote{1} & 18,840 & \$152,430.00 \\
                    17-2141 & Mechanical Engineers & STEM\TblrNote{1} & 277,560 & \$95,300.00 \\ 
                    13-2041 & Credit Analysts & Finance & 71,960 & \$77,440.00 \\ 
                    35-9021 & Dishwashers & Hospitality & 431,840 & \$28,130.00 \\ 
                \bottomrule
            \end{talltblr}
    \end{table}
} 

% Mechanical Engineers
\newcommand{\MatchesEngineers}{
    \begin{longtblr}[
        caption = {Best Career Matches -- Mechanical Engineers},
        label = {tab:similarity_engineers},
        remark{Note} = {Similarity scores estimated with Euclidean matching and linear weights.},
        remark{Source} = {Author's calculation based on \cite{onet2023data}.}
    ]{
        colspec = {X[l, m]X[0, c, m]},
        width = \linewidth,
        rowhead = 1,
        rowfoot = 0
    }
        \toprule
            Comparison Occupation & Similarity \\ 
        \midrule
            Mechanical Engineers & 1.00 \\ 
            Aerospace Engineers & 0.90 \\ 
            Marine Engineers and Naval Architects & 0.88 \\ 
            Nuclear Engineers & 0.88 \\ 
            Mechatronics Engineers & 0.87 \\ 
            Microsystems Engineers & 0.86 \\ 
            Photonics Engineers & 0.86 \\ 
            Fuel Cell Engineers & 0.86 \\ 
            Automotive Engineers & 0.86 \\ 
            Materials Engineers & 0.86 \\ 
            Mining and Geological Engineers, Including Mining Safety Engineers & 0.86 \\
        \bottomrule
    \end{longtblr}
}

% Physicists
\newcommand{\MatchesPhysicists}{
    \begin{longtblr}[
        caption = {Best Career Matches -- Physicists},
        label = {tab:similarity_physicists},
        remark{Note} = {Similarity scores estimated with Euclidean matching and linear weights.},
        remark{Source} = {Author's calculation based on \cite{onet2023data}.}
    ]{
        colspec = {X[l, m]X[0, c, m]},
        width = \linewidth,
        rowhead = 1,
        rowfoot = 0
    }
        \toprule
            Comparison Occupation & Similarity \\ 
        \midrule
            Physicists & 1.00 \\ 
            Astronomers & 0.85 \\ 
            Mathematicians & 0.80 \\ 
            Physics Teachers, Postsecondary & 0.80 \\ 
            Materials Scientists & 0.79 \\ 
            Engineering Teachers, Postsecondary & 0.79 \\ 
            Nanosystems Engineers & 0.78 \\ 
            Aerospace Engineers & 0.77 \\ 
            Biochemists and Biophysicists & 0.77 \\ 
            Nuclear Engineers & 0.77 \\ 
            Computer and Information Research Scientists & 0.77 \\ 
        \bottomrule
    \end{longtblr}
}

% Credit Analysts
\newcommand{\MatchesCreditAnalysts}{
    \begin{longtblr}[
        caption = {Best Career Matches -- Credit Analysts},
        label = {tab:similarity_credit_analysts},
        remark{Note} = {Similarity scores estimated with Euclidean matching and linear weights.},
        remark{Source} = {Author's calculation based on \cite{onet2023data}.}
    ]{
        colspec = {X[l, m]X[0, c, m]},
        width = \linewidth,
        rowhead = 1,
        rowfoot = 0
    }
        \toprule
            Comparison Occupation & Similarity \\ 
        \midrule
            Credit Analysts & 1.00 \\ 
            Accountants and Auditors & 0.87 \\ 
            Loan Interviewers and Clerks & 0.86 \\ 
            Budget Analysts & 0.85 \\ 
            Securities, Commodities, and Financial Services Sales Agents & 0.85 \\ 
            Financial Examiners & 0.85 \\ 
            Business Intelligence Analysts & 0.85 \\ 
            Tax Preparers & 0.85 \\ 
            Insurance Underwriters & 0.84 \\ 
            Bookkeeping, Accounting, and Auditing Clerks & 0.84 \\ 
            Tax Examiners and Collectors, and Revenue Agents & 0.83 \\ 
        \bottomrule
    \end{longtblr}
}

% Dishwashers
\newcommand{\MatchesDishwashers}{
    \begin{longtblr}[
        caption = {Best Career Matches -- Dishwashers},
        label = {tab:similarity_dishwashers},
        remark{Note} = {Similarity scores estimated with Euclidean matching and linear weights.},
        remark{Source} = {Author's calculation based on \cite{onet2023data}.}
    ]{
        colspec = {X[l, m]X[0, c, m]},
        width = \linewidth,
        rowhead = 1,
        rowfoot = 0
    }
        \toprule
            Comparison Occupation & Similarity \\ 
        \midrule
            Dishwashers & 1.00 \\ 
            Janitors and Cleaners, Except Maids and Housekeeping Cleaners & 0.88 \\ 
            Maids and Housekeeping Cleaners & 0.88 \\ 
            Pressers, Textile, Garment, and Related Materials & 0.87 \\ 
            Models & 0.87 \\ 
            Postal Service Mail Sorters, Processors, and Processing Machine Operators & 0.87 \\ 
            Sewing Machine Operators & 0.86 \\ 
            Laundry and Dry-Cleaning Workers & 0.86 \\ 
            Orderlies & 0.86 \\ 
            Cutters and Trimmers, Hand & 0.86 \\ 
            Shoe and Leather Workers and Repairers & 0.86 \\
        \bottomrule
    \end{longtblr}
}

% Similarity matrix
\newcommand{\SimilarityMatrix}{
    \begin{table}[H]
        \centering
        \begin{talltblr}[
            caption = {Similarity Matrix},
            label = {tab:similarity_matrix},
            remark{Note 1} = {Similarity scores estimated with Euclidean matching and linear weights.},
            remark{Note 2} = {Comparison occupations on the horizontal axis.},
            remark{Source} = {Author's calculation based on \cite{onet2023data}.}
        ]{
            colspec = {X[c, m]X[c, m]X[c, m]X[c, m]X[c, m]},
            width = \linewidth
        }
            \toprule
                & Mechanical Engineers & Physicists & Credit Analysts & Dishwashers \\ 
            \midrule
                Mechanical Engineers & 1.00 & 0.75 & 0.54 & 0.38 \\ 
                Physicists & 0.80 & 1.00 & 0.57 & 0.34 \\ 
                Credit Analysts & 0.75 & 0.66 & 1.00 & 0.54 \\ 
                Dishwashers & 0.62 & 0.53 & 0.66 & 1.00 \\ 
            \bottomrule
        \end{talltblr}
    \end{table}
}

% Detailed skill sets
\newcommand{\DetailedSkillSets}{
    \begin{longtblr}[
        caption = {Detailed Skill Sets},
        label = {tab:detailed_skill_sets},
        remark{Source} = {\cite{onet2023data}.}
    ]{
        % colspec = {X[1.5, c, m]X[1, c, m]X[1, c, m]X[1, c, m]X[1, c, m]},
        colspec = {X[0.4, l, m]X[0.2, c, m]X[0.2, c, m]X[0.2, c, m]X[0.2, c, m]},
        width = \linewidth,
        rowhead = 1,
        rowfoot = 0
    }
        \toprule
            Competency & Mechanical Engineers & Physicists & Credit Analysts & Dishwashers \\
        \midrule
            Engineering and Technology & 87 & 86 & 13 & 0 \\ 
            Industrial Design & 84 & 45 & 0 & 0 \\ 
            Applied Mathematics & 79 & 94 & 68 & 0 \\ 
            Mechanical & 78 & 28 & 0 & 12 \\ 
            Physics & 73 & 97 & 0 & 0 \\ 
            Oral Comprehension & 70 & 84 & 59 & 32 \\ 
            Reading Comprehension & 70 & 84 & 57 & 30 \\ 
            Oral Expression & 70 & 82 & 59 & 30 \\ 
            Deductive Reasoning & 70 & 79 & 64 & 29 \\ 
            Pure Mathematics & 70 & 79 & 59 & 14 \\ 
            Mathematical Reasoning & 68 & 84 & 59 & 16 \\ 
            Written Comprehension & 68 & 79 & 59 & 29 \\ 
            Complex Problem-Solving & 68 & 70 & 43 & 30 \\ 
            Computers and Electronics & 67 & 83 & 43 & 0 \\ 
            Natural Science & 66 & 82 & 5 & 0 \\ 
            Operations Analysis & 66 & 52 & 34 & 0 \\ 
            Active Learning & 63 & 79 & 52 & 29 \\ 
            Information Ordering & 63 & 71 & 52 & 30 \\ 
            Number Facility & 61 & 79 & 63 & 16 \\ 
            Critical Thinking & 61 & 70 & 57 & 30 \\ 
            Monitoring & 61 & 57 & 43 & 30 \\ 
            Written Expression & 59 & 80 & 57 & 23 \\ 
            Inductive Reasoning & 59 & 79 & 59 & 29 \\ 
            Category Flexibility & 59 & 73 & 50 & 29 \\ 
            Problem Sensitivity & 59 & 70 & 57 & 29 \\ 
            Judgment and Decision & 59 & 64 & 54 & 30 \\ 
            Originality & 57 & 75 & 37 & 25 \\ 
            Writing & 57 & 71 & 55 & 25 \\ 
            Native Language & 57 & 70 & 60 & 38 \\ 
            Active Listening & 57 & 68 & 57 & 32 \\ 
            Near Vision & 57 & 57 & 61 & 34 \\ 
            Administration and Management & 57 & 54 & 42 & 37 \\ 
            Quality Control Analysis & 57 & 46 & 11 & 29 \\ 
            Chemistry & 56 & 61 & 0 & 21 \\ 
            Fluency of Ideas & 55 & 73 & 41 & 27 \\ 
            Speaking & 55 & 68 & 57 & 30 \\ 
            Visualization & 55 & 61 & 25 & 29 \\ 
            Systems Analysis & 55 & 57 & 43 & 16 \\ 
            Systems Evaluation & 55 & 55 & 45 & 16 \\ 
            Speech Recognition & 54 & 57 & 50 & 30 \\ 
            Technology Design & 54 & 48 & 0 & 0 \\ 
            Operations Monitoring & 54 & 34 & 14 & 30 \\ 
            Instructing & 52 & 66 & 34 & 20 \\ 
            Time Management & 52 & 52 & 43 & 30 \\ 
            Troubleshooting & 52 & 29 & 0 & 29 \\ 
            Education and Training & 51 & 72 & 29 & 41 \\ 
            Administrative & 51 & 40 & 59 & 0 \\ 
            Learning Strategies & 50 & 71 & 36 & 18 \\ 
            Selective Attention & 50 & 57 & 41 & 30 \\ 
            Persuasion & 50 & 50 & 36 & 27 \\ 
            Coordination & 50 & 48 & 36 & 30 \\ 
            Production and Processing & 49 & 34 & 29 & 0 \\ 
            Speech Clarity & 48 & 71 & 45 & 27 \\ 
            Flexibility of Closure & 48 & 55 & 45 & 29 \\ 
            Perceptual Speed & 48 & 46 & 45 & 30 \\ 
            Customer and Personal Service & 48 & 37 & 39 & 38 \\ 
            Far Vision & 46 & 52 & 30 & 32 \\ 
            Management of Personnel Resources & 45 & 45 & 27 & 29 \\ 
            Speed of Closure & 43 & 55 & 41 & 29 \\ 
            Negotiation & 43 & 39 & 39 & 29 \\ 
            Social Perceptiveness & 41 & 45 & 41 & 29 \\ 
            Service Orientation & 41 & 43 & 43 & 29 \\ 
            Visual Color Discrimination & 41 & 41 & 21 & 29 \\ 
            Management of Material Resources & 41 & 32 & 13 & 13 \\ 
            Multitasking & 39 & 41 & 29 & 29 \\ 
            Management of Financial Resources & 39 & 30 & 27 & 11 \\ 
            Hearing Sensitivity & 39 & 30 & 16 & 29 \\ 
            Memorization & 37 & 48 & 34 & 21 \\ 
            Auditory Attention & 37 & 32 & 21 & 29 \\ 
            Telecommunications & 37 & 23 & 7 & 0 \\ 
            Equipment Selection & 37 & 21 & 0 & 29 \\ 
            Installation & 37 & 0 & 0 & 0 \\ 
            Building and Construction & 36 & 15 & 16 & 0 \\ 
            Personnel and Human Resources & 35 & 38 & 23 & 0 \\ 
            Public Safety and Security & 35 & 26 & 15 & 26 \\ 
            Programming & 34 & 55 & 14 & 0 \\ 
            Depth Perception & 34 & 30 & 5 & 29 \\ 
            Operation and Control & 34 & 9 & 0 & 30 \\ 
            Sales and Marketing & 33 & 20 & 29 & 0 \\ 
            Finger Dexterity & 32 & 29 & 25 & 34 \\ 
            Law and Government & 32 & 28 & 54 & 28 \\ 
            Reaction Time & 30 & 0 & 0 & 29 \\ 
            Transportation & 28 & 12 & 0 & 33 \\ 
            Communications and Media & 25 & 46 & 23 & 19 \\ 
            Economics and Accounting & 25 & 26 & 74 & 0 \\ 
            Arm-Hand Steadiness & 25 & 0 & 11 & 41 \\ 
            Repairing & 25 & 0 & 0 & 29 \\ 
            Equipment Maintenance & 23 & 0 & 0 & 29 \\ 
            Medicine and Dentistry & 23 & 0 & 0 & 0 \\ 
            Geography & 21 & 15 & 25 & 0 \\ 
            Biology & 20 & 16 & 0 & 0 \\ 
            Trunk Strength & 14 & 14 & 14 & 43 \\ 
            Manual Dexterity & 14 & 0 & 9 & 52 \\ 
            Wrist-Finger Speed & 9 & 0 & 0 & 21 \\ 
            Rate Control & 5 & 0 & 0 & 32 \\ 
            Spatial Orientation & 5 & 0 & 0 & 14 \\ 
            Psychology & 0 & 28 & 10 & 0 \\ 
            Therapy and Counseling & 0 & 21 & 0 & 0 \\ 
            Sociology and Anthropology & 0 & 0 & 16 & 0 \\ 
            History and Archeology & 0 & 0 & 13 & 0 \\ 
            Control Precision & 0 & 0 & 7 & 32 \\ 
            Extent Flexibility & 0 & 0 & 0 & 43 \\ 
            Static Strength & 0 & 0 & 0 & 39 \\ 
            Multilimb Coordination & 0 & 0 & 0 & 37 \\ 
            Stamina & 0 & 0 & 0 & 32 \\ 
            Gross Body Coordination & 0 & 0 & 0 & 30 \\ 
            Speed of Limb Movement & 0 & 0 & 0 & 30 \\ 
            Response Orientation & 0 & 0 & 0 & 29 \\ 
            Dynamic Strength & 0 & 0 & 0 & 25 \\ 
            Gross Body Equilibrium & 0 & 0 & 0 & 21 \\ 
            Glare Sensitivity & 0 & 0 & 0 & 21 \\ 
            Sound Localization & 0 & 0 & 0 & 14 \\ 
            Night Vision & 0 & 0 & 0 & 14 \\ 
            Peripheral Vision & 0 & 0 & 0 & 14 \\ 
            Explosive Strength & 0 & 0 & 0 & 11 \\ 
            Dynamic Flexibility & 0 & 0 & 0 & 0 \\ 
            Foreign Language & 0 & 0 & 0 & 0 \\ 
            Philosophy and Theology & 0 & 0 & 0 & 0 \\ 
            Fine Arts & 0 & 0 & 0 & 0 \\ 
            Food Production & 0 & 0 & 0 & 0 \\
        \bottomrule
    \end{longtblr}
}

% References
\addbibresource{references.bib}

% Title
\title{
    Introduction to Quantitative Career Matching:\\
    How to Statistically Pick the Right Occupation
}
\author{Cao Bittencourt}
\date{\today}

\begin{document}

\maketitle

\newpage

% Abstract
\begin{abstract}
    \noindent
    In this introductory paper, we address the crucial, but highly neglected, subject of data-driven career choice and development. We begin by sketching out the most basic approach to quantitative career matching, and test our model by calculating career compatibility coefficients with data from the United States' primary source of occupational information, the O*NET framework. We, then, illustrate the results with a sample of occupations from O*NET's database, and describe the best matches for each selected career. Despite the model's simplicity, we found career matches to be very accurate, although not as precise as they would be had we employed superior methods. In the future, we expect to cover some advanced material, deliberately left out of this introduction, and detail more sophisticated and performant models.\\

    \noindent
    \textbf{Keywords:} Career matching; Career compatibility; Career choice; Career development; Vocational choice; Occupational Information Network.
\end{abstract}

% Table of contents
\newpage
\tableofcontents

% List of tables
\newpage
\listoftables

% Introduction
\newpage
\section{Introduction}
Choosing the right occupation is one of the most important decisions in life.
In fact, a considerable portion of a person's life -- generally, a third of it
-- is actually spent working. Additionally, finding a career that is suitable
for one's skill set and natural aptitudes greatly impacts job performance and
overall satisfaction. Having sufficient compensation to feed a family, and
adequate work-life balance, are also crucial. And, for some individuals, we
could even cite self-fulfillment and ethical reasons as major factors
contributing to the significance of work.

Given this overarching importance of career choice and development, we would
expect to find many quantitative approaches to the issue in the scientific
literature. Nonetheless, despite some competent efforts by psychometricians and
personality theorists
\parencites[e.g.][]{holland1959theory, holland1997making}{schein1978careerdynamics,
    schein1993careeranchors}, not much has been done to address this problem in a
purely data-driven manner.

In the late 1990s, however, a notable achievement in this area was brought
about by the Occupational Information Network \parencite{onet1998data}. Sponsored by the United States Department of Labor, this
program set out to quantify job requirements in a standardized fashion. Their
research resulted in a robust database of career paths, which includes numeric
estimates of required competence levels for dozens of abilities; as well as
several other characteristics, such as the frequency of typical work
activities, the seriousness of expected job hazards, etc. Their catalog of
occupations is vast, and their data have been updated frequently for over two
decades.

But even with such rich information publicly available, there is still a lack
of statistical modeling of it. In this series of articles, we aim to fix this.
Starting with the most immediate concepts, we expect to document all
mathematical methods we have devised to tackle the study of career choice and
development. Here, we detail a basic approach to quantitative career matching
based on the estimation of compatibility scores. In the future, we shall cover
more complex matching algorithms, and advanced topics, like identifying
occupations' core attributes, assessing the generality of career paths (to
distinguish between generalist and specialist workers), calculating
employability coefficients, constructing optimal strategies for career
planning, talent acquisition, and much more.

% Methods
\section{Methods}
\subsection{The Occupational Information Network}
The Occupational Information Network (O*NET) is an American primary source of
occupational information. This program has been sponsored by the Employment and
Training Administration of the U.S. Department of Labor (USDOL/ETA) with the
aim of facilitating the development and maintenance of a skilled workforce \parencite{onet2024about}.

Essentially, the O*NET model provides a thorough, quantitative, description of
hundreds of SOC \parencite[Standard Occupational Classification, see][]{bls2018soc} career paths in
a vectorized format. O*NET accomplishes this task by systematically evaluating
the competencies of actual people in the labor market, on a 0 to 100 scale, in
order to derive numeric guidelines on what attributes are required for each
position.

Therefore, the resulting ``career profiles" from O*NET's surveys are meant to
be taken as ``canonical representations" of SOC occupations: they ``map out"
what, say, an engineer, an accountant, or any other occupation tends to ``look
like", in terms of professional attributes.

\subsection{Skill Sets}
With the convenient structure of the O*NET framework in mind, we begin by
defining and measuring the professional attributes that characterize each
economic agent in the labor market. To do so, we establish a bounded, uniform
scale from 0 (complete incompetence) to 100 (complete mastery), such that:
\EqnAttribute where $a_{i}^{k}$ is the $i$-th professional attribute of a
person $k$. Moreover, let us denote as a ``skill set", or ``career profile",
the collection of their $m$ attributes, like so: \EqnSkillSet

Thus, all individuals in the economy -- whether employed, or not -- are
mathematically described by their own skill sets, or career profiles; and this
applies to regular SOC occupations as well, like the ones included in the O*NET
database.

\subsection{A Euclidean Approach to Career Matching}
Finally, individuals and occupations are said to be similar to the exact degree
to which their vectors of professional attributes, or skill sets, overlap.

To measure this compatibility, we can utilize several matching methods. Some of
these are complicated, and require lengthy explanations, while others are much
easier to introduce. As the scope of this article is very limited, we only
cover the most basic of methods: the Euclidean approach to career matching.

\subsubsection{An Initial Model for Euclidean Career Matching}
In this approach to career matching, compatibility is estimated by the inverse
function of Euclidean distance. Hence, we first define this concept:
\EqnEuclideanDistance

Equation (\ref{eqn:euclidean_distance}) measures the absolute distance from the
professional attribute vector $\vb{a_{\boldsymbol{k}}}$ to a comparison vector
$\vb{a_{q}}$; in other words, it tells us how these two skill sets are
\textit{dissimilar}.

But, of course, our interest here is career compatibility, not its opposite.
Furthermore, similarity is typically expressed as a percentage; and for this,
we need dissimilarity to be bounded to a known interval. Therefore, in order to
convert the Euclidean distance (\ref{eqn:euclidean_distance}) to a similarity
metric, we need to employ some sort of normalization procedure, like the
following: \EqnNormalizedDistance

The denominator in equation (\ref{eqn:normalized_distance}) is the maximum
theoretical distance to the $\vb{a_q}$ comparison skill set. That is, here we
normalize distance (\ref{eqn:euclidean_distance}) by calculating the distance
to $\vb{a_q}$ from its most dissimilar vector. So, for each coordinate of the
$\vb{a_{q}}$ skill set, we measure which distance would be greater: that from
the scale's lower bound (viz. 0), or that from the upper bound (viz. 100), as
any other distance has to be less than those to the scale's limits. This way,
the denominator corresponds to the maximum distance to $\vb{a_{q}}$; and,
consequently, the normalized distance $\tilde{d}(\vb{a_{\boldsymbol{k}}},
    \vb{a_{q}}) \in [0,1] \ \forall \ \vb{a_{\boldsymbol{k}}}, \vb{a_{q}}$.

At last, similarity is easily derived from the normalized dissimilarity:
\EqnSimilarity

As opposed to equations (\ref{eqn:euclidean_distance}) and
(\ref{eqn:normalized_distance}), this formula describes the measure to which
career profiles are \textit{similar}. Again, it is evident
$s(\vb{a_{\boldsymbol{k}}}, \vb{a_{q}}) \in [0,1] \ \forall \
    \vb{a_{\boldsymbol{k}}}, \vb{a_{q}}$. Thus, we can estimate, as a percentage,
how compatible a person is with an occupation, and whether they are alike in
terms of their competencies, or not.

\subsubsection{A Weighted Euclidean Approach to Career Matching}
Although straightforward, the above method is far too simplistic, for each and
every competency is given the same importance in matching; and this is,
clearly, not reasonable. In reality, some professional attributes are,
definitely, more important to some occupations, while to others they are less
or even not important.

Consider, for instance, the absurdity of a barber who knows a lot about the
arts, and fashion, and entertaining customers, and even accounting and
bookkeeping, and yet does not know how to actually cut hair. One could call
them artistic, or stylish, but hardly a barber. For despite the usefulness of
all these other things for the barber, they are not nearly as important as
their main activity (viz. cutting hair).

This is the issue of weighting professional competencies, and it is crucial for
adequate career matching. In fact, to the degree to which attributes are said
to be ``central", or ``indispensable", to certain career paths (like being able
to cut hair is for a barber), so too weighting these attributes is
indispensable for career matching algorithms to function properly.

Now, as it is with matching methods, here also we find a variety of manners of
weighting professional attributes. The first, which we cover below, is to use
linear weights. This said, quadratic, logistic, or any other sort of increasing
weights can be employed as well. In our more complex career matching
algorithms, for example, we make use of a specific weighting function to
determine the ``indispensability" of human capital (i.e. competencies' relative
importance), and this vastly improves models' performance. For illustrative
purposes, however, linear weights seem to suffice.

\subsubsection{Linear Weights for Euclidean Career Matching}
When implementing linear weights with the Euclidean approach, each distance
between career profiles is multiplied by the attributes of the comparison skill
set (viz. that to which compatibility is to be calculated). Thus, distances to
the most important competencies are emphasized, while the remaining receive
less importance.

Mathematically, the weighted metrics (with subscript $w$) for similarity scores
and normalized distances are as follows: \EqnWeightedSimilarity where
\EqnWeightedDistance

With these adjusted equations, Euclidean distance is weighted proportionally to
the professional attribute levels of the $\vb{a_{q}}$ career profile. This
means that the matching algorithm optimizes itself for each comparison skill
set, as irrelevant competencies are left out of the analysis, and the
``indispensable" ones (e.g. cutting hair for a barber, dentistry for a dentist)
are given their full importance.

\subsubsection{Adjusting for Overqualification}
Another positive side effect of making use of weights in matching is an
implicit correction for the algorithmic penalties associated with
overqualification. For, if an individual has additional competencies from other
fields, which are not necessary at a certain position, the normalized
unadjusted distance (\ref{eqn:normalized_distance}) yields less favorable
matching results: \EqnDistanceDerivative

This overqualification problem is highly detrimental for matching accuracy, as
it tends to ``punish" people with many professional attributes, whether they
are skilled generalists or merely hobbyists. But, of course, as the absurd
scenario of the useless barber mentioned above, this too is not reasonable at
all. Indeed, no one would say, for instance, that an airline pilot is less of a
pilot if they also know how to cut hair, just as we do not say a barber is less
of a barber if they cannot fly an airplane. Put another way, competencies that
have ``nothing to do" with an occupation should not be a limiting factor to
career compatibility.

The application of matching weights helps to mitigate this, as the less
important competencies are, rightly, given less importance; and those distances
to completely irrelevant attributes (viz. with competence levels of 0) are even
nullified entirely, so that: \EqnUnweightedWeightedDistances

Therefore, if a person has additional unnecessary skills, these weighting
techniques provide a correction for the penalties of overqualification,
decreasing the normalized Euclidean distance and increasing compatibility.

\subsection{Data}
The version of the O*NET database utilized contains 873 unique career profiles,
even including a few variants on top of SOC occupations. For brevity's sake,
though, they are succinctly described below in terms of general clusters:
\GeneralOccupationalStatistics

As it concerns matching procedures themselves, career compatibility
coefficients were estimated for each one of the 873 available occupations.
Nevertheless, limitations of scope appropriate for an introductory article did
not allow for all these results to be displayed.

Hence, as the O*NET database contains too many career paths, for this exercise,
we only present four occupations. These are: mechanical engineers, physicists,
credit analysts, and dishwashers. Their main characteristics are summarized in
Table \ref{tab:summary_occupations}. \SummaryOccupations

The career profiles, or skill sets, used for matching were composed of three
broad categories of professional attributes from the O*NET framework, namely:
``Skills", ``Abilities", and ``Fields of Knowledge". These three categories
combined sum to exactly 120 competencies, and are detailed in the Appendix. The
remaining categories of attributes (e.g. ``Interests", ``Work Values") do not
constitute ``competencies" in the strict sense of the word, so they were not
included in the analysis. Additionally, a few of the original attribute names
were changed to reduce ambiguity (e.g. from ``Design" to ``Industrial Design").

% Results
\section{Results}

\subsection{Most Compatible Occupations}
Again, despite the small selection of four occupations, there are still too
many career matches to be displayed here. Thus, we only illustrate the results
of the weighted Euclidean approach with the ten most compatible occupations.
\MatchesEngineers \MatchesPhysicists \MatchesCreditAnalysts \MatchesDishwashers

\subsection{Similarity Matrix}
We also found it useful to build a similarity matrix, with which one can easily
compare occupations' compatibility coefficients: \SimilarityMatrix

As noted in the similarity matrix, comparison occupations are on the horizontal
axis, meaning that Table \ref{tab:similarity_matrix} should be read vertically.
This is because the normalization (\ref{eqn:normalized_distance}) and weighting
(\ref{eqn:weighted_distance}) techniques employed imply the characteristic
symmetry of Euclidean distance is, in most cases, lost.

Therefore, the table shows, for example, that mechanical engineers have 62\%
compatibility with dishwashers, while dishwashers have only 38\% compatibility
with mechanical engineers; that is to say engineers are sufficiently qualified
to perform the job activities of dishwashers, but not the other way around.

% Discussion
\section{Discussion}
Assessing the matching results in Tables
\ref{tab:similarity_engineers}--\ref{tab:similarity_dishwashers}, we could say
this Euclidean model is quite accurate: mechanical engineers are matched with
ten other types of engineers; while theoretical physicists have high similarity
with astronomers, biophysicists, mathematicians, and scientific roles, in
general; credit analysts are, adequately, matched to occupations in the
financial industry; and dishwashers, in turn, are compatible with janitors,
cleaners, and other so-called ``blue-collar" positions.

When we analyze Table \ref{tab:similarity_matrix}, we find results to be
likewise accurate. Firstly, the S.T.E.M. career paths are fairly close to one
another, with around 80\% similarity. This is expected, since engineers are, in
a way, applied physicists. Dishwashers, on the other hand, are not really
similar to either one, but are a little more compatible with credit analysts.
And, finally, credit analysts, have roughly 60\% similarity with the other
three occupations, indicating they are somewhat compatible, and
undifferentiated in this respect. It also stands out, as already mentioned,
that the more ``highly qualified" profiles have high compatibility scores with
the ``less qualified", or ``simpler", ones, while the reverse is not true.

As regards precision, however, this model is evidently suboptimal. For one
thing, similarity scores are too high, specially when comparing occupations
that have little or nothing to do with one another, as we have in this
exercise. That is, though the model's career matching predictions ``make
sense'', the degree of similarity, in some matches, is grossly overestimated.

Indeed, the compatibility of, say, dishwashers with scientific occupations is
relatively low, which is realistic; but, at almost 40\% similarity, it is
still, objectively, not low enough. And the same applies to credit analysts, as
well. For as an entry-level position in the field of Finance, credit analysts
do seem to be some sort of ``middle ground" between the other selected roles:
in fact, their job activities are mostly mathematical in nature, yet are
neither as demanding and abstract as those of physicists, nor as direct and
concrete as the repetitive manual labor of dishwashers. So, in this sense, the
estimated compatibility scores are correct. Nevertheless, the actual numeric
values of these similarities are, definitely, out of proportion; and more
reasonable figures would be in the range of 30 to 40\%, much lower than the
coefficients obtained with the Euclidean matching method.

A second, more encompassing, deficiency of this approach, that could even help
to explain the first one, has to do with factors that we deliberately left out
of the model. Educational attainment, for instance, is one of them, and it
could drastically change matching results, as several career paths have basic
requirements in the form of specific degrees, certifications, years of
experience, and so on and so forth.

Of course, these other aspects are all very important, and a full analysis of
career compatibility should take them into consideration. That being said, here
we decided to exclude them for two main reasons. The first is that adding such
variables alongside the 120 competencies without modification to the matching
method does not work, in purely practical terms, because the algorithm employed
is not suitable to perform binary filtering based on key, necessary conditions,
but rather to compute the distance between vectors.

The second reason is that these requirements are, actually, best conceptualized
as \textit{independent} coefficients themselves, that can, then, be multiplied,
or otherwise combined with any kind of function to limit career compatibility
scores. Moreover, skill set similarity, the sole object of this initial model,
does not imply -- or is even implied by -- educational attainment, years of
experience, etc. In other words, as these things are separate in the real
world, so too they should separate in statistical models. And this ``separation
of concerns" is more parsimonious and leads to a ``cleaner", theoretically
consistent, framework.

At last, and most importantly, we must repeat that this Euclidean approach to
career matching is an oversimplification for illustrative purposes; and that we
have already developed more advanced methods to account for its problems. In
fact, it is easy to derive better matching estimates only by tweaking the model
presented above.

An immediate improvement, for example, would be to adjust weighting procedures
to further emphasize the core competencies of comparison occupations by
substituting the linear weights with a polarizing, logistic-like function. And
one could also apply scaling functions to normalized Euclidean distances, in
order to correct the similarity metric, like so: \SimilarityScaling where
$f(.)$ is an increasing function of $\tilde{d}_w(\vb{a_{\boldsymbol{k}}},
    \vb{a_{q}})$, whether it be a square root, some linear interpolation, or any
appropriate scaling function.
% An immediate improvement, for example, would be to adjust weighting procedures
% to further emphasize the core competencies of comparison occupations by
% substituting the linear weights with a polarizing, logistic-like function. And
% one could also apply scaling functions to normalized Euclidean distances, in
% order to correct the similarity metric, like so: \SimilarityScaling where
% $f(.)$ is an increasing function of $\tilde{d}_w(\vb{a_{\boldsymbol{k}}},
%     \vb{a_{q}})$, whether it be a square root, some linear interpolation, or any
% other appropriate scaling function.

Having said this, beyond some ``quick fixes", more dramatic improvements
require rewriting the entire matching algorithm to another format. But then
again, the goal of this article was to provide an introduction to the crucial
issue of quantitative career matching, not to exhaust the subject. In future
works, we expect to be able to explain at length our sophisticated and
performant models.

% Conclusion
\section{Conclusion}
In this paper, we described the simplest approach to quantitative career
matching, and illustrated its application with a sample of data from a robust
source of occupational information, namely the O*NET framework. We also
discussed a few of the fundamental issues involved in assessing career path
compatibility, like competency weighting, and the problem of overqualification.
In an effort to keep the article concise and accessible, however, we
deliberately ignored many of the other complexities surrounding this subject.

Even so, we found our results to be mostly accurate, though not as precise as
they would be had we employed superior methods. We highlight that the sample
occupations were matched as ``expected": engineers with engineers; physicists
with scientists; credit analysts with other roles in the financial sector; and
dishwashers with janitors, and ``blue-collar" positions. Nevertheless, we
noted, as well, that similarity scores were too high for some of these matches.

To tackle this inconsistency, we proposed an ideal, parsimonious solution, in
which a number of independent coefficients ought to be calculated to separately
account for the various aspects affecting compatibility. In addition, we also
proposed a couple of more immediate ``quick fixes" to adjust matching
estimates.

Overall, as suggested by the article's title, this was an initial sketch to
address a most important topic: that of rationally choosing a career. And,
despite the basic model presented here not allowing to fully develop what
``picking the right occupation" looks like statistically, we do hope and think
it is, at least, a step in the right direction.

% References
\newpage
\printbibliography[
    heading=bibintoc,
    title={References}
]

% Appendices
\newpage
\begin{appendices}
    \section{Detailed Skill Sets}
    \DetailedSkillSets
\end{appendices}

\end{document}