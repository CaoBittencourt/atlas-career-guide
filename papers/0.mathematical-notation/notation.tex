%SECTION: SETUP
% region: document class 
% \documentclass{article}
\documentclass{elsarticle} %elsevier
% \documentclass[twocolumn]{elsarticle} %elsevier
% \documentclass{apa7} %american psychological association
% \documentclass[doc]{apa7} %american psychological association
% \documentclass{aomart} %annals of mathematics
% \documentclass{imsproc} %institute of mathematics and statistics
% \documentclass{nwejmart} %north-western european journal of mathematics
% \documentclass{paper} 
% \documentclass{sr-vorl} %springer book
% \documentclass{ecca} %economica
% \documentclass{ecta} %econometrica
% \documentclass{apecon} %applied economics

% endregion

% region: packages 
\usepackage{xr}
\usepackage{physics}
\usepackage{amsmath}
\usepackage{float}
\usepackage{enumitem}
\setlist[enumerate]{label*=\arabic*.}
\usepackage[hidelinks]{hyperref}

% endregion

% region: title 
% Title
\title{
    % The Career Atlas: Mathematical Notation
    The Career Atlas:\\
    Mathematical Notation
}
\author{Cao Bittencourt}
% \affiliation{
%     {B. Sc. in Economics from EPGE (FGV), RJ, Brazil.}\\
%     {Statistician at Atlas Career Guide Inc., FL, USA.}
% }
\affiliation{{B. Sc. in Economics from EPGE (FGV), RJ, Brazil.}}
\affiliation{{Statistician at Atlas Career Guide Inc., FL, USA.}}
\date{\today}

% endregion

% region: display breaks 
\allowdisplaybreaks

% endregion

%SECTION: PAPER
\begin{document}

% \maketitle

% region: abstract 
\begin{abstract}
    \noindent
    This is a brief document to define statistical methods for data-driven career choice and development. It deals with topics such as: career matching (i.e. vocational choice); estimation of competence, or overall skill level; estimation of skill set generality; versatility; skill set profitability; employability; labor market competitiveness; labor market taxonomy; optimal human resources acquistion and allocation; and so on and so forth. Each concept shall be explained at length in separate articles.

    \noindent
    \\ \textbf{Keywords:} Career choice; Career development; Matching algorithms; Competence; Similarity.
\end{abstract}
% endregion

% region: title 
\maketitle

% endregion

% region: table of contents 
\tableofcontents
\newpage
% endregion

% region: basic definitions 
\section{Basic Definitions}
\subsection{Skill Sets}
The $i$-th professional attribute, or competency, of a person $k$ is defined
as:
\begin{gather}
    a_{i}^{k}
    \in [0, 100]
    ,
\end{gather}
where the interval $[0,100]$ determines the bounds for every competency.\footnote[1]{
    More generally, these could be defined as $a_\text{lb}$ (the lower bound) and $a_\text{ub}$ (the upper bound). Here, the interval $[0,100]$ is used because of its ease of interpretation.
}

The skill set, or career profile, of a person $k$ is the vector of their $m$
attributes:
\begin{gather}
    \boldsymbol{a_k} = (a_{1}^{k}, \dots, a_{m}^{k})
    .
\end{gather}

The skill set matrix, or career profile matrix, is the collection of all $n$
skill sets in the economy:
\begin{gather}
    \textbf{A} =
    \begin{bmatrix}
        a_{1}^{1} & \hdots & a_{m}^{1} \\
        \vdots    & \ddots & \vdots    \\
        a_{1}^{n} & \hdots & a_{m}^{n} \\
    \end{bmatrix}
    .
\end{gather}

\subsection{Skill Set Normalization}
Normalization by the scale bounds is defined by the tilde operator:
\begin{gather}
    \tilde{a}_{i}^{k} =
    \frac{a_{i}^{k} - 0}{100 - 0} =
    \frac{a_{i}^{k}}{100}
    \in [0,1]
    ;\\
    \boldsymbol{\tilde{a}_{k}} =
    (\tilde{a}_{1}^{k}, \dots, \tilde{a}_{m}^{k})
    ;\\
    \textbf{Ã} =
    \begin{bmatrix}
        \tilde{a}_{1}^{1} & \hdots & \tilde{a}_{m}^{1} \\
        \vdots            & \ddots & \vdots            \\
        \tilde{a}_{1}^{n} & \hdots & \tilde{a}_{m}^{n} \\
    \end{bmatrix}
    .
\end{gather}

Normalization by a skill set's highest attribute is defined by the hat
operator:
\begin{gather}
    \hat{a}_{i}^{k} =
    \frac{a_{i}^{k}}{\max{a_{j}^{k}}}
    \in [0,1]
    ;\\
    \boldsymbol{\hat{a}_{k}} =
    (\hat{a}_{1}^{k}, \dots, \hat{a}_{m}^{k})
    ;\\
    \textbf{Â} =
    \begin{bmatrix}
        \hat{a}_{1}^{1} & \hdots & \hat{a}_{m}^{1} \\
        \vdots          & \ddots & \vdots          \\
        \hat{a}_{1}^{n} & \hdots & \hat{a}_{m}^{n} \\
    \end{bmatrix}
    .
\end{gather}

\subsection{Preferences}
\begin{enumerate}
    \item individual attribute preference $\upsilon_{i}^{k}$
    \item attribute preference vector $\boldsymbol{\upsilon_k}$
    \item attribute preference matrix $\boldsymbol{\Upsilon}$
    \item most attribute operations apply to preferences
    \item preference-adjusted skill sets
\end{enumerate}

% endregion

% region: basic skill set models 
\section{Basic Skill Set Models}
\subsection{Skill Set Generality}
The generality of a skill set is the mean of its maxima-normalized attributes:
\begin{gather}
    \gamma_{k} = \left(\frac{1}{m}\right)\sum_{i=1}^{m}{\hat{a}_{i}^{k}}
    \in [0,1]
    .
\end{gather}

People with high $\gamma_{k}$ scores are called \textit{generalists}.
Conversely, those with low $\gamma_{k}$ scores are called \textit{specialists}.
Career profiles that are neither broad nor specialized are said to be
\textit{balanced}.

The generality vector of all $n$ skill sets in the economy is:
\begin{gather}
    \boldsymbol{\gamma} = (\gamma_{1}, \dots, \gamma_{n})
    .
\end{gather}

\subsection{Attribute Equivalence}
The attribute equivalence of a particular attribute in a skill set measures the
importance of that attribute relative to the skill set's highest attribute. It
is calculated using skill set generality as both a midpoint and scaling
parameter in the following linear-logistic classification function:
\begin{gather}
    \text{aeq}(\hat{a}_{i}^{k}, \gamma_{k}) =
    \hat{a}_{i}^{k}
    \left[
        1 + \gamma_{k} (1 - \hat{a}_{i}^{k})
        \exp\bigg(
        % -\frac{\hat{a}_{i}^{k} - \gamma_{k}}{1 - \gamma_{k}}
        \frac{\hat{a}_{i}^{k} - \gamma_{k}}{\gamma_{k} - 1}
        % \frac{\gamma_{k} - \hat{a}_{i}^{k}}{1 - \gamma_{k}}
        \bigg)
        \right] ^ {
    % -\frac{1}{\hat{a}_{i}^{k}}
    -\frac{\gamma_{k}}{\hat{a}_{i}^{k}}
    % -{\gamma_{k} / \hat{a}_{i}^{k}}
    }
    \in [0,1]
    .
\end{gather}

For a short-hand notation, the attribute equivalence can be denoted by the
umlaut operator:
\begin{gather}
    \ddot{a}_{i}^{k} = \text{aeq}(\hat{a}_{i}^{k}, \gamma_{k})
    .
\end{gather}

This is not to be confused with the Netwonian dot notation for partial
derivatives, which we do not employ, instead preferring the more explicit
$\partialderivative{x}{y}$ derivative operator of Leibniz.

At any rate, attributes with high levels of $\ddot{a}_{i}^{k}$ are said to be
equivalent to the skill set's most important attribute. These are a career
profile's \textit{core} competencies. The remaining competencies are classified
as either \textit{important}, \textit{auxiliary}, \textit{minor}, or
\textit{unimportant}.

The attribute equivalence vector of a skill set is given by the collection of
their $m$ umlauted attributes:
\begin{gather}
    \boldsymbol{\ddot{a}_{k}} = (\ddot{a}_{1}^{k}, ..., \ddot{a}_{m}^{k})
    .
\end{gather}

Finally, the attribute equivalence matrix is the collection of all attribute
equivalence vectors in the economy:
\begin{gather}
    \textbf{Ä} =
    \begin{bmatrix}
        \ddot{a}_{1}^{1} & \hdots & \ddot{a}_{m}^{1} \\
        \vdots           & \ddots & \vdots           \\
        \ddot{a}_{1}^{n} & \hdots & \ddot{a}_{m}^{n} \\
    \end{bmatrix}
    .
\end{gather}

\subsection{Skill Set Competence}
The overall competence of a skill set is the mean of its scale-normalized
attributes, weighted by each attribute's importance (i.e. its attribute
equivalence):
\begin{gather}
    c_k =
    \frac{
    \sum_{i=1}^{m}{
    \ddot{a}_{i}^{k}
    \tilde{a}_{i}^{k}
    }
    }{
    \sum_{i=1}^{m}{
    \ddot{a}_{i}^{k}
    }
    }
    \in [0,1]
    .
\end{gather}

Career profiles with high $c_k$ are said to be competent. However, this
adjective can be seen as offensive to some people; and, most importantly, it
could also be misleading, because competence is often defined relative to the
specific requirements of a particular job. Therefore, we opt for the more
generic competence classification of \textit{high level}, \textit{mid level},
and \textit{low level}, which is somewhat less ambiguous.

The competence vector of all $n$ skill sets in the economy is:
\begin{gather}
    \boldsymbol{c} = (c_1, \dots, c_n)
    .
\end{gather}

% endregion

% region: additional definitions 
\section{Additional Definitions}
\subsection{Dumber Axiomatic Overview}
\begin{enumerate}
    \item Skill Set Sufficiency Axiom:
        \begin{enumerate}
            \item having the required skill level of an attribute is sufficient to perform its activities at that level
            \item skills are sufficient to perform activities
            \item $x \implies T_x$
            \item $x \iff T_x$
        \end{enumerate}
    \item Skill Set Composition Axiom:
        \begin{enumerate}
            \item rational economic agents can naturally compose multiple skills to accomplish complex tasks (i.e. tasks that require multiple skills)
            \item skills are composable to perform complex activities
            \item $x,y \iff T_{xy}$
            \item p.s.: it does not matter \textit{how} skills are composed to perform a complex activity. the point of the axiom is that any rational person which is also sufficiently qualified can ``figure out'' how to ``piece together'', that is compose, the required skills to perform a complex activity:
            \begin{gather*}
                (x,y \iff T_{xy}) \iff (\ x,y \iff T_{yx})
                .
            \end{gather*} 
        \end{enumerate}
    \item Occupational Reducibility Axiom:
        \begin{enumerate}
            \item from a practical standpoint, occupations are just a collection of job activities
            \item occupations can be reduced to their job activities
            \item $\boldsymbol{q} \iff \boldsymbol{T_q}$
        \end{enumerate}
    \item Occupational Skill-Sufficiency Corollary:
        \begin{enumerate}
            \item from axioms: Reducibility, Sufficiency
            \item as occupations are just a collection of job activities and skills are sufficient to perform activities, therefore occupations are just a collection of skills
            \item occupations can be reduced to their skill set
            \item $\boldsymbol{q} \iff \boldsymbol{T_q} \iff \boldsymbol{a_q}$
        \end{enumerate}
    \item Occupational Divisibility Corollary 1:
        \begin{enumerate}
            \item from axioms: Reducibility, Sufficiency, Composition
            \item as occupations are just a collection of simple and complex job activities, and skills are sufficient to perform activities, and rational agents can naturally compose skills to accomplish complex activities, therefore occupations can be ``broken down'' into ``suboccupations'', each with their own subset of the ``main'' occupation's activities and respective skill ``subsets''
            \item occupations can be divided into suboccupations, each with its own skill subset
            \item $
            \boldsymbol{q} \iff \boldsymbol{T_q} 
            \text{ and } 
            \boldsymbol{T_q} = \{T_x, T_y, T_z, ..., T_{xy}, T_{yx}, T_{xz}, T_{xzy}\}
            \implies \\
            \boldsymbol{T_{q_{xyz}}} = \{T_x, T_y, T_z, ..., T_{xy}, T_{yx}, T_{xz}, T_{xzy}\},\\
            \boldsymbol{T_{q_{xy}}} = \{T_x, T_y, ..., T_{xy}, T_{yx}\},\\
            \boldsymbol{T_{q_{xz}}} = \{T_x, T_z, ..., T_{xz}\},\\
            \boldsymbol{T_{q_{yz}}} = \{T_y, T_z, ...\},\\
            \boldsymbol{T_{q_x}} = T_x,\\
            \boldsymbol{T_{q_y}} = T_y,\\
            \boldsymbol{T_{q_z}} = T_z
            .
            $
            \item notice: we are not saying \textit{how} an occupation's complex activities are composed. what the corollary states is only that an occupation, which is just a collection of activities, can \textit{potentially} be divided into suboccupations, based on how their job activities are composed.
        \end{enumerate}
    \item Occupational Divisibility Corollary 2:
        \begin{enumerate}
            \item from corollary: Divisibility 1
            \item from axioms: Reducibility, Sufficiency, Composition
            \item to the extent to which an occupation is divisible, a sufficiently qualified skill set does not need to be a perfect match with an occupation's entire skill set in order to perform at least a portion of their job activities
            \item partial matching: people can perform isolated subsets of an occupation's job activities
            \item $
            \boldsymbol{q} \iff \boldsymbol{T_q}
            \text{ and } 
            \boldsymbol{T_q} = \{T_x, T_y, T_z, ..., T_{xy}, T_{yx}, T_{xz}, T_{xzy}\}
            \implies \\
            \boldsymbol{T_{q_{xyz}}} = \{T_x, T_y, T_z, ..., T_{xy}, T_{yx}, T_{xz}, T_{xzy}\} \implies (x,y,z \implies \boldsymbol{T_{q_{xyz}}}),\\
            \boldsymbol{T_{q_{xy}}} = \{T_x, T_y, ..., T_{xy}, T_{yx}\} \implies (x,y \implies \boldsymbol{T_{q_{xy}}}),\\
            \boldsymbol{T_{q_{xz}}} = \{T_x, T_z, ..., T_{xz}\} \implies (x,z \implies \boldsymbol{T_{q_{xz}}}),\\
            \boldsymbol{T_{q_{yz}}} = \{T_y, T_z, ...\} \implies (y,z \implies \boldsymbol{T_{q_{yz}}}),\\
            \boldsymbol{T_{q_x}} = T_x \implies (x \implies \boldsymbol{T_{q_{x}}}),\\
            \boldsymbol{T_{q_y}} = T_y \implies (y \implies \boldsymbol{T_{q_{y}}}),\\
            \boldsymbol{T_{q_z}} = T_z \implies (z \implies \boldsymbol{T_{q_{z}}})
            .
            $
            \item again, we are not stating \textit{how} an occupation is divided, only that, because of skill set sufficiency, composition, and occupational reducibility, occupations are, therefore, potentially divisible into suboccupations (with skill subsets), and to the extent to which they are divisible, therefore, any sufficiently qualified person can perform the skill subset for which they are qualified
        \end{enumerate}
    \item Occupational [] Axiom:
        \begin{enumerate}
            \item if necessary, define an axiom to address the issue above in the Divisibility Corollary
        \end{enumerate}
\end{enumerate}

\subsection{Axiomatic Overview}
\begin{enumerate}
    \item Skill Set Sufficiency Axiom:
          \begin{enumerate}
            \item pragmatic explanation: activities are just the application of skill. no skill, no activity.
            \item explanation: having a certain skill level is sufficient to perform an activity that requires that skill at that difficulty level
              \item intuition: action follows virtue, for virtue, itself, is the capacity to act
              \item mathematical definition (weak version):
                    \begin{gather*}
                        a_{i}^{k} \geq
                        \bar{a}_{i}^{\ell}
                        \implies \{
                        T_{i}^{1}, \dots, T_{i}^{\ell}
                        \}
                        ,\ {\ell} > 1
                        ,
                    \end{gather*}
                    where $a_{i}^{k}$ is the skill level person $k$ has at attribute $i$;
                    $a_{i}^{\ell}$ is the skill level of that attribute required to perform a task $T_{i}^{\ell}$ of difficulty level $\ell$;
                    and $\{T_{i}^{1}, \dots, T_{i}^{\ell}\}$ is the set of all that attribute's tasks of difficulty level less or equal to $\ell$.
              \item mathematical definition (strong version):
                    \begin{gather*}
                        a_{i}^{k} \geq
                        \bar{a}_{i}^{\ell}
                        \iff \{
                        T_{i}^{1}, \dots, T_{i}^{\ell}
                        \}
                        ,\ {\ell} > 1
                        .
                    \end{gather*}
              \item commentary: both versions of the axiom are reasonable.
              \item use: strong version.
          \end{enumerate}
          \item Skill Set Composition Axiom:
          \begin{enumerate}
            \item pragmatic explanation: rational economic agents naturally know how to compose (i.e. ``piece together'') multiple skills to accomplish complex tasks.
            \item explanation: just as activities that require one skill follow directly from having the required skill level, so too cross-functional, or composite, activities, which require multiple skills, follow directly from having the required skill level at each skill.
            \item intuition: 
              \item mathematical definition (weak version):
                    \begin{gather*}
                        ,
                    \end{gather*}
              \item mathematical definition (strong version):
                    \begin{gather*}
                        .
                    \end{gather*}
              \item commentary: .
              \item use: .
          \end{enumerate}
\end{enumerate}

\subsection{Fundamental Axioms}
In order for derive quantitative career choice and career development methods,
we need some basic assumptions.
% \begin{enumerate}
%     \item the difficulty of a task is \textit{measureable} in terms of the other tasks associated with a particular skill (i.e. tasks that require a skill level of $a_i \geq \bar{a}_1$ are objectively more difficult than tasks that require a skill level $\bar{a}_2 < \bar{a}_1$)
%     \item the skill scale is uniform
%     \begin{enumerate}
%         \item each competency has both easy and difficult levels
%         \item the difficulty of each scale is the same as the difficulty of the other scales
%         \item i.e. performing the tasks associated with attribute $a_i$ at the level $a_i = \bar{a}$ is, objectively, just as difficult as performing the tasks of attribute $a_j$ at the same $a_i = a_j = \bar{a}$ level
%         \item this has nothing to do with the distribution of task difficulty for each attribute: it only means that tasks at the same difficulty level are just as hard; it does not mean that some attributes are not more \textit{frequently} harder than other attributes
%     \end{enumerate}
%     \item the skill scale is truncated
%     \begin{enumerate}
%         \item 
%     \end{enumerate}
% \end{enumerate}

\subsubsection{Skill Set Sufficiency Axiom}
\begin{enumerate}
    \item SSSA (weak): skill $\implies$ activities?
    \item SSSA (strong): skill $\iff$ activities?
\end{enumerate}

there appears to be two ideas inside this and other axioms related to what
constitutes the (practical) essence of a skill set or occupation. are skills
sufficient to define an occupation?

strong version: skills are sufficient to define an occupation or career
profile. weak version: skills are not sufficient to define an occupation or
career profile semantically, but pragmatically skills are sufficient to define
an occupation, because skill $\iff$ activities and activities $\iff$
occupation.

The Skill Set Sufficiency Axiom (SSSA) defines that the activities a person is
capable of performing follow directly from their skill set. That is to say,
activities are essentially the application of an underlying capacity, so that
having a certain skill level in a competency necessarily translates into being
able to perform every task associated with that competency at that level (and,
of course, at the previous levels).

The most intuitive way to understand and mathematically guarantee the Skill Set
Sufficiency Axiom is by conceptualizing human capital as a sum of successes on
binary outcome variables representing tasks of progressive difficulty:
% \begin{gather}
%     a_{i}^{k} = \sum_{\ell=0}^{\ell_i}{_{0}^{1}[ T_{i}^{k} | T_{i}^{\ell}]}
%     ,\\
%     _{0}^{1}[ T_{i}^{k} | T_{i}^{\ell}] = 
%     \begin{cases}
%         1, \text{if $k$ succedes at difficulty level $\ell$;} \\
%         0, \text{otherwise.}
%     \end{cases}
% \end{gather}
\begin{gather}
    a_{i}^{k} = \sum_{\ell=0}^{\ell_i}T_{i_{\ell}}^{k}
    ,\\
    T_{i_{\ell}}^{k} =
    \begin{cases}
        1, \text{if $k$ succedes at difficulty level $\ell$;} \\
        0, \text{otherwise.}
    \end{cases}
\end{gather}

This, in turn, pressuposes the following:
\begin{enumerate}
    \item task difficulty is objectively measureable (see Fundamental Axioms)
    \item the skill scale is uniform (see Fundamental Axioms)
          \begin{enumerate}
              \item each competency has both easy and difficult levels
              \item the difficulty of each scale is the same as the difficulty of the other scales
              \item i.e. performing the tasks associated with attribute $a_i$ at the level $a_i =
                        \bar{a}_{\ell}$ is, objectively, just as difficult as performing the tasks of
                    attribute $a_j$ at the same $a_i = a_j = \bar{a}_{\ell}$ level
              \item this has nothing to do with the distribution of task difficulty for each
                    attribute: it only means that tasks at the same difficulty level are just as
                    hard; it does not mean that some attributes are not more \textit{frequently}
                    harder than other attributes
          \end{enumerate}
    \item the skill scale is truncated (see Fundamental Axioms)
    \item anyone who can perform a task of difficulty $\bar{a}_{\ell_1}$ can also perform
          a task of difficulty $\bar{a}_{\ell_2} \leq \bar{a}_{\ell_1}$ (see Fundamental
          Axioms)
    \item a person $k$ can only perform a task of difficulty $\bar{a}_{\ell}$ if they
          have a skill level of $a_{i}^{k} \geq \bar{a}_{\ell}$
\end{enumerate}

Thus, we can define a person's skill level as the sum of their successful
trials on tasks of increasing difficulty. And assuming the scales are
truncated, we can also interpret these metrics as the portion of tasks one is
able to perform out of all difficulty levels for that skill.

\begin{enumerate}
    \item e.g. skill level = 0: cannot perform even the most basic of tasks associated
          with that skill
    \item e.g. skill level = 1: can perform only the bottom 1\% of tasks associated with
          that skill (in order of difficulty)
    \item e.g. skill level = 10: can perform the bottom 10\% of tasks associated with
          that skill (in order of difficulty)
    \item e.g. skill level = 50: can perform the (easiest) half of the tasks associated
          with that skill
    \item e.g. skill level = 100: can perform all the tasks associated with that skill
\end{enumerate}

Additionally, we can interpret skill levels as the normalization of task
difficulty by the most objectively difficult task for each skill.
\begin{enumerate}
    \item e.g. skill level = 0: cannot perform even the most basic of tasks associated
          with that skill
    \item e.g. skill level = 1: can only perform tasks of up to 1\% the difficulty of
          that skill's most difficult task
    \item e.g. skill level = 10: can only perform tasks of up to 10\% the difficulty of
          that skill's most difficult task
    \item e.g. skill level = 50: can perform tasks of up to half of the difficulty of
          that skill's most difficult task
    \item e.g. skill level = 100: can perform all the tasks associated with that skill
\end{enumerate}

Now, because we assume the scales are truncated, this latter interpretation
implies, and is implied, by the former. For if a task is of the same difficulty
as another, then they are proportionately just as difficult in relation to that
skill's most difficult task (i.e. they require the same percentage of the
scale's upper limit), and, likewise, are also included in the same difficulty
``bracket" (i.e. they are equivalent to the same skill test in the previous
aggregate binary outcome interpretation).

So, however one decides to interpret skill levels, the conclusion remains the
same: to be skilled in an attribute is to be able to perform the activities
associated with that attribute. Put simply, the capacity to act follows virtue,
for virtue is, itself, the capacity to act.

\subsubsection{Skill Set Composition Axiom}
combination of skills into composite activities

if an activity is associated with more than one skill (i.e. it is a
cross-functional, or composite activity), then any rational person with the
required skill set can naturally ``piece the skills together'' in order to
perform the composite activity

in this particular sense, the whole is the sum of its parts.

\subsubsection{[Alternative/Additional] Skill Set [Unity] Axiom?}
probably conflicts with Occupational Divisibility Axiom / Skill Set Composition
Axiom (actually, this is not opposite to the Composition Axiom and, perhaps
even follows from it, but rather it appears to contradict the Divisibility
Axiom).

A skill set's activities can be grouped together into a single, coherent,
``general task'', which is the skill set's operation.

everything a skill set does is its operation, whole and indivisible.

Saint Thomas Aquinas: ``Act follows being.'' A thing's operation describes what
kind of thing it is.

e.g. a person who cuts hair professionally is called a barber. the operation of
a barber is to cut hair. additional skills are also part of the barber's
toolkit and are essential to a barber's operation in proportion to their
equivalence to the most important attribute (viz. cutting hair, which, itself,
is a composite activity, that, thus, requires cross-functional skills).

therefore, to the degree one can approximate the ``canonical representation''
of the ``barber's operation'', the more one can perform their job activities
and be hireable as a barber

?i.e. it is not $(a_{i}^{q}, a_{j}^{q})$, but rather $([a_{i}^{q} | a_{j}^{q}], [a_{j}^{q} | a_{i}^{q}])$.

``manual dexterity + arts + fashion + entertaining customers + etc etc = operation of a barber'' is a coherent whole, so that in this interpration (which may or may not contradict the other axioms), there is only one barber activity, namely ``barber stuff'', and a person can only do ``barber stuff'' if they are sufficiently qualified in all attributes required to perform ``barber stuff''. otherwise, they cannot output but a single unit of ``barber stuff'', or, alternatively, can only output imperfect approximations of ``barber stuff''.

in this interpretation (which, again, may or may not contradict the other
axioms), individual skills have to be simultaneously present in order to be
effective at producing the intended outcome.

likewise, if the same skills are present in the same quantitities, the outcome
is the same.

two occupations are only differentiable if they have different skills.

(perhaps Strong Skill Set Sufficiency Axiom is too strong)

in other words, to be able to do ``barber stuff'', it is not sufficient to have
``barber skills'': one has to be interchangeable with a barber (i.e.
sufficiently similar)

from this follows that $s_{kq}^{\text{hr}} = f(\textit{\ss}_{kq})$

in this particular sense, the whole is greater than the sum of its parts.

(this probably does not make sense. the reductionist approach is probably better.)

(or define a weak axiom that encompasses both of these definitions?)

\subsection{Occupational Axioms}
\subsubsection{Occupational Reducibility Axiom}
an occupation is nothing more than a series of job activities. corollary: an
occupation = its job activities $\implies$ an occupation = its skill set

\subsubsection{Occupational Divisibility Axiom}
since act follows virtue, and even composite activities follow from a
sufficiently qualified skill set, and since an occupation is nothing more than
a series of activities, then:

to have an overlap, or sufficient qualification, of $s_{kq}^{\geq}$ with an
occupation's skill set is to be able to perform $s_{kq}^{\geq}$ \% of that
occupation's job activities

if person $k$ has a sufficient qualification of $s_{kq}^{\geq} < 1$ with
occupation $q$, then they can partially perform occupation $q$'s job activities

if person $k$ has a sufficient qualification of $s_{kq}^{\geq} = 1$ with
occupation $q$, then they can completely perform occupation $q$'s job
activities

now, the Occupational Divisibility Axiom (ODA) states that as an occupation is
nothing but its skill set, then both its simple and composite activities can be
``broken down'' into skill \textit{subsets}. that is, an occupation is
``divisible'' into ``suboccupations'': a (uppercase) ``Job" is a series of
(lowercase) ``jobs''.

\subsubsection{Occupational Hireability Axiom}
and because an occupation is divisible into ``suboccupations'', and any skilled
person can perform a task, even a composite (cross-functional) task, if they
are sufficiently qualified, then any portion of an occupation (i.e. its
``suboccupations''), or skill set (i.e. its skill ``subsets''), can be
partially or completely outsourced to any sufficiently qualified person. in
other words, hireability does not require a perfect skill set match.

Weak Version of the Occupational Hireability Axiom (WOHA): provided a skill set
is sufficiently similar, etc etc, hireable, as an occupation, then a
proportional set of that occupation's job activities can be outsourced to the
alternative skill set

Strong Version of the Occupational Hireability Axiom (SOHA): any subset of an
occupation's activities can be outsourced to an alternative skill set to the
exact measure of the alternative skill set's qualification relative to that
occupation (e.g. if they have sqa of 0\%, then they cannot be hired to perform
not even the most basic task of that occupation; if they have sqa of 1\%, then
they can be hired to perform the bottom 1\% of that occupation's tasks; if they
have sqa of 50\%, then they can be hired to perform the easiest half of that
occupation's tasks; and so on and so forth).

The strong version of this axiom is not very realistic, for it removes all
occupational distinctions, and therefore ``liquifies'' all jobs in the labor
market into one big ``Job'', of which every occupation, with its respective
activities, is a part of. In other words, if SOHA were true, it would, then,
follow that every worker, and every action of every worker, on every job, in
every field, is interchangeable with any other action of any other worker, on
any other job, in any other field, provided they refer to the same skill at the
same difficulty level. In a way, this would defeat the purpose of the Skill Set
Composition Axiom, for every skill subset would, itself, be a subset of the
all-encompassing ``skill set" of ``The economy''. But, of course, this absurd.
So we do not employ the strong version of the Hireability Axiom, thus allowing
for qualitative distinctions between occupations.

The weak version of the Occupational Hireability Axiom, however, is quite
reasonable and, indeed, useful. [explain why it makes sense; does not
        ``dismember'' occupations]. Furthermore, some version of the Hireability Axiom
is required for us to write any theoretically coherent aggregate comparative
metrics, so as to allow us to go, for instance, from ``person $k$ is $x$
compatible with occupation $q$" to ``person $k$ is hireable at $x$ percent of
occupation $q$'s job posts''; and, likewise, to further aggregate such metrics
on the labor market level and, even, to perform simulations of competition in
the labor market.

Mathematically,

\begin{gather}
    _{0}^{1}[\text{hr}_{kq} \ |\ \bar{\text{hr}}_{kq}] =
    \begin{cases}
        1, \text{$\text{hr}_{kq} \geq \bar{\text{hr}}_{kq}$}; \\
        0, \text{otherwise}.
    \end{cases}
    ,
\end{gather}
and
\begin{gather}
    _{0}^{1}[\text{hr}_{kq}] =
    \begin{cases}
        1, \text{$\text{hr}_{kq} \geq 0.5$}; \\
        0, \text{otherwise}.
    \end{cases}
\end{gather}
are the binary indicators of the $\text{hr}_{kq} \in [0,1]$ hireability statistic. This metric tells us \textit{whether} a person $k$ can be hired to perform at least a portion of occupation $q$'s job activities (in accordance with the Divisibility Axiom).

Additionally, we define the hireability similarity metric $s_{kq}^{\text{hr}}
    \in [0,1]$, which determines the \textit{percentage} of a job's activities one
could be hired to perform.

When multiplied, these two coefficients quantify if, and to what extent, a
skill set can be hired to perform the job activities of another skill set:
\begin{gather}
    _{0}^{1}[\text{hr}_{kq} \ |\ \bar{\text{hr}}_{kq}] \times s_{kq}^{\text{hr}}
    \in [0,1]
    .
\end{gather}

The ``radical'' SOHA defines these metrics as:
\begin{gather}
    \text{hr}_{kq} = 1 \ \forall \ kq;\\
    s_{kq}^{\text{hr}} = s_{kq}^{\geq}
    \in [0,1]
    .
\end{gather}

However, the more reasonable WOHA does not make any strong assumptions about
these coefficients, but only that they should be weaker than the SOHA's
pressupositions:
\begin{gather}
    \text{hr}_{kq} = h(\boldsymbol{a_{k}}, \boldsymbol{a_{q}}, ...);\\
    s_{kq}^{\text{hr}} \leq s_{kq}^{\geq}
    \in [0,1]
    ,
\end{gather}
where $h(.)$ is any monotonic hireability function, a much weaker pressuposition than the SOHA's hireability \textit{constant} $\text{hr}_{kq} = 1$. In addition, the second assumption is that a skill set can be hired to perform \textit{up to} $s_{kq}^{\geq}$ of another skill set's job activities, but not necessarily all $s_{kq}^{\geq}$ percent of them. That is, they cannot be hired to perform more activities than they are capable of performing. Furthermore, additional assumptions can be made about the hireability threshold $\bar{\text{hr}}_{kq}$.

Example metrics that satisfy the Weak Hireability Axiom are:
\begin{gather}
    \text{hr}_{kq} =
    s_{kq}^{\geq} \times
    \ddot{\tau}_{kq}^{\theta};\\
    s_{kq}^{\text{hr}} = \textit{\ss}_{kq}
\end{gather}

\subsubsection{Occupational Aggregation Axiom}
``person $k$ is $s_{kq}$ compatible with occupation $q$" to ``person $k$ is hireable at $s_{kq}^{h}$ percent of occupation $q$'s job posts''

% endregion

% region: comparative models 
\section{Comparative Models}
\subsection{Pairwise Comparative Models}
\subsubsection{Similarity}
The most basic comparative model is that of Euclidean matching with linear
weights:
\begin{gather}
    s_{kq} =
    s(\boldsymbol{a_k}, \boldsymbol{a_q}) =
    1 - \tilde{d}(\boldsymbol{a_k}, \boldsymbol{a_q})
    \in [0,1]
    ,
\end{gather}
where
\begin{gather}
    \tilde{d}_{kq} =
    \tilde{d}(\boldsymbol{a_k}, \boldsymbol{a_q}) =
    \sqrt{
    \frac{
    \sum_{i = 1}^{m}{
    a_{i}^{q} (a_{i}^{k} - a_{i}^{q})^2
    }
    }{
    \sum_{i = 1}^{m}{
    a_{i}^{q} \max(100 - a_{i}^{q}, a_{i}^{q})^2
    }
    }
    }
    \in [0,1]
    .
\end{gather}

In this model, we compare a skill set $\boldsymbol{a_k}$ to a skill set
$\boldsymbol{a_q}$ by calculating the weighted Euclidean distance from
$\boldsymbol{a_k}$ to $\boldsymbol{a_q}$ normalized by the maximum theoretical
distance to $\boldsymbol{a_q}$.

Other weighting systems can be employed in this type of matching model. We
could, for instance, substitute the linear weights with either quadratic
weights,
\begin{gather}
    {a_{i}^{q}} ^ 2
    \in [0,1]
    ,
\end{gather}
or speciality-root weights,
\begin{gather}
    {a_{i}^{q}} ^ {
        \frac{1}{1 - \gamma_k}
    }
    \in [0,1]
    .
\end{gather}

But the best and most interpretable results are obtained using attribute
equivalence as the weighting function:
\begin{gather}
    \tilde{d}_{kq} =
    \tilde{d}(\boldsymbol{a_k}, \boldsymbol{a_q}) =
    \sqrt{
    \frac{
    \sum_{i = 1}^{m}{
    \ddot{a}_{i}^{q} (a_{i}^{k} - a_{i}^{q})^2
    }
    }{
    \sum_{i = 1}^{m}{
    \ddot{a}_{i}^{q} \max(100 - a_{i}^{q}, a_{i}^{q})^2
    }
    }
    }
    \in [0,1]
    .
\end{gather}

We could also employ other matching methods instead of the ``baseline" weighted
Euclidean model. [detail each method later]:
\begin{enumerate}
    \item logit regression matching
    \item probit regression matching
    \item bvls regression matching
    \item tobit regression matching
    \item pearson correlation matching
    \item kendal nonparametric correlation matching
    \item spearman nonparametric correlation matching
\end{enumerate}

At last, similarity and normalized distance metrics determine the respective
vectors and matrices, as follows:
\begin{gather}
    \boldsymbol{s_k} = (s_{k,1}, \dots, s_{k,n})
    ;\\
    \boldsymbol{\tilde{d}_k} = (\tilde{d}_{k,1}, \dots, \tilde{d}_{k,n})
    ;\\
    \textbf{S} =
    \begin{bmatrix}
        s_{1,1} & \hdots & s_{n,1} \\
        \vdots  & \ddots & \vdots  \\
        s_{1,n} & \hdots & s_{n,n} \\
    \end{bmatrix} =
    \begin{bmatrix}
        1       & \hdots & s_{k,1} & \hdots & s_{n,1} \\
        \vdots  & \ddots & \vdots  & \ddots & \vdots  \\
        s_{1,k} & \hdots & 1       & \hdots & s_{n,k} \\
        \vdots  & \ddots & \vdots  & \ddots & \vdots  \\
        s_{1,n} & \hdots & s_{k,n} & \hdots & 1       \\
    \end{bmatrix}
    ;\\
    \textbf{D} =
    \begin{bmatrix}
        \tilde{d}_{1,1} & \hdots & \tilde{d}_{n,1} \\
        \vdots          & \ddots & \vdots          \\
        \tilde{d}_{1,n} & \hdots & \tilde{d}_{n,n} \\
    \end{bmatrix} =
    \begin{bmatrix}
        0               & \hdots & \tilde{d}_{k,1} & \hdots & \tilde{d}_{n,1} \\
        \vdots          & \ddots & \vdots          & \ddots & \vdots          \\
        \tilde{d}_{1,k} & \hdots & 0               & \hdots & \tilde{d}_{n,k} \\
        \vdots          & \ddots & \vdots          & \ddots & \vdots          \\
        \tilde{d}_{1,n} & \hdots & \tilde{d}_{k,n} & \hdots & 0               \\
    \end{bmatrix}
    .
\end{gather}

\subsubsection{Qualification}
A closely related concept to matching is the qualification comparative model.
In this family of functions, however, Euclidean matching is mandatory, as other
matching methods do not make sense for this specific type of calculation. The
reason for this is at that, here, we are not particularly interested in
matching (i.e. a typical classification problem), but rather in the actual
distances between comparison skill sets.

To define these models, we first have to define the gap function, which
measures only positive competency gaps:
\begin{gather}
    \delta_{kq}^{i} =
    \delta(a_{i}^{k}, a_{i}^{q}) =
    \max(
    a_{i}^{k} - a_{i}^{q}
    , 0
    )
    \in [0, 100]
    .
\end{gather}

Having defined the gap function, we can write the underqualification model:
\begin{gather}
    \tilde{\delta}_{kq}^{<} =
    \text{uqa}(\boldsymbol{a_k}, \boldsymbol{a_q}) =
    \sqrt{
    \frac{
    \sum_{i=1}^{m}{
    \ddot{a}_{i}^{q} {\delta(a_{i}^{q}, a_{i}^{k})} ^ 2
    }
    }{
    \sum_{i=1}^{m}{
    \ddot{a}_{i}^{q} {\delta(a_{i}^{q}, 0)} ^ 2
    }
    }
    } =
    \sqrt{
    \frac{
    \sum_{i=1}^{m}{
    \ddot{a}_{i}^{q} {\delta(a_{i}^{q}, a_{i}^{k})} ^ 2
    }
    }{
    \sum_{i=1}^{m}{
    \ddot{a}_{i}^{q} {a_{i}^{q}} ^ 2
    }
    }
    }
    .
\end{gather}

And, analogously, the overqualification model is given by:
\begin{gather}
    \tilde{\delta}_{kq}^{\geq} =
    \text{oqa}(\boldsymbol{a_k}, \boldsymbol{a_q}) =
    \sqrt{
    \frac{
    \sum_{i=1}^{m}{
    \ddot{a}_{i}^{q} {\delta(a_{i}^{k}, a_{i}^{q})} ^ 2
    }
    }{
    \sum_{i=1}^{m}{
    \ddot{a}_{i}^{q} {\delta(100, a_{i}^{q})} ^ 2
    }
    }
    } =
    \sqrt{
    \frac{
    \sum_{i=1}^{m}{
    \ddot{a}_{i}^{q} {\delta(a_{i}^{k}, a_{i}^{q})} ^ 2
    }
    }{
    \sum_{i=1}^{m}{
    \ddot{a}_{i}^{q} {(100 - a_{i}^{q})} ^ 2
    }
    }
    }
    .
\end{gather}

The final set of ``sufficient qualification" is, evidently, the complement of
the underqualification model:
\begin{gather}
    s_{kq}^{\geq} =
    \text{sqa}(\boldsymbol{a_k}, \boldsymbol{a_q}) =
    1 -
    \sqrt{
    \frac{
    \sum_{i=1}^{m}{
    \ddot{a}_{i}^{q} {\delta(a_{i}^{q}, a_{i}^{k})} ^ 2
    }
    }{
    \sum_{i=1}^{m}{
    \ddot{a}_{i}^{q} {\delta(a_{i}^{q}, 0)} ^ 2
    }
    }
    } =
    1 - \text{uqa}(\boldsymbol{a_k}, \boldsymbol{a_q})
    .
\end{gather}

As with the similarity and normalized distance statistics described above,
these three qualification models are bounded to the $[0,1]$ interval. Likewise,
they also determine qualification vectors:
\begin{gather}
    \boldsymbol{\tilde{\delta}_{k}^{<}} = (\tilde{\delta}_{k,1}^{<}, \dots, \tilde{\delta}_{k,n}^{<})
    ;\\
    \boldsymbol{\tilde{\delta}_{k}^{\geq}} = (\tilde{\delta}_{k,1}^{\geq}, \dots, \tilde{\delta}_{k,n}^{\geq})
    ;\\
    \boldsymbol{s_{k}^{\geq}} = (s_{k,1}^{\geq}, \dots, s_{k,n}^{\geq})
    ;
\end{gather}
and matrices
\begin{gather}
    \textbf{$\tilde{\Delta}$}_{\boldsymbol{<}} =
    \begin{bmatrix}
        \tilde{\delta}_{1,1}^{<} & \hdots & \tilde{\delta}_{n,1}^{<} \\
        \vdots                   & \ddots & \vdots                   \\
        \tilde{\delta}_{1,n}^{<} & \hdots & \tilde{\delta}_{n,n}^{<} \\
    \end{bmatrix} =
    \begin{bmatrix}
        0                        & \hdots & \tilde{\delta}_{k,1}^{<} & \hdots & \tilde{\delta}_{n,1}^{<} \\
        \vdots                   & \ddots & \vdots                   & \ddots & \vdots                   \\
        \tilde{\delta}_{1,k}^{<} & \hdots & 0                        & \hdots & \tilde{\delta}_{n,k}^{<} \\
        \vdots                   & \ddots & \vdots                   & \ddots & \vdots                   \\
        \tilde{\delta}_{1,n}^{<} & \hdots & \tilde{\delta}_{k,n}^{<} & \hdots & 0                        \\
    \end{bmatrix}
    ;\\
    \textbf{$\tilde{\Delta}$}_{\boldsymbol{\geq}} =
    \begin{bmatrix}
        \tilde{\delta}_{1,1}^{\geq} & \hdots & \tilde{\delta}_{n,1}^{\geq} \\
        \vdots                      & \ddots & \vdots                      \\
        \tilde{\delta}_{1,n}^{\geq} & \hdots & \tilde{\delta}_{n,n}^{\geq} \\
    \end{bmatrix} =
    \begin{bmatrix}
        0                           & \hdots & \tilde{\delta}_{k,1}^{\geq} & \hdots & \tilde{\delta}_{n,1}^{\geq} \\
        \vdots                      & \ddots & \vdots                      & \ddots & \vdots                      \\
        \tilde{\delta}_{1,k}^{\geq} & \hdots & 0                           & \hdots & \tilde{\delta}_{n,k}^{\geq} \\
        \vdots                      & \ddots & \vdots                      & \ddots & \vdots                      \\
        \tilde{\delta}_{1,n}^{\geq} & \hdots & \tilde{\delta}_{k,n}^{\geq} & \hdots & 0                           \\
    \end{bmatrix}
    ;\\
    \textbf{S}_{\boldsymbol{\geq}} =
    \begin{bmatrix}
        s_{1,1}^{\geq} & \hdots & s_{n,1}^{\geq} \\
        \vdots         & \ddots & \vdots         \\
        s_{1,n}^{\geq} & \hdots & s_{n,n}^{\geq} \\
    \end{bmatrix} =
    \begin{bmatrix}
        1              & \hdots & s_{k,1}^{\geq} & \hdots & s_{n,1}^{\geq} \\
        \vdots         & \ddots & \vdots         & \ddots & \vdots         \\
        s_{1,k}^{\geq} & \hdots & 1              & \hdots & s_{n,k}^{\geq} \\
        \vdots         & \ddots & \vdots         & \ddots & \vdots         \\
        s_{1,n}^{\geq} & \hdots & s_{k,n}^{\geq} & \hdots & 1              \\
    \end{bmatrix}
    .
\end{gather}

P.S.: should qualification be unweighted?
\begin{gather}
    \tilde{\delta}_{kq}^{<} =
    \text{uqa}(\boldsymbol{a_k}, \boldsymbol{a_q}) =
    \sqrt{
    \frac{
    \sum_{i=1}^{m}{
    {\delta(a_{i}^{q}, a_{i}^{k})} ^ 2
    }
    }{
    \sum_{i=1}^{m}{
    {\delta(a_{i}^{q}, 0)} ^ 2
    }
    }
    } =
    \sqrt{
    \frac{
    \sum_{i=1}^{m}{
    {\delta(a_{i}^{q}, a_{i}^{k})} ^ 2
    }
    }{
    \sum_{i=1}^{m}{
    {a_{i}^{q}} ^ 2
    }
    }
    }
    ;\\
    \tilde{\delta}_{kq}^{\geq} =
    \text{oqa}(\boldsymbol{a_k}, \boldsymbol{a_q}) =
    \sqrt{
    \frac{
    \sum_{i=1}^{m}{
    {\delta(a_{i}^{k}, a_{i}^{q})} ^ 2
    }
    }{
    \sum_{i=1}^{m}{
    {\delta(100, a_{i}^{q})} ^ 2
    }
    }
    } =
    \sqrt{
    \frac{
    \sum_{i=1}^{m}{
    {\delta(a_{i}^{k}, a_{i}^{q})} ^ 2
    }
    }{
    \sum_{i=1}^{m}{
    {(100 - a_{i}^{q})} ^ 2
    }
    }
    }
    ;\\
    s_{kq}^{\geq} =
    \text{sqa}(\boldsymbol{a_k}, \boldsymbol{a_q}) =
    1 -
    \sqrt{
    \frac{
    \sum_{i=1}^{m}{
    {\delta(a_{i}^{q}, a_{i}^{k})} ^ 2
    }
    }{
    \sum_{i=1}^{m}{
    {\delta(a_{i}^{q}, 0)} ^ 2
    }
    }
    } =
    1 - \text{uqa}(\boldsymbol{a_k}, \boldsymbol{a_q})
    .
\end{gather}

Or even

\begin{gather}
    \tilde{\delta}_{kq}^{<} =
    \text{uqa}(\boldsymbol{a_k}, \boldsymbol{a_q}) =
    \frac{
    \sum_{i=1}^{m}{
    {\delta(a_{i}^{q}, a_{i}^{k})}
    }
    }{
    \sum_{i=1}^{m}{
    {\delta(a_{i}^{q}, 0)}
    }
    } =
    \frac{
    \sum_{i=1}^{m}{
    {\delta(a_{i}^{q}, a_{i}^{k})}
    }
    }{
    \sum_{i=1}^{m}{
    {a_{i}^{q}}
    }
    }
    ;\\
    \tilde{\delta}_{kq}^{\geq} =
    \text{oqa}(\boldsymbol{a_k}, \boldsymbol{a_q}) =
    \frac{
    \sum_{i=1}^{m}{
    {\delta(a_{i}^{k}, a_{i}^{q})}
    }
    }{
    \sum_{i=1}^{m}{
    {\delta(100, a_{i}^{q})}
    }
    } =
    \frac{
    \sum_{i=1}^{m}{
    {\delta(a_{i}^{k}, a_{i}^{q})}
    }
    }{
    \sum_{i=1}^{m}{
    {(100 - a_{i}^{q})}
    }
    }
    ;\\
    s_{kq}^{\geq} =
    \text{sqa}(\boldsymbol{a_k}, \boldsymbol{a_q}) =
    1 -
    \frac{
    \sum_{i=1}^{m}{
    {\delta(a_{i}^{q}, a_{i}^{k})}
    }
    }{
    \sum_{i=1}^{m}{
    {\delta(a_{i}^{q}, 0)}
    }
    } =
    1 - \text{uqa}(\boldsymbol{a_k}, \boldsymbol{a_q})
    ?
\end{gather}

P.S.: unweighted qualification metrics can be overly strict. probably use
umlauted weights when calculating qualification for the hireability
coefficient. however, absolute qualification metrics (i.e. unweighted) can be
useful, specially in career development. the strictness of unweighted
qualification only becomes a probably in assessing hireability. therefore, it
seems appropriate to make a weighted / unweighted qualification metric
distinction. for example:
\begin{gather}
    \ddot{s}_{kq}^{\geq} =
    1 -
    \sqrt{
    \frac{
    \sum_{i=1}^{m}{
    \ddot{a}_{i}^{q} {\delta(a_{i}^{q}, a_{i}^{k})} ^ 2
    }
    }{
    \sum_{i=1}^{m}{
    \ddot{a}_{i}^{q} {a_{i}^{q}} ^ 2
    }
    }
    } =
    1 - \ddot{\delta}_{kq}^{<}
    \\
    s_{kq}^{\geq} =
    \text{sqa}(\boldsymbol{a_k}, \boldsymbol{a_q}) =
    1 -
    \frac{
    \sum_{i=1}^{m}{
    {\delta(a_{i}^{q}, a_{i}^{k})}
    }
    }{
    \sum_{i=1}^{m}{
    a_{i}^{q}
    }
    } =
    1 - \tilde{\delta}_{kq}^{<}
\end{gather}
or, instead, define the weighted model as the default, and the unweighted as the ``scale-normalized'' model (as it is normalized by the actual theoretical limit, not the weighted theoretical limit):
\begin{gather}
    s_{kq}^{\geq} =
    \text{sqa}(\boldsymbol{a_k}, \boldsymbol{a_q}, \boldsymbol{\ddot{a}_k}, \boldsymbol{\ddot{a}_q}) =
    1 -
    \sqrt{
    \frac{
    \sum_{i=1}^{m}{
    \ddot{a}_{i}^{q} {\delta(a_{i}^{q}, a_{i}^{k})} ^ 2
    }
    }{
    \sum_{i=1}^{m}{
    \ddot{a}_{i}^{q} {a_{i}^{q}} ^ 2
    }
    }
    } =
    1 - \delta_{kq}^{<}
    \\
    \tilde{s}_{kq}^{\geq} =
    \text{sqa}(\boldsymbol{a_k}, \boldsymbol{a_q}) =
    1 -
    \frac{
    \sum_{i=1}^{m}{
    {\delta(a_{i}^{q}, a_{i}^{k})}
    }
    }{
    \sum_{i=1}^{m}{
    a_{i}^{q}
    }
    } =
    1 - \tilde{\delta}_{kq}^{<}
\end{gather}

\subsubsection{Interchangeability}
\begin{enumerate}
    \item sufficient similarity
    \item scaled similarity
    \item similarity in the strong sense
    \item s.s. $\rightarrow$ $\textit{\ss}_{kq}$
    \item alternatively, $\textit{\ss}_{kq} = \text{seq(...)} = \ddot{s}_{kq}$
    \item interchangeability vector $\boldsymbol{\textbf{\textit{\ss}}_k}$
    \item interchangeability matrix $\boldsymbol{\textbf{\ss}}$
\end{enumerate}
alternatively,
\begin{align*}
    \textit{\ss}_{kq} & =
    \text{seq}(\boldsymbol{a_k}, \boldsymbol{a_q}) =
    s_{kq} \left\{
    1 + c_q \tilde{d}_{kq} \exp\left[
        -\frac{
            d_{\max}^{q}
            (\tilde{d}_{kq} - c_q)
            % - (\tilde{d}_{kq} - c_q)
            % d_{\max}^{q}
        }{
            1 - c_q
        }
        \right]
    \right\} ^ {-\frac{c_q}{\tilde{d}_{kq}}} \\
                       & =
    s_{kq} \left\{
    1 + c_q (1 - s_{kq}) \exp\left[
        \frac{
            c_q d_{\max}^{q} - d_{kq}
            % c_q d_{\text{ub}}^{q} - d_{kq}
            % c_q \bar{d}_{q} - d_{kq}
        }{
            1 - c_q
        }
        \right]
    \right\} ^ {
    - c_q / \tilde{d}_{kq}
    }                                         \\
                       & =
    (1 - \tilde{d}_{kq})
    \times
    \left[
        1 + c_q \tilde{d}_{kq} \exp\left(
        \frac{
            % c_q d_{\max}^{q} - d_{kq}
            % c_q d_{\text{ub}}^{q} - d_{kq}
            c_q \bar{d}_{q} - d_{kq} }{ 1 - c_q } \right) \right] ^ { - c_q /
    \tilde{d}_{kq} }                         \\  & = (1 - \tilde{d}_{kq}) \left[ 1 + c_q \tilde{d}_{kq}
        \exp\left( \frac{
               % c_q d_{\max}^{q} - d_{kq}
               % c_q d_{\text{ub}}^{q} - d_{kq}
               c_q \bar{d}_{q} - d_{kq} }{ 1 - c_q } \right) \right] ^ { - c_q /
    \tilde{d}_{kq} }                         \\  & = s_{kq} \left[ 1 + c_q \tilde{d}_{kq} \exp\left(
           \frac{
               % c_q d_{\max}^{q} - d_{kq}
               % c_q d_{\text{ub}}^{q} - d_{kq}
               c_q \bar{d}_{q} - d_{kq} }{ 1 - c_q } \right) \right] ^ { - c_q /
    \tilde{d}_{kq} }                         \\  & = s_{kq} \left[ 1 + c_q \tilde{d}_{kq} \exp\left(
           \frac{
               % c_q d_{\max}^{q} - d_{kq}
               c_q d_{\text{ub}}^{q} - d_{kq}
               % c_q d_{q}^{\text{ub}} - d_{kq}
               % c_q d_{q}^{ub} - d_{kq}
               % c_q d_{ub}^{q} - d_{kq}
               % c_q \bar{d}_{q} - d_{kq}
               }{
               1 - c_q
               }
           \right)
        \right] ^ {
           - c_q / \tilde{d}_{kq}
    }                                         \\
                       & =
    s_{kq} \left\{
    1 + c_q \tilde{d}_{kq} \exp\left[
        -\frac{
            \bar{d}_{q}
            (\tilde{d}_{kq} - c_q)
            % - (\tilde{d}_{kq} - c_q)
            % d_{\max}^{q}
        }{
            1 - c_q
        }
        \right]
    \right\} ^ {-\frac{c_q}{\tilde{d}_{kq}}} \\
                       & =
    s_{kq} \left\{
    1 + c_q \tilde{d}_{kq} \exp\left[
        \frac{
            \bar{d}_{q}
            (c_q - \tilde{d}_{kq})
            % - (\tilde{d}_{kq} - c_q)
            % d_{\max}^{q}
        }{
            1 - c_q
        }
        \right]
    \right\} ^ {-\frac{c_q}{\tilde{d}_{kq}}} \\
                       & =
    s_{kq} \left\{
    1 + c_q \tilde{d}_{kq} \exp\left[
        \frac{
            \bar{d}_{q}
            (c_q - \tilde{d}_{kq})
            % - (\tilde{d}_{kq} - c_q)
            % d_{\max}^{q}
        }{
            1 - c_q
        }
        \right]
    \right\} ^ {- c_q / \tilde{d}_{kq}}      \\
                       & =
    s_{kq} \left\{
    1 + c_q (1 - s_{kq}) \exp\left[
        (
        % c_q d_{\max}^{q} - d_{kq}
        % c_q d_{\text{ub}}^{q} - d_{kq}
        c_q \bar{d}_{q} - d_{kq} ) / (1 - c_q) \right] \right\} ^ { - c_q /
    \tilde{d}_{kq} }
\end{align*}

\begin{gather}
    \textit{\ss}_{kq} =
    \text{seq}(s_{kq}, c_q) =
    s_{kq} ^ {\frac{1}{1 - c_q}}
\end{gather}

\subsubsection{Education and Experience Equivalence}
\begin{enumerate}
    \item years of education $\tau_k$, $\tau_q$
    \item years of education vector $\boldsymbol{\tau}$
    \item $\ddot{\tau}_{kq} = \text{eeq}(...) = \text{teq}(\tau_k, \tau_q, ...)$
    \item $\ddot{\tau}_{kq}^{\theta} = \text{eeq}(...) = \text{teq}(\tau_k, \tau_q, ...) \times s_{kq}^{\theta}$
    \item field similarity (direction): $s_{kq}^{\theta} = \cos\theta_{kq} = \frac{
                  \boldsymbol{\ddot{a}_{k} \ \dotproduct\ \boldsymbol{\ddot{a}_{q}}} }{
                  \norm{\boldsymbol{\ddot{a}_{k}}} \norm{\boldsymbol{\ddot{a}_{q}}} } $
    \item $s_{kq}^{\theta} \in [0,1]$, because cosine cannot be negative, as attributes are truncated
    \item interpretation: equivalent education and experience in the same field
    \item vector: $\boldsymbol{s_{k}^{\theta}}$
    \item matrix: $\textbf{S}_{\mathbf{\Theta}}$
    \item eeq vector: $\boldsymbol{\ddot{\tau}_{k}^{\theta}}$
    \item eeq matrix: $\boldsymbol{\ddot{\textbf{T}}_{\mathbf{\Theta}}}$
\end{enumerate}
\begin{gather}
    \ddot{\tau}_{kq} =
    \text{teq}(\tau_k,\tau_q,c_q) =
    \big\{
    1 + c_q \tau_q
    \exp\big[
        -\tau_q (
        \tau_k - \tau_q + 1 - c_q
        )
        \big]
    \big\} ^ {-c_q}
\end{gather}

\begin{gather}
    s_{kq}^{\theta} =
    \cos\theta_{kq} =
    \frac{
        \boldsymbol{\ddot{a}_k}
        \dotproduct
        \boldsymbol{\ddot{a}_q}
    }{
        \norm{\boldsymbol{\ddot{a}_k}}
        \norm{\boldsymbol{\ddot{a}_k}}
    } =
    \frac{
    \sum_{i=1}^{m}{
    \ddot{a}_{i}^{k}
    \ddot{a}_{i}^{q}
    }
    }{
    \sqrt{
    \sum_{i=1}^{m}{
    \ddot{a}_{i}^{{k} ^ 2}
    }
    }
    \sqrt{
    \sum_{i=1}^{m}{
    \ddot{a}_{i}^{{q} ^ 2}
    }
    }
    }
\end{gather}

\begin{gather}
    \ddot{\tau}_{kq}^{\theta} = \ddot{\tau}_{kq} \times s_{kq}^{\theta}
\end{gather}

\subsubsection{Hireability}
\begin{enumerate}
    \item to be hireable is to be:
    \item sq: sufficiently qualified (basic skill level)
    \item se: sufficiently educated / experienced (in the field)
    \item ss: sufficiently similar (good fit, same shape)
    \item i.e. hireability pressuposes all the previous metrics
    \item if $\textit{\ss}$ is similarity in the strong sense, then hireability is
          similarity in the ``strongest sense", as it even pressuposes interchangeability
          ($\textit{\ss}$)
    \item $
              s_{kq}^{h} =
              \text{sq}(\boldsymbol{a_k},\boldsymbol{a_q}) \times
              \text{se}(\boldsymbol{a_k},\boldsymbol{a_q}) \times
              \text{ss}(\boldsymbol{a_k},\boldsymbol{a_q}) = \\
              \text{sqa}(\boldsymbol{a_k},\boldsymbol{a_q}) \times
              \text{eeq}(\boldsymbol{a_k},\boldsymbol{a_q}) \times
              \text{seq}(\boldsymbol{a_k},\boldsymbol{a_q}) = \\
              s_{kq}^{\geq} \times
              \ddot{\tau}_{kq}^{\theta} \times
              \textit{\ss}_{kq} = \\
              s_{kq}^{\geq} \times
              \ddot{\tau}_{kq} \times
              s_{kq}^{\theta} \times
              \ddot{s}_{kq}
          $
    \item p.s.: this metric is too strict. suppose a candidate scored a 0.9 on every
          metric: $0.9 ^ 4 = 0.6561$
    \item candidate is hireable if $_{0}^{1}[s_{kq}^{h}]$
    \item candidate is hireable for $_{0}^{1}[s_{kq}^{h}] \times s_{kq}^{h}$ percent of
          jobs (requires an axiom)
    \item hireability vector: $\boldsymbol{s_{k}^{h}}$
    \item hireability matrix: $\boldsymbol{\textbf{S}_{h}}$
    \item evaluated hireability vector: $_{0}^{1}[\boldsymbol{s_{k}^{h}}]$
    \item evaluated hireability matrix: $_{0}^{1}[\boldsymbol{\textbf{S}_{h}}]$
    \item ?percent of hireable jobs vector: $_{0}^{1}[\boldsymbol{s_{k}^{h}}]\boldsymbol{s_{k}^{h}}$
    \item ?percent of hireable jobs matrix: $_{0}^{1}[\boldsymbol{\textbf{S}_{h}}]^{\top}\boldsymbol{\textbf{S}_{h}}$
\end{enumerate}

\subsubsection{Utility Equivalence}
\begin{enumerate}
    \item normalized wages $\hat{y}_k = y_k / \max y_q$
    \item normalized wages vector $\boldsymbol{\hat{y}}$
    \item wage utility function $u(\hat{y}_k)$
    \item utility equivalence function $\ddot{u}_{kq} = \text{ueq}(u(\hat{y}_k),
              u(\hat{y}_q), ...)$
    \item static utility $u_k$ vs comparative utility $u_{kq}$
    \item incorporate preferences in the utility function like in the eeq function? (def
          = 1 for O*NET career profiles)
    \item e.g. $\ddot{u}_{kq} = \text{ueq}(u(\hat{y}_k), u(\hat{y}_q), ...) \times
              s_{kq}^{\upsilon}$
\end{enumerate}

\subsection{Aggregate Comparative Models}
\subsubsection{Employability}
\begin{enumerate}
    \item employment levels $w_k, w_q$
    \item total workforce $W = \sum_{q=1}^{n}{w_q}$
    \item $\Lambda_k = \left(\frac{1}{W}\right) \sum_{q=1}^{n}{_{0}^{1}[s_{kq}^{h}] s_{kq}^{h} w_q}$
    \item alternatively, $\Lambda_k = \left(\frac{1}{W}\right)
              \sum_{q=1}^{n}{_{0}^{1}[s_{kq}^{h}] s_{kq}^{\geq} w_q}$
    \item employability = percentage of jobs in the economy on which a candidate could be
          hired
    \item employability can also be defined at specific aggregation levels of the economy
          (see Labor Economic Models)
    \item $\Lambda_{k}^{?} = \left(\frac{1}{W}\right) \sum_{?}{_{0}^{1}[s_{kq}^{h}] s_{kq}^{h} w_q}$
    \item ? is the economic aggregation level (e.g. sector, industry, market, role)
    \item employability vector $\boldsymbol{\Lambda}$
    \item need to define an axiom to go from x percent of hireability in one job of a
          class to hireability in x percent of jobs of that class
\end{enumerate}

\subsubsection{Competitiveness}
\begin{enumerate}
    \item employment levels $w_k, w_q$
    \item total workforce $W = \sum_{q=1}^{n}{w_q}$
    \item $\text{vs}_k = \left(\frac{1}{W}\right) \sum_{q=1}^{n}{_{0}^{1}[\ddot{u}_{qk}] _{0}^{1}[s_{kq}^{h}] s_{kq}^{h} w_q}$
    \item alternatively, $\text{vs}_k = \left(\frac{1}{W}\right)
              \sum_{q=1}^{n}{_{0}^{1}[\ddot{u}_{qk}] _{0}^{1}[s_{kq}^{h}] s_{qk}^{\geq}
              w_q}$
    \item competitiveness (versus) = percentage of workers in the economy that could (and
          would like to) be hired for a particular job
    \item competitiveness can also be defined at specific aggregation levels of the
          economy (see Labor Economic Models)
    \item $\text{vs}_{k}^{?} = \left(\frac{1}{W}\right) \sum_{?}{_{0}^{1}[\ddot{u}_{qk}] _{0}^{1}[s_{kq}^{h}] s_{kq}^{h} w_q}$
    \item ? is the economic aggregation level (e.g. sector, industry, market, role)
    \item competitiveness vector $\textbf{vs}$
\end{enumerate}

\subsection{Labor Economic Models}
\subsubsection{Economic Taxonomy}
\begin{enumerate}
    \item hierarchical clustering of hireability coefficients
    \item number of levels = 1 (i.e. trivial, economy) + L levels, where L is the number
          for which the levels L and L - 1 (i.e. the previous aggregation) are identical
    \item alternatively, use optimization methods to choose number of aggregation levels
    \item alternatively, use the same number of aggregation levels as popular economic
          taxonomies (e.g. ISIC, NAICS, SIC, NACE, OKVED)
\end{enumerate}
\subsubsection{Competition in Labor Markets}
\begin{enumerate}
    \item every hireable skill set competes at every level / market on which it is
          hireable
    \item market value, industry value, etc = percent of money in an aggregation level
    \item market value, industry employability, etc = weighted employability of workers
          at that level
    \item market value, industry competitiveness, etc = weighted competitiveness of
          workers at that level
    \item etc etc (define every variable for labor market aggregations)
\end{enumerate}

% endregion

% region: microeconomic models 
\section{Microeconomic Models}
\subsection{Marginal Human Capital Models}
\subsubsection{Marginal Compensation or Market Prices}
\begin{enumerate}
    \item wages $y_k$
    \item NNLS attributes vs wages $\rightarrow p_i$
    \item $p_i$ is the marginal compensation (i.e. market price) of a point in an attribute
    \item market prices vector $\textbf{p}$
\end{enumerate}

\subsubsection{Marginal Time Investment}
\begin{enumerate}
    \item years of education and experience $\tau_k$
    \item NNLS attributes vs years of education and experience $\rightarrow \eta_i$
    \item $\eta_i$ is the marginal time investment (expected time of arrival) to gain a point in an attribute
    \item marginal time investment (ETA) vector $\boldsymbol{\eta}$
\end{enumerate}

\subsubsection{Micro-Flexibility}
\begin{enumerate}
    \item NNLS attributes vs attributes $\rightarrow \phi_{ij}$
    \item micro-flexibility = expected gain on other attributes for an additional point
          in an attribute
    \item !requires additional statistical pressupositions for regression coefficients to be interpretable as $\phi_{ij}$
    \item !potential for recursion
    \item !affects ETA
    \item define a geometric progression learning model?
    \item attribute micro-flexibility vector $\boldsymbol{\phi_{i}}$
    \item attribute micro-flexibility matrix $\boldsymbol{\phi}$
\end{enumerate}

\subsection{Aggregate Human Capital Models}
\subsubsection{Skill Set Value}
\begin{enumerate}
    \item expected compensation based on market prices $\textbf{p}$ and skill set
          $\boldsymbol{a_k}$
    \item $y_{k}^{\text{\tiny{E}}} = \text{E}[y_k | \boldsymbol{a_k}, \textbf{p}] = \sum_{i=1}^{m}{a_{i}^{k} p_i}$
          % \item $\text{E}[ \ y_k \ | \ \textbf{p}, \boldsymbol{a_k}\ ]$
          % \item $\text{E}(y_k | \textbf{p}, \boldsymbol{a_k})$
    \item use another notation for this?
    \item probably define a value variable
\end{enumerate}

\subsubsection{Macro-Flexibility}
\begin{enumerate}
    \item attribute macro-flexibility $\Phi_i = \left( \frac{1}{W} \right)
              \sum_{k=1}^{n}{\tilde{a}_{i}^{k} w_k}$
    \item attribute macro-flexibility vector $\boldsymbol{\Phi}$
\end{enumerate}

% endregion

% region: additional comparative models 
\section{Additional Comparative Models}
\subsection{Total Time Investment (ETA)}
\begin{enumerate}
    \item aggregate expected time investment to fill competency gaps
    \item uppercase eta $\eta$
    \item $\text{H}_{kq} = \sum_{i=1}^{m}{
              \delta(a_{i}^{k}, a_{i}^{q})
              \eta_i
              }$
    \item vector $\textbf{H}\boldsymbol{_k}$
    \item matrix $\textbf{H}$
    \item p.s.: notation conflicts with hireability matrix (probably change hireability
          notation)
\end{enumerate}

\subsection{Letter-Shaped Skill Sets}
\begin{enumerate}
    \item a generalization of ``T-Shaped Skills''
    \item refer to the atlas.letters R package
\end{enumerate}

% endregion

% region: additional skill set models 
\section{Additional Skill Set Models}
\subsection{Versatility}
\begin{enumerate}
    \item the versatility of a skill set is the weighted sum of its macro-flexibility
          % \item  $\Phi_k = \frac{
          %           \sum_{i=1}^{m}{\ddot{a}_{i}^{k} \Phi_i}
          %           }{
          %           \sum_{i=1}^{m}{\ddot{a}_{i}^{k}}
          %           }
          %       $
    \item $\Phi_k = \frac{ \sum_{i=1}^{m}{a_{i}^{k} \Phi_i} }{
              \sum_{i=1}^{m}{a_{i}^{k}} }$
\end{enumerate}

\subsection{Leverage? [aggregate micro-flexibility]}
\begin{enumerate}
    \item define a name for aggregate micro-flexibility (leverage?)
    \item same procedure as above
          % \item $\phi_k = \frac{ \sum_{i=1}^{m}{a_{i}^{k} \phi_i} }{
          %     \sum_{i=1}^{m}{a_{i}^{k}} }$
\end{enumerate}

% endregion

% region: roadmap models 
\section{Roadmap Models}
\subsection{Career Recommendation}
\begin{enumerate}
    \item this theory needs to be further developed, as I never got the chance to write
          it down completely
    \item see quantitative roadmap sketch for an idea of what this should be
\end{enumerate}

\subsubsection{Macro-Strategies}
\subsubsection{Preference-Adjusted Strategic Matching}

\subsection{Competency Training Recommendation}
\subsubsection{Micro-Strategies}
\subsubsection{Preference-Adjusted Strategic Training}

% endregion

% region: factor-analytic models 

\section{Factor-Analytic Models}
\subsection{Exploratory Factor Analysis of the O*NET Database}
\begin{enumerate}
    \item use standard factor-analytic notation (I already reserved many latin and greek
          characters for this reason)
\end{enumerate}

\subsection{Psychometric Questionnaire Optimization}
\subsubsection{Minimal Factor Representation}
\subsubsection{Retained Variance}
\subsubsection{Retained Matching Accuracy and Precision}

\subsection{Factor-Analytic Intelligence Quotient (IQ) Approximation}
\begin{enumerate}
    \item in psychometrics, general intelligence (IQ) is denoted as $g$
    \item define a expected value notation (also solves the expected compensation, or
          skill set value, problem above)
    \item $g_{k}^{\text{\tiny{E}}} = \text{E}[g_k | \boldsymbol{a_k}, \boldsymbol{\Psi_?}] =
              \left(
              \frac{1}{m_\psi?}
              \right)
              \sum_{\psi?}{
                  \tilde{a}_{i}^{k}
              }
          $
    \item $\psi?$ are the subset of items which are proxies for intelligence
    \item alternatively, write this coefficient as factor scores (if there is a notation
          for factor scores)
\end{enumerate}

\subsection{Factor-Analytic Comparative Statics}
\begin{enumerate}
    \item there isn't a mathematical notation for this method yet
    \item but most or all of its variables and functions have already been defined above
    \item refer to the atlas.fstatics R package to understand how this model works
\end{enumerate}
\subsubsection{Unbounded Exogenous Impacts}
\subsubsection{Truncated Exogenous Impacts}
\subsection{Career Type}
\begin{enumerate}
    \item refer to the atlas.acti R package
\end{enumerate}
% endregion

% region: writing plan 
\section{Writing Plan}
\subsection{Introduction to Quantitative Career Matching: How to Statistically Pick the Right Occupation}
\begin{enumerate}
    \item introduce the much neglected topic of data-driven career choice and development
    \item estimate the baseline Euclidean matching model
    \item for simplicity's sake
    \item to benchmark matching methods
\end{enumerate}

% \subsection{Equivalence Estimators for Quantitative Career Matching}
% \begin{enumerate}
%     \item aeq, seq functions
%     \item eeq function?
%     \item take $\gamma_k$ and $c_k$ parameters as given
%     \item explain how to calculate generality and competence on the next paper
%     \item recalculate the Euclidean matching model from paper 1 with these new methods
% \end{enumerate}

\subsection{Generalists vs Specialists: Who Are the Most Competent Workers?}
\begin{enumerate}
    \item briefly introduce umlaut notation here (ä)
    \item explain how to calculate generality and competence
    \item weighted correlation between generality and competence
\end{enumerate}

\subsection{Alternative/Advanced Career Matching Methods}
\begin{enumerate}
    \item explain all career matching methods besides the baseline Euclidean model
    \item compare all models against one another
    \item benchmark models against the baseline Euclidean model
    \item select one matching method
\end{enumerate}

\subsection{Employability and Competitiveness}
\begin{enumerate}
    \item having defined all the statistical pressupositions for these models in the
          previous papers, calculate employability and competitiveness coefficients
    \item choose another name for this paper?
\end{enumerate}

\subsection{A Hierarchical Clustering Labor Market Taxonomy}
\begin{enumerate}
    \item define the hireability taxonomic model
    \item choose another name for this paper?
\end{enumerate}

\subsection{Competition in Labor Markets}
\begin{enumerate}
    \item continue from where the taxonomy paper left off
    \item choose another name for this paper
\end{enumerate}

\subsection{Human Capital Flexibility}
\begin{enumerate}
    \item calculate macro-flexibility
    \item calculate micro-flexibility
    \item aggregate macro-flexibility for each skill set (versatility)
    \item aggregate micro-flexibility for each skill set (?)
\end{enumerate}

\subsection{Time Investment and Expected Returns of Human Capital}
\begin{enumerate}
    \item choose another name for this paper?
    \item calculate marginal compensation of human capital
    \item calculate marginal time investment of human capital
    \item aggregate marginal compensation of human capital (skill set value)
    \item aggregate marginal time investment of human capital (ETA)
    \item calculate which careers are most efficient to aim for in the short run
          (starting from the average skill set)
    \item calculate which careers are most effective to aim for in the long run (starting
          from the average skill set)
\end{enumerate}

\subsection{The Career Roadmap: A Mathematical Map to Career Choice and Career Development}
\begin{enumerate}
    \item choose another name for this paper?
    \item The Career Roadmap: Statistical Methods for Optimal Career Choice and Career
          Development?
    \item define the concept of macro-strategies
    \item combine all statistical metrics so far as the career recommendation coefficient
    \item define the concept of micro-strategies
    \item combine all statistical metrics so far as the training recommendation
          coefficient
    \item calculate which careers are most efficient to aim for in the short run
          (starting from the average skill set)
    \item calculate which careers are most effective to aim for in the long run (starting
          from the average skill set)
\end{enumerate}

\subsection{Letter-Shaped Career Profiles: A Generalization of ``T-Shaped Skills''}
\begin{enumerate}
    \item define the methodology for converting hersheys fonts to career profiles
    \item match career profiles against the latin, greek, and cyrillic alphabets
    \item find a way to include ``from A-to-Z'' in this paper's name
    \item or from ``A-to-$\Omega$"
\end{enumerate}

\subsection{Factor-Analyzing the Occupational Information Network Database}
\begin{enumerate}
    \item EFA on O*NET db
\end{enumerate}

\subsection{Career Types for the Occupational Information Network Database}
\begin{enumerate}
    \item calculate career types based on the factor model from the EFA paper and the
          statistics already defined in the previous papers
    \item career type molecules
\end{enumerate}

\subsection{Factor-Analytic Comparative Statics: Estimating Exogenous Impacts on the Labor Market}
\begin{enumerate}
    \item ai impact analysis
    \item some other impact analysis
    \item aggregate on the attribute level
    \item aggregate on the occupation level
    \item aggregate on each taxonomic level defined in the economic taxonomy paper
          (including the entire economy)
\end{enumerate}

\subsection{Factor-Analytic Intelligence Quotient Approximation}
\begin{enumerate}
    \item based on the factor model from the EFA paper, choose a proxy
    \item calculate factor-analytic approximation of IQ
    \item compare against data
\end{enumerate}
% endregion

\end{document}