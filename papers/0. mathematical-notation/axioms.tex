%SECTION: SETUP
% region: document class 
% \documentclass{article}
\documentclass{elsarticle} %elsevier
% \documentclass[twocolumn]{elsarticle} %elsevier
% \documentclass{apa7} %american psychological association
% \documentclass[doc]{apa7} %american psychological association
% \documentclass{aomart} %annals of mathematics
% \documentclass{imsproc} %institute of mathematics and statistics
% \documentclass{nwejmart} %north-western european journal of mathematics
% \documentclass{paper} 
% \documentclass{sr-vorl} %springer book
% \documentclass{ecca} %economica
% \documentclass{ecta} %econometrica
% \documentclass{apecon} %applied economics

% endregion

% region: packages 
\usepackage{xr}
\usepackage{physics}
\usepackage{amsmath}
\usepackage{float}
\usepackage{enumitem}
\setlist[enumerate]{label*=\arabic*.}
\usepackage[hidelinks]{hyperref}

% endregion

% region: title 
% Title
\title{
    Derivation of the Employability Theorem
}
\author{Cao Bittencourt}
\affiliation{{B. Sc. in Economics from EPGE (FGV), RJ, Brazil.}}
\affiliation{{Statistician at Atlas Career Guide Inc., FL, USA.}}
\date{\today}

% endregion

% region: display breaks 
\allowdisplaybreaks

% endregion

%SECTION: PAPER
\begin{document}

% \maketitle

% region: abstract 
\begin{abstract}
    \noindent
    This is document demonstrates the employability theorem from a set of fairly tautological axioms, which are pressuposed in quantitative career choice and career development methods.
\end{abstract}
% endregion

% region: title 
\maketitle

% endregion

% region: table of contents
\tableofcontents
\newpage
% endregion

% region: premises 
\section{Premises of the Employability Theorem}
\subsection{Skill Set Sufficiency Axiom}
\begin{enumerate}
    \item having the required skill level of an attribute is sufficient to perform its
          activities at that level
    \item skills are sufficient to perform activities
    \item $x \implies T_x$
    \item $x \iff T_x$
\end{enumerate}
\subsection{Skill Set Composition Axiom}
\begin{enumerate}
    \item rational economic agents can naturally compose multiple skills to accomplish
          complex tasks (i.e. tasks that require multiple skills)
    \item skills are composable to perform complex activities
    \item $x,y \iff T_{xy}$
    \item p.s.: it does not matter \textit{how} skills are composed to perform a complex
          activity. the point of the axiom is that any rational person which is also
          sufficiently qualified can ``figure out'' how to ``piece together'', that is
          compose, the required skills to perform a complex activity:
          \begin{gather*}
              (x,y \iff T_{xy}) \iff (\ x,y \iff T_{yx})
              .
          \end{gather*}
\end{enumerate}
\subsection{Occupational Reducibility Axiom}
\begin{enumerate}
    \item from a practical standpoint, occupations are just a collection of job
          activities
    \item occupations can be reduced to their job activities
    \item $\boldsymbol{q} \iff \boldsymbol{T_q}$
\end{enumerate}
\subsection{Occupational Skill-Sufficiency Corollary}
\begin{enumerate}
    \item from axioms: Reducibility, Sufficiency
    \item as occupations are just a collection of job activities and skills are
          sufficient to perform activities, therefore occupations are just a collection
          of skills
    \item occupations can be reduced to their skill set
    \item $\boldsymbol{q} \iff \boldsymbol{T_q} \iff \boldsymbol{a_q}$
\end{enumerate}

% endregion

% region: demonstration 
\section{Demonstration of the Employability Theorem}

% endregion

% region: additional corollaries 
\section{Additional Corollaries}
\subsection{Occupational Divisibility Corollary 1 (Suboccupations)}
\begin{enumerate}
    \item from axioms: Reducibility, Sufficiency, Composition
    \item as occupations are just a collection of simple and complex job activities, and
          skills are sufficient to perform activities, and rational agents can naturally
          compose skills to accomplish complex activities, therefore occupations can be
          ``broken down'' into ``suboccupations'', each with their own subset of the
          ``main'' occupation's activities and respective skill ``subsets''
    \item occupations can be divided into suboccupations, each with its own skill subset
    \item $
              \boldsymbol{q} \iff \boldsymbol{T_q}
              \text{ and }
              \boldsymbol{T_q} = \{T_x, T_y, T_z, ..., T_{xy}, T_{yx}, T_{xz}, T_{xzy}\}
              \implies \\
              \boldsymbol{T_{q_{xyz}}} = \{T_x, T_y, T_z, ..., T_{xy}, T_{yx}, T_{xz}, T_{xzy}\},\\
              \boldsymbol{T_{q_{xy}}} = \{T_x, T_y, ..., T_{xy}, T_{yx}\},\\
              \boldsymbol{T_{q_{xz}}} = \{T_x, T_z, ..., T_{xz}\},\\
              \boldsymbol{T_{q_{yz}}} = \{T_y, T_z, ...\},\\
              \boldsymbol{T_{q_x}} = T_x,\\
              \boldsymbol{T_{q_y}} = T_y,\\
              \boldsymbol{T_{q_z}} = T_z
              .
          $
    \item notice: we are not saying \textit{how} an occupation's complex activities are
          composed. what the corollary states is only that an occupation, which is just a
          collection of activities, can \textit{potentially} be divided into
          suboccupations, based on how their job activities are composed.
\end{enumerate}
\subsection{Occupational Divisibility Corollary 2 (Skill Subsets)}
\begin{enumerate}
    \item from corollary: Divisibility 1
    \item from axioms: Reducibility, Sufficiency, Composition
    \item to the extent to which an occupation is divisible, a sufficiently qualified
          skill set does not need to be a perfect match with an occupation's entire skill
          set in order to perform at least a portion of their job activities
    \item partial matching: people can perform isolated subsets of an occupation's job
          activities
    \item $
              \boldsymbol{q} \iff \boldsymbol{T_q}
              \text{ and }
              \boldsymbol{T_q} = \{T_x, T_y, T_z, ..., T_{xy}, T_{yx}, T_{xz}, T_{xzy}\}
              \implies \\
              \boldsymbol{T_{q_{xyz}}} = \{T_x, T_y, T_z, ..., T_{xy}, T_{yx}, T_{xz}, T_{xzy}\} \implies (x,y,z \implies \boldsymbol{T_{q_{xyz}}}),\\
              \boldsymbol{T_{q_{xy}}} = \{T_x, T_y, ..., T_{xy}, T_{yx}\} \implies (x,y \implies \boldsymbol{T_{q_{xy}}}),\\
              \boldsymbol{T_{q_{xz}}} = \{T_x, T_z, ..., T_{xz}\} \implies (x,z \implies \boldsymbol{T_{q_{xz}}}),\\
              \boldsymbol{T_{q_{yz}}} = \{T_y, T_z, ...\} \implies (y,z \implies \boldsymbol{T_{q_{yz}}}),\\
              \boldsymbol{T_{q_x}} = T_x \implies (x \implies \boldsymbol{T_{q_{x}}}),\\
              \boldsymbol{T_{q_y}} = T_y \implies (y \implies \boldsymbol{T_{q_{y}}}),\\
              \boldsymbol{T_{q_z}} = T_z \implies (z \implies \boldsymbol{T_{q_{z}}})
              .
          $
    \item again, we are not stating \textit{how} an occupation is divided, only that,
          because of skill set sufficiency, composition, and occupational reducibility,
          occupations are, therefore, potentially divisible into suboccupations (with
          skill subsets), and to the extent to which they are divisible, therefore, any
          sufficiently qualified person can perform the skill subset for which they are
          qualified
\end{enumerate}
\subsection{Occupational Divisibility Corollary 3 (Partial Employment)}

% endregion

\end{document}