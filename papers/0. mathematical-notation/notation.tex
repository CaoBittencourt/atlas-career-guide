%SECTION: SETUP
% region: document class 
% \documentclass{article}
\documentclass{elsarticle} %elsevier
% \documentclass[twocolumn]{elsarticle} %elsevier
% \documentclass{apa7} %american psychological association
% \documentclass[doc]{apa7} %american psychological association
% \documentclass{aomart} %annals of mathematics
% \documentclass{imsproc} %institute of mathematics and statistics
% \documentclass{nwejmart} %north-western european journal of mathematics
% \documentclass{paper} 
% \documentclass{sr-vorl} %springer book
% \documentclass{ecca} %economica
% \documentclass{ecta} %econometrica
% \documentclass{apecon} %applied economics

% endregion

% region: packages 
\usepackage{xr}
\usepackage{physics}
\usepackage{amsmath}
\usepackage{float}

% endregion

% region: title 
% Title
\title{
    % The Career Atlas: Mathematical Notation
    The Career Atlas:\\
    Mathematical Notation
}
\author{Cao Bittencourt}
% \affiliation{
%     {B. Sc. in Economics from EPGE (FGV), RJ, Brazil.}\\
%     {Statistician at Atlas Career Guide Inc., FL, USA.}
% }
\affiliation{{B. Sc. in Economics from EPGE (FGV), RJ, Brazil.}}
\affiliation{{Statistician at Atlas Career Guide Inc., FL, USA.}}
\date{\today}

% endregion

% region: display breaks 
\allowdisplaybreaks

% endregion

%SECTION: PAPER
\begin{document}

% \maketitle

% region: abstract 
\begin{abstract}
    \noindent
    This is a brief document to define statistical methods for data-driven career choice and development. It deals with topics such as: career matching (i.e. vocational choice); estimation of competence, or overall skill level; estimation of skill set generality; versatility; skill set profitability; employability; labor market competitiveness; labor market taxonomy; optimal human resources acquistion and allocation; and so on and so forth. Each concept shall be explained at length in separate articles.

    \noindent
    \\ \textbf{Keywords:} Career choice; Career development; Matching algorithms; Competence; Similarity.
\end{abstract}
% endregion

% region: title 
\maketitle

% endregion

% region: table of contents
\tableofcontents
\newpage
% endregion

% region: basic definitions
\section{Basic Definitions}
\subsection{Skill Sets}
The $i$-th professional attribute, or competency, of a person $k$ is defined
as:
\begin{gather}
    a_{i}^{k}
    \in [0, 100]
    ,
\end{gather}
where the interval $[0,100]$ determines the bounds for every competency.\footnote[1]{
    More generally, these could be defined as $a_\text{lb}$ (the lower bound) and $a_\text{ub}$ (the upper bound). Here, the interval $[0,100]$ is used because of its ease of interpretation.
}

The skill set, or career profile, of a person $k$ is the vector of their $m$
attributes:
\begin{gather}
    \boldsymbol{a_k} = (a_{1}^{k}, \dots, a_{m}^{k})
    .
\end{gather}

The skill set matrix, or career profile matrix, is the collection of all $n$
skill sets in the economy:
\begin{gather}
    \textbf{A} =
    \begin{bmatrix}
        a_{1}^{1} & \hdots & a_{m}^{1} \\
        \vdots    & \ddots & \vdots    \\
        a_{1}^{n} & \hdots & a_{m}^{n} \\
    \end{bmatrix}
    .
\end{gather}

\subsection{Skill Set Normalization}
Normalization by the scale bounds is defined by the tilde operator:
\begin{gather}
    \tilde{a}_{i}^{k} =
    \frac{a_{i}^{k} - 0}{100 - 0} =
    \frac{a_{i}^{k}}{100}
    \in [0,1]
    ;\\
    \boldsymbol{\tilde{a}_{k}} =
    (\tilde{a}_{1}^{k}, \dots, \tilde{a}_{m}^{k})
    ;\\
    \textbf{Ã} =
    \begin{bmatrix}
        \tilde{a}_{1}^{1} & \hdots & \tilde{a}_{m}^{1} \\
        \vdots            & \ddots & \vdots            \\
        \tilde{a}_{1}^{n} & \hdots & \tilde{a}_{m}^{n} \\
    \end{bmatrix}
    .
\end{gather}

Normalization by a skill set's highest attribute is defined by the hat
operator:
\begin{gather}
    \hat{a}_{i}^{k} =
    \frac{a_{i}^{k}}{\max{a_{j}^{k}}}
    \in [0,1]
    ;\\
    \boldsymbol{\hat{a}_{k}} =
    (\hat{a}_{1}^{k}, \dots, \hat{a}_{m}^{k})
    ;\\
    \textbf{Â} =
    \begin{bmatrix}
        \hat{a}_{1}^{1} & \hdots & \hat{a}_{m}^{1} \\
        \vdots          & \ddots & \vdots          \\
        \hat{a}_{1}^{n} & \hdots & \hat{a}_{m}^{n} \\
    \end{bmatrix}
    .
\end{gather}

\subsection{Preferences}
\begin{enumerate}
    \item individual attribute preference $\upsilon_{i}^{k}$
    \item attribute preference vector $\boldsymbol{\upsilon_k}$
    \item attribute preference matrix $\boldsymbol{\Upsilon}$
    \item most attribute operations apply to preferences
    \item preference-adjusted skill sets
\end{enumerate}

% endregion

% region: basic skill set models 
\section{Basic Skill Set Models}
\subsection{Skill Set Generality}
The generality of a skill set is the mean of its maxima-normalized attributes:
\begin{gather}
    \gamma_{k} = \left(\frac{1}{m}\right)\sum_{i=1}^{m}{\hat{a}_{i}^{k}}
    \in [0,1]
    .
\end{gather}

People with high $\gamma_{k}$ scores are called \textit{generalists}.
Conversely, those with low $\gamma_{k}$ scores are called \textit{specialists}.
Career profiles that are neither broad nor specialized are said to be
\textit{balanced}.

The generality vector of all $n$ skill sets in the economy is:
\begin{gather}
    \boldsymbol{\gamma} = (\gamma_{1}, \dots, \gamma_{n})
    .
\end{gather}

\subsection{Attribute Equivalence}
The attribute equivalence of a particular attribute in a skill set measures the
importance of that attribute relative to the skill set's highest attribute. It
is calculated using skill set generality as both a midpoint and scaling
parameter in the following linear-logistic classification function:
\begin{gather}
    \text{aeq}(\hat{a}_{i}^{k}, \gamma_{k}) =
    \hat{a}_{i}^{k}
    \left[
        1 + \gamma_{k} (1 - \hat{a}_{i}^{k})
        \exp\bigg(
        % -\frac{\hat{a}_{i}^{k} - \gamma_{k}}{1 - \gamma_{k}}
        \frac{\hat{a}_{i}^{k} - \gamma_{k}}{\gamma_{k} - 1}
        % \frac{\gamma_{k} - \hat{a}_{i}^{k}}{1 - \gamma_{k}}
        \bigg)
        \right] ^ {
    -\frac{\gamma_{k}}{\hat{a}_{i}^{k}}
    % -{\gamma_{k} / \hat{a}_{i}^{k}}
    }
    \in [0,1]
    .
\end{gather}

For a short-hand notation, the attribute equivalence can be denoted by the
umlaut operator:
\begin{gather}
    \ddot{a}_{i}^{k} = \text{aeq}(\hat{a}_{i}^{k}, \gamma_{k})
    .
\end{gather}

This is not to be confused with the Netwonian dot notation for partial
derivatives, which we do not employ, instead preferring the more explicit
$\partialderivative{x}{y}$ derivative operator of Leibniz.

At any rate, attributes with high levels of $\ddot{a}_{i}^{k}$ are said to be
equivalent to the skill set's most importante attribute. These are a career
profile's \textit{core} competencies. The remaining competencies are classified
as either \textit{important}, \textit{auxiliary}, \textit{minor}, or
\textit{unimportant}.

The attribute equivalence vector of a skill set is given by the collection of
their $m$ umlauted attributes:
\begin{gather}
    \boldsymbol{\ddot{a}_{k}} = (\ddot{a}_{1}^{k}, ..., \ddot{a}_{m}^{k})
    .
\end{gather}

Finally, the attribute equivalence matrix is the collection of all attribute
equivalence vectors in the economy:
\begin{gather}
    \textbf{Ä} =
    \begin{bmatrix}
        \ddot{a}_{1}^{1} & \hdots & \ddot{a}_{m}^{1} \\
        \vdots           & \ddots & \vdots           \\
        \ddot{a}_{1}^{n} & \hdots & \ddot{a}_{m}^{n} \\
    \end{bmatrix}
    .
\end{gather}

\subsection{Skill Set Competence}
The overall competence of a skill set is the mean of its scale-normalized
attributes, weighted by each attribute's importance (i.e. its attribute
equivalence):
\begin{gather}
    c_k =
    \frac{
    \sum_{i=1}^{m}{
    \ddot{a}_{i}^{k}
    \tilde{a}_{i}^{k}
    }
    }{
    \sum_{i=1}^{m}{
    \ddot{a}_{i}^{k}
    }
    }
    \in [0,1]
    .
\end{gather}

Career profiles with high $c_k$ are said to be competent. However, this
adjective can be seen as offensive to some people; and, most importantly, it
could also be misleading, because competence is often defined relative to the
specific requirements of a particular job. Therefore, we opt for the more
generic competence classification of \textit{high level}, \textit{mid level},
and \textit{low level}, which is somewhat less ambiguous.

The competence vector of all $n$ skill sets in the economy is:
\begin{gather}
    \boldsymbol{c} = (c_1, \dots, c_n)
    .
\end{gather}

% endregion

% region: comparative models 
\section{Comparative Models}
\subsection{Pairwise Comparative Models}
\subsubsection{Similarity}
The most basic comparative model is that of Euclidean matching with linear
weights:
\begin{gather}
    s_{k,q} =
    s(\boldsymbol{a_k}, \boldsymbol{a_q}) =
    1 - \tilde{d}(\boldsymbol{a_k}, \boldsymbol{a_q})
    \in [0,1]
    ,
\end{gather}
where
\begin{gather}
    \tilde{d}_{k,q} =
    \tilde{d}(\boldsymbol{a_k}, \boldsymbol{a_q}) =
    \sqrt{
    \frac{
    \sum_{i = 1}^{m}{
    a_{i}^{q} (a_{i}^{k} - a_{i}^{q})^2
    }
    }{
    \sum_{i = 1}^{m}{
    a_{i}^{q} \max(100 - a_{i}^{q}, a_{i}^{q})^2
    }
    }
    }
    \in [0,1]
    .
\end{gather}

In this model, we compare a skill set $\boldsymbol{a_k}$ to a skill set
$\boldsymbol{a_q}$ by calculating the weighted Euclidean distance from
$\boldsymbol{a_k}$ to $\boldsymbol{a_q}$ normalized by the maximum theoretical
distance to $\boldsymbol{a_q}$.

Other weighting systems can be employed in this type of matching model. We
could, for instance, substitute the linear weights with either quadratic
weights,
\begin{gather}
    {a_{i}^{q}} ^ 2
    \in [0,1]
    ,
\end{gather}
or speciality-root weights,
\begin{gather}
    {a_{i}^{q}} ^ {
        \frac{1}{1 - \gamma_k}
    }
    \in [0,1]
    .
\end{gather}

But the best and most interpretable results are obtained using attribute
equivalence as the weighting function:
\begin{gather}
    \tilde{d}_{k,q} =
    \tilde{d}(\boldsymbol{a_k}, \boldsymbol{a_q}) =
    \sqrt{
    \frac{
    \sum_{i = 1}^{m}{
    \ddot{a}_{i}^{q} (a_{i}^{k} - a_{i}^{q})^2
    }
    }{
    \sum_{i = 1}^{m}{
    \ddot{a}_{i}^{q} \max(100 - a_{i}^{q}, a_{i}^{q})^2
    }
    }
    }
    \in [0,1]
    .
\end{gather}

We could also employ other matching methods instead of the ``baseline" weighted
Euclidean model. [detail each method later]:
\begin{enumerate}
    \item logit regression matching
    \item probit regression matching
    \item bvls regression matching
    \item tobit regression matching
    \item pearson correlation matching
    \item kendal nonparametric correlation matching
    \item spearman nonparametric correlation matching
\end{enumerate}

At last, similarity and normalized distance metrics determine the respective
vectors and matrices, as follows:
\begin{gather}
    \boldsymbol{s_k} = (s_{k,1}, \dots, s_{k,n})
    ;\\
    \boldsymbol{\tilde{d}_k} = (\tilde{d}_{k,1}, \dots, \tilde{d}_{k,n})
    ;\\
    \textbf{S} =
    \begin{bmatrix}
        s_{1,1} & \hdots & s_{n,1} \\
        \vdots  & \ddots & \vdots  \\
        s_{1,n} & \hdots & s_{n,n} \\
    \end{bmatrix} =
    \begin{bmatrix}
        1       & \hdots & s_{k,1} & \hdots & s_{n,1} \\
        \vdots  & \ddots & \vdots  & \ddots & \vdots  \\
        s_{1,k} & \hdots & 1       & \hdots & s_{n,k} \\
        \vdots  & \ddots & \vdots  & \ddots & \vdots  \\
        s_{1,n} & \hdots & s_{k,n} & \hdots & 1       \\
    \end{bmatrix}
    ;\\
    \textbf{D} =
    \begin{bmatrix}
        \tilde{d}_{1,1} & \hdots & \tilde{d}_{n,1} \\
        \vdots          & \ddots & \vdots          \\
        \tilde{d}_{1,n} & \hdots & \tilde{d}_{n,n} \\
    \end{bmatrix} =
    \begin{bmatrix}
        0               & \hdots & \tilde{d}_{k,1} & \hdots & \tilde{d}_{n,1} \\
        \vdots          & \ddots & \vdots          & \ddots & \vdots          \\
        \tilde{d}_{1,k} & \hdots & 0               & \hdots & \tilde{d}_{n,k} \\
        \vdots          & \ddots & \vdots          & \ddots & \vdots          \\
        \tilde{d}_{1,n} & \hdots & \tilde{d}_{k,n} & \hdots & 0               \\
    \end{bmatrix}
    .
\end{gather}

\subsubsection{Qualification}
A closely related concept to matching is the qualification comparative model.
In this family of functions, however, Euclidean matching is mandatory, as other
matching methods do not make sense for this specific type of calculation. The
reason for this is at that, here, we are not particularly interested in
matching (i.e. a typical classification problem), but rather in the actual
distances between comparison skill sets.

To define these models, we first have to define the gap function, which
measures only positive competency gaps:
\begin{gather}
    \delta_{k,q}^{i} =
    \delta(a_{i}^{k}, a_{i}^{q}) =
    \max(
    a_{i}^{k} - a_{i}^{q}
    , 0
    )
    \in [0, 100]
    .
\end{gather}

Having defined the gap function, we can write the underqualification model:
\begin{gather}
    \tilde{\delta}_{k,q}^{<} =
    \text{uqa}(\boldsymbol{a_k}, \boldsymbol{a_q}) =
    \sqrt{
    \frac{
    \sum_{i=1}^{m}{
    \ddot{a}_{i}^{q} {\delta(a_{i}^{q}, a_{i}^{k})} ^ 2
    }
    }{
    \sum_{i=1}^{m}{
    \ddot{a}_{i}^{q} {\delta(a_{i}^{q}, 0)} ^ 2
    }
    }
    } =
    \sqrt{
    \frac{
    \sum_{i=1}^{m}{
    \ddot{a}_{i}^{q} {\delta(a_{i}^{q}, a_{i}^{k})} ^ 2
    }
    }{
    \sum_{i=1}^{m}{
    \ddot{a}_{i}^{q} {a_{i}^{q}} ^ 2
    }
    }
    }
    .
\end{gather}

And, analogously, the overqualification model is given by:
\begin{gather}
    \tilde{\delta}_{k,q}^{\geq} =
    \text{oqa}(\boldsymbol{a_k}, \boldsymbol{a_q}) =
    \sqrt{
    \frac{
    \sum_{i=1}^{m}{
    \ddot{a}_{i}^{q} {\delta(a_{i}^{k}, a_{i}^{q})} ^ 2
    }
    }{
    \sum_{i=1}^{m}{
    \ddot{a}_{i}^{q} {\delta(100, a_{i}^{q})} ^ 2
    }
    }
    } =
    \sqrt{
    \frac{
    \sum_{i=1}^{m}{
    \ddot{a}_{i}^{q} {\delta(a_{i}^{k}, a_{i}^{q})} ^ 2
    }
    }{
    \sum_{i=1}^{m}{
    \ddot{a}_{i}^{q} {(100 - a_{i}^{q})} ^ 2
    }
    }
    }
    .
\end{gather}

The final set of ``sufficient qualification" is, evidently, the complement of
the underqualification model:
\begin{gather}
    s_{k,q}^{\geq} =
    \text{sqa}(\boldsymbol{a_k}, \boldsymbol{a_q}) =
    1 -
    \sqrt{
    \frac{
    \sum_{i=1}^{m}{
    \ddot{a}_{i}^{q} {\delta(a_{i}^{q}, a_{i}^{k})} ^ 2
    }
    }{
    \sum_{i=1}^{m}{
    \ddot{a}_{i}^{q} {\delta(a_{i}^{q}, 0)} ^ 2
    }
    }
    } =
    1 - \text{uqa}(\boldsymbol{a_k}, \boldsymbol{a_q})
    .
\end{gather}

As with the similarity and normalized distance statistics described above,
these three qualification models are bounded to the $[0,1]$ interval. Likewise,
they also determine qualification vectors:
\begin{gather}
    \boldsymbol{\tilde{\delta}_{k}^{<}} = (\tilde{\delta}_{k,1}^{<}, \dots, \tilde{\delta}_{k,n}^{<})
    ;\\
    \boldsymbol{\tilde{\delta}_{k}^{\geq}} = (\tilde{\delta}_{k,1}^{\geq}, \dots, \tilde{\delta}_{k,n}^{\geq})
    ;\\
    \boldsymbol{s_{k}^{\geq}} = (s_{k,1}^{\geq}, \dots, s_{k,n}^{\geq})
    ;
\end{gather}
and matrices
\begin{gather}
    \textbf{$\tilde{\Delta}$}_{\boldsymbol{<}} =
    \begin{bmatrix}
        \tilde{\delta}_{1,1}^{<} & \hdots & \tilde{\delta}_{n,1}^{<} \\
        \vdots                   & \ddots & \vdots                   \\
        \tilde{\delta}_{1,n}^{<} & \hdots & \tilde{\delta}_{n,n}^{<} \\
    \end{bmatrix} =
    \begin{bmatrix}
        0                        & \hdots & \tilde{\delta}_{k,1}^{<} & \hdots & \tilde{\delta}_{n,1}^{<} \\
        \vdots                   & \ddots & \vdots                   & \ddots & \vdots                   \\
        \tilde{\delta}_{1,k}^{<} & \hdots & 0                        & \hdots & \tilde{\delta}_{n,k}^{<} \\
        \vdots                   & \ddots & \vdots                   & \ddots & \vdots                   \\
        \tilde{\delta}_{1,n}^{<} & \hdots & \tilde{\delta}_{k,n}^{<} & \hdots & 0                        \\
    \end{bmatrix}
    ;\\
    \textbf{$\tilde{\Delta}$}_{\boldsymbol{\geq}} =
    \begin{bmatrix}
        \tilde{\delta}_{1,1}^{\geq} & \hdots & \tilde{\delta}_{n,1}^{\geq} \\
        \vdots                      & \ddots & \vdots                      \\
        \tilde{\delta}_{1,n}^{\geq} & \hdots & \tilde{\delta}_{n,n}^{\geq} \\
    \end{bmatrix} =
    \begin{bmatrix}
        0                           & \hdots & \tilde{\delta}_{k,1}^{\geq} & \hdots & \tilde{\delta}_{n,1}^{\geq} \\
        \vdots                      & \ddots & \vdots                      & \ddots & \vdots                      \\
        \tilde{\delta}_{1,k}^{\geq} & \hdots & 0                           & \hdots & \tilde{\delta}_{n,k}^{\geq} \\
        \vdots                      & \ddots & \vdots                      & \ddots & \vdots                      \\
        \tilde{\delta}_{1,n}^{\geq} & \hdots & \tilde{\delta}_{k,n}^{\geq} & \hdots & 0                           \\
    \end{bmatrix}
    ;\\
    \textbf{S}_{\boldsymbol{\geq}} =
    \begin{bmatrix}
        s_{1,1}^{\geq} & \hdots & s_{n,1}^{\geq} \\
        \vdots         & \ddots & \vdots         \\
        s_{1,n}^{\geq} & \hdots & s_{n,n}^{\geq} \\
    \end{bmatrix} =
    \begin{bmatrix}
        1              & \hdots & s_{k,1}^{\geq} & \hdots & s_{n,1}^{\geq} \\
        \vdots         & \ddots & \vdots         & \ddots & \vdots         \\
        s_{1,k}^{\geq} & \hdots & 1              & \hdots & s_{n,k}^{\geq} \\
        \vdots         & \ddots & \vdots         & \ddots & \vdots         \\
        s_{1,n}^{\geq} & \hdots & s_{k,n}^{\geq} & \hdots & 1              \\
    \end{bmatrix}
    .
\end{gather}

P.S.: should qualification be unweighted?
\begin{gather}
    \tilde{\delta}_{k,q}^{<} =
    \text{uqa}(\boldsymbol{a_k}, \boldsymbol{a_q}) =
    \sqrt{
    \frac{
    \sum_{i=1}^{m}{
    {\delta(a_{i}^{q}, a_{i}^{k})} ^ 2
    }
    }{
    \sum_{i=1}^{m}{
    {\delta(a_{i}^{q}, 0)} ^ 2
    }
    }
    } =
    \sqrt{
    \frac{
    \sum_{i=1}^{m}{
    {\delta(a_{i}^{q}, a_{i}^{k})} ^ 2
    }
    }{
    \sum_{i=1}^{m}{
    {a_{i}^{q}} ^ 2
    }
    }
    }
    ;\\
    \tilde{\delta}_{k,q}^{\geq} =
    \text{oqa}(\boldsymbol{a_k}, \boldsymbol{a_q}) =
    \sqrt{
    \frac{
    \sum_{i=1}^{m}{
    {\delta(a_{i}^{k}, a_{i}^{q})} ^ 2
    }
    }{
    \sum_{i=1}^{m}{
    {\delta(100, a_{i}^{q})} ^ 2
    }
    }
    } =
    \sqrt{
    \frac{
    \sum_{i=1}^{m}{
    {\delta(a_{i}^{k}, a_{i}^{q})} ^ 2
    }
    }{
    \sum_{i=1}^{m}{
    {(100 - a_{i}^{q})} ^ 2
    }
    }
    }
    ;\\
    s_{k,q}^{\geq} =
    \text{sqa}(\boldsymbol{a_k}, \boldsymbol{a_q}) =
    1 -
    \sqrt{
    \frac{
    \sum_{i=1}^{m}{
    {\delta(a_{i}^{q}, a_{i}^{k})} ^ 2
    }
    }{
    \sum_{i=1}^{m}{
    {\delta(a_{i}^{q}, 0)} ^ 2
    }
    }
    } =
    1 - \text{uqa}(\boldsymbol{a_k}, \boldsymbol{a_q})
    .
\end{gather}

Or even

\begin{gather}
    \tilde{\delta}_{k,q}^{<} =
    \text{uqa}(\boldsymbol{a_k}, \boldsymbol{a_q}) =
    \frac{
    \sum_{i=1}^{m}{
    {\delta(a_{i}^{q}, a_{i}^{k})}
    }
    }{
    \sum_{i=1}^{m}{
    {\delta(a_{i}^{q}, 0)}
    }
    } =
    \frac{
    \sum_{i=1}^{m}{
    {\delta(a_{i}^{q}, a_{i}^{k})}
    }
    }{
    \sum_{i=1}^{m}{
    {a_{i}^{q}}
    }
    }
    ;\\
    \tilde{\delta}_{k,q}^{\geq} =
    \text{oqa}(\boldsymbol{a_k}, \boldsymbol{a_q}) =
    \frac{
    \sum_{i=1}^{m}{
    {\delta(a_{i}^{k}, a_{i}^{q})}
    }
    }{
    \sum_{i=1}^{m}{
    {\delta(100, a_{i}^{q})}
    }
    } =
    \frac{
    \sum_{i=1}^{m}{
    {\delta(a_{i}^{k}, a_{i}^{q})}
    }
    }{
    \sum_{i=1}^{m}{
    {(100 - a_{i}^{q})}
    }
    }
    ;\\
    s_{k,q}^{\geq} =
    \text{sqa}(\boldsymbol{a_k}, \boldsymbol{a_q}) =
    1 -
    \frac{
    \sum_{i=1}^{m}{
    {\delta(a_{i}^{q}, a_{i}^{k})}
    }
    }{
    \sum_{i=1}^{m}{
    {\delta(a_{i}^{q}, 0)}
    }
    } =
    1 - \text{uqa}(\boldsymbol{a_k}, \boldsymbol{a_q})
    ?
\end{gather}


\subsubsection{Interchangeability}
\begin{enumerate}
    \item sufficient similarity
    \item scaled similarity
    \item similarity in the strong sense
    \item s.s. $\rightarrow$ $\textit{\ss}_{k,q}$
    \item alternatively, $\textit{\ss}_{k,q} = \text{seq(...)} = \ddot{s}_{k,q}$
    \item interchangeability vector $\boldsymbol{\textbf{\textit{\ss}}_k}$
    \item interchangeability matrix $\boldsymbol{\textbf{\ss}}$
\end{enumerate}
alternatively,
\begin{align*}
    \textit{\ss}_{k,q} &= 
    \text{seq}(\boldsymbol{a_k}, \boldsymbol{a_q}) = 
    s_{k,q} \left\{
        1 + c_q \tilde{d}_{k,q} \exp\left[
            -\frac{
                d_{\max}^{q}
                (\tilde{d}_{k,q} - c_q)
                % - (\tilde{d}_{k,q} - c_q)
                % d_{\max}^{q}
            }{
                1 - c_q
            }
        \right]
    \right\} ^ {-\frac{c_q}{\tilde{d}_{k,q}}}\\
    &=
    s_{k,q} \left\{
        1 + c_q (1 - s_{k,q}) \exp\left[
            \frac{
                c_q d_{\max}^{q} - d_{k,q}
            % c_q d_{\text{ub}}^{q} - d_{k,q}
            % c_q \bar{d}_{q} - d_{k,q}
            }{
                1 - c_q
            }
        \right]
    \right\} ^ {
        - c_q / \tilde{d}_{k,q}
    }\\
    &=
    (1 - \tilde{d}_{k,q})
    \times
    \left[
        1 + c_q \tilde{d}_{k,q} \exp\left(
            \frac{
                % c_q d_{\max}^{q} - d_{k,q}
            % c_q d_{\text{ub}}^{q} - d_{k,q}
            c_q \bar{d}_{q} - d_{k,q}
            }{
                1 - c_q
            }
        \right)
    \right] ^ {
        - c_q / \tilde{d}_{k,q}
    }\\
    &=
    (1 - \tilde{d}_{k,q})
    \left[
        1 + c_q \tilde{d}_{k,q} \exp\left(
            \frac{
                % c_q d_{\max}^{q} - d_{k,q}
            % c_q d_{\text{ub}}^{q} - d_{k,q}
            c_q \bar{d}_{q} - d_{k,q}
            }{
                1 - c_q
            }
        \right)
    \right] ^ {
        - c_q / \tilde{d}_{k,q}
    }\\
    &=
    s_{k,q}
    \left[
        1 + c_q \tilde{d}_{k,q} \exp\left(
            \frac{
                % c_q d_{\max}^{q} - d_{k,q}
            % c_q d_{\text{ub}}^{q} - d_{k,q}
            c_q \bar{d}_{q} - d_{k,q}
            }{
                1 - c_q
            }
        \right)
    \right] ^ {
        - c_q / \tilde{d}_{k,q}
    }\\
    &=
    s_{k,q}
    \left[
        1 + c_q \tilde{d}_{k,q} \exp\left(
            \frac{
                % c_q d_{\max}^{q} - d_{k,q}
            c_q d_{\text{ub}}^{q} - d_{k,q}
            % c_q d_{q}^{\text{ub}} - d_{k,q}
            % c_q d_{q}^{ub} - d_{k,q}
            % c_q d_{ub}^{q} - d_{k,q}
            % c_q \bar{d}_{q} - d_{k,q}
            }{
                1 - c_q
            }
        \right)
    \right] ^ {
        - c_q / \tilde{d}_{k,q}
    }\\
    &=
    s_{k,q} \left\{
        1 + c_q \tilde{d}_{k,q} \exp\left[
            -\frac{
                \bar{d}_{q}
                (\tilde{d}_{k,q} - c_q)
                % - (\tilde{d}_{k,q} - c_q)
                % d_{\max}^{q}
            }{
                1 - c_q
            }
        \right]
    \right\} ^ {-\frac{c_q}{\tilde{d}_{k,q}}}\\
    &=
    s_{k,q} \left\{
        1 + c_q \tilde{d}_{k,q} \exp\left[
            \frac{
                \bar{d}_{q}
                (c_q - \tilde{d}_{k,q})
                % - (\tilde{d}_{k,q} - c_q)
                % d_{\max}^{q}
            }{
                1 - c_q
            }
        \right]
    \right\} ^ {-\frac{c_q}{\tilde{d}_{k,q}}}\\
    &=
    s_{k,q} \left\{
        1 + c_q \tilde{d}_{k,q} \exp\left[
            \frac{
                \bar{d}_{q}
                (c_q - \tilde{d}_{k,q})
                % - (\tilde{d}_{k,q} - c_q)
                % d_{\max}^{q}
            }{
                1 - c_q
            }
        \right]
    \right\} ^ {- c_q / \tilde{d}_{k,q}}\\
    &=
    s_{k,q} \left\{
        1 + c_q (1 - s_{k,q}) \exp\left[
            (
                % c_q d_{\max}^{q} - d_{k,q}
            % c_q d_{\text{ub}}^{q} - d_{k,q}
            c_q \bar{d}_{q} - d_{k,q}
            ) / (1 - c_q)
        \right]
    \right\} ^ {
        - c_q / \tilde{d}_{k,q}
    }
\end{align*}

\subsubsection{Education and Experience Equivalence}
\begin{enumerate}
    \item years of education $\tau_k$, $\tau_q$
    \item years of education vector $\boldsymbol{\tau}$
    \item $\ddot{\tau}_{k,q} = \text{eeq}(...) = \text{teq}(\tau_k, \tau_q, ...)$
    \item $\ddot{\tau}_{k,q}^{\theta} = \text{eeq}(...) = \text{teq}(\tau_k, \tau_q, ...) \times s_{k,q}^{\theta}$
    \item field similarity (direction): $s_{k,q}^{\theta} = \cos\theta_{k,q} = \frac{
                  \boldsymbol{\ddot{a}_{k} \ \dotproduct\ \boldsymbol{\ddot{a}_{q}}} }{
                  \norm{\boldsymbol{\ddot{a}_{k}}} \norm{\boldsymbol{\ddot{a}_{q}}} } $
    \item $s_{k,q}^{\theta} \in [0,1]$, because cosine cannot be negative, as attributes are truncated
    \item interpretation: equivalent education and experience in the same field
    \item vector: $\boldsymbol{s_{k}^{\theta}}$
    \item matrix: $\textbf{S}_{\mathbf{\Theta}}$
    \item eeq vector: $\boldsymbol{\ddot{\tau}_{k}^{\theta}}$
    \item eeq matrix: $\boldsymbol{\ddot{\textbf{T}}_{\mathbf{\Theta}}}$
\end{enumerate}
\begin{gather}
    \ddot{\tau}_{k,q} =
    \text{teq}(\tau_k,\tau_q,c_q) =
    \big\{
        1 + c_q \tau_q 
        \exp\big[
            -\tau_q (
                    \tau_k - \tau_q + 1 - c_q
                )
        \big]
    \big\} ^ {-c_q}
\end{gather}

\begin{gather}
    s_{k,q}^{\theta} = 
    \cos\theta_{k,q} = 
    \frac{
        \boldsymbol{\ddot{a}_k}
        \dotproduct
        \boldsymbol{\ddot{a}_q}
    }{
        \norm{\boldsymbol{\ddot{a}_k}}
        \norm{\boldsymbol{\ddot{a}_k}}
    } =
    \frac{
        \sum_{i=1}^{m}{
            \ddot{a}_{i}^{k}
            \ddot{a}_{i}^{q}
        }
    }{
        \sqrt{
            \sum_{i=1}^{m}{
                \ddot{a}_{i}^{{k} ^ 2}
            }
        }
        \sqrt{
            \sum_{i=1}^{m}{
                \ddot{a}_{i}^{{q} ^ 2}
            }
        }
    }
\end{gather}

\begin{gather}
    \ddot{\tau}_{k,q}^{\theta} = \ddot{\tau}_{k,q} \times s_{k,q}^{\theta}
\end{gather}

\subsubsection{Hireability}
\begin{enumerate}
    \item to be hireable is to be:
    \item sq: sufficiently qualified (basic skill level)
    \item se: sufficiently educated / experienced (in the field)
    \item ss: sufficiently similar (good fit, same shape)
    \item i.e. hireability pressuposes all the previous metrics
    \item if $\textit{\ss}$ is similarity in the strong sense, then hireability is similarity
          in the ``strongest sense", as it even pressuposes interchangeability ($\textit{\ss}$)
    \item $
              s_{k,q}^{h} =
              \text{sq}(\boldsymbol{a_k},\boldsymbol{a_q}) \times
              \text{se}(\boldsymbol{a_k},\boldsymbol{a_q}) \times
              \text{ss}(\boldsymbol{a_k},\boldsymbol{a_q}) = \\
              \text{sqa}(\boldsymbol{a_k},\boldsymbol{a_q}) \times
              \text{eeq}(\boldsymbol{a_k},\boldsymbol{a_q}) \times
              \text{seq}(\boldsymbol{a_k},\boldsymbol{a_q}) = \\
              s_{k,q}^{\geq} \times
              \ddot{\tau}_{k,q}^{\theta} \times
              \textit{\ss}_{k,q} = \\
              s_{k,q}^{\geq} \times
              \ddot{\tau}_{k,q} \times
              s_{k,q}^{\theta} \times
              \ddot{s}_{k,q}
          $
    \item candidate is hireable if $_{0}^{1}[s_{k,q}^{h}]$
    \item candidate is hireable for $_{0}^{1}[s_{k,q}^{h}] \times s_{k,q}^{h}$ percent of jobs (requires an axiom)
    \item hireability vector: $\boldsymbol{s_{k}^{h}}$
    \item hireability matrix: $\boldsymbol{\textbf{S}_{h}}$
    \item evaluated hireability vector: $_{0}^{1}[\boldsymbol{s_{k}^{h}}]$
    \item evaluated hireability matrix: $_{0}^{1}[\boldsymbol{\textbf{S}_{h}}]$
    \item ?percent of hireable jobs vector: $_{0}^{1}[\boldsymbol{s_{k}^{h}}]\boldsymbol{s_{k}^{h}}$
    \item ?percent of hireable jobs matrix: $_{0}^{1}[\boldsymbol{\textbf{S}_{h}}]^{\top}\boldsymbol{\textbf{S}_{h}}$
\end{enumerate}

\subsubsection{Utility Equivalence}
\begin{enumerate}
    \item normalized wages $\hat{y}_k = y_k / \max y_q$
    \item normalized wages vector $\boldsymbol{\hat{y}}$
    \item wage utility function $u(\hat{y}_k)$
    \item utility equivalence function $\ddot{u}_{k,q} = \text{ueq}(u(\hat{y}_k),
              u(\hat{y}_q), ...)$
    \item static utility $u_k$ vs comparative utility $u_{k,q}$
    \item incorporate preferences in the utility function like in the eeq function? (def
          = 1 for O*NET career profiles)
    \item e.g. $\ddot{u}_{k,q} = \text{ueq}(u(\hat{y}_k), u(\hat{y}_q), ...) \times
              s_{k,q}^{\upsilon}$
\end{enumerate}

\subsection{Aggregate Comparative Models}
\subsubsection{Employability}
\begin{enumerate}
    \item employment levels $w_k, w_q$
    \item total workforce $W = \sum_{q=1}^{n}{w_q}$
    \item $\Lambda_k = \left(\frac{1}{W}\right) \sum_{q=1}^{n}{_{0}^{1}[s_{k,q}^{h}] s_{k,q}^{h} w_q}$
    \item alternatively, $\Lambda_k = \left(\frac{1}{W}\right)
              \sum_{q=1}^{n}{_{0}^{1}[s_{k,q}^{h}] s_{k,q}^{\geq} w_q}$
    \item employability = percentage of jobs in the economy on which a candidate could be
          hired
    \item employability can also be defined at specific aggregation levels of the economy
          (see Labor Economic Models)
    \item $\Lambda_{k}^{?} = \left(\frac{1}{W}\right) \sum_{?}{_{0}^{1}[s_{k,q}^{h}] s_{k,q}^{h} w_q}$
    \item ? is the economic aggregation level (e.g. sector, industry, market, role)
    \item employability vector $\boldsymbol{\Lambda}$
    \item need to define an axiom to go from x percent of hireability in one job of a
          class to hireability in x percent of jobs of that class
\end{enumerate}

\subsubsection{Competitiveness}
\begin{enumerate}
    \item employment levels $w_k, w_q$
    \item total workforce $W = \sum_{q=1}^{n}{w_q}$
    \item $\text{vs}_k = \left(\frac{1}{W}\right) \sum_{q=1}^{n}{_{0}^{1}[\ddot{u}_{q,k}] _{0}^{1}[s_{k,q}^{h}] s_{k,q}^{h} w_q}$
    \item alternatively, $\text{vs}_k = \left(\frac{1}{W}\right)
              \sum_{q=1}^{n}{_{0}^{1}[\ddot{u}_{q,k}] _{0}^{1}[s_{k,q}^{h}] s_{q,k}^{\geq} w_q}$
    \item competitiveness (versus) = percentage of workers in the economy that could (and
          would like to) be hired for a particular job
    \item competitiveness can also be defined at specific aggregation levels of the
          economy (see Labor Economic Models)
    \item $\text{vs}_{k}^{?} = \left(\frac{1}{W}\right) \sum_{?}{_{0}^{1}[\ddot{u}_{q,k}] _{0}^{1}[s_{k,q}^{h}] s_{k,q}^{h} w_q}$
    \item ? is the economic aggregation level (e.g. sector, industry, market, role)
    \item competitiveness vector $\textbf{vs}$
\end{enumerate}

\subsection{Labor Economic Models}
\subsubsection{Economic Taxonomy}
\begin{enumerate}
    \item hierarchical clustering of hireability coefficients
    \item number of levels = 1 (i.e. trivial, economy) + L levels, where L is the number
          for which the levels L and L - 1 (i.e. the previous aggregation) are identical
    \item alternatively, use optimization methods to choose number of aggregation levels
    \item alternatively, use the same number of aggregation levels as popular economic
          taxonomies (e.g. ISIC, NAICS, SIC, NACE, OKVED)
\end{enumerate}
\subsubsection{Competition in Labor Markets}
\begin{enumerate}
    \item every hireable skill set competes at every level / market on which it is
          hireable
    \item market value, industry value, etc = percent of money in an aggregation level
    \item market value, industry employability, etc = weighted employability of workers
          at that level
    \item market value, industry competitiveness, etc = weighted competitiveness of
          workers at that level
    \item etc etc (define every variable for labor market aggregations)
\end{enumerate}

% endregion

% region: microeconomic models 
\section{Microeconomic Models}
\subsection{Marginal Human Capital Models}
\subsubsection{Marginal Compensation or Market Prices}
\begin{enumerate}
    \item wages $y_k$
    \item NNLS attributes vs wages $\rightarrow p_i$
    \item $p_i$ is the marginal compensation (i.e. market price) of a point in an attribute
    \item market prices vector $\textbf{p}$
\end{enumerate}

\subsubsection{Marginal Time Investment}
\begin{enumerate}
    \item years of education and experience $\tau_k$
    \item NNLS attributes vs years of education and experience $\rightarrow \eta_i$
    \item $\eta_i$ is the marginal time investment (expected time of arrival) to gain a point in an attribute
    \item marginal time investment (ETA) vector $\boldsymbol{\eta}$
\end{enumerate}

\subsubsection{Micro-Flexibility}
\begin{enumerate}
    \item NNLS attributes vs attributes $\rightarrow \phi_{i,j}$
    \item micro-flexibility = expected gain on other attributes for an additional point
          in an attribute
    \item !requires additional statistical pressupositions for regression coefficients to be interpretable as $\phi_{i,j}$
    \item !potential for recursion
    \item !affects ETA
    \item define a geometric progression learning model?
    \item attribute micro-flexibility vector $\boldsymbol{\phi_{i}}$
    \item attribute micro-flexibility matrix $\boldsymbol{\phi}$
\end{enumerate}

\subsection{Aggregate Human Capital Models}
\subsubsection{Skill Set Value}
\begin{enumerate}
    \item expected compensation based on market prices $\textbf{p}$ and skill set
          $\boldsymbol{a_k}$
    \item $y_{k}^{\text{\tiny{E}}} = \text{E}[y_k | \boldsymbol{a_k}, \textbf{p}] = \sum_{i=1}^{m}{a_{i}^{k} p_i}$
          % \item $\text{E}[ \ y_k \ | \ \textbf{p}, \boldsymbol{a_k}\ ]$
          % \item $\text{E}(y_k | \textbf{p}, \boldsymbol{a_k})$
    \item use another notation for this?
    \item probably define a value variable 
\end{enumerate}

\subsubsection{Macro-Flexibility}
\begin{enumerate}
    \item attribute macro-flexibility $\Phi_i = \left( \frac{1}{W} \right)
              \sum_{k=1}^{n}{\tilde{a}_{i}^{k} w_k}$
    \item attribute macro-flexibility vector $\boldsymbol{\Phi}$
\end{enumerate}

% endregion

% region: additional comparative models 
\section{Additional Comparative Models}
\subsection{Total Time Investment (ETA)}
\begin{enumerate}
    \item aggregate expected time investment to fill competency gaps
    \item uppercase eta $\eta$
    \item $\text{H}_{k,q} = \sum_{i=1}^{m}{
              \delta(a_{i}^{k}, a_{i}^{q})
              \eta_i
              }$
    \item vector $\textbf{H}\boldsymbol{_k}$
    \item matrix $\textbf{H}$
    \item p.s.: notation conflicts with hireability matrix (probably change hireability
          notation)
\end{enumerate}

\subsection{Letter-Shaped Skill Sets}
\begin{enumerate}
    \item a generalization of ``T-Shaped Skills''
    \item refer to the atlas.letters R package
\end{enumerate}

% endregion

% region: additional skill set models 
\section{Additional Skill Set Models}
\subsection{Versatility}
\begin{enumerate}
    \item the versatility of a skill set is the weighted sum of its macro-flexibility
          % \item  $\Phi_k = \frac{
          %           \sum_{i=1}^{m}{\ddot{a}_{i}^{k} \Phi_i}
          %           }{
          %           \sum_{i=1}^{m}{\ddot{a}_{i}^{k}}
          %           }
          %       $
    \item $\Phi_k = \frac{ \sum_{i=1}^{m}{a_{i}^{k} \Phi_i} }{
              \sum_{i=1}^{m}{a_{i}^{k}} }$
\end{enumerate}

\subsection{Leverage? [aggregate micro-flexibility]}
\begin{enumerate}
    \item define a name for aggregate micro-flexibility (leverage?)
    \item same procedure as above
          % \item $\phi_k = \frac{ \sum_{i=1}^{m}{a_{i}^{k} \phi_i} }{
          %     \sum_{i=1}^{m}{a_{i}^{k}} }$
\end{enumerate}

% endregion

% region: roadmap models 
\section{Roadmap Models}
\subsection{Career Recommendation}
\begin{enumerate}
    \item this theory needs to be further developed, as I never got the chance to write
          it down completely
    \item see quantitative roadmap sketch for an idea of what this should be
\end{enumerate}

\subsubsection{Macro-Strategies}
\subsubsection{Preference-Adjusted Strategic Matching}

\subsection{Competency Training Recommendation}
\subsubsection{Micro-Strategies}
\subsubsection{Preference-Adjusted Strategic Training}

% endregion

% region: factor-analytic models 

\section{Factor-Analytic Models}
\subsection{Exploratory Factor Analysis of the O*NET Database}
\begin{enumerate}
    \item use standard factor-analytic notation (I already reserved many latin and greek
          characters for this reason)
\end{enumerate}

\subsection{Psychometric Questionnaire Optimization}
\subsubsection{Minimal Factor Representation}
\subsubsection{Retained Variance}
\subsubsection{Retained Matching Accuracy and Precision}

\subsection{Factor-Analytic Intelligence Quotient (IQ) Approximation}
\begin{enumerate}
    \item in psychometrics, general intelligence (IQ) is denoted as $g$
    \item define a expected value notation (also solves the expected compensation, or
          skill set value, problem above)
    \item $g_{k}^{\text{\tiny{E}}} = \text{E}[g_k | \boldsymbol{a_k}, \boldsymbol{\Psi_?}] = 
    \left(
              \frac{1}{m_\psi?}
              \right)
              \sum_{\psi?}{
                  \tilde{a}_{i}^{k}
              }
          $
    \item $\psi?$ are the subset of items which are proxies for intelligence
    \item alternatively, write this coefficient as factor scores (if there is a notation
          for factor scores)
\end{enumerate}

\subsection{Factor-Analytic Comparative Statics}
\begin{enumerate}
    \item there isn't a mathematical notation for this method yet
    \item but most or all of its variables and functions have already been defined above
    \item refer to the atlas.fstatics R package to understand how this model works
\end{enumerate}
\subsubsection{Unbounded Exogenous Impacts}
\subsubsection{Truncated Exogenous Impacts}
\subsection{Career Type}
\begin{enumerate}
    \item refer to the atlas.acti R package
\end{enumerate}
% endregion

% region: writing plan 
\section{Writing Plan}
\subsection{Introduction to Quantitative Career Matching: How to Statistically Pick the Right Occupation}
\begin{enumerate}
    \item introduce the much neglected topic of data-driven career choice and development
    \item estimate the baseline Euclidean matching model
    \item for simplicity's sake
    \item to benchmark matching methods
\end{enumerate}

\subsection{Equivalence Estimators for Quantitative Career Matching}
\begin{enumerate}
    \item aeq, seq functions
    \item eeq function?
    \item take $\gamma_k$ and $c_k$ parameters as given
    \item explain how to calculate generality and competence on the next paper
    \item recalculate the Euclidean matching model from paper 1 with these new methods
\end{enumerate}

\subsection{Generalists vs Specialists: Who Are the Most Competent Workers?}
\begin{enumerate}
    \item explain how to calculate generality and competence
    \item weighted correlation between generality and competence
\end{enumerate}

\subsection{Alternative/Advanced Career Matching Methods}
\begin{enumerate}
    \item explain all career matching methods besides the baseline Euclidean model
    \item compare all models against one another
    \item benchmark models against the baseline Euclidean model
    \item select one matching method
\end{enumerate}

\subsection{Employability and Competitiveness}
\begin{enumerate}
    \item having defined all the statistical pressupositions for these models in the
          previous papers, calculate employability and competitiveness coefficients
    \item choose another name for this paper?
\end{enumerate}

\subsection{A Hierarchical Clustering Labor Market Taxonomy}
\begin{enumerate}
    \item define the hireability taxonomic model
    \item choose another name for this paper?
\end{enumerate}

\subsection{Competition in Labor Markets}
\begin{enumerate}
    \item continue from where the taxonomy paper left off
    \item choose another name for this paper
\end{enumerate}

\subsection{Human Capital Flexibility}
\begin{enumerate}
    \item calculate macro-flexibility
    \item calculate micro-flexibility
    \item aggregate macro-flexibility for each skill set (versatility)
    \item aggregate micro-flexibility for each skill set (?)
\end{enumerate}

\subsection{Time Investment and Expected Returns of Human Capital}
\begin{enumerate}
    \item choose another name for this paper?
    \item calculate marginal compensation of human capital
    \item calculate marginal time investment of human capital
    \item aggregate marginal compensation of human capital (skill set value)
    \item aggregate marginal time investment of human capital (ETA)
    \item calculate which careers are most efficient to aim for in the short run
          (starting from the average skill set)
    \item calculate which careers are most effective to aim for in the long run (starting
          from the average skill set)
\end{enumerate}

\subsection{The Career Roadmap: A Mathematical Map to Career Choice and Career Development}
\begin{enumerate}
    \item choose another name for this paper?
    \item The Career Roadmap: Statistical Methods for Optimal Career Choice and Career
          Development?
    \item define the concept of macro-strategies
    \item combine all statistical metrics so far as the career recommendation coefficient
    \item define the concept of micro-strategies
    \item combine all statistical metrics so far as the training recommendation
          coefficient
    \item calculate which careers are most efficient to aim for in the short run
          (starting from the average skill set)
    \item calculate which careers are most effective to aim for in the long run (starting
          from the average skill set)
\end{enumerate}

\subsection{Letter-Shaped Career Profiles: A Generalization of ``T-Shaped Skills''}
\begin{enumerate}
    \item define the methodology for converting hersheys fonts to career profiles
    \item match career profiles against the latin, greek, and cyrillic alphabets
    \item find a way to include ``from A-to-Z'' in this paper's name
    \item or from ``A-to-$\Omega$"
\end{enumerate}

\subsection{Factor-Analyzing the Occupational Information Network Database}
\begin{enumerate}
    \item EFA on O*NET db
\end{enumerate}

\subsection{Career Types for the Occupational Information Network Database}
\begin{enumerate}
    \item calculate career types based on the factor model from the EFA paper and the
          statistics already defined in the previous papers
    \item career type molecules
\end{enumerate}

\subsection{Factor-Analytic Comparative Statics: Estimating Exogenous Impacts on the Labor Market}
\begin{enumerate}
    \item ai impact analysis
    \item some other impact analysis
    \item aggregate on the attribute level
    \item aggregate on the occupation level
    \item aggregate on each taxonomic level defined in the economic taxonomy paper
          (including the entire economy)
\end{enumerate}

\subsection{Factor-Analytic Intelligence Quotient Approximation}
\begin{enumerate}
    \item based on the factor model from the EFA paper, choose a proxy
    \item calculate factor-analytic approximation of IQ
    \item compare against data
\end{enumerate}
% endregion

\end{document}\text{vs}