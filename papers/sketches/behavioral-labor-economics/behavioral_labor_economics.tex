%SECTION: SETUP
% region: document class 
% \documentclass{article}
\documentclass{elsarticle} %elsevier
% \documentclass[twocolumn]{elsarticle} %elsevier
% \documentclass{apa7} %american psychological association
% \documentclass[doc]{apa7} %american psychological association
% \documentclass{aomart} %annals of mathematics
% \documentclass{imsproc} %institute of mathematics and statistics
% \documentclass{nwejmart} %north-western european journal of mathematics
% \documentclass{paper} 
% \documentclass{sr-vorl} %springer book
% \documentclass{ecca} %economica
% \documentclass{ecta} %econometrica
% \documentclass{apecon} %applied economics

% endregion

% region: packages 
\usepackage{xr}
\usepackage{physics}
\usepackage{amsmath}
\usepackage{float}
\usepackage{enumitem}
\setlist[enumerate]{label*=\arabic*.}
\usepackage[hidelinks]{hyperref}

% endregion

% region: title 
% Title
\title{
    Behavioral Labor Economics
    % :\\ Bridging the Gap Between Economic Science and Psychometrics
}
\author{Cao Bittencourt}
\affiliation{{B. Sc. in Economics from EPGE (FGV), RJ, Brazil.}}
\affiliation{{Statistician at Atlas Career Guide Inc., FL, USA.}}
\date{\today}

% endregion

% region: display breaks 
\allowdisplaybreaks

% endregion

%SECTION: PAPER
\begin{document}

% \maketitle

% region: abstract 
\begin{abstract}
    \noindent
    Psychometricians tend to restrict themselves to the theoretical and practical aspects of measuring psychological phenomena, and are generally more concerned with deriving reliable and interpretable scales to describe human behavior for its own sake. That is to say, their focus is specialized on the foundational problems of measurement, often on a more individualized level (i.e. closer to real human subjects) and in a somewhat more ``concrete'' fashion (though theory is crucial for psychometric modelling, as well).
    In a sense, Psychometricians are actually more ``micro-oriented'' than Microeconomists. For even these (Microeconomists), begin their investigation of human activity on a higher, and much more abstracted, level than Psychometricians. Likewise, in the field of Labor Economics, analyses are usually restricted to Macroeconomic ``heights'', drastically simplifying social and psychological variables, which are only the object of Economic Science in so far as they allow for a better understanding of market dynamics. A complex topic like human capital, for instance, is reduced to aggregate quantities of efficiency, productivity, and educational attainment, instead of modelled on a more detailed -- let us say, ``micro-microeconomic'' -- scale, as they would be in Psychometrics.
    Of course, these procedures are both appropriate for each of the respective disciplines. For, indeed, it makes sense to heavily abstract when dealing with Macroeconomic phenomena, just as it also makes sense to be precise when it comes to psychological measurement. In Macroeconomics, it is, in fact, correct to assume complexity ``evens itself out'' somewhere within a lower layer of specificity. On the other hand, Psychometricians devote most of their time studying these very realities Economists ``take for granted'' and rarely ``venture'' beyond the spheres of measurement, model validation, and psychological theory, into more abstracted and sometimes more quantitatively heavy fields, such as Economics.
    We propose, however, that this isolation of both disciplines fails to realize some of their own applications. Indeed, we find there to be a very interesting, important, and yet mostly uncharted, intersection between Psychometrics, Labor Economics, and Statistics in general, which we call ``Behavioral Labor Economics''.
\end{abstract}
% endregion

% region: title 
\maketitle

% endregion

\end{document}