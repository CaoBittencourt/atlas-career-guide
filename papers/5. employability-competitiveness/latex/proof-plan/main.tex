%SECTION: SETUP 
% region: document class 
\documentclass[hidelinks, nonatbib]{elsarticle}
\makeatletter
\let\c@author\relax
\makeatother

% endregion

% region: packages 
\usepackage{amsmath}
\usepackage{amssymb}
\usepackage{amsthm}
\usepackage{float}
\usepackage{tabularray}
\usepackage{enumitem}
\setlist[enumerate]{label*=\arabic*.}
\usepackage[backend=biber, style=apa]{biblatex}
\usepackage{hyperref}

% endregion

% region: source files 
% equations
% macros
% time to complete a task
\newcommand{\ttc}{\text{ttc}}

% time allocation
\newcommand{\ta}{\text{ta}}

% time allocation antiderivative
\newcommand{\TA}{\text{TA}}

% time allocation inverse antiderivative
\newcommand{\invTA}{\text{TA}^{-1}}

% operational output
\newcommand{\oo}{\mho}

% productivity
\newcommand{\pctt}{\tilde{T}}

% minimum required productivity
\newcommand{\mrp}{\underaccent{\bar}{\pctt}}

% sufficiently qualified workers
\newcommand{\wsq}{
    \boldsymbol{\ddot{w}_q}(
        \boldsymbol{w_q}
        ,\boldsymbol{\mrp_q}
        ,\boldsymbol{\pctt_q}
    )
}

% figures
\input{figures.tex}

% tables
\input{tables.tex}

% references
\addbibresource{references.bib}

% endregion

% region: display breaks 
\allowdisplaybreaks

% endregion

% region: title 
\title{
    The Employability Theorem
}
\author{Cao Bittencourt}
\affiliation{{B. Sc. in Economics from EPGE (FGV), RJ, Brazil.}}
\affiliation{{Statistician at Atlas Career Guide Inc., FL, USA.}}
\date{\today}

% endregion

%SECTION: PAPER 
\begin{document}

% region: abstract 
\begin{abstract}
    In this document, the Employability Theorem is demonstrated from a set of fairly tautological axioms, which are presupposed in quantitative career choice and career development methods.
\end{abstract}

\begin{keyword}
    Employability theorem; 
    Career choice; 
    Career development; 
    Vocational choice; 
    Occupational Information Network; 
    O*NET.
\end{keyword}

% endregion

% region: title 
\maketitle

% endregion

% region: proof plan 
\section{Proof Plan}
\begin{enumerate}
    \item basic presuppositions
    \item basic lemmas
    \item complex tasks
    \item occupations are but tasks
    \item occupations' tasks are complex
    \item occupations' tasks are holistic (operation)
    \begin{enumerate}
        \item more difficult tasks presuppose the easier tasks have been accomplished
        \item i.e. $l \in [0,1]$ is a ``progress bar'' of an occupation's operation
        \item strongly holistic: each task $l \geq \bar{l}$ requires all the previous $l \in [0, \bar{l}], \bar{l} \in [0,1]$ difficulty levels to be accomplished. in addition, if all $l \in [0,1]$ levels are not all accomplished, the whole effort is vain and the operation is not completed (i.e. round down $\mho_q$ when calculating operational output). furthermore, each and every $l \in [0,1]$ difficulty level cannot be outsourced (i.e. only a perfectly qualified worker can output a unit of the occupation's operation).
        \item moderately holistic: each task $l \geq \bar{l}$ requires all the previous $l \in [0, \bar{l}], \bar{l} \in [0,1]$ difficulty levels to be accomplished. in addition, if all $l \in [0,1]$ levels are not all accomplished, the whole effort is vain and the operation is not completed (i.e. round down $\mho_q$ when calculating operational output). however, each and every $l \in [0,1]$ difficulty level can be outsourced (i.e. workers can output partial units of the occupation's operation, which contribute to the operation's completion).
        \item weakly holistic: each task $l \geq \bar{l}$ requires all the previous $l \in [0, \bar{l}], \bar{l} \in [0,1]$ difficulty levels to be accomplished. however, if not all $l \in [0,1]$ levels are accomplished, the whole effort is not vain and the operation is partially completed (i.e. do not round $\mho_q$ when calculating operational output). furthermore, each and every $l \in [0,1]$ difficulty level can be outsourced (i.e. workers can output partial units of the occupation's operation, which contribute to the operation's completion).
    \end{enumerate}
    \item assume weak occupational complexity axiom (the other versions are too strict)
\end{enumerate}

% endregion

\end{document}