%SECTION: SETUP 
% region: document class 
\documentclass[hidelinks, nonatbib]{elsarticle}
\makeatletter
\let\c@author\relax
\makeatother

% endregion

% region: packages 
\usepackage{amsmath}
\usepackage{amssymb}
\usepackage{amsthm}
\usepackage{float}
\usepackage{tabularray}
\usepackage{enumitem}
\setlist[enumerate]{label*=\arabic*.}
\usepackage[backend=biber, style=apa]{biblatex}
\usepackage{hyperref}

% endregion

% region: source files 
% equations
% equivalence function
\newcommand{\eq}{\text{eq}}

% skill set interchangeability
\newcommand{\esset}{\textit{\ss}}

% skill set interchangeability vector
\newcommand{\essetvec}{\textbf{\textit{\ss}}}

% skill set interchangeability matrix
\newcommand{\Esset}{\textbf{\ss}}

% attribute equivalence
% \newcommand{\aeq}{\textit{ä}}
\newcommand{\aeq}{{\"a}}

% linear-logistic trigonometrically-scaled equivalence estimator
\newcommand{\EquivalenceEstimator}{
% \begin{gather}
%     \label{eqn:equivalence_estimator}
%     \eq(x,M) = 
%         x\left\{
%             1 + M (1 - x) \exp[-
%                 \tan\bigg(
%                     \frac{\pi}{2}
%                     \cos
%                         \bigg(
%                             \frac{\pi}{2} 
%                             % M (1 - M)
%                             x (1 - M)
%                         \bigg)
%                 \bigg)
%                 (x - M)
%             ]
%         \right\} ^ {
%             -\frac{M}{x}
%     },\\
%     x, M \in [0,1].
% \end{gather}
% 
\begin{gather}
    \label{eqn:equivalence_estimator}
    \eq(x,M) = 
        x\left\{
            1 + M (1 - x) \exp[-
                b(x - M)
            ]
        \right\} ^ {
            -\frac{M}{x}
    },\\
    b = 
    \tan\bigg[
        \frac{\pi}{2}
        \cos^{M(1 - M)}
        % \cos^{x(1 - M)}
            \bigg(
                \frac{\pi}{2} 
                % M (1 - M)
                x (1 - M)
            \bigg)
    \bigg],\\
    x, M \in [0,1].
\end{gather}
% 
% \begin{gather}
%     \label{eqn:equivalence_estimator}
%     \eq(x,M) = 
%         x\left\{
%             1 + M (1 - x) \exp[-
%                 \tan\bigg(
%                     \frac{\pi}{2}
%                         \cos^{x(1 - M)}
%                         \bigg(
%                             \frac{\pi}{2} 
%                             x (1 - M)
%                         \bigg)
%                     \bigg)
%                 (x - M)
%             ]
%         \right\} ^ {
%             -\frac{M}{x}
%     },\\
%     x, M \in [0,1].
% \end{gather}
}

% skill set interchangeability
\newcommand{\InterchangeabilityFunction}{
    \begin{gather}
        \esset_{k,q} = 
        \esset(s(\boldsymbol{a_k}, \boldsymbol{a_q}), M) = 
        \eq(s(\boldsymbol{a_k}, \boldsymbol{a_q}), M)
    \end{gather}
}

% interchangeability matrix
\newcommand{\InterchangeabilityMatrix}{
    \begin{gather}
        \Esset = 
        \begin{bmatrix}
            \textit{\ss}_{1,1} & \hdots & \textit{\ss}_{n,1}\\
            \vdots & \ddots & \vdots\\
            \textit{\ss}_{1,n} & \hdots & \textit{\ss}_{n,n}\\
        \end{bmatrix} = 
        \begin{bmatrix}
            1 & \hdots & \textit{\ss}_{k,1} & \hdots & \textit{\ss}_{n,1}\\
            \vdots & \ddots & \vdots & \vdots & \vdots\\
            \textit{\ss}_{1,k} & \hdots & 1 & \hdots & \textit{\ss}_{n,k}\\
            \vdots & \vdots & \vdots & \ddots & \vdots\\
            \textit{\ss}_{1,n} & \hdots & \textit{\ss}_{k,n} & \hdots & 1\\
        \end{bmatrix}\\
    \end{gather}
}

% binary interchangeability (hireability)
\newcommand{\SignalFunction}{
    \begin{gather}
        h_{k,q} = 
        h(\esset_{k,q}) = 
        \begin{cases}
            1, \text{if } \esset_{k,q} \geq 0.5.\\
            0, \text{otherwise.}
        \end{cases}
    \end{gather}
}

% attribute equivalence matrix
\newcommand{\AttributeEquivalence}{
    \begin{gather}
        \aeq_{i}^{k} = 
        \aeq(\boldsymbol{a_k}, M) = 
        \eq\left(
            \frac{
                a_{i}^{k}
            }{
                \max{a_{j}^{k}}
            }
            , M
        \right)
    \end{gather}
}

% figures
\input{figures.tex}

% tables
% General Occupational Statistics
\newcommand{\GeneralOccupationalStatistics}{
    \begin{longtblr}[
        caption = {General Occupational Statistics},
        label = {tab:onet_careers},
        note{1} = {Number of SOC occupations plus variants.},
        note{2} = {Total employment levels in the United States in 2022.},
        note{3} = {Employment-weighted mean wages in the United States in 2022.},
        note{4} = {Total cluster wages as a percentage of total wages in the United States in 2022.},
        remark{Source} = {\cites{bls2022data}{onet2023data}.}
    ]{
        colspec = {X[0.5, l, m]X[0, c, m]X[0.25, c, m]X[0.25, c, m]X[0, c, m]},
        width = \linewidth,
        rowhead = 1,
        rowfoot = 0
    }
        \toprule
            Cluster & N\TblrNote{1} & Employment\TblrNote{2} & Wage\TblrNote{3} & Market Share\TblrNote{4} \\ 
        \midrule
            Business Management \& Administration &  65 & 26,639,371 & \$59,924.00 & 21.05\% \\ 
            Health Science &  99 & 16,856,404 & \$62,753.00 & 13.95\% \\ 
            Marketing &  26 & 13,961,830 & \$45,512.00 & 8.38\% \\ 
            Manufacturing & 138 & 13,644,785 & \$44,169.00 & 7.95\% \\ 
            Hospitality &  42 & 17,416,290 & \$28,965.00 & 6.65\% \\ 
            Transportation, Distribution \& Logistics &  62 & 11,018,200 & \$45,511.00 & 6.61\% \\ 
            Architecture \& Construction &  86 & 9,005,327 & \$53,142.00 & 6.31\% \\ 
            Education \& Training &  64 & 7,266,480 & \$62,621.00 & 6.00\% \\ 
            Finance &  22 & 5,255,514 & \$70,325.00 & 4.87\% \\ 
            Law, Public Safety, Corrections \& Security &  35 & 4,790,620 & \$61,985.00 & 3.92\% \\ 
            Information Technology &  22 & 3,588,065 & \$82,061.00 & 3.88\% \\ 
            Human Services &  38 & 7,823,620 & \$36,809.00 & 3.80\% \\ 
            Science, Technology, Engineering \& Mathematics &  70 & 2,207,815 & \$100,683.00 & 2.93\% \\ 
            Agriculture, Food \& Natural Resources &  40 & 2,140,244 & \$51,134.00 & 1.44\% \\ 
            Arts, Audio/Video Technology \& Communications &  40 & 1,692,270 & \$57,836.00 & 1.29\% \\ 
            Government \& Public Administration &  24 & 1,166,588 & \$62,636.00 & 0.96\% \\
        \bottomrule
    \end{longtblr}
}

% Summary of Occupations
\newcommand{\SummaryOccupations}{
    \begin{table}[H]
        \centering
            \begin{talltblr}[
            caption = {Summary of Sample Occupations},
            label = {tab:summary_occupations},
            note{1} = {``STEM" stands for ``Science, Technology, Engineering, and Mathematics".},
            note{2} = {Total employment levels in the United States in 2022.},
            note{3} = {Median wages in the United States in 2022.},
            remark{Source} = {\cites{bls2022data}{onet2023data}.}
            ]{
            colspec = {ccccc},
            % colspec = {X[c, m]Q[c, m]X[c, m]X[c, m]X[c, m]},
            width = \linewidth
            }
                \toprule
                    SOC & Occupation & Cluster & Employment\TblrNote{2} & Wage\TblrNote{3} \\ 
                \midrule
                    19-2012 & Physicists & STEM\TblrNote{1} & 18,840 & \$152,430.00 \\
                    17-2141 & Mechanical Engineers & STEM\TblrNote{1} & 277,560 & \$95,300.00 \\ 
                    13-2041 & Credit Analysts & Finance & 71,960 & \$77,440.00 \\ 
                    35-9021 & Dishwashers & Hospitality & 431,840 & \$28,130.00 \\ 
                \bottomrule
            \end{talltblr}
    \end{table}
} 

% Mechanical Engineers
\newcommand{\MatchesEngineers}{
    \begin{longtblr}[
        caption = {Best Career Matches -- Mechanical Engineers},
        label = {tab:similarity_engineers},
        remark{Note} = {Similarity scores estimated with Euclidean matching and linear weights.},
        remark{Source} = {Author's calculation based on \cite{onet2023data}.}
    ]{
        colspec = {X[l, m]X[0, c, m]},
        width = \linewidth,
        rowhead = 1,
        rowfoot = 0
    }
        \toprule
            Comparison Occupation & Similarity \\ 
        \midrule
            Mechanical Engineers & 1.00 \\ 
            Aerospace Engineers & 0.90 \\ 
            Marine Engineers and Naval Architects & 0.88 \\ 
            Nuclear Engineers & 0.88 \\ 
            Mechatronics Engineers & 0.87 \\ 
            Microsystems Engineers & 0.86 \\ 
            Photonics Engineers & 0.86 \\ 
            Fuel Cell Engineers & 0.86 \\ 
            Automotive Engineers & 0.86 \\ 
            Materials Engineers & 0.86 \\ 
            Mining and Geological Engineers, Including Mining Safety Engineers & 0.86 \\
        \bottomrule
    \end{longtblr}
}

% Physicists
\newcommand{\MatchesPhysicists}{
    \begin{longtblr}[
        caption = {Best Career Matches -- Physicists},
        label = {tab:similarity_physicists},
        remark{Note} = {Similarity scores estimated with Euclidean matching and linear weights.},
        remark{Source} = {Author's calculation based on \cite{onet2023data}.}
    ]{
        colspec = {X[l, m]X[0, c, m]},
        width = \linewidth,
        rowhead = 1,
        rowfoot = 0
    }
        \toprule
            Comparison Occupation & Similarity \\ 
        \midrule
            Physicists & 1.00 \\ 
            Astronomers & 0.85 \\ 
            Mathematicians & 0.80 \\ 
            Physics Teachers, Postsecondary & 0.80 \\ 
            Materials Scientists & 0.79 \\ 
            Engineering Teachers, Postsecondary & 0.79 \\ 
            Nanosystems Engineers & 0.78 \\ 
            Aerospace Engineers & 0.77 \\ 
            Biochemists and Biophysicists & 0.77 \\ 
            Nuclear Engineers & 0.77 \\ 
            Computer and Information Research Scientists & 0.77 \\ 
        \bottomrule
    \end{longtblr}
}

% Credit Analysts
\newcommand{\MatchesCreditAnalysts}{
    \begin{longtblr}[
        caption = {Best Career Matches -- Credit Analysts},
        label = {tab:similarity_credit_analysts},
        remark{Note} = {Similarity scores estimated with Euclidean matching and linear weights.},
        remark{Source} = {Author's calculation based on \cite{onet2023data}.}
    ]{
        colspec = {X[l, m]X[0, c, m]},
        width = \linewidth,
        rowhead = 1,
        rowfoot = 0
    }
        \toprule
            Comparison Occupation & Similarity \\ 
        \midrule
            Credit Analysts & 1.00 \\ 
            Accountants and Auditors & 0.87 \\ 
            Loan Interviewers and Clerks & 0.86 \\ 
            Budget Analysts & 0.85 \\ 
            Securities, Commodities, and Financial Services Sales Agents & 0.85 \\ 
            Financial Examiners & 0.85 \\ 
            Business Intelligence Analysts & 0.85 \\ 
            Tax Preparers & 0.85 \\ 
            Insurance Underwriters & 0.84 \\ 
            Bookkeeping, Accounting, and Auditing Clerks & 0.84 \\ 
            Tax Examiners and Collectors, and Revenue Agents & 0.83 \\ 
        \bottomrule
    \end{longtblr}
}

% Dishwashers
\newcommand{\MatchesDishwashers}{
    \begin{longtblr}[
        caption = {Best Career Matches -- Dishwashers},
        label = {tab:similarity_dishwashers},
        remark{Note} = {Similarity scores estimated with Euclidean matching and linear weights.},
        remark{Source} = {Author's calculation based on \cite{onet2023data}.}
    ]{
        colspec = {X[l, m]X[0, c, m]},
        width = \linewidth,
        rowhead = 1,
        rowfoot = 0
    }
        \toprule
            Comparison Occupation & Similarity \\ 
        \midrule
            Dishwashers & 1.00 \\ 
            Janitors and Cleaners, Except Maids and Housekeeping Cleaners & 0.88 \\ 
            Maids and Housekeeping Cleaners & 0.88 \\ 
            Pressers, Textile, Garment, and Related Materials & 0.87 \\ 
            Models & 0.87 \\ 
            Postal Service Mail Sorters, Processors, and Processing Machine Operators & 0.87 \\ 
            Sewing Machine Operators & 0.86 \\ 
            Laundry and Dry-Cleaning Workers & 0.86 \\ 
            Orderlies & 0.86 \\ 
            Cutters and Trimmers, Hand & 0.86 \\ 
            Shoe and Leather Workers and Repairers & 0.86 \\
        \bottomrule
    \end{longtblr}
}

% Similarity matrix
\newcommand{\SimilarityMatrix}{
    \begin{table}[H]
        \centering
        \begin{talltblr}[
            caption = {Similarity Matrix},
            label = {tab:similarity_matrix},
            remark{Note 1} = {Similarity scores estimated with Euclidean matching and linear weights.},
            remark{Note 2} = {Comparison occupations on the horizontal axis.},
            remark{Source} = {Author's calculation based on \cite{onet2023data}.}
        ]{
            colspec = {X[c, m]X[c, m]X[c, m]X[c, m]X[c, m]},
            width = \linewidth
        }
            \toprule
                & Mechanical Engineers & Physicists & Credit Analysts & Dishwashers \\ 
            \midrule
                Mechanical Engineers & 1.00 & 0.75 & 0.54 & 0.38 \\ 
                Physicists & 0.80 & 1.00 & 0.57 & 0.34 \\ 
                Credit Analysts & 0.75 & 0.66 & 1.00 & 0.54 \\ 
                Dishwashers & 0.62 & 0.53 & 0.66 & 1.00 \\ 
            \bottomrule
        \end{talltblr}
    \end{table}
}

% Detailed skill sets
\newcommand{\DetailedSkillSets}{
    \begin{longtblr}[
        caption = {Detailed Skill Sets},
        label = {tab:detailed_skill_sets},
        remark{Source} = {\cite{onet2023data}.}
    ]{
        % colspec = {X[1.5, c, m]X[1, c, m]X[1, c, m]X[1, c, m]X[1, c, m]},
        colspec = {X[0.4, l, m]X[0.2, c, m]X[0.2, c, m]X[0.2, c, m]X[0.2, c, m]},
        width = \linewidth,
        rowhead = 1,
        rowfoot = 0
    }
        \toprule
            Competency & Mechanical Engineers & Physicists & Credit Analysts & Dishwashers \\
        \midrule
            Engineering and Technology & 87 & 86 & 13 & 0 \\ 
            Industrial Design & 84 & 45 & 0 & 0 \\ 
            Applied Mathematics & 79 & 94 & 68 & 0 \\ 
            Mechanical & 78 & 28 & 0 & 12 \\ 
            Physics & 73 & 97 & 0 & 0 \\ 
            Oral Comprehension & 70 & 84 & 59 & 32 \\ 
            Reading Comprehension & 70 & 84 & 57 & 30 \\ 
            Oral Expression & 70 & 82 & 59 & 30 \\ 
            Deductive Reasoning & 70 & 79 & 64 & 29 \\ 
            Pure Mathematics & 70 & 79 & 59 & 14 \\ 
            Mathematical Reasoning & 68 & 84 & 59 & 16 \\ 
            Written Comprehension & 68 & 79 & 59 & 29 \\ 
            Complex Problem-Solving & 68 & 70 & 43 & 30 \\ 
            Computers and Electronics & 67 & 83 & 43 & 0 \\ 
            Natural Science & 66 & 82 & 5 & 0 \\ 
            Operations Analysis & 66 & 52 & 34 & 0 \\ 
            Active Learning & 63 & 79 & 52 & 29 \\ 
            Information Ordering & 63 & 71 & 52 & 30 \\ 
            Number Facility & 61 & 79 & 63 & 16 \\ 
            Critical Thinking & 61 & 70 & 57 & 30 \\ 
            Monitoring & 61 & 57 & 43 & 30 \\ 
            Written Expression & 59 & 80 & 57 & 23 \\ 
            Inductive Reasoning & 59 & 79 & 59 & 29 \\ 
            Category Flexibility & 59 & 73 & 50 & 29 \\ 
            Problem Sensitivity & 59 & 70 & 57 & 29 \\ 
            Judgment and Decision & 59 & 64 & 54 & 30 \\ 
            Originality & 57 & 75 & 37 & 25 \\ 
            Writing & 57 & 71 & 55 & 25 \\ 
            Native Language & 57 & 70 & 60 & 38 \\ 
            Active Listening & 57 & 68 & 57 & 32 \\ 
            Near Vision & 57 & 57 & 61 & 34 \\ 
            Administration and Management & 57 & 54 & 42 & 37 \\ 
            Quality Control Analysis & 57 & 46 & 11 & 29 \\ 
            Chemistry & 56 & 61 & 0 & 21 \\ 
            Fluency of Ideas & 55 & 73 & 41 & 27 \\ 
            Speaking & 55 & 68 & 57 & 30 \\ 
            Visualization & 55 & 61 & 25 & 29 \\ 
            Systems Analysis & 55 & 57 & 43 & 16 \\ 
            Systems Evaluation & 55 & 55 & 45 & 16 \\ 
            Speech Recognition & 54 & 57 & 50 & 30 \\ 
            Technology Design & 54 & 48 & 0 & 0 \\ 
            Operations Monitoring & 54 & 34 & 14 & 30 \\ 
            Instructing & 52 & 66 & 34 & 20 \\ 
            Time Management & 52 & 52 & 43 & 30 \\ 
            Troubleshooting & 52 & 29 & 0 & 29 \\ 
            Education and Training & 51 & 72 & 29 & 41 \\ 
            Administrative & 51 & 40 & 59 & 0 \\ 
            Learning Strategies & 50 & 71 & 36 & 18 \\ 
            Selective Attention & 50 & 57 & 41 & 30 \\ 
            Persuasion & 50 & 50 & 36 & 27 \\ 
            Coordination & 50 & 48 & 36 & 30 \\ 
            Production and Processing & 49 & 34 & 29 & 0 \\ 
            Speech Clarity & 48 & 71 & 45 & 27 \\ 
            Flexibility of Closure & 48 & 55 & 45 & 29 \\ 
            Perceptual Speed & 48 & 46 & 45 & 30 \\ 
            Customer and Personal Service & 48 & 37 & 39 & 38 \\ 
            Far Vision & 46 & 52 & 30 & 32 \\ 
            Management of Personnel Resources & 45 & 45 & 27 & 29 \\ 
            Speed of Closure & 43 & 55 & 41 & 29 \\ 
            Negotiation & 43 & 39 & 39 & 29 \\ 
            Social Perceptiveness & 41 & 45 & 41 & 29 \\ 
            Service Orientation & 41 & 43 & 43 & 29 \\ 
            Visual Color Discrimination & 41 & 41 & 21 & 29 \\ 
            Management of Material Resources & 41 & 32 & 13 & 13 \\ 
            Multitasking & 39 & 41 & 29 & 29 \\ 
            Management of Financial Resources & 39 & 30 & 27 & 11 \\ 
            Hearing Sensitivity & 39 & 30 & 16 & 29 \\ 
            Memorization & 37 & 48 & 34 & 21 \\ 
            Auditory Attention & 37 & 32 & 21 & 29 \\ 
            Telecommunications & 37 & 23 & 7 & 0 \\ 
            Equipment Selection & 37 & 21 & 0 & 29 \\ 
            Installation & 37 & 0 & 0 & 0 \\ 
            Building and Construction & 36 & 15 & 16 & 0 \\ 
            Personnel and Human Resources & 35 & 38 & 23 & 0 \\ 
            Public Safety and Security & 35 & 26 & 15 & 26 \\ 
            Programming & 34 & 55 & 14 & 0 \\ 
            Depth Perception & 34 & 30 & 5 & 29 \\ 
            Operation and Control & 34 & 9 & 0 & 30 \\ 
            Sales and Marketing & 33 & 20 & 29 & 0 \\ 
            Finger Dexterity & 32 & 29 & 25 & 34 \\ 
            Law and Government & 32 & 28 & 54 & 28 \\ 
            Reaction Time & 30 & 0 & 0 & 29 \\ 
            Transportation & 28 & 12 & 0 & 33 \\ 
            Communications and Media & 25 & 46 & 23 & 19 \\ 
            Economics and Accounting & 25 & 26 & 74 & 0 \\ 
            Arm-Hand Steadiness & 25 & 0 & 11 & 41 \\ 
            Repairing & 25 & 0 & 0 & 29 \\ 
            Equipment Maintenance & 23 & 0 & 0 & 29 \\ 
            Medicine and Dentistry & 23 & 0 & 0 & 0 \\ 
            Geography & 21 & 15 & 25 & 0 \\ 
            Biology & 20 & 16 & 0 & 0 \\ 
            Trunk Strength & 14 & 14 & 14 & 43 \\ 
            Manual Dexterity & 14 & 0 & 9 & 52 \\ 
            Wrist-Finger Speed & 9 & 0 & 0 & 21 \\ 
            Rate Control & 5 & 0 & 0 & 32 \\ 
            Spatial Orientation & 5 & 0 & 0 & 14 \\ 
            Psychology & 0 & 28 & 10 & 0 \\ 
            Therapy and Counseling & 0 & 21 & 0 & 0 \\ 
            Sociology and Anthropology & 0 & 0 & 16 & 0 \\ 
            History and Archeology & 0 & 0 & 13 & 0 \\ 
            Control Precision & 0 & 0 & 7 & 32 \\ 
            Extent Flexibility & 0 & 0 & 0 & 43 \\ 
            Static Strength & 0 & 0 & 0 & 39 \\ 
            Multilimb Coordination & 0 & 0 & 0 & 37 \\ 
            Stamina & 0 & 0 & 0 & 32 \\ 
            Gross Body Coordination & 0 & 0 & 0 & 30 \\ 
            Speed of Limb Movement & 0 & 0 & 0 & 30 \\ 
            Response Orientation & 0 & 0 & 0 & 29 \\ 
            Dynamic Strength & 0 & 0 & 0 & 25 \\ 
            Gross Body Equilibrium & 0 & 0 & 0 & 21 \\ 
            Glare Sensitivity & 0 & 0 & 0 & 21 \\ 
            Sound Localization & 0 & 0 & 0 & 14 \\ 
            Night Vision & 0 & 0 & 0 & 14 \\ 
            Peripheral Vision & 0 & 0 & 0 & 14 \\ 
            Explosive Strength & 0 & 0 & 0 & 11 \\ 
            Dynamic Flexibility & 0 & 0 & 0 & 0 \\ 
            Foreign Language & 0 & 0 & 0 & 0 \\ 
            Philosophy and Theology & 0 & 0 & 0 & 0 \\ 
            Fine Arts & 0 & 0 & 0 & 0 \\ 
            Food Production & 0 & 0 & 0 & 0 \\
        \bottomrule
    \end{longtblr}
}

% references
\addbibresource{references.bib}

% endregion

% region: display breaks 
\allowdisplaybreaks

% endregion

% region: title 
\title{
    The Employability Theorem
}
\author{Cao Bittencourt}
\affiliation{{B. Sc. in Economics from EPGE (FGV), RJ, Brazil.}}
\affiliation{{Statistician at Atlas Career Guide Inc., FL, USA.}}
\date{\today}

% endregion

%SECTION: PAPER 
\begin{document}

% region: abstract 
\begin{abstract}
    In this document, the Employability Theorem is demonstrated from a set of fairly tautological axioms, which are presupposed in quantitative career choice and career development methods.
\end{abstract}

\begin{keyword}
    Employability theorem; 
    Career choice; 
    Career development; 
    Vocational choice; 
    Occupational Information Network; 
    O*NET.
\end{keyword}

% endregion

% region: title 
\maketitle

% endregion

% endregion

% region: proof plan 
\section{Proof Plan}
\begin{enumerate}
    \item basic presuppositions
    \item basic lemmas
    \item complex tasks
    \item occupations are but tasks
    \item occupations' tasks are complex
    \item occupations' tasks are holistic (operation)
    \begin{enumerate}
        \item more difficult tasks presuppose the easier tasks have been accomplished
        \item i.e. $l \in [0,1]$ is a ``progress bar'' of an occupation's operation
        \item strongly holistic: each task $l \geq \bar{l}$ requires all the previous $l \in [0, \bar{l}], \bar{l} \in [0,1]$ difficulty levels to be accomplished. in addition, if all $l \in [0,1]$ levels are not all accomplished, the whole effort is vain and the operation is not completed (i.e. round down $\mho_q$ when calculating operational output). furthermore, each and every $l \in [0,1]$ difficulty level cannot be outsourced (i.e. only a perfectly qualified worker can output a unit of the occupation's operation).
        \begin{enumerate}
            \item individual's time constraint is spent entirely on trying to accomplish the complex holistic task by themselves. therefore, there is no optimization to be done.
            \item 
            \begin{align*}
                \mho_q 
                = 
                \sum_{k=1}^{n}{
                    \left\lfloor
                        [k \in \Lambda_q]
                        \times
                        \mho_{q}^{k}
                    \right\rfloor
                }
                = 
                \sum_{k=1}^{n}{
                    \left\lfloor
                        [k \in \Lambda_q]
                        \times
                        \int_{0}^{1}{
                            T_{q}(l,l_{q}^{k})
                            dl
                        }
                    \right\rfloor
                }
            \end{align*}
        \end{enumerate} 
        \item moderately holistic: each task $l \geq \bar{l}$ requires all the previous $l \in [0, \bar{l}], \bar{l} \in [0,1]$ difficulty levels to be accomplished. in addition, if all $l \in [0,1]$ levels are not all accomplished, the whole effort is vain and the operation is not completed (i.e. round down $\mho_q$ when calculating operational output). however, each and every $l \in [0,1]$ difficulty level can be outsourced (i.e. workers can output partial units of the occupation's operation, which contribute to the operation's completion).
        \begin{enumerate}
            \item because of outsourcing, individual's time constraint is spent working from where another worker ``left off'', so that even if a worker cannot accomplish the entire operation by themselves, they can still contribute to the operation's completion by reducing the time highly skilled workers will have to spend on relatively more trivial tasks.
            
            The first worker spends their entire unitary time allowance trying their hardest to accomplish the highest amount of tasks they can. When they hit their skill cap, they restart their efforts, so as to spend their entire time allowance helping out the next worker:
            \begin{gather*}
                \int_{0}^{\tilde{T}_{q}^{k}}{
                    T_{q}(l,l_{q}^{k})
                    \times
                    \text{ta}_{q}(l)
                    dl
                }
                +
                \int_{0}^{\bar{l}}{
                    T_{q}(l,l_{q}^{k})
                    \times
                    \text{ta}_{q}(l)
                    dl
                }
                = 1
                \\
                \int_{0}^{\bar{l}}{
                    1
                    \times
                    \text{ta}_{q}(l)
                    dl
                }
                = 1 -
                \int_{0}^{\tilde{T}_{q}^{k}}{
                    1
                    \times
                    \text{ta}_{q}(l)
                    dl
                }
                \\
                \int_{0}^{\bar{l}}{
                    \text{ta}_{q}(l)
                    dl
                }
                = 
                \int_{\tilde{T}_{q}^{k}}^{1}{
                    \text{ta}_{q}(l)
                    dl
                }
                \\
                \text{TA}_{q}(\bar{l}) -
                \text{TA}_{q}(0)
                = 
                \text{TA}_{q}(1) -
                \text{TA}_{q}(\tilde{T}_{q}^{k})
                \\
                \text{TA}_{q}(\bar{l})
                = 
                \text{TA}_{q}(1) -
                \text{TA}_{q}(\tilde{T}_{q}^{k})
                \\
                \bar{l}
                = 
                \text{TA}_{q}^{-1}
                \left(
                    \text{TA}_{q}(1) -
                    \text{TA}_{q}(\tilde{T}_{q}^{k})
                \right)
                ,
            \end{gather*}
            so that $k$ accomplishes tasks of difficulty levels $0$ through $\tilde{T}_{q}^{k}$ on their ``first run'', and restarts their effort to provide additional $l \in \left[0,\text{TA}_{q}^{-1}
            \left(
                \text{TA}_{q}(1) -
                \text{TA}_{q}(\tilde{T}_{q}^{k})
            \right)
            \right]$ levels worth of complex tasks. Thus, the next worker does not need to start from zero, but rather from where $k$ ``left off'': either $\tilde{T}_{q}^{k}$, $\bar{l}$, or some $l \in [0, \tilde{T}_{q}^{k}]$.
            \item 
            \begin{align*}
                \mho_q 
                &= 
                \left\lfloor
                \sum_{k=1}^{n}{
                    [k \in \Lambda_q]
                    \times
                    \mho_{q}^{k}
                }
                \right\rfloor
                \\
                &= 
                \left\lfloor
                \sum_{k=1}^{n}{
                    [k \in \Lambda_q]
                    \times
                    \int_{0}^{1}{
                        T_{q}(l,l_{q}^{k})
                        dl
                    }
                }
                \right\rfloor
            \end{align*}
        \end{enumerate}
        \item weakly holistic: each task $l \geq \bar{l}$ requires all the previous $l \in [0, \bar{l}], \bar{l} \in [0,1]$ difficulty levels to be accomplished. however, if not all $l \in [0,1]$ levels are accomplished, the whole effort is not vain and the operation is partially completed (i.e. do not round $\mho_q$ when calculating operational output). furthermore, each and every $l \in [0,1]$ difficulty level can be outsourced (i.e. workers can output partial units of the occupation's operation, which contribute to the operation's completion).
    \end{enumerate}
    \item assume weak occupational complexity axiom (the other versions are too strict)
    \item perhaps posit an even weaker version of occupational complexity:
    \begin{enumerate}
        \item
        \begin{gather}
            \frac{
                \partial \mho_{q}
            }{
                \partial l
            } > 0
            ,\\
            \frac{
                \partial^2 \mho_{q}
            }{
                \partial l^2
            } < 0
            ,
        \end{gather}
        so that even though tasks of a particular level are not required for the operation to ``count'' (i.e. partial delivery), it is still detrimental to focus too much on one subset of tasks, that is, employers are incentivised to produce the entire spectrum of difficulty levels, because marginal productivity increases when a tasks of a particular difficulty level have not been accomplished yet.

        (actually, we need a indicator variable for the amount of tasks accomplished for a difficulty level, something analogous to $T_q(l)$)
    \end{enumerate}
    \item now, because of weak occupational complexity, employers will maximize operational output by attempting to produce the entire spectrum of difficulty levels for the complex tasks of an occupation.
    \item this can be done either by having only perfectly qualified employees work on the operation individually from beginning to end, or by splitting responsabilities into two, or more, types of jobs, thus allowing for less qualified, ``junior'' employees, to work alongside more qualified and perfectly qualified, ``senior'' employees towards the common goal of accomplishing the entire occupational operation.
    \item additionally, because there are skill differences among workers in the labor market, any rational employer will always, and rightly, expect their employees to be of varying skill levels, rather than all perfectly qualified, so that splitting responsabilities into separate positions will not only be an alternative mode of hiring and producing, but in fact the optimal one.
    \item therefore, given expected and actual skill differences among workers, employers will split job posts based on the required skill level. thus, there will be ``junior'' job posts and ``senior'' job posts, each dedicated to accomplishing a particular subset of complex tasks with difficulty levels appropriate for employees' respective capacity.
    \item notice this does not mean all people working on ``junior'' positions will, necessarily, be ``junior'' employees themselves, that is, less qualified. indeed, if talent is abundant in the labor market, these ``junior'' positions will have to be filled by more qualified, or even perfectly qualified, ``senior'' employees. for if there were only one type of job, spanning the entire difficulty level spectrum, highly qualified workers would already have to accomplish these ``junior'' tasks themselves, in order to maximize operational output. however, by having two, or more, types of jobs, split by minimum required competence, highly qualified workers may specialize to the measure that there are less qualified workers available to accomplish the easier tasks. but, if there are none, they will, again, have to work on these themselves.
    \item analogously, from the employers' perspective, it does not matter who accomplishes ``junior'' tasks, so long as they are accomplished. thus, if highly qualified workers are abundant in a particular time period of a labor market, production is not hindered when allocating ``seniors'' to ``junior'' positions, for in these circumstances talent is not wasted. that is, because only highly qualified workers can accomplish highly demanding tasks, rational employers will generally not hire them to work on ``junior'' tasks, thus ``saving'' their talent for more difficult tasks, which a ``junior'' would not be able to accomplish. but, if there is enough talent to output the optimal quantity of ``senior'' tasks, it can actually be more productive to employ the remaining ``seniors'' to ``junior'' positions.
    \item furthermore, in a continuous setting, rational employers will maximize their hiring pool by offering more than only two types of jobs. thus, there will not only be ``senior'' and ``junior'' positions, but several levels in a production hierarchy, each responsible for a particular subinterval of task difficulty, which will approximate a continuum of ``seniority'' as the number of workers becomes large enough.
    \item now, as for employees' work routine, rational employers will have them work over their responsability spectrum in a proportional and optimal matter, thus avoiding wasting production (i.e. uncompleted ``loops'' over the responsability spectrum). [this means each employee will spend their entire time allowance producing a partial operational output, that is a multiple of the difficulty subinterval they were hired to accomplish, which will, in turn, contribute, alonside the partial outputs of other employees, to accomplish the entire occupational operation.]
    \item the reason this avoids wasting production is because [...].
    \item Weak Skill Differences Axiom (WSDA)
    \begin{enumerate}
        \item There are, or there could be, skill differences among people in the workforce (i.e. workers are not all ``clones'' of one another or equally competent). Thus, the expected value of productivity is:
        \begin{gather}
            \mathbb{E}[
                \tilde{T}_{q}^{k}
            ]
            \in
            [0,1]
            ,
        \end{gather}
        instead of 
        \begin{gather}
            \mathbb{E}[
                \tilde{T}_{q}^{k}
            ]
            =
            \tilde{T}_{q}^{k}
            = 1
            ,
        \end{gather}
        for all $k,q \in \{1, \dots, n\}$. This means employers do not expected every worker to be perfectly qualified and will adjust their hiring and production strategies accordingly.
    \end{enumerate}
    \item (Binary Employability Theorem) thus, in the binary case, ``junior'' productive output will be given by:
    \begin{gather}
        \mho_{q}^{\text{Jr}} =
        \frac{
            1
        }{
            \int_{0}^{\tilde{T}_{q}^{\text{Jr}}}
            \text{ta}_q(l)
            dl
        } = 
        \left(
            \int_{0}^{\tilde{T}_{q}^{\text{Jr}}}
                \text{ta}_q(l)
                dl
        \right) ^ {-1}
        ,
    \end{gather}
    where $\text{ta}_q(l)$ is the time allocation function of occupation $q$'s complex tasks, and time allowance (the numerator) is set to one.
    \item analogously, ``senior'' productive output is:
    \begin{gather}
        \mho_{q}^{\text{Sr}} =
        \frac{
            1
        }{
            \int_{\tilde{T}_{q}^{\text{Jr}}}^{1}
            \text{ta}_q(l)
            dl
        } = 
        \left(
            \int_{\tilde{T}_{q}^{\text{Jr}}}^{1}
                \text{ta}_q(l)
                dl
        \right) ^ {-1}
        .
    \end{gather}
    \item finally, as a mismatch in productive output due to time allocation differences between ``junior'' and ``senior'' tasks would result in wasted production, a rational employer will optimally ``orchestrate'' the productive effort by offering just enough ``senior'' job posts in the labor market to meet ``junior'' productivity. thus, by setting ``junior'' job posts to $w_{q}^{\text{Jr}} > 0$ and ``senior'' job posts to $w_{q}^{\text{Sr}} > 0$, we get the ratio between ``junior'' and ``senior'' positions required to output any level of occupation $q$'s operation:
    \begin{gather}
        w_{q}^{\text{Sr}} \times 
        \mho_{q}^{\text{Sr}} = 
        w_{q}^{\text{Jr}} \times 
        \mho_{q}^{\text{Jr}}
        \\
        \therefore
        w_{q}^{\text{Sr}} \times 
        \left(
            \int_{\tilde{T}_{q}^{\text{Jr}}}^{1}
                \text{ta}_q(l)
                dl
        \right) ^ {-1}
        = 
        w_{q}^{\text{Jr}} \times 
        \left(
            \int_{0}^{\tilde{T}_{q}^{\text{Jr}}}
                \text{ta}_q(l)
                dl
        \right) ^ {-1}
        \\
        \therefore
        w_{q}^{\text{Sr}}
        = 
        w_{q}^{\text{Jr}}
        \times
        \left(
            \frac{
                \int_{\tilde{T}_{q}^{\text{Jr}}}^{1}
                    \text{ta}_q(l)
                    dl
            }{
                \int_{0}^{\tilde{T}_{q}^{\text{Jr}}}
                    \text{ta}_q(l)
                    dl
            }
        \right)
        .
    \end{gather}

    with this, ``senior'' employability (i.e. the percentage of job posts for which they could be hired) is
    \begin{gather}
        \tilde{w}_{q}^{\text{Sr}} =
        \frac{
            w_{q}^{\text{Jr}} + 
            w_{q}^{\text{Sr}}
        }{
            w_{q}^{\text{Jr}} + 
            w_{q}^{\text{Sr}}
        }
        = 1
    \end{gather}
    and ``junior'' employability is
    \begin{align}
        \tilde{w}_{q}^{\text{Jr}} 
        &=
        \frac{
            w_{q}^{\text{Jr}}
        }{
            w_{q}^{\text{Jr}} + 
            w_{q}^{\text{Sr}}
        }
        \\&=
        \frac{
            w_{q}^{\text{Jr}}
        }{
            w_{q}^{\text{Jr}} + 
            w_{q}^{\text{Jr}} \times
            \left(
                \frac{
                    \int_{\tilde{T}_{q}^{\text{Jr}}}^{1}
                        \text{ta}_q(l)
                        dl
                }{
                    \int_{0}^{\tilde{T}_{q}^{\text{Jr}}}
                        \text{ta}_q(l)
                        dl
                }
            \right)
        }
        \\&=
        \left(
            1 + 
            \frac{
                \int_{\tilde{T}_{q}^{\text{Jr}}}^{1}
                    \text{ta}_q(l)
                    dl
            }{
                \int_{0}^{\tilde{T}_{q}^{\text{Jr}}}
                    \text{ta}_q(l)
                    dl
            }
        \right) ^ {-1}
        \\&=
        \left(
            1 + 
            \frac{
                \int_{0}^{1}
                    \text{ta}_q(l)
                    dl    
                    -
                \int_{0}^{\tilde{T}_{q}^{\text{Jr}}}
                    \text{ta}_q(l)
                    dl
            }{
                \int_{0}^{\tilde{T}_{q}^{\text{Jr}}}
                    \text{ta}_q(l)
                    dl
            }
        \right) ^ {-1}
        \\&=
        \left(
            1 + 
            \frac{
                1 -
                \int_{0}^{\tilde{T}_{q}^{\text{Jr}}}
                    \text{ta}_q(l)
                    dl
            }{
                \int_{0}^{\tilde{T}_{q}^{\text{Jr}}}
                    \text{ta}_q(l)
                    dl
            }
        \right) ^ {-1}
        \\&=
        \left(
            1 + 
            \frac{
                1
            }{
                \int_{0}^{\tilde{T}_{q}^{\text{Jr}}}
                    \text{ta}_q(l)
                    dl
            }
            -
            \frac{
                \int_{0}^{\tilde{T}_{q}^{\text{Jr}}}
                    \text{ta}_q(l)
                    dl
            }{
                \int_{0}^{\tilde{T}_{q}^{\text{Jr}}}
                    \text{ta}_q(l)
                    dl
            }
        \right) ^ {-1}
        \\&=
        \left(
            1 + 
            \frac{
                1
            }{
                \int_{0}^{\tilde{T}_{q}^{\text{Jr}}}
                    \text{ta}_q(l)
                    dl
            }
            -
            1
        \right) ^ {-1}
        \\&=
        \left(
            \frac{
                1
            }{
                \int_{0}^{\tilde{T}_{q}^{\text{Jr}}}
                    \text{ta}_q(l)
                    dl
            }
        \right) ^ {-1}
        \\&=
        \int_{0}^{\tilde{T}_{q}^{\text{Jr}}}
            \text{ta}_q(l)
            dl
        .
    \end{align}

    thus, the employability of a partially qualified worker, that is a ``junior'', is precisely the percentage of an operation's total time duration their skill set allows them to accomplish (i.e. the inverse of their operational output).

Maximum-Monotonic Labor Stratification Definition

Now, to generalize this conclusion, we shall define notation in terms of maximum labor stratification, a productive arrangement where there are several job subtypes, indeed as many as there are jobs themselves, each with a limited spectrum of responsabilities.
% Now, to generalize this conclusion for other economic configurations, we shall define notation in terms of maximum labor stratification, a productive arrangement where there are several job subtypes, indeed as many as there are jobs themselves, each with a limited spectrum of responsabilities.

Hence, mathematically,
\begin{gather}
    l \in \left[
    \ell_{v-1}
    ,
    \ell_{v}
    \right]
    ,
\end{gather}
with
\begin{align}
    \ell_{v} \in [0,1]
    \
    \forall
    \
    &v \in 
    \left\{
        1, \dots, w_q
    \right\}
    ,
    \\
    \ell_{w_q} &:= 1
    ,
    \\
    \ell_0 &:= 0
\end{align}
is one of $w_q$ responsability spectra in a maximally stratified labor market, in which employment levels are unitary, or given by
\begin{gather}
    \sum_{v=1}^{w_q}
    1
    =
    w_q
    ,
\end{gather}
so that any available position is its own job subtype and covers only a restrictive range of task difficulty, accounting for
\begin{gather}
    \Omega_{q}^{v} := 
    \frac{1}{\mho_{q}^{v}} = 
    \int_{
        \ell_{v-1}
    }^{
        \ell_{v}
    }
    \text{ta}(l)dl
    \in [0,1]
\end{gather}
of an operation's total time duration,
\begin{gather}
    \sum_{v=1}^{w_q}
    \Omega_{q}^{v} 
    = 
    \sum_{v=1}^{w_q}
    \int_{
        \ell_{v-1}
    }^{
        \ell_{v}
    }
    \text{ta}(l)dl
    =
    \int_{0}^{1}
    \text{ta}(l)dl
    =
    1
    .
\end{gather}

Intuitively speaking, we would say production in a maximally and monotonically stratified labor market is not ``independent'', in the sense that employees do not work on an occupation's operation from beginning to end. This means each of them will spend all their time allowance producing a partial operational output, that is a multiple of a difficulty subinterval of complex tasks, which will, in turn, contribute, alonside the partial outputs of other employees, to accomplish[ing?] the occupational operation in its entirety.

However, in a maximum labor stratification setting, these partial operational outputs will not be produced merely via ``senior'' and ``junior'' positions, as previously, but rather within a myriad of levels in a production hierarchy, approximating a continuum of ``seniority'' as the workforce becomes large enough.
% However, in a maximally stratified labor market, these partial operational outputs will not be produced merely via ``senior'' and ``junior'' positions, as previously, but rather within a myriad of levels in a production hierarchy, approximating a continuum of ``seniority'' as the workforce becomes large enough.

Again, this does not mean employees are, themselves, more or less competent, only that available job posts are preemptively stratified with respect to task difficulty, in order to maximize employers' hiring pool and safeguard production in the case workers are not sufficiently qualified to produce the whole responsability spectrum independently (see ``Maximum-Monotonic Labor Stratification Lemma'' below).

[end definition of maximum-monotonic labor stratification]

Having understood what maximum-monotonic labor stratification is, we shall demonstrate that, given our axioms, such an economic configuration is, in fact, the only optimal production strategy and, so, holds in the labor market. But, to do so, we must first derive the maximum operational output, irrespective of productive arrangement, to serve as our ``benchmark''.

Maximum Operational Output Lemma (MOOL)

    In any labor market, the maximum operational output is exactly the number of employees in its workforce:
    \begin{gather}
        \mho_{q}^{*}
        =
        \mho(
            \boldsymbol{w_{q}^{*}},
            \boldsymbol{\mho_q}
        ) = 
        \min(
            \boldsymbol{w_{q}^{*}}
            \times
            \boldsymbol{\mho_q}
        )
        = w_q
        ,
    \end{gather}
    where $\boldsymbol{w_{q}^{*}}$ is the vector of optimal employment levels in a labor market with $w_q$ employees; and $\boldsymbol{\mho_q}$, the vector of partial operational outputs. Or, assuming maximum labor stratification with unitary employment levels,
    \begin{gather}
        \mho_{q}^{*}
        =
        \mho(
            \boldsymbol{1},
            \boldsymbol{\mho_q}(
                \boldsymbol{\ell_{q}^{*}}
            )
        )
        =
        \min(
            \boldsymbol{1}
            \times
            \boldsymbol{\mho_q}(
                \boldsymbol{\ell_{q}^{*}}
            )
        )
        = w_q
        ,
    \end{gather}
    where $\boldsymbol{\ell_{q}^{*}}$ are optimal stratification bounds for the responsability spectra of occupation $q$'s job posts (see ``Optimal Stratification Lemma'' below).
    
    Moreover, when optimizing employment levels, this maximum production can only be attained when the percentage of each position relative to the entire workforce respects the Proportional Employment Condition (PEC):
    \begin{gather}
        \boldsymbol{\tilde{w}_{q}^{*}}
        =
        \frac{
            \boldsymbol{w_{q}^{*}}
        }{
            w_q
        }
        =
        \boldsymbol{\Omega_{q}}
        ,
    \end{gather}
    which determines the ratio, or proportion, of a particular job subtype in a labor market is the percentage of an operation's total time duration,
    \begin{gather}
        \boldsymbol{1} ^ {\top}
        \cdot
        \boldsymbol{\Omega_{q}}
        = 1
        ,
    \end{gather}
    accounted by it.
    
    \textit{Proof}:

    We begin with the most trivial of economic configurations, that of independent production with perfectly qualified workers. In this scenario, each employee devotes their unitary time allowance, which coincides with the total time duration of occupation $q$'s operation,
    \begin{gather}
        \int_{0}^{1}{
            \text{ta}(l)
            dl
        }
        =
        1
        ,
    \end{gather}
    to output exactly one productive unit:
    \begin{gather}
        1
        \times
        \left(
            \int_{0}^{1}{
                \text{ta}(l)dl
            }
        \right) ^ {-1}
        =
        1
        .
    \end{gather}

    Therefore, $w_q$ of such employees working in parallel, yield an output of
    \begin{align}
        w_q 
        \times 
        \left(
            \int_{0}^{1}{
                \text{ta}(l)dl
            }
        \right) ^ {-1}
        =
        w_q
        .
    \end{align}

    Here, we have taken occupation $q$'s responsability spectrum $l \in [0,1]$ as a whole, or as if it were a single task, so that the minimum amount produced of this ``holistic task'', covering all occupation $q$'s responsabilities, is one unit per worker, or $w_q$ aggregate units.

    However, it can be easier to understand this result if we analyze responsability spectra individually, as if a perfectly qualified, independent, employee worked on a series of tasks, which sum to their time allowance,
    \begin{gather}
        \boldsymbol{1} ^ {\top}
        \cdot
        \boldsymbol{\Omega_{q}}
        = 
        1
        .
    \end{gather}
    
    With this, we note that, as each worker's time allowance is the same as operations' total duration, failing to output any single task by overemphasizing another would nullify the whole productive effort. Hence, the optimal choice of hours to allocate to any responsability spectrum has to be the minimum time required to complete it, or
    % With this, we note that, as each worker's time allowance is the same as operations' total time requirements, failing to output any single task by overemphasizing another would nullify the whole operation, when production is independent. Hence, the optimal choice of hours to allocate to any responsability spectrum has to be the minimum time requirement to complete it, or
    \begin{gather}
        \Omega_{q}^{\ell} 
        \in
        [0,1]
        .
    \end{gather}

    Furthermore, by the definition of partial operational output (ref) above, one outputs $\mho_{q}^{\ell}$ when spending their unitary time allowance to produce a responsability spectrum. So, the output, with only $\Omega_{q}^{\ell}$ time units, is:
    % Furthermore, by the definition of partial operational output (ref) above, one outputs $\mho_{q}^{\ell}$ when spending their unitary time allowance to produce some responsability spectrum. So, operational output, with only $\Omega_{q}^{\ell}$ time units, is:
    % Furthermore, by the definition of partial operational output (ref) above, one outputs $\mho_{q}^{\ell}$ when spending their unitary time allowance to produce some responsability spectrum. Therefore, with only $\Omega_{q}^{\ell}$ time units,  operational output is:
    \begin{gather}
        \Omega_{q}^{\ell}
        \mho_{q}^{\ell}
        =
        \left(
            \frac{1}{\mho_{q}^{\ell}}
        \right)
        \times
        \mho_{q}^{\ell}
        =
        1
        .
    \end{gather}
    
    Finally, as Weak Occupational Complexity implies the production function is homothetic, the aggregate operational output of $w_q$ perfectly qualified employees working independently is:
    \begin{gather}
        \mho_{q}^{*}
        =
        \min(
            \boldsymbol{\Omega_q}
            \times
            \boldsymbol{\mho_q}
        )
        \times
        w_q
        =
        \Omega_{q}^{\ell}
        \mho_{q}^{\ell}
        \times
        w_q
        =
        1
        \times
        w_q
        =
        w_q
        .
    \end{gather}
    Hence, a perfectly qualified employee working full-time and independently can output one unit of an occupation's complex tasks with one unit of their time (i.e. their entire time allowance). And, likewise, a workforce with $w_q$ employees identical to this one produces $w_q$ units of operational output. Or, to put it simply, a maximally productive person acchieves maximum production.
    
    We, now, proceed with the binary setting presented above, where employees choose a $\tilde{w}_{q}^{\text{Jr}} \in [0,1]$ percentage of less qualified (i.e. ``junior'') job posts to offer, which determine the remaining $\tilde{w}_{q}^{\text{Sr}} = 1 - \tilde{w}_{q}^{\text{Jr}} \in [0,1]$ percentage of perfectly qualified (or ``senior'') job posts. In this case,
    \begin{align}
        \mho(\tilde{w}_{q}^{\text{Jr}})
        &=
        \min\left(
            \tilde{w}_{q}^{\text{Jr}}
            % \times
            \mho_{q}^{\text{Jr}}
            ,
            \tilde{w}_{q}^{\text{Sr}}
            % \times
            \mho_{q}^{\text{Sr}}
        \right)
        \\
        &=
        \min\left(
            \frac{
                \tilde{w}_{q}^{\text{Jr}}
            }{
                \int_{0}^{
                    \tilde{T}_{q}^{\text{Jr}}
                }{
                    \text{ta}(l)
                    dl
                }
            }
            ,
            \frac{
                1 - \tilde{w}_{q}^{\text{Jr}}
            }{
                \int_{
                    \tilde{T}_{q}^{\text{Jr}}
                }^{1}{
                    \text{ta}(l)
                    dl
                }
            }
        \right)
        \\
        &=
        \min\left(
            \frac{
                \tilde{w}_{q}^{\text{Jr}}
            }{
                \int_{0}^{
                    \tilde{T}_{q}^{\text{Jr}}
                }{
                    \text{ta}(l)
                    dl
                }
            }
            ,
            \frac{
                1 - \tilde{w}_{q}^{\text{Jr}}
            }{
                \int_{0}^{1}{
                    \text{ta}(l)
                    dl
                }
                -
                \int_{0}^{
                    \tilde{T}_{q}^{\text{Jr}}
                }{
                    \text{ta}(l)
                    dl
                }
            }
        \right)
        \\
        &=
        \min\left(
            \frac{
                \tilde{w}_{q}^{\text{Jr}}
            }{
                \Omega_{q}^{\text{Jr}}
            }
            ,
            \frac{
                1 - \tilde{w}_{q}^{\text{Jr}}
            }{
                1 - \Omega_{q}^{\text{Jr}}
            }
        \right)
        ,
    \end{align}
    whereas the operational output of employing $\Omega_{q}^{\text{Jr}} \in [0,1]$ is
    % whereas the operational output of $\Omega_{q}^{\text{Jr}} \in [0,1]$ is
    \begin{align}
        \mho(\Omega_{q}^{\text{Jr}}) 
        =
        \min\left(
            \frac{
                \Omega_{q}^{\text{Jr}}
            }{
                \Omega_{q}^{\text{Jr}}
            }
            ,
            \frac{
                1 - \Omega_{q}^{\text{Jr}}
            }{
                1 - \Omega_{q}^{\text{Jr}}
            }
        \right)
        =
        \frac{
            \Omega_{q}^{\text{Jr}}
        }{
            \Omega_{q}^{\text{Jr}}
        }
        =
        \frac{
            1 - \Omega_{q}^{\text{Jr}}
        }{
            1 - \Omega_{q}^{\text{Jr}}
        }
        =
        1
        .
    \end{align}

    With this, if $\tilde{w}_{q}^{\text{Jr}}$ is set to $\tilde{w}_{q}^{\text{Jr}} > \Omega_{q}^{\text{Jr}}$, then
    \begin{align}
        &
        1 - \tilde{w}_{q}^{\text{Jr}}
        <
        1 - \Omega_{q}^{\text{Jr}}
        \\
        &\therefore
        \frac{
            \tilde{w}_{q}^{\text{Jr}}
        }{
            \Omega_{q}^{\text{Jr}}
        }
        >
        1
        >
        \frac{
            1 - \tilde{w}_{q}^{\text{Jr}}
        }{
            1 - \Omega_{q}^{\text{Jr}}
        }
        \\
        &\therefore
        \mho(\tilde{w}_{q}^{\text{Jr}}) 
        =
        \min\left(
            \frac{
                \tilde{w}_{q}^{\text{Jr}}
            }{
                \Omega_{q}^{\text{Jr}}
            }
            ,
            \frac{
                1 - \tilde{w}_{q}^{\text{Jr}}
            }{
                1 - \Omega_{q}^{\text{Jr}}
            }
        \right)
        =
        \frac{
            1 - \tilde{w}_{q}^{\text{Jr}}
        }{
            1 - \Omega_{q}^{\text{Jr}}
        }
        <
        1
        \\
        &\implies
        \mho(\tilde{w}_{q}^{\text{Jr}})
        <
        \mho(\Omega_{q}^{\text{Jr}})
        ;
    \end{align}
    and, if $\tilde{w}_{q}^{\text{Jr}} < \Omega_{q}^{\text{Jr}}$,
    \begin{align}
        &
        1 - \tilde{w}_{q}^{\text{Jr}}
        >
        1 - \Omega_{q}^{\text{Jr}}
        \\
        &\therefore
        \frac{
            \tilde{w}_{q}^{\text{Jr}}
        }{
            \Omega_{q}^{\text{Jr}}
        }
        <
        1
        <
        \frac{
            1 - \tilde{w}_{q}^{\text{Jr}}
        }{
            1 - \Omega_{q}^{\text{Jr}}
        }
        \\
        &\therefore
        \mho(\tilde{w}_{q}^{\text{Jr}}) 
        =
        \min\left(
            \frac{
                \tilde{w}_{q}^{\text{Jr}}
            }{
                \Omega_{q}^{\text{Jr}}
            }
            ,
            \frac{
                1 - \tilde{w}_{q}^{\text{Jr}}
            }{
                1 - \Omega_{q}^{\text{Jr}}
            }
        \right)
        =
        \frac{
            \tilde{w}_{q}^{\text{Jr}}
        }{
            \Omega_{q}^{\text{Jr}}
        }
        <
        1
        \\
        &\implies
        \mho(\tilde{w}_{q}^{\text{Jr}})
        <
        \mho(\Omega_{q}^{\text{Jr}})
        .
    \end{align}

    Therefore, 
    \begin{gather}
        \mho(\tilde{w}_{q}^{\text{Jr}}) < 
        \mho(\Omega_{q}^{\text{Jr}}) 
        = 1
        \
        \forall
        \
        \tilde{w}_{q}^{\text{Jr}}
        \neq
        \Omega_{q}^{\text{Jr}}
        \in [0,1]
        .
    \end{gather}

    Analogously, with multiple job subtypes, optimal operational output is:
    \begin{gather}
        \mho(\boldsymbol{\Omega_q})
        = 
        \min(
            \boldsymbol{\Omega_q}
            \times
            \boldsymbol{\mho_q}
        )
        =
        \frac{
            \Omega_{q}^{\ell}
        }{
            \Omega_{q}^{\ell}
        }
        =
        1
        ,
    \end{gather}
    for, again, since
    \begin{gather}
        \boldsymbol{1} ^ {\top}
        \cdot
        \boldsymbol{\tilde{w}_{q}}
        =
        \boldsymbol{1} ^ {\top}
        \cdot
        \boldsymbol{\Omega_{q}}
        = 1
        ,
    \end{gather}
    choosing any $\tilde{w}_{q}^{\ell} \neq \Omega_{q}^{\ell}$ implies the proportion of at least one position, say $\tilde{w}_{q}^{r}$, is impacted, and aggregate output along with it, either because
    % choosing any $\tilde{w}_{q}^{\ell} \neq \Omega_{q}^{\ell}$ implies the proportion of at least one position, say $\tilde{w}_{q}^{r}$, is impacted, and thus aggregate output along with it, either because
    % choosing any $\tilde{w}_{q}^{\ell} \neq \Omega_{q}^{\ell} \in [0,1]$ implies the proportion of at least one position, say $\tilde{w}_{q}^{r}$, is impacted, and thus aggregate output along with it, either because
    \begin{align}
        &
        \tilde{w}_{q}^{\ell} > \Omega_{q}^{\ell}
        \\
        &\therefore
        \frac{
            \tilde{w}_{q}^{\ell}
        }{
            \Omega_{q}^{\ell}
        }
        > 
        1
        > 
        \frac{
            \tilde{w}_{q}^{r}
        }{
            \Omega_{q}^{r}
        }
        \\
        &\therefore
        \mho(\boldsymbol{\tilde{w}_q})
        = 
        \min(
            \boldsymbol{\tilde{w}_q}
            \times
            \boldsymbol{\mho_q}
        )
        =
        \frac{
            \tilde{w}_{q}^{r}
        }{
            \Omega_{q}^{r}
        }
        < 
        1
        \\
        &\implies
        \mho(\boldsymbol{\tilde{w}_q})
        <
        \mho(\boldsymbol{\Omega_q})
        ;
    \end{align}
    or, alternatively, because
    \begin{align}
        &
        \tilde{w}_{q}^{\ell} < \Omega_{q}^{\ell}
        \\
        &\therefore
        \frac{
            \tilde{w}_{q}^{\ell}
        }{
            \Omega_{q}^{\ell}
        }
        < 
        1
        < 
        \frac{
            \tilde{w}_{q}^{r}
        }{
            \Omega_{q}^{r}
        }
        \\
        &\therefore
        \mho(\boldsymbol{\tilde{w}_q})
        = 
        \min(
            \boldsymbol{\tilde{w}_q}
            \times
            \boldsymbol{\mho_q}
        )
        =
        \frac{
            \tilde{w}_{q}^{\ell}
        }{
            \Omega_{q}^{\ell}
        }
        < 
        1
        \\
        &\implies
        \mho(\boldsymbol{\tilde{w}_q})
        <
        \mho(\boldsymbol{\Omega_q})
        .
    \end{align}

    Thus, 
    \begin{align}
        % &
        \mho(
            \boldsymbol{\tilde{w}_{q}},
            \boldsymbol{\mho_q}
        ) 
        &< 
        \mho(
            \boldsymbol{\Omega_{q}},
            \boldsymbol{\mho_q}
        ) 
        = 1
        \\
        \therefore
        % &\therefore
        \mho(
            \boldsymbol{w_{q}},
            \boldsymbol{\mho_q}
        )
        &< 
        \mho(
            w_{q}
            \boldsymbol{\Omega_{q}},
            \boldsymbol{\mho_q}
        ) 
        = w_q
        \\
        % &
        \forall
        \
        \boldsymbol{\tilde{w}_{q}}
        &\neq
        \boldsymbol{\Omega_{q}}
        % ,
        % \\
        % &
        % \boldsymbol{\tilde{w}_{q}}
        % ,
        % \boldsymbol{\Omega_{q}}
        % > 
        % \boldsymbol{0}
        % ,
        % \\
        % &
        % \boldsymbol{1} ^ {\top}
        % \cdot
        % \boldsymbol{\tilde{w}_{q}}
        % =
        % \boldsymbol{1} ^ {\top}
        % \cdot
        % \boldsymbol{\Omega_{q}}
        % = 1
        % .
    \end{align}

    % As every worker in the labor market is assumed to have the same unitary time allowance, which coincides with operations' total time duration,
    %     % \item As every worker in the labor is assumed to have the same unitary time allowance, which coincides with operations' total time duration,
    %     \begin{gather}
    %         \int_{0}^{1}{
    %             \text{ta}(l)
    %             dl
    %         }
    %         =
    %         1
    %         ,
    %     \end{gather}
    %     there cannot be, even in theory, any productive arrangement yielding operational output higher than that obtained when perfectly qualified employees work independently. For, indeed, if employers establish other modes of production, workers' time allowances and tasks' time requirements will not change. In addition, splitting responsabilities across multiple job subtypes does not affect operational output (WOCA). Therefore, perfectly qualified independent production is the ``ceiling'' or ``benchmark'' for all production strategies.
        
    %     \begin{align}
    %         \int_{0}^{1}{
    %             \text{ta}(l)dl
    %         }
    %         =
    %         1
    %         \\
    %         \therefore
    %         \mho_{q}^{*}
    %         =
    %         1 \times \left(
    %             \int_{0}^{1}{
    %                 \text{ta}(l)dl
    %             }
    %         \right) ^ {-1}
    %         =
    %         1
    %         \\
    %         \therefore
    %         w_q \times
    %         \mho_{q}^{*}
    %         =
    %         w_q \times \left(
    %             \int_{0}^{1}{
    %                 \text{ta}(l)dl
    %             }
    %         \right) ^ {-1}
    %         =
    %         w_q
    %         \\
    %         \because
    %         \sum_{v=1}^{w_q}
    %         \int_{0}^{1}{
    %                 \text{ta}(l)dl
    %             }
    %         =
    %         \sum_{v=1}^{w_q}
    %         1
    %         = 
    %         w_q \times 1
    %         = 
    %         w_q
    %     \end{align}

    %     The mathematical proof for this is quite straighforward. Because the aggregate production function is of the Leontief functional form (for simplicity's sake, here considering only two factors of production, namely an easier responsability spectrum and a harder responsability spectrum), that is
    %     \begin{gather}
    %         \mho_{q}(\tilde{w}_{q}^{k})
    %         = 
    %         \min\left[
    %             \tilde{w}_{q}^{k}
    %             \left(
    %                 \int_{0}^{\tilde{T}_{q}^{k}}{
    %                     \text{ta}(l)dl
    %                 }
    %             \right) ^ {-1}
    %             ,
    %             (1 - \tilde{w}_{q}^{k})
    %             \left(
    %                 \int_{\tilde{T}_{q}^{k}}^{1}{
    %                     \text{ta}(l)dl
    %                 }
    %             \right) ^ {-1}
    %         \right]
    %         ,
    %     \end{gather}
    %     and because time allowances are unitary and coincide with total time requirements, optimal relative employment levels are trivial and have to be
    %     \begin{gather}
    %         \tilde{w}_{q}^{k}
    %         =
    %         \int_{0}^{\tilde{T}_{q}^{k}}{
    %             \text{ta}(l)dl
    %         }
    %         \in [0,1]
    %     \end{gather}
    %     and
    %     \begin{align}
    %         1 - \tilde{w}_{q}^{k}
    %         &=
    %         \int_{\tilde{T}_{q}^{k}}^{1}{
    %             \text{ta}(l)dl
    %         }
    %         \\
    %         &=
    %         \int_{0}^{1}{
    %             \text{ta}(l)dl
    %         }
    %         -
    %         \int_{0}^{\tilde{T}_{q}^{k}}{
    %             \text{ta}(l)dl
    %         }
    %         \\
    %         &=
    %         1
    %         -
    %         \int_{0}^{\tilde{T}_{q}^{k}}{
    %             \text{ta}(l)dl
    %         }
    %         \in [0,1]
    %     \end{align}
    %     to avoid hindering production, so that
    %     \begin{align}
    %         \mho_{q}(\tilde{w}_{q}^{k})
    %         &= 
    %         \min\left[
    %             \tilde{w}_{q}^{k}
    %             \left(
    %                 \int_{0}^{\tilde{T}_{q}^{k}}{
    %                     \text{ta}(l)dl
    %                 }
    %             \right) ^ {-1}
    %             ,
    %             (1 - \tilde{w}_{q}^{k})
    %             \left(
    %                 \int_{\tilde{T}_{q}^{k}}^{1}{
    %                     \text{ta}(l)dl
    %                 }
    %             \right) ^ {-1}
    %         \right]
    %         \\
    %         &= 
    %         \tilde{w}_{q}^{k}
    %         \left(
    %             \int_{0}^{\tilde{T}_{q}^{k}}{
    %                 \text{ta}(l)dl
    %             }
    %         \right) ^ {-1}
    %         \\
    %         &= 
    %         (1 - \tilde{w}_{q}^{k})
    %         \left(
    %             \int_{\tilde{T}_{q}^{k}}^{1}{
    %                 \text{ta}(l)dl
    %             }
    %         \right) ^ {-1}
    %         \\
    %         &= 
    %         \left(
    %             \int_{0}^{\tilde{T}_{q}^{k}}{
    %                 \text{ta}(l)dl
    %             }
    %         \right)
    %         \left(
    %             \int_{0}^{\tilde{T}_{q}^{k}}{
    %                 \text{ta}(l)dl
    %             }
    %         \right) ^ {-1}
    %         \\
    %         &= 
    %         \left(
    %             \int_{\tilde{T}_{q}^{k}}^{1}{
    %                 \text{ta}(l)dl
    %             }
    %         \right)
    %         \left(
    %             \int_{\tilde{T}_{q}^{k}}^{1}{
    %                 \text{ta}(l)dl
    %             }
    %         \right) ^ {-1}
    %         \\
    %         &= 
    %         1
    %     \end{align}
    %     it, then, follows that, if we set any relative employment level above the optimal one $\tilde{v}_{q}^{k} > \tilde{w}_{q}^{k}, \tilde{v}_{q}^{k}, \tilde{w}_{q}^{k} \in [0,1]$, operational output will be lower, for
    %     \begin{align}
    %         \mho_{q}(\tilde{v}_{q}^{k})
    %         = 
    %         \min\left[
    %             \tilde{v}_{q}^{k}
    %             \left(
    %                 \int_{0}^{\tilde{T}_{q}^{k}}{
    %                     \text{ta}(l)dl
    %                 }
    %             \right) ^ {-1}
    %             ,
    %             (1 - \tilde{v}_{q}^{k})
    %             \left(
    %                 \int_{\tilde{T}_{q}^{k}}^{1}{
    %                     \text{ta}(l)dl
    %                 }
    %             \right) ^ {-1}
    %         \right]
    %         \\
    %         = 
    %         \min\left[
    %             \frac{
    %                 \tilde{v}_{q}^{k}
    %             }{
    %                 \int_{0}^{\tilde{T}_{q}^{k}}{
    %                     \text{ta}(l)dl
    %                 }
    %             }
    %             ,
    %             \frac{
    %                 1 - \tilde{v}_{q}^{k}
    %             }{
    %                 \int_{\tilde{T}_{q}^{k}}^{1}{
    %                     \text{ta}(l)dl
    %                 }
    %             }
    %         \right]
    %         \\
    %         = 
    %         \min\left[
    %             \frac{
    %                 \tilde{v}_{q}^{k}
    %             }{
    %                 \int_{0}^{\tilde{T}_{q}^{k}}{
    %                     \text{ta}(l)dl
    %                 }
    %             }
    %             ,
    %             \frac{
    %                 1 - \tilde{v}_{q}^{k}
    %             }{
    %                 1 - 
    %                 \int_{0}^{\tilde{T}_{q}^{k}}{
    %                     \text{ta}(l)dl
    %                 }
    %             }
    %         \right]
    %         \\
    %         = 
    %         \min\left[
    %             \frac{
    %                 \tilde{v}_{q}^{k}
    %             }{
    %                 \tilde{w}_{q}^{k}
    %             }
    %             ,
    %             \frac{
    %                 1 - \tilde{v}_{q}^{k}
    %             }{
    %                 1 - 
    %                 \tilde{w}_{q}^{k}
    %             }
    %         \right]
    %         \therefore
    %         \\
    %         \tilde{v}_{q}^{k} > 
    %         \tilde{w}_{q}^{k} 
    %         \land
    %         \tilde{v}_{q}^{k}, 
    %         \tilde{w}_{q}^{k} 
    %         \in [0,1]
    %         \implies
    %         \frac{
    %             \tilde{v}_{q}^{k}
    %         }{
    %             \tilde{w}_{q}^{k}
    %         } 
    %         > 
    %         \frac{
    %             1 - \tilde{v}_{q}^{k}
    %         }{
    %             1 - \tilde{w}_{q}^{k}
    %         }
    %         \\
    %         \therefore
    %         \min\left[
    %             \frac{
    %                 \tilde{v}_{q}^{k}
    %             }{
    %                 \tilde{w}_{q}^{k}
    %             }
    %             ,
    %             \frac{
    %                 1 - \tilde{v}_{q}^{k}
    %             }{
    %                 1 - 
    %                 \tilde{w}_{q}^{k}
    %             }
    %         \right]
    %         =
    %         \frac{
    %             1 - \tilde{v}_{q}^{k}
    %         }{
    %             1 - \tilde{w}_{q}^{k}
    %         }
    %         \\
    %         =
    %         \frac{
    %             1 - \tilde{v}_{q}^{k}
    %         }{
    %             \int_{\tilde{T}_{q}^{k}}^{1}{
    %                 \text{ta}(l)dl
    %             }
    %         }
    %         \\
    %         \therefore
    %         \frac{
    %             1 - \tilde{v}_{q}^{k}
    %         }{
    %             \int_{\tilde{T}_{q}^{k}}^{1}{
    %                 \text{ta}(l)dl
    %             }
    %         } 
    %         < 
    %         \frac{
    %             1 - \tilde{w}_{q}^{k}
    %         }{
    %             \int_{\tilde{T}_{q}^{k}}^{1}{
    %                 \text{ta}(l)dl
    %             }
    %         }
    %         \\
    %         =
    %         \frac{
    %             \tilde{w}_{q}^{k}
    %         }{
    %             \int_{0}^{\tilde{T}_{q}^{k}}{
    %                 \text{ta}(l)dl
    %             }
    %         }
    %         \\
    %         = 1
    %     \end{align}

    We can derive the same conclusion for a maximally stratified labor market. But here, instead of choosing a $\boldsymbol{w_{q}^{*}}$ vector of employment levels, employers maximize production with optimal $\boldsymbol{\ell_{q}^{*}}$ responsability bounds.

    Let 
    \begin{gather}
        \boldsymbol{\ell_{q}^{*}}
        :=
        (\ell_{0}^{*}, \dots, \ell_{w_q}^{*})
        :=
        (0, \dots, 1) 
        \in [0,1]^{w_q}
        ,
    \end{gather}
    with
    \begin{gather}
    \sum_{v=1}^{w_q}
    \int_{
        \ell_{v-1}^{*}
    }^{
        \ell_{v}^{*}
    }
    \text{ta}(l)dl
    =
    \int_{0}^{1}
    \text{ta}(l)dl
    =
    1
    \end{gather}
    be the vector of optimal responsability bounds that maximizes operational output, such that
    \begin{gather}
        \mho_{q}(\boldsymbol{\ell_{q}^{*}})
        =
        \min(
            \boldsymbol{1}
            \times
            \boldsymbol{\mho_q}(
                \boldsymbol{\ell_{q}^{*}}
            )
        )
        =
        1
        \times
        \left(
            \int_{
                \ell_{v-1}^{*}
            }^{
                \ell_{v}^{*}
            }
            \text{ta}(l)dl
        \right) ^ {-1}
        =
        w_q
        .
    \end{gather}

    If employers were to set some $\ell_v < \ell_{v}^{*}$, the production of this particular job subtype would increase, for
    \begin{align}
    \ell_v < \ell_{v}^{*}
    \implies
        \left(
            \int_{
                \ell_{v-1}^{*}
            }^{
                \ell_{v}
            }
            \text{ta}(l)dl
        \right) ^ {-1}
        >
        \left(
            \int_{
                \ell_{v-1}^{*}
            }^{
                \ell_{v}^{*}
            }
            \text{ta}(l)dl
        \right) ^ {-1}
        =
        w_q
        .
    \end{align}
    
    However, because every worker has the same unitary time allowance, this would also entail that either the missing subinterval of complex tasks $l \in (\ell_{v}, \ell_{v}^{*}]$ would not be produced at all, in which case
    \begin{align}
        \mho_{q}(\boldsymbol{\ell_{q}})
        =
        0
        \times
        \left(
            \int_{
                \ell_{v}
            }^{
                \ell_{v}^{*}
            }
            \text{ta}(l)dl
        \right) ^ {-1}
        =
        0
        ,
    \end{align}
    or that it would be produced with a $1 - \omega_{q}^{v} \in [0,1]$ fraction of a time unit, yielding some quantity
    \begin{align}
        \mho_{q}(
            \boldsymbol{\ell_{q}}
            ,\omega_{q}^{v}
        )
        =
        (1 - \omega_{q}^{v})
        \times
        \left(
            \int_{
                \ell_{v}
            }^{
                \ell_{v}^{*}
            }
            \text{ta}(l)dl
        \right) ^ {-1}
        ,
    \end{align}
    where $\omega_{q}^{v} \in [0,1]$ is the percentage of worker $v$'s time allowance dedicated to the emphasized $l \in [\ell_{v-1}^{*}, \ell_{v}]$ responsability spectrum.

    Furthermore, because aggregate operational output is given by the Leontief production function,
    \begin{gather}
    % \begin{align}
        \mho_{q}(
            \boldsymbol{\ell_{q}}
            ,\boldsymbol{\omega_{q}}
        )
        =
        % &=
        \mho(
            \boldsymbol{1},
            \boldsymbol{\mho_q}(
                \boldsymbol{\ell_{q}}
                ,\boldsymbol{\omega_{q}}
            )
        )
        =
        \min(
            \boldsymbol{1}
            \times
            \boldsymbol{\mho_q}(
                \boldsymbol{\ell_{q}}
                ,\boldsymbol{\omega_{q}}
            )
        )
        ,
        \\
        \boldsymbol{\mho_q}(
            \boldsymbol{\ell_{q}}
            ,\boldsymbol{\omega_{q}}
        )
        =
        \left\{
            \mho_{q}^{v}
            ,
            v \in \{1, \dots, w_q\}
        \right\}
        ,
        \\
        \mho_{q}^{v} = 
        \min
        \left(
                \frac{
                    \omega_{q}^{v}
                }{
                    \int_{\ell_{v-1}^{*}}^{\ell_v}{
                        \text{ta}(l)dl
                    }
                }
                ,
                \frac{
                    1 - \omega_{q}^{v}
                }{
                    \int_{\ell_v}^{\ell_{v}^{*}}{
                        \text{ta}(l)dl
                    }
                }
            \right)
        % \\
        % &=
        % \min(
        %     \boldsymbol{1}
        %     \times
        %     \boldsymbol{\mho_q}(
        %         \boldsymbol{\ell_{q}}
        %         ,\boldsymbol{\omega_{q}}
        %     )
        % )
        % \\
        % &=
        % \min
        % \left(
        %     \frac{
        %         \omega_{q}^{1}
        %     }{
        %         \int_{\ell_{0}^{*}}^{\ell_1}{
        %             \text{ta}(l)dl
        %         }
        %     }
        %     ,
        %     \frac{
        %         1 - \omega_{q}^{1}
        %     }{
        %         \int_{\ell_1}^{\ell_{1}^{*}}{
        %             \text{ta}(l)dl
        %         }
        %     }
        %     ,
        %     \dots
        %     ,
        %     \frac{
        %         \omega_{q}^{w_q}
        %     }{
        %         \int_{\ell_{w_q - 1}^{*}}^{\ell_{w_q}}{
        %             \text{ta}(l)dl
        %         }
        %     }
        %     ,
        %     \frac{
        %         1 - \omega_{q}^{w_q}
        %     }{
        %         \int_{\ell_{w_q}}^{\ell_{w_q}^{*}}{
        %             \text{ta}(l)dl
        %         }
        %     }
        % \right)
        % \\
        % &:=
        % \min
        % \left(
        %     \frac{1}{
        %         \int_{0}^{\ell_1}{
        %             \text{ta}(l)dl
        %         }
        %     }
        %     ,
        %     \dots
        %     ,
        %     \frac{1}{
        %         \int_{\ell_{w_q-1}}^{1}{
        %             \text{ta}(l)dl
        %         }
        %     }
        % \right)
        ,
    \end{gather}
    % \end{align}
    it would be pointless if only a subset of employees were to increase their operational output by themselves; for an occupation's complex tasks are all complementary: they work together to acchieve its operation. Hence, for $\mho_{q}(
        \boldsymbol{\ell_q}
        ,\boldsymbol{\omega_q}
    )$ to be greater than $
    \mho_{q}(
        \boldsymbol{\ell_{q}^{*}}
    )
    =
    w_q$,
    \begin{gather}
        \mho_{q}^{v} > w_q
        \
        \forall
        \
        v \in \{1, \dots, w_q\}
        ,
    \end{gather}
    which requires partial operational outputs to surpass the following point of equilibrium:
    \begin{align}
        &\mho_{q}(
            \boldsymbol{\ell_q}
            ,\boldsymbol{\omega_q}
        ) 
        =
        \mho_{q}(
            \boldsymbol{\ell_{q}^{*}}
        ) 
        = 
        \min(
            \boldsymbol{1}
            \times
            \boldsymbol{\mho_q}(
                \boldsymbol{\ell_{q}^{*}}
            )
        )
        =
        w_q
        \\
        &\iff
        \omega_{q}^{v}
        \mho_{q}^{v}(
            \ell_{v-1}^{*}
            ,\ell_{v}
        )
        =
        (1-\omega_{q}^{v})
        \mho_{q}^{v}(
            \ell_{v}
            ,\ell_{v}^{*}
        )
        =
        w_q
        \
        \forall
        \
        v \in \{1, \dots, w_q\}
        \\
        &\iff
        \omega_{q}^{v}
        =
        \frac{w_q}{
            \mho_{q}^{v}(
                \ell_{v-1}^{*}
                ,\ell_{v}
            )
        }
        :=
        w_q
        \Omega_{q}^{v}(
            \ell_{v-1}^{*}
            ,\ell_{v}
        )
        :=
        w_q
        \int_{
            \ell_{v-1}^{*}
        }^{
            \ell_{v}
        }{
            \text{ta}(l)
            dl
        }
        \\
        &\land
        1 - \omega_{q}^{v}
        =
        \frac{w_q}{
            \mho_{q}^{v}(
                \ell_{v}
                ,\ell_{v}^{*}
            )
        }
        :=
        w_q
        \Omega_{q}^{v}(
            \ell_{v}
            ,\ell_{v}^{*}
        )
        :=
        w_q
        \int_{
            \ell_{v}
        }^{
            \ell_{v}^{*}
        }{
            \text{ta}(l)
            dl
        }
        ,
    \end{align}
    where $\mho_{q}^{v}(\ell_{v},\ell_{r})$, with
    \begin{gather}
        \mho_{q}^{v}
        =
        \min
        \left[
            \omega_{q}^{v}
            \mho_{q}^{v}(
                \ell_{v-1}^{*}
                ,\ell_{v}
            )
            ,
            (1 - \omega_{q}^{v})
            \mho_{q}^{v}(
                \ell_{v}
                ,\ell_{v}^{*}
            )
        \right]
        ,
    \end{gather}
    and $\Omega_{q}^{v}(\ell_{v},\ell_{r}), v, r \in \{1, \dots, w_q\}$, are short-hand notations for partial operational output and its inverse, total time allocation.

    Now, if any single $\omega_{q}^{v} \in [0,1], v \in \{1, \dots, w_q\}$ is set to
    \begin{gather}
        \omega_{q}^{v} > w_q
        \Omega_{q}^{v}(
            \ell_{v-1}^{*}
            ,\ell_{v}
        )
        ,
    \end{gather}
    then, indeed,
    \begin{gather}
        \omega_{q}^{v}
        \mho_{q}^{v}(
            \ell_{v-1}^{*}
            ,\ell_{v}
        )
        >
        w_q
        ,
    \end{gather}
    but also
    \begin{align}
        &
        (1-\omega_{q}^{v})
        \mho_{q}^{v}(
            \ell_{v}
            ,\ell_{v}^{*}
        )
        <
        w_q
        \\
        &\implies
        \mho_{q}^{v} = 
        \min\left[
            \frac{
                \omega_{q}^{v}
            }{
                \Omega_{q}^{v}(
                    \ell_{v-1}^{*}
                    ,\ell_{v}
                )
            }
            ,
            \frac{
                1-\omega_{q}^{v}
            }{
                \Omega_{q}^{v}(
                    \ell_{v}
                    ,\ell_{v}^{*}
                )
            }
        \right]
        =
        \frac{
            1-\omega_{q}^{v}
        }{
            \Omega_{q}^{v}(
                \ell_{v}
                ,\ell_{v}^{*}
            )
        }
        <
        w_q
        \\
        &\therefore
        % &\implies
        \mho_{q}(
            \boldsymbol{\ell_q}
            ,\boldsymbol{\omega_q}
        ) 
        <
        \mho_{q}(
            \boldsymbol{\ell_{q}^{*}}
        )
        =
        w_q
        ,
    \end{align}
    and, conversely, 
    \begin{align}
        &
        \omega_{q}^{v} < 
        w_q
        \Omega_{q}^{v}(
            \ell_{v-1}^{*}
            ,\ell_{v}
        )
        \\
        &\implies
        \omega_{q}^{v}
        \mho_{q}^{v}(
            \ell_{v-1}^{*}
            ,\ell_{v}
        )
        % \frac{
        %     \omega_{q}^{v}
        % }{
        %     \Omega_{q}^{v}(
        %         \ell_{v-1}^{*}
        %         ,\ell_{v}
        %     )
        % }
        < 
        w_q
        <
        (1-\omega_{q}^{v})
        \mho_{q}^{v}(
            \ell_{v}
            ,\ell_{v}^{*}
        )
        % \frac{
        %     1-\omega_{q}^{v}
        % }{
        %     \Omega_{q}^{v}(
        %         \ell_{v}
        %         ,\ell_{v}^{*}
        %     )
        % }
        \\
        &\implies
        \mho_{q}^{v} = 
        \min\left[
            \frac{
                \omega_{q}^{v}
            }{
                \Omega_{q}^{v}(
                    \ell_{v-1}^{*}
                    ,\ell_{v}
                )
            }
            ,
            \frac{
                1-\omega_{q}^{v}
            }{
                \Omega_{q}^{v}(
                    \ell_{v}
                    ,\ell_{v}^{*}
                )
            }
        \right]
        =
        \frac{
            \omega_{q}^{v}
        }{
            \Omega_{q}^{v}(
                \ell_{v-1}^{*}
                ,\ell_{v}
            )
        }
        <
        w_q
        \\
        &\therefore
        % &\implies
        \mho_{q}(
            \boldsymbol{\ell_q}
            ,\boldsymbol{\omega_q}
        ) 
        <
        \mho_{q}(
            \boldsymbol{\ell_{q}^{*}}
        )
        =
        w_q
        ,
    \end{align}
    so that 
    \begin{gather}
        \nexists
        \
        \boldsymbol{\ell_q}
        % \neq
        % \boldsymbol{\ell_{q}^{*}}
        ,
        \boldsymbol{\omega_q}
        \in
        [0,1]^{w_q}
        \
        \big|
        \
        \mho_{q}(
            \boldsymbol{\ell_q}
            ,\boldsymbol{\omega_q}
        ) 
        >
        \mho_{q}(
            \boldsymbol{\ell_{q}^{*}}
        )
        =
        \min(
            \boldsymbol{1}
            \times
            \boldsymbol{\mho_q}(
                \boldsymbol{\ell_{q}^{*}}
            )
        )
        =
        w_q
        ,
        \\
        \sum_{v=1}^{w_q}
        \int_{
            \ell_{v-1}
        }^{
            \ell_{v}
        }
        \text{ta}(l)dl
        =
        \sum_{v=1}^{w_q}
        \int_{
            \ell_{v-1}^{*}
        }^{
            \ell_{v}^{*}
        }
        \text{ta}(l)dl
        =
        \int_{0}^{1}
        \text{ta}(l)dl
        =
        1
        .
    \end{gather}

    Thus, we have demonstrated there cannot be, in any productive arrangement, a higher aggregate operational output than $w_q$, that is the number of employees in a particular labor market, for all attempts to yield more output, actually, end up hindering production
        In other words, one day is sufficient to produce a ``day's work''. Or, put another way, the most one can produce in a day is a ``day's work''.
        % \begin{align}
        %     \mho_{q}(\tilde{v}_{q}^{k})
        %     &= 
        %     \min\left[
        %         \tilde{v}_{q}^{k}
        %         \left(
        %             \int_{0}^{\tilde{T}_{q}^{k}}{
        %                 \text{ta}(l)dl
        %             }
        %         \right) ^ {-1}
        %         ,
        %         (1 - \tilde{v}_{q}^{k})
        %         \left(
        %             \int_{\tilde{T}_{q}^{k}}^{1}{
        %                 \text{ta}(l)dl
        %             }
        %         \right) ^ {-1}
        %     \right]
        %     \\
        %     &= 
        %     (1 - \tilde{v}_{q}^{k})
        %     \left(
        %         \int_{\tilde{T}_{q}^{k}}^{1}{
        %             \text{ta}(l)dl
        %         }
        %     \right) ^ {-1}
        %     \\
        %     \therefore
        %     \mho_{q}(\tilde{w}_{q}^{k})
        %     &= 
        %     \min\left[
        %         \tilde{w}_{q}^{k}
        %         \left(
        %             \int_{0}^{\tilde{T}_{q}^{k}}{
        %                 \text{ta}(l)dl
        %             }
        %         \right) ^ {-1}
        %         ,
        %         (1 - \tilde{w}_{q}^{k})
        %         \left(
        %             \int_{\tilde{T}_{q}^{k}}^{1}{
        %                 \text{ta}(l)dl
        %             }
        %         \right) ^ {-1}
        %     \right]
        %     \\
        %     &= 
        %     \tilde{w}_{q}^{k}
        %     \left(
        %         \int_{0}^{\tilde{T}_{q}^{k}}{
        %             \text{ta}(l)dl
        %         }
        %     \right) ^ {-1}
        %     \\
        %     &= 
        %     (1 - \tilde{w}_{q}^{k})
        %     \left(
        %         \int_{\tilde{T}_{q}^{k}}^{1}{
        %             \text{ta}(l)dl
        %         }
        %     \right) ^ {-1}
        %     \\
        %     &>
        %     (1 - \tilde{v}_{q}^{k})
        %     \left(
        %         \int_{\tilde{T}_{q}^{k}}^{1}{
        %             \text{ta}(l)dl
        %         }
        %     \right) ^ {-1}
        %     ,
        % \end{align}
    % \end{enumerate}

Optimal Stratification Lemma (OSL)
% Maximally Stratified Production Lemma (MSPL)

Because in a maximally and monotonically stratified labor market every position is its own job subtype (for, again, employment levels are unitary), optimal production is, then, obtained not by choosing how many workers to allocate to tasks of varying difficulty levels, but instead by setting appropriate responsability ranges for each position (i.e. which tasks to allocate \textit{to} workers). The bounds for these ranges are:
\begin{gather}
    \ell_v
    =
    \text{TA}^{-1}\left(
        \frac{v}{w_q}
        +
        \text{TA}(0)
    \right)
    \
    \forall
    \
    v \in 
    \left\{
        1, \dots, w_q
    \right\}
    .
\end{gather}

\textit{Proof}:

We have just demonstrated that the maximum operational output in any labor market, with or without unique, unitary, positions, is exactly 
% We have just demonstrated that the maximum operational output in any labor market is exactly 
\begin{gather}
    \mho_{q}^{*}
        =
        \min(
            \boldsymbol{w_{q}^{*}}
            \times
            \boldsymbol{\mho_q}
        )
        = 
        \min(
            \boldsymbol{1}
            \times
            \boldsymbol{\mho_q}(
                \boldsymbol{\ell_{q}^{*}}
            )
        )
        =
        w_q
    % \mho_{q}^{*} =
    % \min\left(
    %     \boldsymbol{w_q}
    %     \times
    %     \boldsymbol{\mho_{q}}
    % \right)
    % =
    % w_q
    ,
\end{gather}
or the number of employees in its workforce.

Therefore, optimal bounds for responsability spectra can be calculated by equating partial operational outputs with maximum production, for if maximum-monotonic labor stratification is to be optimal, it must yield the same partial outputs as any efficient production strategy.
% Therefore, optimal responsability ranges can be calculated by equating partial operational outputs with maximum production, for if maximum-monotonic labor stratification is to be optimal, it must yield the same partial outputs as any efficient production strategy.

So, for the first job subtype,
\begin{gather}
    1 \times \left(
        \int_{\ell_0}^{\ell_1}
        \text{ta}(l)
        dl
    \right) ^ {-1}
    =
    1 \times \left(
        \int_{0}^{\ell_1}
        \text{ta}(l)
        dl
    \right) ^ {-1}
    =
    w_q
    ,
\end{gather}
which means the partial operational output of the first worker, whose tasks range from $\ell_0 = 0$ to $\ell_1 \in [0,1]$ exclusively, should produce the same amount of the $l \in [0,\ell_1]$ responsability spectrum as would be produced in an economic configuration with maximum operational output (e.g. with $w_q$ perfectly qualified employees working independently).

Thus, solving for $\ell_1$, we get:
\begin{align}
    1 
    &\times 
    \left(
        \int_{0}^{\ell_1}
        \text{ta}(l)
        dl
    \right) ^ {-1}
    =
    w_q
    \\
    &\therefore
    \int_{0}^{\ell_1}
    \text{ta}(l)
    dl
    =
    \frac{1}{w_q}
    \\
    &\therefore
    \text{TA}(l)
    \big|_{0}^{\ell_1}
    =
    \text{TA}(\ell_1)
    -
    \text{TA}(0)
    =
    \frac{1}{w_q}
    \\
    &\therefore
    \text{TA}^{-1}(
        \text{TA}(\ell_1)
    )
    =
    \text{TA}^{-1}\left(
        \frac{1}{w_q}
        +
        \text{TA}(0)
    \right)
    \\
    &\therefore
    \ell_1
    =
    \text{TA}^{-1}\left(
        \frac{1}{w_q}
        +
        \text{TA}(0)
    \right)
    .
\end{align}

Likewise, for the second worker,
\begin{align}
    1 
    &\times 
    \left(
        \int_{\ell_1}^{\ell_2}
        \text{ta}(l)
        dl
    \right) ^ {-1}
    =
    w_q
    \\
    &\therefore
    \int_{\ell_1}^{\ell_2}
    \text{ta}(l)
    dl
    =
    \frac{1}{w_q}
    \\
    &\therefore
    \text{TA}(l)
    \big|_{\ell_1}^{\ell_2}
    =
    \text{TA}(\ell_2)
    -
    \text{TA}(\ell_1)
    =
    \frac{1}{w_q}
    \\
    &\therefore
    \text{TA}^{-1}(
        \text{TA}(\ell_2)
    )
    =
    \text{TA}^{-1}\left(
        \frac{1}{w_q}
        +
        \text{TA}(\ell_1)
    \right)
    \\
    &\therefore
    \text{TA}^{-1}(
        \text{TA}(\ell_2)
    )
    =
    \text{TA}^{-1}\left(
        \frac{1}{w_q}
        +
        \frac{1}{w_q}
        +
        \text{TA}(0)
    \right)
    \\
    &\therefore
    \ell_2
    =
    \text{TA}^{-1}\left(
        \frac{2}{w_q}
        +
        \text{TA}(0)
    \right)
    .
\end{align}

For the third worker,
\begin{align}
    1 
    &\times 
    \left(
        \int_{\ell_2}^{\ell_3}
        \text{ta}(l)
        dl
    \right) ^ {-1}
    =
    w_q
    \\
    &\therefore
    \int_{\ell_2}^{\ell_3}
    \text{ta}(l)
    dl
    =
    \frac{1}{w_q}
    \\
    &\therefore
    \text{TA}(l)
    \big|_{\ell_2}^{\ell_3}
    =
    \text{TA}(\ell_3)
    -
    \text{TA}(\ell_2)
    =
    \frac{1}{w_q}
    \\
    &\therefore
    \text{TA}^{-1}(
        \text{TA}(\ell_3)
    )
    =
    \text{TA}^{-1}\left(
        \frac{1}{w_q}
        +
        \text{TA}(\ell_2)
    \right)
    \\
    &\therefore
    \text{TA}^{-1}(
        \text{TA}(\ell_3)
    )
    =
    \text{TA}^{-1}\left(
        \frac{1}{w_q}
        +
        \frac{1}{w_q}
        +
        \frac{1}{w_q}
        +
        \text{TA}(0)
    \right)
    \\
    &\therefore
    \ell_3
    =
    \text{TA}^{-1}\left(
        \frac{3}{w_q}
        +
        \text{TA}(0)
    \right)
    .
\end{align}

And so on and so forth, up to the very last worker:
\begin{align}
    1 
    &\times 
    \left(
        \int_{\ell_{w_q - 1}}^{\ell_{w_q}}
        \text{ta}(l)
        dl
    \right) ^ {-1}
    =
    w_q
    \\
    &\therefore
    \int_{\ell_{w_q - 1}}^{\ell_{w_q}}
    \text{ta}(l)
    dl
    =
    \frac{1}{w_q}
    \\
    &\therefore
    \text{TA}(l)
    \big|_{\ell_{w_q - 1}}^{\ell_{w_q}}
    =
    \text{TA}(\ell_{w_q})
    -
    \text{TA}(\ell_{w_q - 1})
    =
    \frac{1}{w_q}
    \\
    &\therefore
    \text{TA}^{-1}(
        \text{TA}(\ell_{w_q})
    )
    =
    \text{TA}^{-1}\left(
        \frac{1}{w_q}
        +
        \text{TA}(\ell_{w_q - 1})
    \right)
    \\
    &\therefore
    \text{TA}^{-1}(
        \text{TA}(\ell_{w_q})
    )
    =
    \text{TA}^{-1}\left(
        \frac{1}{w_q}
        +
        \dots
        +
        \frac{1}{w_q}
        +
        \text{TA}(0)
    \right)
    \\
    &\therefore
    \ell_{w_q}
    =
    \text{TA}^{-1}\left(
        \frac{w_q}{w_q}
        +
        \text{TA}(0)
    \right)
    := 1
    \\
    &\iff
    \text{TA}^{-1}\left(
        \frac{w_q}{w_q}
        +
        \text{TA}(0)
    \right)
    =
    \text{TA}^{-1}\left(
        1
        +
        \text{TA}(0)
    \right)
    =
    1
    \\
    &\iff
    \text{TA}\left(
        \text{TA}^{-1}\left(
        1
        +
        \text{TA}(0)
    \right)
    \right)
    =
    \text{TA}(1)
    \\
    &\iff
    \text{TA}(1)
    -
    \text{TA}(0)
    =
    \int_{0}^{1}{
        \text{ta}(l)
        dl
    }
    =
    1
    ,
\end{align}
which is true, by definition,
\begin{align}
    &\because
    \text{ta}(l) 
    :=
    \text{ttc}(l)
    \times
    \left(
        \int_{0}^{1}{
            \text{ttc}(l)
            dl
        }
    \right) ^ {-1}
    \\
    % &\implies
    &\therefore
    \int_{0}^{1}{
        \text{ta}(l)
        dl
    }
    =
    \left(
        \int_{0}^{1}
        \text{ttc}(l)
        dl
    \right) ^ {-1}
    \times
    \int_{0}^{1}{
        \text{ttc}(l)
        dl
    }
    =
    1
    .
\end{align}

% And, with this condition met, we can finally arrive, by the induction above, to a general form of optimal responsability bounds:
And, with this condition met, we can finally arrive, by the induction above, to a general form of optimal responsability ranges:
\begin{gather}
    \ell_v
    =
    \text{TA}^{-1}\left(
        \frac{v}{w_q}
        +
        \text{TA}(0)
    \right)
    \
    \forall
    \
    v \in 
    \left\{
        1, \dots, w_q
    \right\}
    .
\end{gather}

% Hence, mathematically,
% \begin{gather}
%     l \in \left[
%     \frac{\ell - 1}{w_q}
%     ,
%     \frac{\ell}{w_q}
%     \right]
%     ,
%     \ell \in 
%     \left\{
%         1, \dots, w_q
%     \right\}
% \end{gather}
% is a responsability spectrum in a maximally stratified labor market, in which employment levels are given by
% \begin{gather}
%     \sum_{\ell=1}^{w_q}
%     w_{q}^{\ell}
%     =
%     w_q
%     ,
% \end{gather}
% so that any available position is its own job subtype and covers only a restrictive range of task difficulty, accounting for
% \begin{gather}
%     \Omega_{q}^{\ell} = 
%     \frac{1}{\mho_{q}^{\ell}} = 
%     \int_{
%         \frac{\ell - 1}{w_q}
%     }^{
%         \frac{\ell}{w_q}
%     }
%     \text{ta}(l)dl
% \end{gather}
% of an operation's total time duration
% \begin{gather}
%     \sum_{\ell=1}^{w_q}
%     \Omega_{q}^{\ell} 
%     = 
%     \sum_{\ell=1}^{w_q}
%     \int_{
%         \frac{\ell - 1}{w_q}
%     }^{
%         \frac{\ell}{w_q}
%     }
%     \text{ta}(l)dl
%     =
%     \int_{0}^{1}
%     \text{ta}(l)dl
%     =
%     1
%     .
% \end{gather}

% Intuitively speaking, we would say production in a maximally and monotonically stratified labor market is not ``independent'', in the sense that employees do not work on an occupation's operation from beginning to end. This means each of them will spend all their time allowance producing a partial operational output, that is a multiple of a difficulty subinterval of complex tasks, which will, in turn, contribute, alonside the partial outputs of other employees, to accomplish the occupational operation in its entirety.

% However, in a maximum labor stratification setting, these partial operational outputs will not be produced merely via ``senior'' and ``junior'' positions, as previously, but rather within a myriad of levels in a production hierarchy, approximating a continuum of ``seniority'' as the workforce becomes large enough.
% % However, in a maximally stratified labor market, these partial operational outputs will not be produced merely via ``senior'' and ``junior'' positions, as previously, but rather within a myriad of levels in a production hierarchy, approximating a continuum of ``seniority'' as the workforce becomes large enough.

% Again, this does not mean employees are, actually, more or less competent, only that the available positions are preemptively stratified with respect to task difficulty, in order to maximize employers' hiring pool and safeguard production in the case workers are not sufficiently qualified to produce the whole responsability spectrum by themselves (see ``Maximum-Monotonic Labor Stratification Lemma'' below).

% [end definition of maximum-monotonic labor stratification]

% \item Independent Operational Output Lemma (IOOL)
% \begin{enumerate}
%     \item From Weak Occupational Complexity Axiom (WOCA)
%     % \item Now, as the Weak Occupational Complexity Axiom states there can be no gain, nor losses, to production when labor is efficiently stratified, we can derive a generalization of the conclusion (ref) above to any economic configuration besides the binary case by comparison. 
    
%     Let us, then, first consider an independent production strategy with maximum worker productivity (i.e. the ``ceiling'' or ``benchmark'' for all other production strategies). Furthermore, because we want to compare this production strategy to that of maximum-monotonic labor stratification, we shall analyze operational output in a stratified fashion as well.

%     In this scenario, the optimal choice of hours to allocate to a responsability spectrum is trivial:
%     \begin{gather}
%         w_{q}^{\ell}
%         =
%         \Omega_{q}^{\ell}
%         \in [0,1]
%         .
%     \end{gather}
%     Now, by the definition of partial operational output (ref) above, one produces $\mho_{q}^{\ell}$ when devoting their entire unitary time allowance to the
%     \begin{gather}
%         l \in \left[
%         \frac{\ell - 1}{w_q}
%         ,
%         \frac{\ell}{w_q}
%         \right]
%         ,
%         \ell \in 
%         \left\{
%             1, \dots, w_q
%         \right\}
%     \end{gather}
%     responsability spectrum. Therefore, with only $\Omega_{q}^{\ell} \in [0,1]$ time units, operational output is:
%     \begin{gather}
%         \Omega_{q}^{\ell}
%         \mho_{q}^{\ell}
%         =
%         \left(
%             \frac{1}{\mho_{q}^{\ell}}
%         \right)
%         \times
%         \mho_{q}^{\ell}
%         =
%         1
%         .
%     \end{gather}
%     % as even the weakest version of occupational complexity implies all difficulty levels are complementary and have to be accomplished for optimal operational output.
%     % In this formulation, though, employment $w_{q}^{\ell} \in [0,1]$ is partial, that is: the same worker is hired to accomplish every responsability spectrum. This is because such an independent labor market is not stratified at all.

%     And, as Weak Occupational Complexity implies operational output is homothetic, the aggregate production of $w_q$ perfectly qualified employees working independently is:
%     % Furthermore, as Weak Occupational Complexity implies operational output is homothetic, the aggregate production of $w_q$ perfectly qualified employees working independently is:
%     \begin{gather}
%         w_{q}^{\ell}
%         \mho_{q}^{\ell}
%         \times
%         w_q
%         =
%         \Omega_{q}^{\ell}
%         \mho_{q}^{\ell}
%         \times
%         w_q
%         =
%         1
%         \times
%         w_q
%         =
%         w_q
%         .
%     \end{gather}
%     Hence, a perfectly qualified employee working full-time and independently can output one unit of an occupation's complex tasks with one unit of their time (i.e. their entire time allowance). And, likewise, a workforce with $w_q$ employees identical to this one produces $w_q$ units of operational output. Or, to put it simply, a maximally productive person acchieves maximum production.
%     % Hence, a perfectly qualified employee working full-time and independently can output one unit of an occupation's complex tasks with one unit of their time (i.e. their entire time allowance). And, likewise, a workforce with $w_q$ employees identical to this one produces $w_q$ units of operational output. Or, to put it simply, a maximally productive person acchieves maximum production. Or, even simpler: one day is sufficient to produce a ``day's work''.
% \end{enumerate}

    \item Productivity Sufficiency Lemma (PSL)
    \begin{enumerate}
        \item dsds
    \end{enumerate}

    \item Maximum-Monotonic Labor Stratification Lemma (MLSL)
    \begin{enumerate}
        \item From Employer Rationality Axiom, Weak Skill Difference Axiom, and Weak Occupational Complexity Axiom.
        \begin{enumerate}
            \item Now, to generalize this conclusion for other economic configurations, we shall define notation in terms of maximum labor stratification, that is a productive arrangement in which there are not one (``homogeneous'' or ``independent''), nor two (``juniors'' and ``seniors''), but rather several job subtypes, indeed as many as there are jobs themselves, each with a limited spectrum of responsabilities. Furthermore, we shall demonstrate that, given our axioms, such an economic configuration is, in fact, the only optimal production strategy and, so, holds in the labor market.
            % \item Now, to work towards a generalization of this conclusion for other economic configurations, we shall define our notation in terms of maximum labor stratification, that is a productive arrangement in which there are not one (``homogeneous'' or ``independent''), nor two (``juniors'' and ``seniors''), but rather several job subtypes, indeed as many as there are jobs themselves, each with a very limited spectrum of responsabilities. Furthermore, we shall demonstrate that, given our axioms, such an economic configuration is, in fact, the only optimal production strategy and, so, holds in the labor market.
            
            Hence, the Maximum-Monotonic Labor Stratification Lemma (MLSL) states that a perfectly rational employer (ERA), which expects there could be skill differences in the workforce (WSDA), and can split operational output without either gain or loss to production (WOCA), will, therefore, strategically stratify their job offers monotonically, and even maximally, so that, if indeed there happens to be skill differences in the labor market, they can, then, allocate less competent workers to easier roles, and avoid wasting talent, thus ``saving their best'' for the most demanding tasks.
            
            Mathematically,
            \begin{gather}
                l \in \left[
                \frac{\ell - 1}{w_q}
                ,
                \frac{\ell}{w_q}
                \right]
                ,
                \ell \in 
                \left\{
                    1, \dots, w_q
                \right\}
            \end{gather}
            is a responsability spectrum in a maximally stratified labor market, in which employment levels are given by
            \begin{gather}
                \sum_{\ell=1}^{w_q}
                w_{q}^{\ell}
                =
                w_q
                ,
            \end{gather}
            so that any available position is its own job subtype and covers only a restrictive range of task difficulty, accounting for
            \begin{gather}
                \Omega_{q}^{\ell} = 
                \frac{1}{\mho_{q}^{\ell}} = 
                \int_{
                    \frac{\ell - 1}{w_q}
                }^{
                    \frac{\ell}{w_q}
                }
                \text{ta}(l)dl
            \end{gather}
            of an operation's total time duration
            \begin{gather}
                \sum_{\ell=1}^{w_q}
                \Omega_{q}^{\ell} 
                = 
                \sum_{\ell=1}^{w_q}
                \int_{
                    \frac{\ell - 1}{w_q}
                }^{
                    \frac{\ell}{w_q}
                }
                \text{ta}(l)dl
                =
                \int_{0}^{1}
                \text{ta}(l)dl
                =
                1
                .
            \end{gather}

            Intuitively speaking, we would say production in a maximally and monotonically stratified labor market is not ``independent'', [in the sense that each employee does not work on the entire operation from beginning to end. This means each employee] will spend all their time allowance producing a partial operational output, that is a multiple of the difficulty subinterval they were hired to accomplish, which will, in turn, contribute, alonside the partial outputs of other employees, to accomplish the complete occupational operation.
            
            However, in a maximally stratified labor market, these partial operational outputs, will not be produced merely via ``senior'' and ``junior'' positions, as previously, but rather within a myriad of levels in a production hierarchy, each responsible for a particular subinterval of task difficulty, approximating a continuum of ``seniority'' as the workforce becomes large enough.
            
            Again, this does not mean employees are, actually, more or less competent, only that the available positions are [preemptively] stratified with respect to task difficulty. 

            And, indeed, regardless of workers' actual capacity, this arrangement must hold, because the expected value of operational output -- and, therefore, of producers' revenue -- is higher and constant when applying a stratified production strategy when compared to an independent production strategy; for in such a strategy, production is more easily limited by skill differences, and so the expected value of operational output is potentially lower, but never higher. That is, 
            \begin{gather}
                \mathbb{E}[
                    \tilde{T}_{q}^{k}
                ]
                \in [0,1]
                \
                \forall
                \
                k,q \in \{1, \dots, n\}
            \end{gather}
            implies that
            \begin{align}
                \int_{0}^{1}{
                    T(l,\tilde{T}_{q}^{k})
                    \text{ta}(l)
                    dl
                } 
                \\
                \leq
                \tilde{v}_{q}^{k}
                \int_{0}^{1}{
                    T(l,\tilde{T}_{q}^{k})
                    \text{ta}(l)
                    dl
                }
                +
                (1 - \tilde{v}_{q}^{k})
                \int_{0}^{1}{
                    T(l,1)
                    \text{ta}(l)
                    dl
                } 
                \\
                \leq
                \tilde{w}_{q}^{k}
                \int_{0}^{\tilde{T}_{q}^{k}}{
                    T(l,\tilde{T}_{q}^{k})
                    \text{ta}(l)
                    dl
                } + 
                (1 - \tilde{w}_{q}^{k})
                \int_{\tilde{T}_{q}^{k}}^{1}{
                    T(l,1)
                    \text{ta}(l)
                    dl
                }
                \\
                =
                \sum_{k=1?}^{n?}
                \sum_{\ell=1}^{w_q}
                \left[
                    k \in \Lambda_{q}^{\ell}
                \right]
                \left[
                    \tilde{T}_{q}^{k}
                    \geq 
                    \frac{\ell}{w_q}
                \right]
                \tilde{w}_{q}^{\ell}
                \int_{
                    \frac{\ell - 1}{w_q}
                }^{
                    \frac{\ell}{w_q}
                }{
                    \text{ta}(l)
                    dl
                }
                \\
                =
                \int_{0}^{1}{
                    T(l,1)
                    \text{ta}(l)
                    dl
                }
                \\
                =
                \int_{0}^{1}{
                    \text{ta}(l)
                    dl
                }
                \\
                =
                1
                ,
            \end{align}
            where $\tilde{v}_{q}^{k} > \tilde{w}_{q}^{k}$, with $\tilde{v}_{q}^{k}, \tilde{w}_{q}^{k} \in [0,1]$, is an inefficient allocation of workers above the optimal relative employment level $\tilde{w}_{q}^{k}$ in a semi-stratified labor market; and the double sum in equation (ref) is the output of a maximally stratified labor market, in which every $\ell \in \{1, \dots, w_q\}$ job subtype is but a fraction of available positions, with a partial workforce of $w_q \times \tilde{w}_{q}^{\ell}$ individuals, all exclusively dedicated to their own responsability spectrum, and identified by $[k \in \Lambda_{q}^{\ell}]$ employment statuses that are evaluated to 1 if they are employed in a particular $\Lambda_{q}^{\ell}$ strata of the labor market, and to 0 for all other job subtypes; while the remaining equations are the maximum operational output of a labor market with $w_q$ perfectly qualified employees working independently on the entire responsability spectrum of occupation $q$'s operation. Or, more succinctly,
            \begin{align}
                &\mathbb{E}[
                    \mho_{q}^{\text{IP}}
                    \
                    |
                    \
                    \mathbb{E}[
                        \tilde{T}_{q}^{k}
                    ]
                    \in [0,1]
                    \
                    \forall
                    \
                    k,q \in \{1, \dots, n\}
                ]
                \\
                &\leq
                \mathbb{E}[
                    \mho_{q}^{\text{IS}}
                    \
                    |
                    \
                    \mathbb{E}[
                        \tilde{T}_{q}^{k}
                    ]
                    \in [0,1]
                    \
                    \forall
                    \
                    k,q \in \{1, \dots, n\}
                ]
                \\
                &\leq
                \mathbb{E}[
                    \mho_{q}^{\text{MS}}
                    \
                    |
                    \
                    \mathbb{E}[
                        \tilde{T}_{q}^{k}
                    ]
                    \in [0,1]
                    \
                    \forall
                    \
                    k,q \in \{1, \dots, n\}
                ]
                \\
                &=
                \mathbb{E}[
                    \mho_{q}^{\text{IP}}
                    \
                    |
                    \
                    \tilde{T}_{q}^{k}
                    = 1
                    \
                    \forall
                    \
                    k,q \in \{1, \dots, n\}
                ]
                ,
            \end{align}
            where each of the terms above represents the expected value of aggregate operational output given the expected productivity in the workforce, for the three production strategies: maximum-monotonic labor stratification (MS), imperfect-monotonic labor stratification (IS), and independent production (IP).

            In other words, splitting responsabilities in accordance with competence is always as productive as the maximum operational output (viz. that which is obtained when employing perfectly qualified workers independently), provided employees are sufficiently qualified for their responsabilities. But, again, this is, by definition, guaranteed by employers' rationality, as well as the simple fact the economy is already producing its current operational output (Operational Equilibrium Lemma, OEL). 
            
            Therefore, employing potentially underqualified workers to output the entire responsability spectrum $l \in [0,1]$ independently can only be as productive as the labor stratification strategy, but never more than it. Indepedent production, then, is a suboptimal strategy when employers expect there to be skill differences in the workforce.
            
            Thus, maximum-monotonic labor stratification follows as an insurance policy against worker's potential underqualification: for if talent is lacking in the labor market, there is nothing to gain by employing individuals which are not sufficiently qualified for a difficult job, whereas if talent is abundant, there is nothing to lose when employing overqualified individuals to a job below their skill level.

            Hence, given the same $w_q$ workforce, operational output in a maximally stratified labor market is always greater or equal to the output of any other economic configuration. It is, therefore, always optimal to monotonically and maximally stratify responsabilities across $w_q$ unique positions, each focused on increasingly demanding tasks.
        \end{enumerate}
        \item Monotonic labor stratification is required and follows logically from employers' perfect rationality axiom.
        \item maximum labor stratification is optional, but also follows logically from employers' perfect rationality axiom.
        \begin{enumerate}
            \item Because the General Employability Theorem (GET) holds true for imperfectly stratified labor markets as well, for less than maximum labor stratification is mathematically equivalent to just a variable change. This said, imperfect labor market stratification leads to inefficiencies in hiring, as the base requirements for each stratum are higher than they would be if labor was maximally stratified.
        \end{enumerate}
    \end{enumerate}

    \item Productivity Sufficiency Lemma (PSL)
    \begin{enumerate}
        \item dsds
    \end{enumerate}

    % Finally, to identify the optimal $\tilde{w}_{q}^{\ell}$ relative employment levels in the labor market, and thus calculate a generalized employability coefficient, we compare the two production strategies that, as demonstrated, guarantee maximum operational output (viz. maximum-monotonic labor stratification and perfectly qualified independent production).
    % % Finally, in order to identify the optimal $\tilde{w}_{q}^{\ell}$ relative employment levels in the labor market, and thus calculate a generalized employability coefficient, we compare the two production strategies that, as demonstrated, guarantee maximum operational output (viz. maximum-monotonic labor stratification and perfectly qualified independent production).

    % \item Equivalent Operational Output Lemma (EOOL)
    % \begin{enumerate}
    %     \item Now, as WOCA states there are no gains nor losses in production due to partial operational outsourcing, the following must hold, in any labor market, even in a maximally stratified one:
    %     \begin{gather}
    %         w_{q}^{\ell}
    %         \mho_{q}^{\ell}
    %         =
    %         w_q
    %         ,
    %     \end{gather}
    %     where $w_q$ is the aggregate operational output of the
    %     \begin{gather}
    %         l \in \left[
    %         \frac{\ell - 1}{w_q}
    %         ,
    %         \frac{\ell}{w_q}
    %         \right]
    %         ,
    %         \ell \in 
    %         \left\{
    %             1, \dots, w_q
    %         \right\}
    %     \end{gather}
    %     responsability spectrum of $q$'s complex tasks when $w_q$ perfectly qualified employees work independently (IOOL); and $w_{q}^{\ell}\mho_{q}^{\ell}$ is the production of $w_{q}^{\ell}$ employees with at least minimum qualification working in a maximally stratified labor market, so that
    %     \begin{gather}
    %         \sum_{\ell=1}^{w_q}
    %         w_{q}^{\ell}
    %         =
    %         w_q
    %         .
    %     \end{gather}
    %     Or, in other words, operational output is equivalent in any production strategy provided the talent employed is sufficiently qualified.

    %     And, with this, we derive the Proportional Employment Condition (PEC):
    %     \begin{gather}
    %         w_{q}^{\ell}
    %         \mho_{q}^{\ell}
    %         =
    %         w_q
    %         \iff
    %         \frac{
    %             w_{q}^{\ell}
    %         }{
    %             w_q
    %         }
    %         =
    %         \Omega_{q}^{\ell}
    %         \in [0,1]
    %         ,
    %     \end{gather}
    %     which determines the ratio, or proportion, of a particular job subtype in a stratified labor market is exactly the percentage of an operation's total time duration accounted by it. Thus, activities that require more time also require more dedicated employees working on them full-time, and vice-versa.
        
    %     Notice, too, this is the same result we had found in the binary stratification example above (of a labor market with only ``junior'' and ``senior'' employees):
    %     \begin{gather}
    %         \tilde{w}_{q}^{\text{Jr}} 
    %         =
    %         \frac{
    %             w_{q}^{\text{Jr}}
    %         }{
    %             w_{q}^{\text{Jr}} + 
    %             w_{q}^{\text{Sr}}
    %         }
    %         =
    %         \frac{
    %             w_{q}^{\text{Jr}}
    %         }{
    %             w_{q}
    %         }
    %         =
    %         \int_{0}^{\tilde{T}_{q}^{\text{Jr}}}
    %             \text{ta}_q(l)
    %             dl
    %         =
    %         \Omega_{q}^{\text{Jr}}
    %         .
    %     \end{gather}
    %     Therefore, this condition must hold in any efficient labor market, irrespective of its stratification's ``granularity''.
    % \end{enumerate}
    \item 
    Definition of aggregate employability in a maximally and monotonically stratified labor market is: 
    \begin{gather}
        \tilde{W}_{k}
        =
        \sum_{q=1}^{n}
        \tilde{W}_{q}^{k}
        ,
    \end{gather}
    where
    \begin{align}
        \tilde{W}_{q}^{k}
        &=
        \sum_{\ell=1}^{w_q}
        \left[
            h_{q}^{k}
            \geq
            \frac{1}{2}
        \right]
        \left[
            \tilde{T}_{q}^{k}
            \geq 
            \frac{\ell}{w_q}
        \right]
        \tilde{w}_{q}^{\ell}
        \\
        &=
        \sum_{\ell=1}^{w_q}
        \left[
            h_{q}^{k}
            \geq
            \frac{1}{2}
        \right]
        \left[
            \tilde{T}_{q}^{k}
            \geq 
            \frac{\ell}{w_q}
        \right]
        \frac{
            {w}_{q}^{\ell}
        }{
            w_q
        }
    \end{align}
    is partial employability, that is one's employability in a particular occupation $q$ in the labor market.

    \begin{align}
        \tilde{W}_k 
        &=
        \left(
            \frac{1}{W}
        \right)
        \sum_{q=1}^{n}
        % \left[
        %     h_{q}^{k}
        %     \geq
        %     \frac{1}{2}
        % \right]
        \sum_{v=1}^{w_q}
        \left[
            h_{q}^{k}
            \geq
            \frac{1}{2}
        \right]
        \left[
            \tilde{T}_{q}^{k}
            \geq
            \ell_v
        \right]
        \\
        &=
        \left(
            \frac{1}{W}
        \right)
        \sum_{q=1}^{n}
        % \left[
        %     h_{q}^{k}
        %     \geq
        %     \frac{1}{2}
        % \right]
        \sum_{v=1}^{w_q}
        \left[
            h_{q}^{k}
            \geq
            \frac{1}{2}
        \right]
        \left[
            \tilde{T}_{q}^{k}
            \geq
            \text{TA}_{q}^{-1}
            \left(
                \frac{v}{w_q}
                +
                \text{TA}_{q}(0)
            \right)
        \right]
    \end{align}
    
    \item Competitiveness in a maximally-monotonically stratified labor market with irregular responsability ranges (i.e. without partial hiring):
    \begin{align}
        \tilde{\text{vs}}_k 
        &=
        \left(
            \frac{1}{W}
        \right)
        \sum_{q=1}^{n}
        \left[
            \ddot{u}_{k}^{q}
            \geq
            \frac{1}{2}
        \right]
        % \left[
        %     h_{k}^{q}
        %     \geq
        %     \frac{1}{2}
        % \right]
        \sum_{v=1}^{w_k}
        \left[
            h_{k}^{q}
            \geq
            \frac{1}{2}
        \right]
        \left[
            \tilde{T}_{k}^{q}
            \geq
            \ell_v
        \right]
        \\
        &=
        \left(
            \frac{1}{W}
        \right)
        \sum_{q=1}^{n}
        \left[
            \ddot{u}_{k}^{q}
            \geq
            \frac{1}{2}
        \right]
        % \left[
        %     h_{k}^{q}
        %     \geq
        %     \frac{1}{2}
        % \right]
        \sum_{v=1}^{w_k}
        \left[
            h_{k}^{q}
            \geq
            \frac{1}{2}
        \right]
        \left[
            \tilde{T}_{k}^{q}
            \geq
            \text{TA}_{k}^{-1}
            \left(
                \frac{v}{w_k}
                +
                \text{TA}_{k}(0)
            \right)
        \right]
    \end{align}
\end{enumerate}

Competitiveness in a maximally-monotonically stratified labor market with irregular responsability ranges (i.e. without partial hiring):
    \begin{align}
        \tilde{\text{vs}}_k 
        &=
        \left(
            \frac{1}{W}
        \right)
        \sum_{q=1}^{n}
        \left[
            \ddot{u}_{k}^{q}
            \geq
            \frac{1}{2}
        \right]
        % \left[
        %     h_{k}^{q}
        %     \geq
        %     \frac{1}{2}
        % \right]
        \sum_{v=1}^{w_k}
        \left[
            h_{k}^{q}
            \geq
            \frac{1}{2}
        \right]
        \left[
            \tilde{T}_{k}^{q}
            \geq
            \ell_v
        \right]
        \\
        &=
        \left(
            \frac{1}{W}
        \right)
        \sum_{q=1}^{n}
        \left[
            \ddot{u}_{k}^{q}
            \geq
            \frac{1}{2}
        \right]
        % \left[
        %     h_{k}^{q}
        %     \geq
        %     \frac{1}{2}
        % \right]
        \sum_{v=1}^{w_k}
        \left[
            h_{k}^{q}
            \geq
            \frac{1}{2}
        \right]
        \left[
            \tilde{T}_{k}^{q}
            \geq
            \text{TA}_{k}^{-1}
            \left(
                \frac{v}{w_k}
                +
                \text{TA}_{k}(0)
            \right)
        \right]
    \end{align}

    P.S.: SSL
    \begin{align}
        a_{i}^{k} := \frac{l_{i}^{k}}{l_{i}} \in [0,1]
        \\
        \tilde{T}_{i}^{k} 
        &= 
        \int_{0}^{l_i}{
            T(l,l_{i}^{k})
            dl
        }
        \left(
            \int_{0}^{l_i}{
                T(l,l_{i})
                dl
            }
        \right) ^ {-1}
        \\
        &= 
        \left(
            \int_{0}^{l_{i}^{k}}{
                1
                \times
                dl
            }
            +
            \int_{l_{i}^{k}}^{l_i}{
                0
                \times
                dl
            }
        \right)
        \times
        \left(
            \int_{0}^{l_i}{
                1
                \times
                dl
            }
        \right) ^ {-1}
        \\
        &= 
        \frac{l_{i}^{k} - 0}{l_{i} - 0}
        \\
        &= 
        \frac{l_{i}^{k}}{l_{i}}
        \\
        &\therefore
        a_{i}^{k} = \tilde{T}_{i}^{k}
    \end{align}

    P.S.: SCL
    \begin{align}
        \boldsymbol{a_k}
        &:= (a_{1}^{k}, \dots, a_{m}^{k})
        ,
        \boldsymbol{a_q}
        := (a_{1}^{q}, \dots, a_{m}^{q})
        \in [0,1]^{m}
        \
        \forall
        \
        k,q \in \{1, \dots, n\}
        \\
        l_{q}^{k}
        &\leq 
        l_{q}^{q}
        \
        \forall
        \
        k,q \in \{1, \dots, n\}
        \\
        \tilde{T}_{q}^{k}
        &= 
        \int_{0}^{l_{q}^{q}}{
            T(l,l_{q}^{k})
            dl
        }
        \left(
            \int_{0}^{l_{q}^{q}}{
                T(l,l_{q}^{q})
                dl
            }
        \right) ^ {-1}
        \\
        &= 
        \left(
            \int_{0}^{l_{q}^{k}}{
                1
                \times
                dl
            }
            +
            \int_{l_{q}^{k}}^{l_{q}^{q}}{
                0
                \times
                dl
            }
        \right)
        \times
        \left(
            \int_{0}^{l_{q}^{q}}{
                1
                \times
                dl
            }
        \right) ^ {-1}
        \\
        &= 
        \frac{l_{q}^{k} - 0}{l_{q}^{q} - 0}
        \\
        &= 
        \frac{l_{q}^{k}}{l_{q}^{q}}
        \in [0,1]
        \\
        &\because
        \neg
        \
        l_{q}^{k}
        >
        l_{q}^{q}
        \\
        &\because
        l_{q}^{k} = 
        f(
            \boldsymbol{l_k},
            \boldsymbol{l_q}
        ) 
        =
        f(
            \boldsymbol{a_k},
            \boldsymbol{a_q}
        )?
        \in [0,1]
        \
        \forall
        \
        k,q \in \{1, \dots, n\}
        \\
        &\therefore
        \tilde{T}_{q}^{k}
        =
        f(
            \boldsymbol{a_k},
            \boldsymbol{a_q}
        )
    \end{align}
    P.S.: Simplified Employability Theorem/Corollary (SET/SEC)

    % We want to show that, as with the Binary Employability Theorem (BET), so too in a maximally and monotonically stratified labor market, employability is the percentage of an operation's total time duration one is capable of accomplishing. Or, mathematically,
    We want to show that, as with the Binary Employability Theorem (BET), so too in a maximally and monotonically stratified labor market, employability is the percentage of an operation's total time duration one is capable of producing. Or, mathematically,
    % We want to show that, as with the Binary Employability Theorem (BET), so too in a maximally and monotonically stratified labor market, employability is the percentage of an operation's total time duration one's productivity allows them to accomplish. Or, mathematically,
    \begin{gather}
        \tilde{W}_{q}^{k}
        =
        \int_{0}^{\tilde{T}_{q}^{k}}
        \text{ta}(l)
        dl
        :=
        \Omega_{q}^{k}
        \in 
        [0,1]
        \
        \forall
        \
        k,q \in \{1, \dots, n\}
        .
    \end{gather}

    % To prove this result, let us consider what would be the employability of person $k$ if they had exactly the minimum required productivity for every job subtype. So, for instance, when $v = 1$,
    To prove this result, let us, then, first consider what would be the employability of person $k$ if they had exactly the minimum required productivity for every job subtype. So, for instance, when $v = 1$,
    \begin{align}
        \tilde{T}_{q}^{k}
        =
        \ell_1
        =
        \text{TA}^{-1}
        \left(
            \frac{1}{w_q}
            +
            \text{TA}(0)
        \right)
        \implies
        \tilde{W}_{q}^{k}
        =
        \frac{1}{w_q}
        ,
    \end{align}
    as a productivity coefficient of $\tilde{T}_{q}^{k} = \ell_1$ is just enough to be hireable on the easiest job in occupation $q$'s labor market, but not on the second, much less on the remaining, more difficult, positions.
    
    Likewise, for other values of $v$, we have
    \begin{gather}
        \tilde{T}_{q}^{k}
        =
        \ell_2
        =
        \text{TA}^{-1}
        \left(
            \frac{2}{w_q}
            +
            \text{TA}(0)
        \right)
        \implies
        \tilde{W}_{q}^{k}
        =
        \frac{2}{w_q}
        ,
        \\
        \tilde{T}_{q}^{k}
        =
        \ell_3
        =
        \text{TA}^{-1}
        \left(
            \frac{3}{w_q}
            +
            \text{TA}(0)
        \right)
        \implies
        \tilde{W}_{q}^{k}
        =
        \frac{3}{w_q}
        ,
        \\
        \vdots
        \\
        \tilde{T}_{q}^{k}
        =
        \ell_{v}
        =
        \text{TA}^{-1}
        \left(
            \frac{v}{w_q}
            +
            \text{TA}(0)
        \right)
        \implies
        \tilde{W}_{q}^{k}
        =
        \frac{v}{w_q}
        ,
    \end{gather}
    so that we may derive the following pattern for any $v \in \{1, \dots, w_q\}$:
    % \begin{align}
    %     \tilde{T}_{q}^{k}
    %     % =
    %     % \text{TA}^{-1}
    %     % \left(
    %     %     \frac{v}{w_q}
    %     %     +
    %     %     \text{TA}(0)
    %     % \right)
    %     % \implies
    %     % \tilde{W}_{q}^{k}
    %     % =
    %     % \frac{v}{w_q}
    %     % \\
    %     % &\therefore
    %     % \tilde{T}_{q}^{k}
    %     &=
    %     \text{TA}^{-1}
    %     \left(
    %         \tilde{W}_{q}^{k}
    %         +
    %         \text{TA}(0)
    %     \right)
    %     \\
    %     \therefore
    %     \text{TA}
    %     (
    %     % \left(
    %         \tilde{T}_{q}^{k}
    %     )
    %     % \right)
    %     &=
    %     \text{TA}
    %     \left(
    %         \text{TA}^{-1}
    %         \left(
    %             \tilde{W}_{q}^{k}
    %             +
    %             \text{TA}(0)
    %         \right)
    %     \right)
    %     \\
    %     \therefore
    %     \text{TA}
    %     (
    %     % \left(
    %         \tilde{T}_{q}^{k}
    %     )
    %     % \right)
    %     &=
    %     \tilde{W}_{q}^{k}
    %     +
    %     \text{TA}(0)
    %     \\
    %     \therefore
    %     \tilde{W}_{q}^{k}
    %     &=
    %     \text{TA}
    %     (
    %     % \left(
    %         \tilde{T}_{q}^{k}
    %     )
    %     % \right)
    %     -
    %     \text{TA}(0)
    %     =
    %     \int_{0}^{\tilde{T}_{q}^{k}}
    %     \text{ta}(l)
    %     dl
    %     :=
    %     \Omega_{q}^{k}
    %     \in 
    %     [0,1]
    %     ,
    % \end{align}
    \begin{align}
        &\tilde{T}_{q}^{k}
        % =
        % \text{TA}^{-1}
        % \left(
        %     \frac{v}{w_q}
        %     +
        %     \text{TA}(0)
        % \right)
        % \implies
        % \tilde{W}_{q}^{k}
        % =
        % \frac{v}{w_q}
        % \\
        % &\therefore
        % \tilde{T}_{q}^{k}
        =
        \text{TA}^{-1}
        \left(
            \tilde{W}_{q}^{k}
            +
            \text{TA}(0)
        \right)
        \\
        &
        \therefore
        \text{TA}
        (
        % \left(
            \tilde{T}_{q}^{k}
        )
        % \right)
        =
        \text{TA}
        \left(
            \text{TA}^{-1}
            \left(
                \tilde{W}_{q}^{k}
                +
                \text{TA}(0)
            \right)
        \right)
        \\
        &
        \therefore
        \text{TA}
        (
        % \left(
            \tilde{T}_{q}^{k}
        )
        % \right)
        =
        \tilde{W}_{q}^{k}
        +
        \text{TA}(0)
        \\
        &
        \therefore
        \tilde{W}_{q}^{k}
        =
        \text{TA}
        (
        % \left(
            \tilde{T}_{q}^{k}
        )
        % \right)
        -
        \text{TA}(0)
        =
        \int_{0}^{\tilde{T}_{q}^{k}}
        \text{ta}(l)
        dl
        :=
        \Omega_{q}^{k}
        \in 
        [0,1]
        ,
    \end{align}
    as we wanted to show.
    
    However, because $\tilde{T}_{q}^{k} \in [0,1]$ is not discretized as responsability ranges $l \in [l_{v-1},l_{v}], v \in \{1, \dots, w_q\}$ are, and because rational employers do not hire insufficiently qualified employees, we must approximate $\tilde{T}_{q}^{k}$ with the closest 
    \begin{gather}
        \tilde{T}_{q}^{\kappa}
        =
        \left(
            \frac{1}{w_q}
        \right)
        \sum_{v=1}^{w_q}
        \left[
            \tilde{T}_{q}^{k}
            \geq
            l_v
        \right]
    \end{gather}
    productivity estimate, such that $\tilde{T}_{q}^{k} \geq \tilde{T}_{q}^{\kappa}$ and $\tilde{T}_{q}^{k} \approx \tilde{T}_{q}^{\kappa}$, where $
    \tilde{T}_{q}^{\kappa} = \ell_{\kappa} \in \{\ell_0, \dots, \ell_{w_q}\}$ determines the most demanding task for which $k$ is still productive. Therefore, the adjusted coefficient is:
    \begin{gather}
        \tilde{W}_{q}^{k}
        =
        \int_{0}^{\tilde{T}_{q}^{\kappa}}
        \text{ta}(l)
        dl
        :=
        \Omega_{q}^{\kappa}
        \approx
        \int_{0}^{\tilde{T}_{q}^{k}}
        \text{ta}(l)
        dl
        \in 
        [0,1]
        ,
    \end{gather}
    when $w_q$ is large enough.

    Of course, this assumes candidate $k$ is evaluated as ``employable'' in accordance with the hireability statistic
    \begin{gather}
        \left[
            h_{q}^{k}
            \geq
            \frac{1}{2}
        \right]
        ,
    \end{gather}
    which accounts for selection criteria besides minimum required productivity. Thus, a more complete formulation would be:
    \begin{gather}
        \tilde{W}_{q}^{k}
        =
        \left[
            h_{q}^{k}
            \geq
            \frac{1}{2}
        \right]
        \int_{0}^{\tilde{T}_{q}^{\kappa}}
        \text{ta}(l)
        dl
        % \in 
        % [0,1]
        ;
    \end{gather}
    or, in the aggregate form,
    \begin{gather}
        \tilde{W}_k 
        = 
        \left(
            \frac{1}{W}
        \right)
        \sum_{q=1}^{n}
        \left[
            h_{q}^{k}
            \geq
            \frac{1}{2}
        \right]
        \int_{0}^{\tilde{T}_{q}^{\kappa}}
        \text{ta}(l)
        dl
        % \\
        :=
        \left(
            \frac{1}{W}
        \right)
        \sum_{q=1}^{n}
        \left[
            h_{q}^{k}
            \geq
            \frac{1}{2}
        \right]
        \Omega_{q}^{\kappa}
        \\
        % &
        \
        \forall
        \
        \tilde{T}_{q}^{\kappa} \in \{
            \ell_{0}, 
            \dots, 
            \ell_{w_q}
            % 
            % \ell_{q}^{0}, 
            % \dots, 
            % \ell_{q}^{w_q}
            % 
            % \ell_{0}^{q}, 
            % \dots, 
            % \ell_{w_q}^{q}
        \}
        % ,
        ;
        k,q \in \{1, \dots, n\}
        .
    \end{gather}
% endregion

\end{document}