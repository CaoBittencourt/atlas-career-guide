%SECTION: SETUP 
% region: document class 
\documentclass[hidelinks, nonatbib]{elsarticle}
\makeatletter
\let\c@author\relax
\makeatother

% endregion

% region: packages 
\usepackage{amsmath}
\usepackage{amssymb}
\usepackage{amsthm}
\usepackage{float}
\usepackage{tabularray}
\usepackage{enumitem}
\setlist[enumerate]{label*=\arabic*.}
\usepackage[backend=biber, style=apa]{biblatex}
\usepackage{hyperref}

% endregion

% region: source files 
% equations
% macros
% time to complete a task
\newcommand{\ttc}{\text{ttc}}

% time allocation
\newcommand{\ta}{\text{ta}}

% time allocation antiderivative
\newcommand{\TA}{\text{TA}}

% time allocation inverse antiderivative
\newcommand{\invTA}{\text{TA}^{-1}}

% operational output
\newcommand{\oo}{\mho}

% productivity
\newcommand{\pctt}{\tilde{T}}

% minimum required productivity
\newcommand{\mrp}{\underaccent{\bar}{\pctt}}

% sufficiently qualified workers
\newcommand{\wsq}{
    \boldsymbol{\ddot{w}_q}(
        \boldsymbol{w_q}
        ,\boldsymbol{\mrp_q}
        ,\boldsymbol{\pctt_q}
    )
}

% figures
\input{figures.tex}

% tables
\input{tables.tex}

% references
\addbibresource{references.bib}

% endregion

% region: display breaks 
\allowdisplaybreaks

% endregion

% region: title 
\title{
    The Employability Theorem
}
\author{Cao Bittencourt}
\affiliation{{B. Sc. in Economics from EPGE (FGV), RJ, Brazil.}}
\affiliation{{Statistician at Atlas Career Guide Inc., FL, USA.}}
\date{\today}

% endregion

%SECTION: PAPER 
\begin{document}

% region: abstract 
\begin{abstract}
    In this document, the Potential Employability Theorem is demonstrated from a set of fairly tautological axioms, which are presupposed in quantitative career choice and career development methods.
\end{abstract}

\begin{keyword}
    Employability theorem; 
    Career choice; 
    Career development; 
    Vocational choice; 
    Occupational Information Network; 
    O*NET.
\end{keyword}

% endregion

% region: title 
\maketitle

% endregion

% region: introduction 
\section{Proof Plan}
% \begin{enumerate}
%     \item Skill Atomicity Axiom
%     \begin{enumerate}
%         \item It is necessary to define some level of attribute atomicity. Otherwise, we are left with having to quantify only the most fundamental levels of human capacity on ``raw'', or basic, organismic functions on a microscopical, even cellular level, which is obviously not practical, nor useful.
%         \item ACTUALLY, THESE ``ATOMICITY LEVELS'' ARE NOT NECESSARY: THESE ARE MORE ARGUMENTS FOR ATOMICITY, RATHER THAN AXIOMS THEMSELVES. WE ONLY NEED TO DEFINE ``SKILL ATOMICITY''. THE ``ATOMICITY LEVELS'' ARE HERE JUST TO ARGUE FOR ATOMICITY.
%         \item Weak Attribute Atomicity: a number of subattributes per attribute, each with its own truncated difficulty distribution, which may or may not be uniform, that can still be satisfactorily grouped under a single attribute, with a uniform difficulty scale and considerable subattribute overlap.
%         \begin{enumerate}
%             \item Questionnaire respondents intuitively and accurately know how to rate themselves on uniform difficulty scales by ``averaging out'' the implicit subscales which are not part of the questionnaire
%         \end{enumerate}
%         \item Moderate Attribute Atomicity (or Symmetric Subattributes): a number of subattributes per attribute, each with a uniform difficulty distribution, that can still be satisfactorily grouped under a single attribute, with a uniform difficulty scale and considerable subattribute overlap.
%         \item Strong Attribute Atomicity: only attributes, and no subattributes. Each attributes measures only one thing and, therefore, has a uniform difficulty scale as a result.
%     \end{enumerate}
%     \item Skill Truncation Axiom
%     \begin{enumerate}
%         \item There is a maximum difficulty level for each skill at each time and, retrospectively, when $\lim_{t \rightarrow \infty}{l_{i}^{t}} = l_i$, an objective, timeless, maximum difficulty level
%     \end{enumerate}
%     \item Skill Definition (by Task Difficulty)
%     \item Skill Sufficiency Lemma (from SAA, STA and Skill Definition)
%     \begin{enumerate}
%         \item ``Skills are sufficient to perform activities''
%         \item $a_{i}^{k} = l_{i}^{k} / l_i$
%         \item $\tilde{T}_{i}^{k} = \int_{0}^{l_i}{T(l, l_{i}^{k}) dl} / \int_{0}^{l_i}{T(l, l_{i}) dl}$
%         \item Therefore, $a_{i}^{k} = \tilde{T}_{i}^{k}$
%     \end{enumerate}
%     \item Complex Task Definitions
%     \item Skill Composition Lemma (from SSL and Complex Tax Definition)
%     \begin{enumerate}
%         \item ``Skills are sufficient to perform complex activities''
%     \end{enumerate}
%     \item Occupational Reducibility Axiom
% \end{enumerate}

% endregion

Time allocation by difficulty level:
\begin{gather}
    \text{ta}_q(\bar{l}) = 
    \frac{
        \text{ttc}(\bar{l})
    }{
        \int_{0}^{1}{
            \text{ttc}(l)
            dl
        }
    }
\end{gather}

Employment by difficulty level:
\begin{gather}
    w_q(\bar{l}) = 
    w_q 
    \times
    \text{ta}_q(\bar{l})
\end{gather}

Employability per occupation:
\begin{align}
    \tilde{w}_{q}^{k} =
    \tilde{w}_q(l_{q}^{k}) &=
    \int_{0}^{1}
    T(l, l_{q}^{k})
    w_q(l)
    dl
    \\
    &=
    \int_{0}^{1}
    T(l, l_{q}^{k})
    w_q 
    \text{ta}_q(l)
    dl
    \\
    &=
    w_q
    \left[
        \int_{0}^{l_{q}^{k}}
            1 \times
            \text{ta}_q(l)
            dl
        + 
        \int_{l_{q}^{k}}^{1}
            0 \times
            \text{ta}_q(l)
            dl
    \right]
    \\
    &=
    w_q
    \int_{0}^{l_{q}^{k}}
    \text{ta}_q(l)
    dl
\end{align}

And with $
l_{q}^{k}
= \tilde{Y}_{q}^{k}
= \tilde{Y}(\boldsymbol{a_k}, \boldsymbol{a_q})
= Y(\boldsymbol{a_k}, \boldsymbol{a_q}) / Y(\boldsymbol{a_q}, \boldsymbol{a_q})$,
\begin{gather}
    \tilde{w}_{q}^{k} =
    w_q
    \int_{0}^{\tilde{Y}_{q}^{k}}
    \text{ta}_q(l)
    dl
\end{gather}

Aggregate employability (entire economy):
\begin{gather}
    \begin{align}
        \tilde{w}_{k} =
        \tilde{w}_{11}^{k} =
        \tilde{w}_{k}(1, 1)
        % &=
        % \sum_{q=1}^{n}{
        %     [
        %         \tilde{Y}_{q}^{k}
        %         \ddot{\tau}_{q}^{k}
        %         s_{kq}^{\theta}
        %         \geq 0.5
        %     ]
        %     \tilde{w}_{q}^{k}
        % }
        % \\
        &=
        \sum_{q=1}^{n}{
            [
                % \tilde{Y}_{q}^{k}
                % \ddot{\tau}_{q}^{k}
                % s_{kq}^{\theta}
                % \geq 0.5
                \tilde{Y}_{q}^{k}
                \ddot{\tau}_{kq}^{\theta}
                \geq 0.5
            ]
            \tilde{w}_{q}^{k}
        }
        \\
        &=
        \sum_{q=1}^{n}{
            \left(
                [
                % \tilde{Y}_{q}^{k}
                % \ddot{\tau}_{q}^{k}
                % s_{kq}^{\theta}
                % \geq 0.5
                \tilde{Y}_{q}^{k}
                \ddot{\tau}_{kq}^{\theta}
                \geq 0.5
                ]
                \int_{0}^{\tilde{Y}_{q}^{k}}{
                    w_q
                    \text{ta}_q(l)
                    dl
                }
            \right)
        }
        ,
    \end{align}
    \\
    \tilde{Y}_{q}^{k} = 
    \tilde{Y}(
        \boldsymbol{a_k},
        \boldsymbol{a_q}
        ) =
        \frac{
            \prod_{i=1}^{m}{
                {
                    \max(
                        1 + a_{i}^{k},
                        1 + a_{i}^{q}
                    )
                } ^ {
                    \sigma_{i}^{q}
                }
            }
        }{
            \prod_{i=1}^{m}{
                {
                    (
                        1 + a_{i}^{q}
                    )
                } ^ {
                    \sigma_{i}^{q}
                }
            }
        }
        ,\\
        \sigma_{i}^{q} = 
        \frac{
            a_{i}^{q}
        }{
            \sum_{i=1}^{m}{
                a_{i}^{q}
            }
        }
\end{gather}

P.S.: think of a notation for economic taxa / aggregation levels.

Aggregate employability (particular subset of the economy):
\begin{gather}
    \begin{align}
        \tilde{w}_{?!}^{k} =
        \tilde{w}_{k}(?, !)
        % &=
        % \sum_{q=1}^{n}{
        %     [
        %         \tilde{Y}_{q}^{k}
        %         \ddot{\tau}_{q}^{k}
        %         s_{kq}^{\theta}
        %         \geq 0.5
        %     ]
        %     \tilde{w}_{q}^{k}
        % }
        % \\
        &=
        \sum_{q=1}^{n}{
            [q \in ?!]
            [
                % \tilde{Y}_{q}^{k}
                % \ddot{\tau}_{q}^{k}
                % s_{kq}^{\theta}
                % \geq 0.5
                \tilde{Y}_{q}^{k}
                \ddot{\tau}_{kq}^{\theta}
                \geq 0.5
            ]
            \tilde{w}_{q}^{k}
        }
        \\
        &=
        \sum_{q=1}^{n}{
            \left(
                [q \in ?!]
                [
                % \tilde{Y}_{q}^{k}
                % \ddot{\tau}_{q}^{k}
                % s_{kq}^{\theta}
                % \geq 0.5
                \tilde{Y}_{q}^{k}
                \ddot{\tau}_{kq}^{\theta}
                \geq 0.5
                ]
                \int_{0}^{\tilde{Y}_{q}^{k}}{
                    w_q
                    \text{ta}_q(l)
                    dl
                }
            \right)
        }
        ,
    \end{align}
    \\
    \tilde{Y}_{q}^{k} = 
    \tilde{Y}(
        \boldsymbol{a_k},
        \boldsymbol{a_q}
        ) =
        \frac{
            \prod_{i=1}^{m}{
                {
                    \max(
                        1 + a_{i}^{k},
                        1 + a_{i}^{q}
                    )
                } ^ {
                    \sigma_{i}^{q}
                }
            }
        }{
            \prod_{i=1}^{m}{
                {
                    (
                        1 + a_{i}^{q}
                    )
                } ^ {
                    \sigma_{i}^{q}
                }
            }
        }
        ,\\
        \sigma_{i}^{q} = 
        \frac{
            a_{i}^{q}
        }{
            \sum_{i=1}^{m}{
                a_{i}^{q}
            }
        }
\end{gather}

Notation for operation output ($\mho$ from IPA's definition of the ``double-o'' sound, e.g. as in the word ``boot''):
\begin{gather}
    \mho_{q}^{k} = 
    \left\lfloor 
        {Y}_{q}^{k}
    \right\rfloor
\end{gather}
\end{document}