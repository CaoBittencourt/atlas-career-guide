%SECTION: SETUP 
% region: document class 
\documentclass[hidelinks, nonatbib]{elsarticle}
\makeatletter
\let\c@author\relax
\makeatother

% endregion

% region: packages 
\usepackage{amsmath}
\usepackage{amssymb}
\usepackage{amsthm}
\usepackage{float}
\usepackage{tabularray}
\usepackage{enumitem}
\setlist[enumerate]{label*=\arabic*.}
\usepackage[backend=biber, style=apa]{biblatex}
\usepackage{hyperref}

% endregion

% region: source files 
% equations
% equivalence function
\newcommand{\eq}{\text{eq}}

% skill set interchangeability
\newcommand{\esset}{\textit{\ss}}

% skill set interchangeability vector
\newcommand{\essetvec}{\textbf{\textit{\ss}}}

% skill set interchangeability matrix
\newcommand{\Esset}{\textbf{\ss}}

% attribute equivalence
% \newcommand{\aeq}{\textit{ä}}
\newcommand{\aeq}{{\"a}}

% linear-logistic trigonometrically-scaled equivalence estimator
\newcommand{\EquivalenceEstimator}{
% \begin{gather}
%     \label{eqn:equivalence_estimator}
%     \eq(x,M) = 
%         x\left\{
%             1 + M (1 - x) \exp[-
%                 \tan\bigg(
%                     \frac{\pi}{2}
%                     \cos
%                         \bigg(
%                             \frac{\pi}{2} 
%                             % M (1 - M)
%                             x (1 - M)
%                         \bigg)
%                 \bigg)
%                 (x - M)
%             ]
%         \right\} ^ {
%             -\frac{M}{x}
%     },\\
%     x, M \in [0,1].
% \end{gather}
% 
\begin{gather}
    \label{eqn:equivalence_estimator}
    \eq(x,M) = 
        x\left\{
            1 + M (1 - x) \exp[-
                b(x - M)
            ]
        \right\} ^ {
            -\frac{M}{x}
    },\\
    b = 
    \tan\bigg[
        \frac{\pi}{2}
        \cos^{M(1 - M)}
        % \cos^{x(1 - M)}
            \bigg(
                \frac{\pi}{2} 
                % M (1 - M)
                x (1 - M)
            \bigg)
    \bigg],\\
    x, M \in [0,1].
\end{gather}
% 
% \begin{gather}
%     \label{eqn:equivalence_estimator}
%     \eq(x,M) = 
%         x\left\{
%             1 + M (1 - x) \exp[-
%                 \tan\bigg(
%                     \frac{\pi}{2}
%                         \cos^{x(1 - M)}
%                         \bigg(
%                             \frac{\pi}{2} 
%                             x (1 - M)
%                         \bigg)
%                     \bigg)
%                 (x - M)
%             ]
%         \right\} ^ {
%             -\frac{M}{x}
%     },\\
%     x, M \in [0,1].
% \end{gather}
}

% skill set interchangeability
\newcommand{\InterchangeabilityFunction}{
    \begin{gather}
        \esset_{k,q} = 
        \esset(s(\boldsymbol{a_k}, \boldsymbol{a_q}), M) = 
        \eq(s(\boldsymbol{a_k}, \boldsymbol{a_q}), M)
    \end{gather}
}

% interchangeability matrix
\newcommand{\InterchangeabilityMatrix}{
    \begin{gather}
        \Esset = 
        \begin{bmatrix}
            \textit{\ss}_{1,1} & \hdots & \textit{\ss}_{n,1}\\
            \vdots & \ddots & \vdots\\
            \textit{\ss}_{1,n} & \hdots & \textit{\ss}_{n,n}\\
        \end{bmatrix} = 
        \begin{bmatrix}
            1 & \hdots & \textit{\ss}_{k,1} & \hdots & \textit{\ss}_{n,1}\\
            \vdots & \ddots & \vdots & \vdots & \vdots\\
            \textit{\ss}_{1,k} & \hdots & 1 & \hdots & \textit{\ss}_{n,k}\\
            \vdots & \vdots & \vdots & \ddots & \vdots\\
            \textit{\ss}_{1,n} & \hdots & \textit{\ss}_{k,n} & \hdots & 1\\
        \end{bmatrix}\\
    \end{gather}
}

% binary interchangeability (hireability)
\newcommand{\SignalFunction}{
    \begin{gather}
        h_{k,q} = 
        h(\esset_{k,q}) = 
        \begin{cases}
            1, \text{if } \esset_{k,q} \geq 0.5.\\
            0, \text{otherwise.}
        \end{cases}
    \end{gather}
}

% attribute equivalence matrix
\newcommand{\AttributeEquivalence}{
    \begin{gather}
        \aeq_{i}^{k} = 
        \aeq(\boldsymbol{a_k}, M) = 
        \eq\left(
            \frac{
                a_{i}^{k}
            }{
                \max{a_{j}^{k}}
            }
            , M
        \right)
    \end{gather}
}

% figures
\input{figures.tex}

% tables
% General Occupational Statistics
\newcommand{\GeneralOccupationalStatistics}{
    \begin{longtblr}[
        caption = {General Occupational Statistics},
        label = {tab:onet_careers},
        note{1} = {Number of SOC occupations plus variants.},
        note{2} = {Total employment levels in the United States in 2022.},
        note{3} = {Employment-weighted mean wages in the United States in 2022.},
        note{4} = {Total cluster wages as a percentage of total wages in the United States in 2022.},
        remark{Source} = {\cites{bls2022data}{onet2023data}.}
    ]{
        colspec = {X[0.5, l, m]X[0, c, m]X[0.25, c, m]X[0.25, c, m]X[0, c, m]},
        width = \linewidth,
        rowhead = 1,
        rowfoot = 0
    }
        \toprule
            Cluster & N\TblrNote{1} & Employment\TblrNote{2} & Wage\TblrNote{3} & Market Share\TblrNote{4} \\ 
        \midrule
            Business Management \& Administration &  65 & 26,639,371 & \$59,924.00 & 21.05\% \\ 
            Health Science &  99 & 16,856,404 & \$62,753.00 & 13.95\% \\ 
            Marketing &  26 & 13,961,830 & \$45,512.00 & 8.38\% \\ 
            Manufacturing & 138 & 13,644,785 & \$44,169.00 & 7.95\% \\ 
            Hospitality &  42 & 17,416,290 & \$28,965.00 & 6.65\% \\ 
            Transportation, Distribution \& Logistics &  62 & 11,018,200 & \$45,511.00 & 6.61\% \\ 
            Architecture \& Construction &  86 & 9,005,327 & \$53,142.00 & 6.31\% \\ 
            Education \& Training &  64 & 7,266,480 & \$62,621.00 & 6.00\% \\ 
            Finance &  22 & 5,255,514 & \$70,325.00 & 4.87\% \\ 
            Law, Public Safety, Corrections \& Security &  35 & 4,790,620 & \$61,985.00 & 3.92\% \\ 
            Information Technology &  22 & 3,588,065 & \$82,061.00 & 3.88\% \\ 
            Human Services &  38 & 7,823,620 & \$36,809.00 & 3.80\% \\ 
            Science, Technology, Engineering \& Mathematics &  70 & 2,207,815 & \$100,683.00 & 2.93\% \\ 
            Agriculture, Food \& Natural Resources &  40 & 2,140,244 & \$51,134.00 & 1.44\% \\ 
            Arts, Audio/Video Technology \& Communications &  40 & 1,692,270 & \$57,836.00 & 1.29\% \\ 
            Government \& Public Administration &  24 & 1,166,588 & \$62,636.00 & 0.96\% \\
        \bottomrule
    \end{longtblr}
}

% Summary of Occupations
\newcommand{\SummaryOccupations}{
    \begin{table}[H]
        \centering
            \begin{talltblr}[
            caption = {Summary of Sample Occupations},
            label = {tab:summary_occupations},
            note{1} = {``STEM" stands for ``Science, Technology, Engineering, and Mathematics".},
            note{2} = {Total employment levels in the United States in 2022.},
            note{3} = {Median wages in the United States in 2022.},
            remark{Source} = {\cites{bls2022data}{onet2023data}.}
            ]{
            colspec = {ccccc},
            % colspec = {X[c, m]Q[c, m]X[c, m]X[c, m]X[c, m]},
            width = \linewidth
            }
                \toprule
                    SOC & Occupation & Cluster & Employment\TblrNote{2} & Wage\TblrNote{3} \\ 
                \midrule
                    19-2012 & Physicists & STEM\TblrNote{1} & 18,840 & \$152,430.00 \\
                    17-2141 & Mechanical Engineers & STEM\TblrNote{1} & 277,560 & \$95,300.00 \\ 
                    13-2041 & Credit Analysts & Finance & 71,960 & \$77,440.00 \\ 
                    35-9021 & Dishwashers & Hospitality & 431,840 & \$28,130.00 \\ 
                \bottomrule
            \end{talltblr}
    \end{table}
} 

% Mechanical Engineers
\newcommand{\MatchesEngineers}{
    \begin{longtblr}[
        caption = {Best Career Matches -- Mechanical Engineers},
        label = {tab:similarity_engineers},
        remark{Note} = {Similarity scores estimated with Euclidean matching and linear weights.},
        remark{Source} = {Author's calculation based on \cite{onet2023data}.}
    ]{
        colspec = {X[l, m]X[0, c, m]},
        width = \linewidth,
        rowhead = 1,
        rowfoot = 0
    }
        \toprule
            Comparison Occupation & Similarity \\ 
        \midrule
            Mechanical Engineers & 1.00 \\ 
            Aerospace Engineers & 0.90 \\ 
            Marine Engineers and Naval Architects & 0.88 \\ 
            Nuclear Engineers & 0.88 \\ 
            Mechatronics Engineers & 0.87 \\ 
            Microsystems Engineers & 0.86 \\ 
            Photonics Engineers & 0.86 \\ 
            Fuel Cell Engineers & 0.86 \\ 
            Automotive Engineers & 0.86 \\ 
            Materials Engineers & 0.86 \\ 
            Mining and Geological Engineers, Including Mining Safety Engineers & 0.86 \\
        \bottomrule
    \end{longtblr}
}

% Physicists
\newcommand{\MatchesPhysicists}{
    \begin{longtblr}[
        caption = {Best Career Matches -- Physicists},
        label = {tab:similarity_physicists},
        remark{Note} = {Similarity scores estimated with Euclidean matching and linear weights.},
        remark{Source} = {Author's calculation based on \cite{onet2023data}.}
    ]{
        colspec = {X[l, m]X[0, c, m]},
        width = \linewidth,
        rowhead = 1,
        rowfoot = 0
    }
        \toprule
            Comparison Occupation & Similarity \\ 
        \midrule
            Physicists & 1.00 \\ 
            Astronomers & 0.85 \\ 
            Mathematicians & 0.80 \\ 
            Physics Teachers, Postsecondary & 0.80 \\ 
            Materials Scientists & 0.79 \\ 
            Engineering Teachers, Postsecondary & 0.79 \\ 
            Nanosystems Engineers & 0.78 \\ 
            Aerospace Engineers & 0.77 \\ 
            Biochemists and Biophysicists & 0.77 \\ 
            Nuclear Engineers & 0.77 \\ 
            Computer and Information Research Scientists & 0.77 \\ 
        \bottomrule
    \end{longtblr}
}

% Credit Analysts
\newcommand{\MatchesCreditAnalysts}{
    \begin{longtblr}[
        caption = {Best Career Matches -- Credit Analysts},
        label = {tab:similarity_credit_analysts},
        remark{Note} = {Similarity scores estimated with Euclidean matching and linear weights.},
        remark{Source} = {Author's calculation based on \cite{onet2023data}.}
    ]{
        colspec = {X[l, m]X[0, c, m]},
        width = \linewidth,
        rowhead = 1,
        rowfoot = 0
    }
        \toprule
            Comparison Occupation & Similarity \\ 
        \midrule
            Credit Analysts & 1.00 \\ 
            Accountants and Auditors & 0.87 \\ 
            Loan Interviewers and Clerks & 0.86 \\ 
            Budget Analysts & 0.85 \\ 
            Securities, Commodities, and Financial Services Sales Agents & 0.85 \\ 
            Financial Examiners & 0.85 \\ 
            Business Intelligence Analysts & 0.85 \\ 
            Tax Preparers & 0.85 \\ 
            Insurance Underwriters & 0.84 \\ 
            Bookkeeping, Accounting, and Auditing Clerks & 0.84 \\ 
            Tax Examiners and Collectors, and Revenue Agents & 0.83 \\ 
        \bottomrule
    \end{longtblr}
}

% Dishwashers
\newcommand{\MatchesDishwashers}{
    \begin{longtblr}[
        caption = {Best Career Matches -- Dishwashers},
        label = {tab:similarity_dishwashers},
        remark{Note} = {Similarity scores estimated with Euclidean matching and linear weights.},
        remark{Source} = {Author's calculation based on \cite{onet2023data}.}
    ]{
        colspec = {X[l, m]X[0, c, m]},
        width = \linewidth,
        rowhead = 1,
        rowfoot = 0
    }
        \toprule
            Comparison Occupation & Similarity \\ 
        \midrule
            Dishwashers & 1.00 \\ 
            Janitors and Cleaners, Except Maids and Housekeeping Cleaners & 0.88 \\ 
            Maids and Housekeeping Cleaners & 0.88 \\ 
            Pressers, Textile, Garment, and Related Materials & 0.87 \\ 
            Models & 0.87 \\ 
            Postal Service Mail Sorters, Processors, and Processing Machine Operators & 0.87 \\ 
            Sewing Machine Operators & 0.86 \\ 
            Laundry and Dry-Cleaning Workers & 0.86 \\ 
            Orderlies & 0.86 \\ 
            Cutters and Trimmers, Hand & 0.86 \\ 
            Shoe and Leather Workers and Repairers & 0.86 \\
        \bottomrule
    \end{longtblr}
}

% Similarity matrix
\newcommand{\SimilarityMatrix}{
    \begin{table}[H]
        \centering
        \begin{talltblr}[
            caption = {Similarity Matrix},
            label = {tab:similarity_matrix},
            remark{Note 1} = {Similarity scores estimated with Euclidean matching and linear weights.},
            remark{Note 2} = {Comparison occupations on the horizontal axis.},
            remark{Source} = {Author's calculation based on \cite{onet2023data}.}
        ]{
            colspec = {X[c, m]X[c, m]X[c, m]X[c, m]X[c, m]},
            width = \linewidth
        }
            \toprule
                & Mechanical Engineers & Physicists & Credit Analysts & Dishwashers \\ 
            \midrule
                Mechanical Engineers & 1.00 & 0.75 & 0.54 & 0.38 \\ 
                Physicists & 0.80 & 1.00 & 0.57 & 0.34 \\ 
                Credit Analysts & 0.75 & 0.66 & 1.00 & 0.54 \\ 
                Dishwashers & 0.62 & 0.53 & 0.66 & 1.00 \\ 
            \bottomrule
        \end{talltblr}
    \end{table}
}

% Detailed skill sets
\newcommand{\DetailedSkillSets}{
    \begin{longtblr}[
        caption = {Detailed Skill Sets},
        label = {tab:detailed_skill_sets},
        remark{Source} = {\cite{onet2023data}.}
    ]{
        % colspec = {X[1.5, c, m]X[1, c, m]X[1, c, m]X[1, c, m]X[1, c, m]},
        colspec = {X[0.4, l, m]X[0.2, c, m]X[0.2, c, m]X[0.2, c, m]X[0.2, c, m]},
        width = \linewidth,
        rowhead = 1,
        rowfoot = 0
    }
        \toprule
            Competency & Mechanical Engineers & Physicists & Credit Analysts & Dishwashers \\
        \midrule
            Engineering and Technology & 87 & 86 & 13 & 0 \\ 
            Industrial Design & 84 & 45 & 0 & 0 \\ 
            Applied Mathematics & 79 & 94 & 68 & 0 \\ 
            Mechanical & 78 & 28 & 0 & 12 \\ 
            Physics & 73 & 97 & 0 & 0 \\ 
            Oral Comprehension & 70 & 84 & 59 & 32 \\ 
            Reading Comprehension & 70 & 84 & 57 & 30 \\ 
            Oral Expression & 70 & 82 & 59 & 30 \\ 
            Deductive Reasoning & 70 & 79 & 64 & 29 \\ 
            Pure Mathematics & 70 & 79 & 59 & 14 \\ 
            Mathematical Reasoning & 68 & 84 & 59 & 16 \\ 
            Written Comprehension & 68 & 79 & 59 & 29 \\ 
            Complex Problem-Solving & 68 & 70 & 43 & 30 \\ 
            Computers and Electronics & 67 & 83 & 43 & 0 \\ 
            Natural Science & 66 & 82 & 5 & 0 \\ 
            Operations Analysis & 66 & 52 & 34 & 0 \\ 
            Active Learning & 63 & 79 & 52 & 29 \\ 
            Information Ordering & 63 & 71 & 52 & 30 \\ 
            Number Facility & 61 & 79 & 63 & 16 \\ 
            Critical Thinking & 61 & 70 & 57 & 30 \\ 
            Monitoring & 61 & 57 & 43 & 30 \\ 
            Written Expression & 59 & 80 & 57 & 23 \\ 
            Inductive Reasoning & 59 & 79 & 59 & 29 \\ 
            Category Flexibility & 59 & 73 & 50 & 29 \\ 
            Problem Sensitivity & 59 & 70 & 57 & 29 \\ 
            Judgment and Decision & 59 & 64 & 54 & 30 \\ 
            Originality & 57 & 75 & 37 & 25 \\ 
            Writing & 57 & 71 & 55 & 25 \\ 
            Native Language & 57 & 70 & 60 & 38 \\ 
            Active Listening & 57 & 68 & 57 & 32 \\ 
            Near Vision & 57 & 57 & 61 & 34 \\ 
            Administration and Management & 57 & 54 & 42 & 37 \\ 
            Quality Control Analysis & 57 & 46 & 11 & 29 \\ 
            Chemistry & 56 & 61 & 0 & 21 \\ 
            Fluency of Ideas & 55 & 73 & 41 & 27 \\ 
            Speaking & 55 & 68 & 57 & 30 \\ 
            Visualization & 55 & 61 & 25 & 29 \\ 
            Systems Analysis & 55 & 57 & 43 & 16 \\ 
            Systems Evaluation & 55 & 55 & 45 & 16 \\ 
            Speech Recognition & 54 & 57 & 50 & 30 \\ 
            Technology Design & 54 & 48 & 0 & 0 \\ 
            Operations Monitoring & 54 & 34 & 14 & 30 \\ 
            Instructing & 52 & 66 & 34 & 20 \\ 
            Time Management & 52 & 52 & 43 & 30 \\ 
            Troubleshooting & 52 & 29 & 0 & 29 \\ 
            Education and Training & 51 & 72 & 29 & 41 \\ 
            Administrative & 51 & 40 & 59 & 0 \\ 
            Learning Strategies & 50 & 71 & 36 & 18 \\ 
            Selective Attention & 50 & 57 & 41 & 30 \\ 
            Persuasion & 50 & 50 & 36 & 27 \\ 
            Coordination & 50 & 48 & 36 & 30 \\ 
            Production and Processing & 49 & 34 & 29 & 0 \\ 
            Speech Clarity & 48 & 71 & 45 & 27 \\ 
            Flexibility of Closure & 48 & 55 & 45 & 29 \\ 
            Perceptual Speed & 48 & 46 & 45 & 30 \\ 
            Customer and Personal Service & 48 & 37 & 39 & 38 \\ 
            Far Vision & 46 & 52 & 30 & 32 \\ 
            Management of Personnel Resources & 45 & 45 & 27 & 29 \\ 
            Speed of Closure & 43 & 55 & 41 & 29 \\ 
            Negotiation & 43 & 39 & 39 & 29 \\ 
            Social Perceptiveness & 41 & 45 & 41 & 29 \\ 
            Service Orientation & 41 & 43 & 43 & 29 \\ 
            Visual Color Discrimination & 41 & 41 & 21 & 29 \\ 
            Management of Material Resources & 41 & 32 & 13 & 13 \\ 
            Multitasking & 39 & 41 & 29 & 29 \\ 
            Management of Financial Resources & 39 & 30 & 27 & 11 \\ 
            Hearing Sensitivity & 39 & 30 & 16 & 29 \\ 
            Memorization & 37 & 48 & 34 & 21 \\ 
            Auditory Attention & 37 & 32 & 21 & 29 \\ 
            Telecommunications & 37 & 23 & 7 & 0 \\ 
            Equipment Selection & 37 & 21 & 0 & 29 \\ 
            Installation & 37 & 0 & 0 & 0 \\ 
            Building and Construction & 36 & 15 & 16 & 0 \\ 
            Personnel and Human Resources & 35 & 38 & 23 & 0 \\ 
            Public Safety and Security & 35 & 26 & 15 & 26 \\ 
            Programming & 34 & 55 & 14 & 0 \\ 
            Depth Perception & 34 & 30 & 5 & 29 \\ 
            Operation and Control & 34 & 9 & 0 & 30 \\ 
            Sales and Marketing & 33 & 20 & 29 & 0 \\ 
            Finger Dexterity & 32 & 29 & 25 & 34 \\ 
            Law and Government & 32 & 28 & 54 & 28 \\ 
            Reaction Time & 30 & 0 & 0 & 29 \\ 
            Transportation & 28 & 12 & 0 & 33 \\ 
            Communications and Media & 25 & 46 & 23 & 19 \\ 
            Economics and Accounting & 25 & 26 & 74 & 0 \\ 
            Arm-Hand Steadiness & 25 & 0 & 11 & 41 \\ 
            Repairing & 25 & 0 & 0 & 29 \\ 
            Equipment Maintenance & 23 & 0 & 0 & 29 \\ 
            Medicine and Dentistry & 23 & 0 & 0 & 0 \\ 
            Geography & 21 & 15 & 25 & 0 \\ 
            Biology & 20 & 16 & 0 & 0 \\ 
            Trunk Strength & 14 & 14 & 14 & 43 \\ 
            Manual Dexterity & 14 & 0 & 9 & 52 \\ 
            Wrist-Finger Speed & 9 & 0 & 0 & 21 \\ 
            Rate Control & 5 & 0 & 0 & 32 \\ 
            Spatial Orientation & 5 & 0 & 0 & 14 \\ 
            Psychology & 0 & 28 & 10 & 0 \\ 
            Therapy and Counseling & 0 & 21 & 0 & 0 \\ 
            Sociology and Anthropology & 0 & 0 & 16 & 0 \\ 
            History and Archeology & 0 & 0 & 13 & 0 \\ 
            Control Precision & 0 & 0 & 7 & 32 \\ 
            Extent Flexibility & 0 & 0 & 0 & 43 \\ 
            Static Strength & 0 & 0 & 0 & 39 \\ 
            Multilimb Coordination & 0 & 0 & 0 & 37 \\ 
            Stamina & 0 & 0 & 0 & 32 \\ 
            Gross Body Coordination & 0 & 0 & 0 & 30 \\ 
            Speed of Limb Movement & 0 & 0 & 0 & 30 \\ 
            Response Orientation & 0 & 0 & 0 & 29 \\ 
            Dynamic Strength & 0 & 0 & 0 & 25 \\ 
            Gross Body Equilibrium & 0 & 0 & 0 & 21 \\ 
            Glare Sensitivity & 0 & 0 & 0 & 21 \\ 
            Sound Localization & 0 & 0 & 0 & 14 \\ 
            Night Vision & 0 & 0 & 0 & 14 \\ 
            Peripheral Vision & 0 & 0 & 0 & 14 \\ 
            Explosive Strength & 0 & 0 & 0 & 11 \\ 
            Dynamic Flexibility & 0 & 0 & 0 & 0 \\ 
            Foreign Language & 0 & 0 & 0 & 0 \\ 
            Philosophy and Theology & 0 & 0 & 0 & 0 \\ 
            Fine Arts & 0 & 0 & 0 & 0 \\ 
            Food Production & 0 & 0 & 0 & 0 \\
        \bottomrule
    \end{longtblr}
}

% references
\addbibresource{references.bib}

% endregion

% region: display breaks 
\allowdisplaybreaks

% endregion

% region: title 
\title{
    The Employability Theorem
}
\author{Cao Bittencourt}
\affiliation{{B. Sc. in Economics from EPGE (FGV), RJ, Brazil.}}
\affiliation{{Statistician at Atlas Career Guide Inc., FL, USA.}}
\date{\today}

% endregion

%SECTION: PAPER 
\begin{document}

% region: abstract 
\begin{abstract}
    In this document, the Potential Employability Theorem is demonstrated from a set of fairly tautological axioms, which are presupposed in quantitative career choice and career development methods.
\end{abstract}

\begin{keyword}
    Employability theorem; 
    Career choice; 
    Career development; 
    Vocational choice; 
    Occupational Information Network; 
    O*NET.
\end{keyword}

% endregion

% region: title 
\maketitle

% endregion

% region: introduction 
\section{Proof Plan}
% \begin{enumerate}
%     \item Skill Atomicity Axiom
%     \begin{enumerate}
%         \item It is necessary to define some level of attribute atomicity. Otherwise, we are left with having to quantify only the most fundamental levels of human capacity on ``raw'', or basic, organismic functions on a microscopical, even cellular level, which is obviously not practical, nor useful.
%         \item ACTUALLY, THESE ``ATOMICITY LEVELS'' ARE NOT NECESSARY: THESE ARE MORE ARGUMENTS FOR ATOMICITY, RATHER THAN AXIOMS THEMSELVES. WE ONLY NEED TO DEFINE ``SKILL ATOMICITY''. THE ``ATOMICITY LEVELS'' ARE HERE JUST TO ARGUE FOR ATOMICITY.
%         \item Weak Attribute Atomicity: a number of subattributes per attribute, each with its own truncated difficulty distribution, which may or may not be uniform, that can still be satisfactorily grouped under a single attribute, with a uniform difficulty scale and considerable subattribute overlap.
%         \begin{enumerate}
%             \item Questionnaire respondents intuitively and accurately know how to rate themselves on uniform difficulty scales by ``averaging out'' the implicit subscales which are not part of the questionnaire
%         \end{enumerate}
%         \item Moderate Attribute Atomicity (or Symmetric Subattributes): a number of subattributes per attribute, each with a uniform difficulty distribution, that can still be satisfactorily grouped under a single attribute, with a uniform difficulty scale and considerable subattribute overlap.
%         \item Strong Attribute Atomicity: only attributes, and no subattributes. Each attributes measures only one thing and, therefore, has a uniform difficulty scale as a result.
%     \end{enumerate}
%     \item Skill Truncation Axiom
%     \begin{enumerate}
%         \item There is a maximum difficulty level for each skill at each time and, retrospectively, when $\lim_{t \rightarrow \infty}{l_{i}^{t}} = l_i$, an objective, timeless, maximum difficulty level
%     \end{enumerate}
%     \item Skill Definition (by Task Difficulty)
%     \item Skill Sufficiency Lemma (from SAA, STA and Skill Definition)
%     \begin{enumerate}
%         \item ``Skills are sufficient to perform activities''
%         \item $a_{i}^{k} = l_{i}^{k} / l_i$
%         \item $\tilde{T}_{i}^{k} = \int_{0}^{l_i}{T(l, l_{i}^{k}) dl} / \int_{0}^{l_i}{T(l, l_{i}) dl}$
%         \item Therefore, $a_{i}^{k} = \tilde{T}_{i}^{k}$
%     \end{enumerate}
%     \item Complex Task Definitions
%     \item Skill Composition Lemma (from SSL and Complex Tax Definition)
%     \begin{enumerate}
%         \item ``Skills are sufficient to perform complex activities''
%     \end{enumerate}
%     \item Occupational Reducibility Axiom
% \end{enumerate}

% endregion

Time allocation by difficulty level:
\begin{gather}
    \text{ta}_q(\bar{l}) = 
    \frac{
        \text{ttc}(\bar{l})
    }{
        \int_{0}^{1}{
            \text{ttc}(l)
            dl
        }
    }
\end{gather}

Employment by difficulty level:
\begin{gather}
    w_q(\bar{l}) = 
    w_q 
    \times
    \text{ta}_q(\bar{l})
\end{gather}

Employability per occupation:
\begin{align}
    \tilde{w}_{q}^{k} =
    \tilde{w}_q(l_{q}^{k}) &=
    \int_{0}^{1}
    T(l, l_{q}^{k})
    w_q(l)
    dl
    \\
    &=
    \int_{0}^{1}
    T(l, l_{q}^{k})
    w_q 
    \text{ta}_q(l)
    dl
    \\
    &=
    w_q
    \left[
        \int_{0}^{l_{q}^{k}}
            1 \times
            \text{ta}_q(l)
            dl
        + 
        \int_{l_{q}^{k}}^{1}
            0 \times
            \text{ta}_q(l)
            dl
    \right]
    \\
    &=
    w_q
    \int_{0}^{l_{q}^{k}}
    \text{ta}_q(l)
    dl
\end{align}

And with $
l_{q}^{k}
= \tilde{Y}_{q}^{k}
= \tilde{Y}(\boldsymbol{a_k}, \boldsymbol{a_q})
= Y(\boldsymbol{a_k}, \boldsymbol{a_q}) / Y(\boldsymbol{a_q}, \boldsymbol{a_q})$,
\begin{gather}
    \tilde{w}_{q}^{k} =
    w_q
    \int_{0}^{\tilde{Y}_{q}^{k}}
    \text{ta}_q(l)
    dl
\end{gather}

Aggregate employability (entire economy):
\begin{gather}
    \begin{align}
        \tilde{w}_{k} =
        \tilde{w}_{11}^{k} =
        \tilde{w}_{k}(1, 1)
        % &=
        % \sum_{q=1}^{n}{
        %     [
        %         \tilde{Y}_{q}^{k}
        %         \ddot{\tau}_{q}^{k}
        %         s_{kq}^{\theta}
        %         \geq 0.5
        %     ]
        %     \tilde{w}_{q}^{k}
        % }
        % \\
        &=
        \sum_{q=1}^{n}{
            [
                % \tilde{Y}_{q}^{k}
                % \ddot{\tau}_{q}^{k}
                % s_{kq}^{\theta}
                % \geq 0.5
                \tilde{Y}_{q}^{k}
                \ddot{\tau}_{kq}^{\theta}
                \geq 0.5
            ]
            \tilde{w}_{q}^{k}
        }
        \\
        &=
        \sum_{q=1}^{n}{
            \left(
                [
                % \tilde{Y}_{q}^{k}
                % \ddot{\tau}_{q}^{k}
                % s_{kq}^{\theta}
                % \geq 0.5
                \tilde{Y}_{q}^{k}
                \ddot{\tau}_{kq}^{\theta}
                \geq 0.5
                ]
                \int_{0}^{\tilde{Y}_{q}^{k}}{
                    w_q
                    \text{ta}_q(l)
                    dl
                }
            \right)
        }
        ,
    \end{align}
    \\
    \tilde{Y}_{q}^{k} = 
    \tilde{Y}(
        \boldsymbol{a_k},
        \boldsymbol{a_q}
        ) =
        \frac{
            \prod_{i=1}^{m}{
                {
                    \max(
                        1 + a_{i}^{k},
                        1 + a_{i}^{q}
                    )
                } ^ {
                    \sigma_{i}^{q}
                }
            }
        }{
            \prod_{i=1}^{m}{
                {
                    (
                        1 + a_{i}^{q}
                    )
                } ^ {
                    \sigma_{i}^{q}
                }
            }
        }
        ,\\
        \sigma_{i}^{q} = 
        \frac{
            a_{i}^{q}
        }{
            \sum_{i=1}^{m}{
                a_{i}^{q}
            }
        }
\end{gather}

P.S.: think of a notation for economic taxa / aggregation levels.

Aggregate employability (particular subset of the economy):
\begin{gather}
    \begin{align}
        \tilde{w}_{?!}^{k} =
        \tilde{w}_{k}(?, !)
        % &=
        % \sum_{q=1}^{n}{
        %     [
        %         \tilde{Y}_{q}^{k}
        %         \ddot{\tau}_{q}^{k}
        %         s_{kq}^{\theta}
        %         \geq 0.5
        %     ]
        %     \tilde{w}_{q}^{k}
        % }
        % \\
        &=
        \sum_{q=1}^{n}{
            [q \in ?!]
            [
                % \tilde{Y}_{q}^{k}
                % \ddot{\tau}_{q}^{k}
                % s_{kq}^{\theta}
                % \geq 0.5
                \tilde{Y}_{q}^{k}
                \ddot{\tau}_{kq}^{\theta}
                \geq 0.5
            ]
            \tilde{w}_{q}^{k}
        }
        \\
        &=
        \sum_{q=1}^{n}{
            \left(
                [q \in ?!]
                [
                % \tilde{Y}_{q}^{k}
                % \ddot{\tau}_{q}^{k}
                % s_{kq}^{\theta}
                % \geq 0.5
                \tilde{Y}_{q}^{k}
                \ddot{\tau}_{kq}^{\theta}
                \geq 0.5
                ]
                \int_{0}^{\tilde{Y}_{q}^{k}}{
                    w_q
                    \text{ta}_q(l)
                    dl
                }
            \right)
        }
        ,
    \end{align}
    \\
    \tilde{Y}_{q}^{k} = 
    \tilde{Y}(
        \boldsymbol{a_k},
        \boldsymbol{a_q}
        ) =
        \frac{
            \prod_{i=1}^{m}{
                {
                    \max(
                        1 + a_{i}^{k},
                        1 + a_{i}^{q}
                    )
                } ^ {
                    \sigma_{i}^{q}
                }
            }
        }{
            \prod_{i=1}^{m}{
                {
                    (
                        1 + a_{i}^{q}
                    )
                } ^ {
                    \sigma_{i}^{q}
                }
            }
        }
        ,\\
        \sigma_{i}^{q} = 
        \frac{
            a_{i}^{q}
        }{
            \sum_{i=1}^{m}{
                a_{i}^{q}
            }
        }
\end{gather}

Notation for operation output ($\mho$ from IPA's definition of the ``double-o'' sound, e.g. as in the word ``boot''):
\begin{gather}
    \mho_{q}^{k} = 
    \left\lfloor 
        {Y}_{q}^{k}
    \right\rfloor
\end{gather}
\end{document}