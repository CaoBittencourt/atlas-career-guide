% region: environments 
% definition
\theoremstyle{definition}
\newtheorem{definition}{Definition}

% subdefinition
\theoremstyle{definition}
\newtheorem{subdefinition}{Definition}[definition]

% axiom
\theoremstyle{definition}
\newtheorem{axiom}{Axiom}

% lemma
\newtheorem{lemma}{Lemma}

% theorem
\newtheorem{theorem}{Theorem}

% potential employability theorem
\newtheorem*{petheorem}{Potential Employability Theorem}

% corollary
\newtheorem{corollary}{Corollary}
% \newtheorem{corollary}{Corollary}[theorem]

% endregion

% region: attribute definition 1 
\newcommand{\SklTasks}{
    \label{skl_tasks}

    % statement
    \begin{definition}[Skill]
        A professional attribute, competency, or skill, of a person $k$ can be conceptualized as a cummulative sum of successes on
        binary outcome variables representing tasks of progressive difficulty which require only that skill:
        \begin{gather}
            a_{i}^{k} = \sum_{l=0}^{l_i}
            T_{i_{l}}^{k}
            ,
        \end{gather}
        where
        \begin{gather}
            T_{i_{l}}^{k} =
            \begin{cases}
                1, \ \text{if $k$ succedes in a task $T_{i}^{l}$ of difficulty level $l$;} \\
                0, \ \text{otherwise.}
            \end{cases}
        \end{gather}
        
        Or, more rigorously,
        \begin{gather}
            a_{i}^{k} = \sum_{l=0}^{l_i}
            T(l, l_{i}^{k})
            ,
        \end{gather}
        where
        \begin{gather}
            T(l, l_{i}^{k}) =
            T_{i_{l}}^{k} =
            \begin{cases}
                1, \ l \leq l_{i}^{k}; \\
                0, \ \text{otherwise.}
            \end{cases}
        \end{gather}
        and $l_{i}^{k} \in [0, l_i]$ is the maximum difficulty level on which $k$ still succeedes. Thus, we can define a person $k$'s skill level in an attribute $i$ as the sum
        of their $T_{i_l}^{k}$ successful trials on a $T_{i} = \{T_{i}^{0}, \dots, T_{i}^{l_i}\}$ set of tasks of increasing
        difficulty.

        Furthermore, as we assume scales are truncated (i.e. there is a maximum
        difficulty level $l_i$, and a trivial difficulty level, which has to be zero),
        we can also interpret $a_{i}^{k}$ as the \textit{portion} of tasks one is able
        to accomplish out of all difficulty levels for that skill. By normalizing $l_i$
        to $100$, for example, we have:
        \begin{enumerate}
            \item $a_{i}^{k} = 0 \iff k$ cannot perform even the most basic of attribute $i$'s tasks;
            \item $a_{i}^{k} = 10 \iff k$ can perform only the bottom 10\% of attribute $i$'s tasks, but nothing more;
            \item $a_{i}^{k} = 50 \iff k$ can perform the easiest half of attribute $i$'s tasks, but not the most difficult half;
            \item $a_{i}^{k} = 100 \iff k$ can perform all of attribute $i$'s tasks.
        \end{enumerate}

        Finally, we can define $a_{i}^{k}$ for a continuum of task difficulty $l \in [0,1]$:
        \begin{gather}
            a_{i}^{k} =
            \int_{0}^{1}{
            T(l, l_{i}^{k})
            dl
            }
            ,
        \end{gather}
        where $T(l, l_{i}^{k})$ is defined as before.
    \end{definition}
}

% endregion

% region: attribute definition 2 
\newcommand{\SklDifficulty}{
    \label{skl_difficulty}

    % statement
    \begin{definition}[Skill]
        Alternatively, we can think of a person $k$'s professional attribute, competency, or skill, as the difficulty of the most difficult task they can accomplish, normalized by the difficulty of the most objectively difficult task of that particular skill:
        \begin{gather}
            a_{i}^{k} =
            \frac{
                l_{i}^{k}
            }{
                l_{i}
            }
            ,
        \end{gather}
        which we normalize by setting $l_{i} = 1$, so that
        \begin{gather}
            a_{i}^{k} =
            \frac{
                l_{i}^{k}
            }{
                1
            }
            =
            l_{i}^{k}
            ,
        \end{gather}
        and $l_{i}^{k} \in [0,1]$. With this normalization, example interpretations of $a_{i}^{k}$ are:
        \begin{enumerate}
            \item $a_{i}^{k} = 0 \iff k$ cannot perform even the most basic of attribute $i$'s tasks;
            \item $a_{i}^{k} = 0.10 \iff k$ can only perform tasks of up to 10\% the difficulty of attribute $i$'s most difficult task, but nothing more;
            \item $a_{i}^{k} = 0.50 \iff k$ can perform tasks of up to half the difficulty of attribute $i$'s most difficult task, but nothing more;
            \item $a_{i}^{k} = 1 \iff k$ can perform all of attribute $i$'s tasks.
        \end{enumerate}

        This is, perhaps, the most natural conceptual model for understanding
        competencies, as, generally, it is more intuitive to think of skill as the
        maximum of one's capacity, rather than the portion of tasks one could
        potentially accomplish.

        But, again, because we assume scales to be truncated, this latter interpretation
        actually implies and is implied by the former. For if a task is of the same
        difficulty as another, then they are just as difficult in relation to that
        skill's most difficult task (i.e. they require the same percentage of the
        scale's upper limit to be performed), and, likewise, are also included in the
        same difficulty ``bracket" (i.e. they are equivalent to the same skill test in
        the aggregate binary outcome interpretation), and, therefore, pressupose the
        same $a_{i}^{k}$ skill level.

        Of course, this equivalence is quite trivial, given that
        \begin{gather}
            \int_{0}^{1}{
            T(l, l_{i}^{k})
            dl
            }
            =
            1 \times \int_{0}^{l_{i}^{k}}{dl} +
            0 \times \int_{l_{i}^{k}}^{1}{dl}
            =
            l_{i}^{k} - 0
            =
            \frac{l_{i}^{k}}{1}
            =
            a_{i}^{k}
            .
        \end{gather}

        This means the percentage of a skill's tasks one can accomplish is also
        the difficulty of the most difficult task one can accomplish relative to that
        skill's most difficult task.

        So, however one decides to interpret skill levels, the conclusion remains the
        same: to be skilled in an attribute is to be able to perform the activities
        associated with that attribute. Put simply, the capacity to act follows virtue,
        for virtue is, itself, the capacity to act.
    \end{definition}
}

% endregion

% region: skill sufficiency lemma (ssl) 
% skills are sufficient to accomplish tasks
\newcommand{\SSL}{
    \label{ssl}

    % preamble
    Now, even though these results are basically tautological, they are still important to guide our intuition. In fact, our first insight towards the Employability Theorem, namely the Skill Sufficiency Lemma (SSL), follows directly from the definitions above.
    % statement
    \begin{lemma}[Skill Sufficiency Lemma]
        According to the SSL, skills are necessary and sufficient to accomplish tasks.
        In particular, to have a skill level of $a_{i}^{k} \in [0,1]$ in attribute $i$ is a necessary and sufficient condition for one to be capable of accomplishing the easier $a_{i}^{k}$ portion of that attribute's tasks.
    \end{lemma}

    % proof
    \begin{proof}
        By definition,
        \begin{gather}
            T(l, l_{i}^{k}) =
            \begin{cases}
                1, \ l \leq l_{i}^{k}; \\
                0, \ \text{otherwise.}
            \end{cases}
        \end{gather}
        is a binary indicator of person $k$'s ability to accomplish a task of difficulty $l \in [0,1]$ which requires only attribute $i$.
        
        With this,
        \begin{gather}
            \tilde{T}_{i}^{k}
            =
            \int_{0}^{1}{
            T(l, l_{i}^{k})
            dl
            }
        \end{gather}
        is the percentage of tasks requiring only attribute $i$ that $k$ can accomplish.

        But both equivalent definitions of $k$'s skill level in attribute $i$, are
        \begin{gather}
            a_{i}^{k} = 
            \int_{0}^{1}{
            T(l, l_{i}^{k})
            dl
            }
            =
            l_{i}^{k}
            ,
        \end{gather}
        which is precisely the $\tilde{T}_{i}^{k}$ aggregation of $T(l, l_{i}^{k})$ in the $[0,1]$ interval.

        Therefore, having a skill level of $a_{i}^{k}$ is a necessary and sufficient condition to be capable of accomplishing the easier $a_{i}^{k}$ portion of that attribute's tasks:
        \begin{gather}
            a_{i}^{k} =
            \tilde{T}_{i}^{k}
            \iff
            \int_{0}^{1}{
            T(l, l_{i}^{k})
            dl
            } =
            \int_{0}^{1}{
            T(l, l_{i}^{k})
            dl
            }
            .
        \end{gather}
    \end{proof}
}

% endregion

% region: complex task definition 
\newcommand{\ComplexTask}{
    \label{complex_task}

    % statement
    \begin{definition}[Complex Task]
        A task is said to be complex if it requires more than one attribute to be accomplished. More precisely, $T_{ij}^{l}$ is a complex task of attributes $i$ and $j$, if its binary outcome indicator is of the form
        \begin{gather}
            T(l, l_{ij}^{k}) =
            \begin{cases}
                1, \ l \leq l_{ij}^{k}; \\
                0, \ \text{otherwise.}
            \end{cases}
        \end{gather}

        \begin{subdefinition}[Weak Complexity]
            dsds
        \end{subdefinition}
        
        \begin{subdefinition}[Moderate Complexity]
            dsds
        \end{subdefinition}

        \begin{subdefinition}[Strong Complexity]
            dsds
        \end{subdefinition}

        and
        \begin{gather}
            l_{ij}^{k} = f(l_{i}^{k}, l_{j}^{k})
        \end{gather}
        meets the following criteria:
        \begin{gather}
            \frac{
                \partial
                f(l_{i}^{k}, l_{j}^{k})
            }{
                \partial 
                l_{i}^{k}
            } \geq 0
            ;\\
            \frac{
                \partial
                f(l_{i}^{k}, l_{j}^{k})
            }{
                \partial 
                l_{j}^{k}
            } \geq 0
            ;\\
            \lim_{
                l_{i}^{k} \rightarrow 0
            }{
                f(l_{i}^{k}, l_{j}^{k})
            } = 0
            ;\\
            \lim_{
                l_{j}^{k} \rightarrow 0
            }{
                f(l_{i}^{k}, l_{j}^{k})
            } = 0
            .
        \end{gather}
        
        Thus, a person $k$'s capacity to perform a complex task is weakly increasing on their capacity to perform the simple tasks of its required attributes, and goes to zero when they are unskilled in at least one of these attributes.

        Additionaly, this definition can be generalized for any complex task $T_{q}^{l}$ of $m$ attributes, requiring an entire skill set $\boldsymbol{a_q} = (a_{1}^{q}, \dots, a_{m}^{q})$ to be accomplished:
        \begin{gather}
            T(l, l_{q}^{k}) =
            \begin{cases}
                1, \ l \leq l_{q}^{k}; \\
                0, \ \text{otherwise.}
            \end{cases}
        \end{gather}
        where
        \begin{gather}
            l_{q}^{k} = f(l_{1}^{k}, \dots, l_{m}^{k})
        \end{gather}
        and 
        \begin{gather}
            \frac{
                \partial
                f(l_{1}^{k}, \dots, l_{m}^{k})
            }{
                \partial 
                l_{i}^{k}
            } \geq 0
            \
            \forall 
            \
            i \in \{1, \dots, m\}
            ;\\
            \lim_{
                l_{i}^{k} \rightarrow 0
            }{
                f(l_{1}^{k}, \dots, l_{m}^{k})
            } = 0
            \
            \forall 
            \
            i \in \{1, \dots, m\}
            .
        \end{gather}

        This means a task is only strictly \textit{complex} if it cannot be reduced to any proper subset of the attributes that it requires. For example, a task of the form
        \begin{gather}
            T(l, l_{ij}^{k}) =
            \begin{cases}
                1, \ l \leq l_{i}^{k} + l_{j}^{k}; \\
                0, \ \text{otherwise.}
            \end{cases}
        \end{gather}
        is not, strictly speaking, a complex task, because if person $k$ is only skilled in either attribute $i$ or $j$, they can still accomplish the task. In particular, if $k$ has precisely zero capacity in either attribute, then $T_{ij}^{l}$ collapses to unidimensional, or simple, tasks $T_{i}^{l}$ when
        \begin{gather}
            T(l, l_{ij}^{k}) =
            \begin{cases}
                1, \ l \leq l_{i}^{k} + 0; \\
                0, \ \text{otherwise.}
            \end{cases}
            = T(l, l_{i}^{k})
            ,\\
        \end{gather}
        or $T_{j}^{l}$ when
        \begin{gather}
            T(l, l_{ij}^{k}) =
            \begin{cases}
                1, \ l \leq 0 + l_{j}^{k}; \\
                0, \ \text{otherwise.}
            \end{cases}
            = T(l, l_{j}^{k})
            ,
        \end{gather}
        in which case $T_{ij}^{l}$ is not \textit{really} complex, but rather a convolution of simple tasks. Notice this has nothing to do with whether $l_{ij}^{k} \in [0,1]$. In fact, if $l_{ij}^{k}$ is given by, say, $l_{ij}^{k} = \sigma_{i} l_{i}^{k} + (1 - \sigma_{i})l_{j}^{k}$, with $\sigma_{i}, l_{i}^{k}, l_{j}^{k} \in [0,1]$, and $k$ has zero capacity in either attribute,
        \begin{gather}
            T(l, l_{ij}^{k}) =
            \begin{cases}
                1, \ l \leq \sigma_{i} l_{i}^{k} + 0; \\
                0, \ \text{otherwise.}
            \end{cases}
            = T(l, \sigma_{i} l_{i}^{k})
        \end{gather}
        and 
        \begin{gather}
            T(l, l_{ij}^{k}) =
            \begin{cases}
                1, \ l \leq 0 + (1 - \sigma_{i}) l_{j}^{k}; \\
                0, \ \text{otherwise.}
            \end{cases}
            = T(l, (1 - \sigma_{i}) l_{j}^{k})
        \end{gather}
        are still simple, not complex tasks, even if they are ``harder versions'' of $T_{i}^{l}$ and $T_{j}^{l}$, respectively, as the maximum difficulty level $k$ can accomplish of task $T_{ij}^{l}$ is only limited by both attributes, not necessarily determined by them -- hence why this formulation is better thought of as two separate tasks.
        
        Of course, this does not mean a degree of substitution between attributes completely invalidates a task's complexity. It only means a complex task must require all of its attributes at \textit{some} level, even if its functional form allows for substitution. However, if it does not require all of its attributes, then it should be broken down into other simple and/or complex tasks. In Economic terms, this is equivalent to saying that competencies, when used to perform complex tasks, are at least weakly complementary.
    \end{definition}
}

% endregion

% region: skill composition axiom / lemma (sca) 
% skills are composable to accomplish complex tasks
\newcommand{\SCA}{
    \label{sca}

    % statement
    \begin{axiom}[Skill Composition Axiom]
        The Skill Composition Axiom (SCA) states that skills are weakly composable to accomplish complex tasks. More precisely, let $T_{ij}^{l}$ be an activity of difficulty level $l$ that requires both attributes $i$ and $j$ (i.e. $T_{ij}^{l}$ is a weakly complex task). With this, we define that any rational and sufficiently qualified economic agent can naturally ``piece together'', that is \textit{compose}, attributes $i$ and $j$ to accomplish the $T_{ij}^{l}$ complex task. Or, mathematically,
        \begin{gather}
            \tilde{T}_{ij}^{k} = 
            \int_{0}^{1}{
                T(l, l_{ij}^{k})
                dl
            }
        \end{gather}
        is the easier portion $k$ can accomplish in a $l \in [0,1]$ continuum of difficulty levels of complex tasks requiring attributes $i$ and $j$, given
        \begin{gather}
            T(l, l_{ij}^{k}) =
            \begin{cases}
                1, \ l \leq l_{ij}^{k}; \\
                0, \ \text{otherwise.}
            \end{cases}
        \end{gather}
        and
        \begin{gather}
            l_{ij}^{k} = 
            f(
                l_{i}^{k},
                l_{j}^{k}
            )
            =
            f(
                a_{i}^{k},
                a_{j}^{k}
            )
            \\
            [f(
                a_{i}^{k},
                a_{j}^{k}
            ) \in [0,1]]
            ,
        \end{gather}
        with
        \begin{gather}
            \frac{
                \partial
                f(
                a_{i}^{k},
                a_{j}^{k}
            )
            }{
                \partial 
                a_{i}^{k}
            } \geq 0
            ,\\
            \frac{
                \partial
                f(
                a_{i}^{k},
                a_{j}^{k}
            )
            }{
                \partial 
                a_{j}^{k}
            } \geq 0
            .
        \end{gather}

        In addition, let us assume that attributes are not perfect complements when performing a complex task, thus allowing for some substitution.
        actually 
        \begin{gather}
            Y_{kq} = 
            \prod_{i=1}^{m}{
                \min(
                    (1 + a_{i}^{k}),
                    (1 + a_{i}^{q})
                ) ^ {\sigma_{i}^{q}}
            }
        \end{gather}
        violates the definition of complex task aggregation function.
        perhaps define task complexity with 
        \begin{gather}
            \frac{
                \partial ^ 2 {f(\boldsymbol{l^{k}})}
            }{
                \partial 
                l_{i}^{k}
                \partial
                l_{j}^{k}
            } \geq 0
        \end{gather}

        task complexity (weak definition): 
        \begin{gather}
            \frac{
                \partial ^ 2 {f(\boldsymbol{l^{k}})}
            }{
                \partial 
                l_{i}^{k}
                \partial
                l_{j}^{k}
            } \geq 0
        \end{gather}
        
        task complexity (moderate definition): 
        \begin{gather}
            \lim_{
                l_{i}^{k} \rightarrow 0
            }{
                f(l_{1}^{k}, \dots, l_{m}^{k})
            } = 0
            \
            \forall 
            \
            i \in \{1, \dots, m\}
        \end{gather}

        task complexity (strong definition): $l_{q}^{k} = \min(l_{1}^{k}, \dots, l_{m}^{k})$
    \end{axiom}

}

% endregion

% region: occupational reducibility axiom (ora) 
% occupations can be reduced to their tasks
\newcommand{\ORA}{
    \label{ora}

    % statement
    \begin{axiom}[Occupational Reducibility Axiom]
        Occupations can be reduced to their tasks.
    \end{axiom}

}

% endregion

% region: occupational weak complexity axiom (oca) 
% all of an occupation's tasks can be thought of as only
% one, indivisible, holistic task, requiring their entire skill set
% we call this an occupation's "operation"
% all an occupation does is its "operation"
% define an aggregation function to compose skills an an "operation"
% operation production: evaluated aggregation function
% max production: occupation's evaluated aggregation function
% productivity: production truncated and normalized by max production
\newcommand{\OCA}{
    \label{oca}

    % statement
    \begin{axiom}[Occupational Complexity Axiom]
        All of an occupation's tasks can be thought of as one indivisible task, which mobilizes their entire skill set. We call this ``holistic task'' an occupation's \textit{operation}.
    \end{axiom}

}

% endregion

% region: occupational composition lemma (ocl) 
% sca, ora, oca => ocl
% skill sets are composable to accomplish occupations' "operations"
\newcommand{\OCL}{
    \label{ocl}

    % statement
    \begin{lemma}[Occupational Composition Lemma]
        Skill sets are composable to accomplish occupations' operations.
    \end{lemma}

    % proof
    \begin{proof}

    \end{proof}

}

% endregion

% region: rationality axiom (ra) 
% employers are rational
% employers will only pay for employees to work on tasks they can accomplish
% if required, employers will outsource parts of an occupation's operation
\newcommand{\RA}{
    \label{ra}

    % statement
    \begin{axiom}[Rationality Axiom]
        Employers are rational and will only pay for employees to work on tasks they can accomplish. Additionaly, employers will outsource parts of an occupation's operation if their employees cannot accomplish the entire operation.
    \end{axiom}

}

% endregion

% region: hireability axiom (ha) 
% hiring is done by evaluating a hireability statistic
\newcommand{\HA}{
    \label{ha}

    % statement
    \begin{axiom}[Hireability Axiom]
        Any rational employer hires employees by evaluating a hireability statistic, which quantifies potential employees' skill set similarity with an occupation, their educational attainment, and years of experience.
    \end{axiom}

}

% endregion

% region: occupational essence lemma (oel) 
% ra, ha => oel
% all of an occupation's job posts are essentially the same
% because all job posts require sufficient similarity,
% and because employers will not hire anyone who is not hireable,
% in accordance with the hireability statistic,
% therefore, all of an occupation's job posts, despite their accidents,
% are sufficiently recognized as a true instance of that occupation
\newcommand{\OEL}{
    \label{oel}

    % statement
    \begin{lemma}[Occupational Essence Lemma]
        All of an occupation's job posts are essentially the same.
    \end{lemma}

    % proof
    \begin{proof}

    \end{proof}

}

% endregion

% region: labor market completeness axiom (lmca) 
% there is sufficient talent in the labor market to outsource difficult tasks
% this is tautological, because the labor market is already producing its current output
\newcommand{\LMCA}{
    \label{lmca}

    % statement
    \begin{axiom}[Labor Market Completeness Axiom]
        There is sufficient talent in the labor market to outsource difficult tasks.
    \end{axiom}

}

% endregion

% region: task duration axiom (tda) 
% define a time to completion function: task difficulty => task duration
\newcommand{\TDA}{
    \label{tda}

    % statement
    \begin{axiom}[Task Duration Axiom]
        More difficult tasks and operations require more time to complete than easier tasks and operations.
    \end{axiom}

}

% endregion

% region: potential employability theorem (pet) 
% the potential employability of a person in a particular occupation
% is the percentage of that occupation's operation total time duration
% that their skill set allows them to accomplish
\newcommand{\PET}{
    \label{pet}

    % statement
    \begin{petheorem}
        The potential employability of a person in a particular occupation is the percentage of that occupation's operation total time duration that their skill set allows them to accomplish.
    \end{petheorem}

    % proof
    \begin{proof}

    \end{proof}

}

% endregion

% region: aggregate employability corollary (aec) 
% a person's employability in a certain subset of the labor market
% is calculated by their average employability on that subset of the labor market,
% weighted by each occupation's employment levels
\newcommand{\AEC}{
    \label{aec}

    % statement
    \begin{corollary}[Aggregate Employability Corollary]
        A person's employability in a certain subset of the labor market is calculated by their average employability on that subset of the labor market, weighted by each occupation's employment levels.
    \end{corollary}

    % proof
    \begin{proof}

    \end{proof}

}

% endregion

% region: occupational competitiveness corollary (occ) 
\newcommand{\OCC}{
    \label{occ}

    % statement
    \begin{corollary}[Occupational Competitiveness Corollary]

    \end{corollary}

    % proof
    \begin{proof}

    \end{proof}

}

% endregion

% region: aggregate competitiveness corollary (acc) 
\newcommand{\ACC}{
    \label{acc}

    % statement
    \begin{corollary}[Aggregate Competitiveness Corollary]

    \end{corollary}

    % proof
    \begin{proof}

    \end{proof}

}

% endregion

% region: dsds 
% occupational divisibility remark 1 (odr 1) on occ
% one can only accomplish the part of an occupation's operation
% that one is sufficiently qualified for
% one can accomplish parts of an operation even if productivity < 1

% occupational divisibility remark 2 (odr 2) on occ
% if one cannot accomplish parts of an occupation's operation,
% then these harder levels of the operation can be outsourced
% though a holistic operation is not divisible, difficulty levels are

% endregion
