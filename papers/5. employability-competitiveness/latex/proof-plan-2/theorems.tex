% definition
\theoremstyle{definition}
\newtheorem{definition}{Definition}

% subdefinition
\theoremstyle{definition}
\newtheorem{subdefinition}{Definition}[definition]

% axiom
\theoremstyle{definition}
\newtheorem{axiom}{Axiom}

% subaxiom
\theoremstyle{axiom}
\newtheorem{subaxiom}{Axiom}[axiom]

% lemma
\newtheorem{lemma}{Lemma}

% theorem
\newtheorem{theorem}{Theorem}

%  employability theorem
\newtheorem*{etheorem}{Employability Theorem}

% corollary
\newtheorem{corollary}{Corollary}
% \newtheorem{corollary}{Corollary}[theorem]

% % region: proof layout 
% \newcommand{\ProofLayout}{
%     \label{proof_layout}

%     % statement
%     \begin{theorem}
%         dsds
%     \end{theorem}

%     % proof
%     \begin{proof}
%         independent perfectly qualified worker
%         \begin{gather}
%             w_{q}^{\ell} = \Omega_{q}^{\ell}
%         \end{gather}
        
%         output of independent perfectly qualified worker
%         \begin{gather}
%             w_{q}^{\ell} 
%             \mho_{q}^{\ell}
%             = 
%             \Omega_{q}^{\ell}
%             \mho_{q}^{\ell}
%             = 1
%         \end{gather}
        
%         aggregate output of $w_q$ independent perfectly qualified workers
%         \begin{gather}
%             w_{q}^{\ell} 
%             \mho_{q}^{\ell}
%             w_q
%             = 
%             \Omega_{q}^{\ell}
%             \mho_{q}^{\ell}
%             w_q
%             = 1 \times w_q
%             = w_q
%         \end{gather}

%         [operational equilibrium lemma/tautology: demonstrate the current employed talent is sufficient to produce the current operational output]

%         max-monotonic labor stratification is the optimal production strategy when there could be skill differences among workers in the labor market:
%         \begin{align}
%             &
%             \mathbb{E}[
%                 \mho_{q}^{\text{IP}}
%                 \
%                 |
%                 \
%                 \mathbb{E}[
%                     \tilde{T}_{q}^{k}
%                     \in
%                     [0,1]
%                     \
%                     \forall
%                     \
%                     k,q
%                     \in
%                     \{1, \dots, n\}
%                 ]
%             ]
%             \\&
%             \leq
%             \mathbb{E}[
%                 \mho_{q}^{\text{MS}}
%                 \
%                 |
%                 \
%                 \mathbb{E}[
%                     \tilde{T}_{q}^{k}
%                     \in
%                     [0,1]
%                     \
%                     \forall
%                     \
%                     k,q
%                     \in
%                     \{1, \dots, n\}
%                 ]
%             ]
%             \\&
%             =
%             \mathbb{E}[
%                 \mho_{q}^{\text{IP}}
%                 \
%                 |
%                 \
%                 \mathbb{E}[
%                     \tilde{T}_{q}^{k}
%                     = 1
%                     \
%                     \forall
%                     \
%                     k,q
%                     \in
%                     \{1, \dots, n\}
%                 ]
%             ]
%         \end{align}

%         operational output in a maximally-monotonically stratified labor market is equivalent to independent perfectly qualified production:
%         \begin{gather}
%             w_{q}^{\ell}
%             \mho_{q}^{\ell}
%             = w_q
%             \iff
%             \frac{
%                 w_{q}^{\ell}
%             }{
%                 w_q
%             }
%             =
%             \Omega_{q}^{\ell}
%             \in [0,1]
%         \end{gather}

%         employability definition in a maximally stratified labor market:
%         \begin{align}
%             \tilde{W}_{q}^{k}
%             &= 
%             \sum_{\ell=1}^{w_q}{
%                 \left[
%                     h_{q}^{k} \geq \frac{1}{2}
%                 \right]
%                 \left[
%                     \tilde{T}_{q}^{k} \geq \frac{\ell}{w_q}
%                 \right]
%                 \tilde{w}_{q}^{\ell}
%             }
%             \\
%             &=
%             \sum_{\ell=1}^{w_q}{
%                 \left[
%                     h_{q}^{k} \geq \frac{1}{2}
%                 \right]
%                 \left[
%                     \tilde{T}_{q}^{k} \geq \frac{\ell}{w_q}
%                 \right]
%                 \frac{w_{q}^{\ell}}{w_q}   
%             }
%         \end{align}

%         therefore,
%         \begin{gather}
%             \tilde{W}_{q}^{k}
%             =
%             \sum_{\ell=1}^{w_q}{
%                 \left[
%                     h_{q}^{k} \geq \frac{1}{2}
%                 \right]
%                 \left[
%                     \tilde{T}_{q}^{k} \geq \frac{\ell}{w_q}
%                 \right]
%                 \Omega_{q}^{\ell}
%             }
%         \end{gather}

%         aggregate employability:
%         \begin{gather}
%             \tilde{W}_{k}
%             =
%             \sum_{q=1}^{n}{
%                 \tilde{W}_{q}^{k}
%             }
%         \end{gather}

%         therefore,
%         \begin{gather}
%             \tilde{W}_{k}
%             =
%             \sum_{q=1}^{n}
%             \sum_{\ell=1}^{w_q}{
%                 \left[
%                     h_{q}^{k} \geq \frac{1}{2}
%                 \right]
%                 \left[
%                     \tilde{T}_{q}^{k} \geq \frac{\ell}{w_q}
%                 \right]
%                 \Omega_{q}^{\ell}
%             }
%         \end{gather}

%         aggregate competitiveness definition:
%         \begin{gather}
%             \text{vs}_{k}
%             =
%             \sum_{q=1}^{n}{
%                 \left[
%                     \ddot{u}_{k}^{q}
%                     \geq
%                     \frac{1}{2}
%                 \right]
%                 \tilde{W}_{k}^{q}
%             }
%         \end{gather}

%         therefore,
%         \begin{gather}
%             \text{vs}_{k}
%             =
%             \sum_{q=1}^{n}{
%                 \left[
%                     \ddot{u}_{k}^{q}
%                     \geq
%                     \frac{1}{2}
%                 \right]
%                 \sum_{\ell=1}^{w_k}{
%                     \left[
%                         h_{k}^{q} \geq \frac{1}{2}
%                     \right]
%                     \left[
%                         \tilde{T}_{k}^{q} \geq \frac{\ell}{w_k}
%                     \right]
%                     \Omega_{k}^{\ell}
%                 }
%             }
%         \end{gather}
%     \end{proof}
% }

% % endregion

% % region: employability theorem (et) 
% % the  employability of a person in a particular occupation
% % is the percentage of that occupation's operation total time duration
% % that their skill set allows them to accomplish
% \newcommand{\ET}{
%     \label{et}

%     % statement
%     \begin{etheorem}
%         The  employability of a person in a particular occupation is the percentage of that occupation's operation total time duration that their skill set allows them to accomplish.
%     \end{etheorem}

%     % proof
%     \begin{proof}
%         Employment by difficulty level:
%         \begin{gather}
%             w_q(\bar{l}) = 
%             w_q 
%             \times
%             \text{ta}_q(\bar{l})
%         \end{gather}
        
%         Employability per occupation:
%         \begin{align}
%             w_{q}^{k} &=
%             [
%                 \tilde{Y}_{q}^{k}
%                 \ddot{\tau}_{q}^{k}
%                 s_{kq}^{\theta}
%                 \geq 0.5
%                 % \tilde{Y}_{q}^{k}
%                 % \ddot{\tau}_{kq}^{\theta}
%                 % \geq 0.5
%             ]
%             \times
%             \int_{0}^{1}
%             T(l, l_{q}^{k})
%             w_q(l)
%             dl
%             \\
%             &=
%             [
%                 \tilde{Y}_{q}^{k}
%                 \ddot{\tau}_{q}^{k}
%                 s_{kq}^{\theta}
%                 \geq 0.5
%                 % \tilde{Y}_{q}^{k}
%                 % \ddot{\tau}_{kq}^{\theta}
%                 % \geq 0.5
%             ]
%             \times
%             \int_{0}^{1}
%             % T(l, l_{q}^{k})
%             [l_{q}^{k} \geq l]
%             % [l \leq l_{q}^{k}]
%             w_q 
%             \text{ta}_q(l)
%             dl
%             \\
%             &=
%             [
%                 \tilde{Y}_{q}^{k}
%                 \ddot{\tau}_{q}^{k}
%                 s_{kq}^{\theta}
%                 \geq 0.5
%                 % \tilde{Y}_{q}^{k}
%                 % \ddot{\tau}_{kq}^{\theta}
%                 % \geq 0.5
%             ]
%             \times
%             w_q
%             \left(
%                 \int_{0}^{l_{q}^{k}}
%                     1 \times
%                     \text{ta}_q(l)
%                     dl
%                 + 
%                 \int_{l_{q}^{k}}^{1}
%                     0 \times
%                     \text{ta}_q(l)
%                     dl
%             \right)
%             \\
%             &=
%             [
%                 \tilde{Y}_{q}^{k}
%                 \ddot{\tau}_{q}^{k}
%                 s_{kq}^{\theta}
%                 \geq 0.5
%                 % \tilde{Y}_{q}^{k}
%                 % \ddot{\tau}_{kq}^{\theta}
%                 % \geq 0.5
%             ]
%             \times
%             w_q
%             \int_{0}^{l_{q}^{k}}
%             \text{ta}_q(l)
%             dl
%         \end{align}
        
%         And with $
%         l_{q}^{k}
%         = \tilde{Y}_{q}^{k}
%         = \tilde{Y}(\boldsymbol{a_k}, \boldsymbol{a_q})
%         = Y(\boldsymbol{a_k}, \boldsymbol{a_q}) / Y(\boldsymbol{a_q}, \boldsymbol{a_q})$,
%         \begin{gather}
%             w_{q}^{k} =
%             \left\lfloor
%                 [
%                     \tilde{Y}_{q}^{k}
%                     \ddot{\tau}_{q}^{k}
%                     s_{kq}^{\theta}
%                     \geq 0.5
%                     % \tilde{Y}_{q}^{k}
%                     % \ddot{\tau}_{kq}^{\theta}
%                     % \geq 0.5
%                 ]
%                 \times
%                 w_q
%                 \int_{0}^{\tilde{Y}_{q}^{k}}
%                 \text{ta}_q(l)
%                 dl
%             \right\rfloor
%         \end{gather}
%     \end{proof}

% }

% % endregion

% % region: aggregate employability corollary (aec) 
% % a person's employability in a certain subset of the labor market
% % is calculated by their average employability on that subset of the labor market,
% % weighted by each occupation's employment levels
% \newcommand{\AEC}{
%     \label{aec}

%     % statement
%     \begin{corollary}[Aggregate Employability Corollary]
%         A person's employability in a certain subset of the labor market is calculated by their average employability on that subset of the labor market, weighted by each occupation's employment levels.
%     \end{corollary}

%     % proof
%     \begin{proof}
%         Aggregate employability (entire economy):
%         \begin{gather}
%             \begin{align}
%                 \tilde{w}_{k} 
%                 % = \tilde{w}_{11}^{k}
%                 % = \tilde{w}_{k}(1, 1)
%                 % &=
%                 % \sum_{q=1}^{n}{
%                 %     [
%                 %         \tilde{Y}_{q}^{k}
%                 %         \ddot{\tau}_{q}^{k}
%                 %         s_{kq}^{\theta}
%                 %         \geq 0.5
%                 %     ]
%                 %     w_{q}^{k}
%                 % }
%                 % \\
%                 &=
%                 \frac{1}{W}
%                 \sum_{q=1}^{n}{
%                     w_{q}^{k}
%                 }
%                 \\
%                 &=
%                 \frac{1}{W}
%                 \sum_{q=1}^{n}{
%                     \left\lfloor
%                         [
%                         \tilde{Y}_{q}^{k}
%                         \ddot{\tau}_{q}^{k}
%                         s_{kq}^{\theta}
%                         \geq 0.5
%                         % \tilde{Y}_{q}^{k}
%                         % \ddot{\tau}_{kq}^{\theta}
%                         % \geq 0.5
%                         ]
%                         \int_{0}^{\tilde{Y}_{q}^{k}}{
%                             w_q
%                             \text{ta}_q(l)
%                             dl
%                         }
%                     \right\rfloor
%                 }
%                 ,
%             \end{align}
%             \\
%             W = \sum_{k=1}^{n}{w_k}
%             ,\\
%             \tilde{Y}_{q}^{k} = 
%             \tilde{Y}(
%                 \boldsymbol{a_k},
%                 \boldsymbol{a_q}
%                 ) =
%                 \frac{
%                     \prod_{i=1}^{m}{
%                         {
%                             \max(
%                                 1 + a_{i}^{k},
%                                 1 + a_{i}^{q}
%                             )
%                         } ^ {
%                             \sigma_{i}^{q}
%                         }
%                     }
%                 }{
%                     \prod_{i=1}^{m}{
%                         {
%                             (
%                                 1 + a_{i}^{q}
%                             )
%                         } ^ {
%                             \sigma_{i}^{q}
%                         }
%                     }
%                 }
%                 ,\\
%                 \sigma_{i}^{q} = 
%                 \frac{
%                     a_{i}^{q}
%                 }{
%                     \sum_{i=1}^{m}{
%                         a_{i}^{q}
%                     }
%                 }
%         \end{gather}
        
%         P.S.: think of a notation for economic taxa / aggregation levels.
        
%         Aggregate employability (particular subset of the economy):
%         \begin{gather}
%             \begin{align}
%                 \tilde{w}_{?!}^{k} =
%                 \tilde{w}_{k}(?, !)
%                 % &=
%                 % \sum_{q=1}^{n}{
%                 %     [
%                 %         \tilde{Y}_{q}^{k}
%                 %         \ddot{\tau}_{q}^{k}
%                 %         s_{kq}^{\theta}
%                 %         \geq 0.5
%                 %     ]
%                 %     w_{q}^{k}
%                 % }
%                 % \\
%                 &=
%                 \sum_{q=1}^{n}{
%                     [q \in ?!]
%                     [
%                         % \tilde{Y}_{q}^{k}
%                         % \ddot{\tau}_{q}^{k}
%                         % s_{kq}^{\theta}
%                         % \geq 0.5
%                         \tilde{Y}_{q}^{k}
%                         \ddot{\tau}_{kq}^{\theta}
%                         \geq 0.5
%                     ]
%                     w_{q}^{k}
%                 }
%                 \\
%                 &=
%                 \sum_{q=1}^{n}{
%                     \left\lfloor
%                         [q \in ?!]
%                         [
%                         % \tilde{Y}_{q}^{k}
%                         % \ddot{\tau}_{q}^{k}
%                         % s_{kq}^{\theta}
%                         % \geq 0.5
%                         \tilde{Y}_{q}^{k}
%                         \ddot{\tau}_{kq}^{\theta}
%                         \geq 0.5
%                         ]
%                         \int_{0}^{\tilde{Y}_{q}^{k}}{
%                             w_q
%                             \text{ta}_q(l)
%                             dl
%                         }
%                     \right\rfloor
%                 }
%                 ,
%             \end{align}
%             \\
%             \tilde{Y}_{q}^{k} = 
%             \tilde{Y}(
%                 \boldsymbol{a_k},
%                 \boldsymbol{a_q}
%                 ) =
%                 \frac{
%                     \prod_{i=1}^{m}{
%                         {
%                             \max(
%                                 1 + a_{i}^{k},
%                                 1 + a_{i}^{q}
%                             )
%                         } ^ {
%                             \sigma_{i}^{q}
%                         }
%                     }
%                 }{
%                     \prod_{i=1}^{m}{
%                         {
%                             (
%                                 1 + a_{i}^{q}
%                             )
%                         } ^ {
%                             \sigma_{i}^{q}
%                         }
%                     }
%                 }
%                 ,\\
%                 \sigma_{i}^{q} = 
%                 \frac{
%                     a_{i}^{q}
%                 }{
%                     \sum_{i=1}^{m}{
%                         a_{i}^{q}
%                     }
%                 }
%         \end{gather}
%     \end{proof}

% }

% % endregion

% % region: occupational competitiveness corollary (occ) 
% \newcommand{\OCC}{
%     \label{occ}

%     % statement
%     \begin{corollary}[Occupational Competitiveness Corollary]

%     \end{corollary}

%     % proof
%     \begin{proof}

%     \end{proof}

% }

% % endregion

% % region: aggregate competitiveness corollary (acc) 
% \newcommand{\ACC}{
%     \label{acc}

%     % statement
%     \begin{corollary}[Aggregate Competitiveness Corollary]

%     \end{corollary}

%     % proof
%     \begin{proof}

%     \end{proof}

% }

% % endregion

% % region: proportionate compensation corollary (pcc) 
% \newcommand{\PCC}{
%     \label{pcc}

%     % statement
%     \begin{corollary}[Proportionate Compensation Corollary]
%         Because employers are rational, employees will be payed the exact portion of the market value they generate with their work.
%     \end{corollary}

%     % proof
%     \begin{proof}

%     \end{proof}

% }

% % endregion

% % region: dsds 
% % occupational divisibility remark 1 (odr 1) on occ
% % one can only accomplish the part of an occupation's operation
% % that one is sufficiently qualified for
% % one can accomplish parts of an operation even if productivity < 1

% % occupational divisibility remark 2 (odr 2) on occ
% % if one cannot accomplish parts of an occupation's operation,
% % then these harder levels of the operation can be outsourced
% % though a holistic operation is not divisible, difficulty levels are

% % endregion
