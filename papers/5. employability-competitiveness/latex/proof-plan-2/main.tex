%SECTION: SETUP 
% region: document class 
\documentclass[hidelinks, nonatbib]{elsarticle}
\makeatletter
\let\c@author\relax
\makeatother

% endregion

% region: packages 
\usepackage{amsmath}
\usepackage{amssymb}
\usepackage{amsthm}
\usepackage{float}
\usepackage{tabularray}
\usepackage{enumitem}
\setlist[enumerate]{label*=\arabic*.}
\usepackage[backend=biber, style=apa]{biblatex}
\usepackage{hyperref}

% endregion

% region: source files 
% theorems
% region: environments 
% skill sufficiency axiom (ssa)
% skills are sufficient to accomplish tasks
\newtheorem{axiom}{Skill Sufficiency Axiom}

% skill composition axiom (sca)
% skills are composable to accomplish complex tasks
\newtheorem{axiom}{Skill Composition Axiom}

% occupational reducibility axiom (ora)
% occupations can be reduced to their tasks
\newtheorem{axiom}{Occupational Reducibility Axiom}

% occupational complexity axiom (oca)
% all of an occupation's tasks can be thought of as only
% one, indivisible, holistic task, requiring their entire skill set
% we call this an occupation's "operation"
% all an occupation does is its "operation"
% define an aggregation function to compose skills an an "operation"
% operation production: evaluated aggregation function
% max production: occupation's evaluated aggregation function
% productivity: production truncated and normalized by max production
\newtheorem{axiom}{Occupational Complexity Axiom}

% occupational composition corollary (occ)
% sca, ora, oca => occ
% skill sets are composable to accomplish occupations' "operations"
\newtheorem{corollary}{Occupational Composition Corollary}

% occupational divisibility remark 1 (odr 1) on occ
% one can only accomplish the part of an occupation's operation
% that one is sufficiently qualified for
% one can accomplish parts of an operation even if productivity < 1

% occupational divisibility remark 2 (odr 2) on occ
% if one cannot accomplish parts of an occupation's operation,
% then these harder levels of the operation can be outsourced
% though a holistic operation is not divisible, difficulty levels are

% rationality axiom (ra)
% employers are rational
% employers will only pay for employees to work on tasks they can accomplish
% if required, employers will outsource parts of an occupation's operation
\newtheorem{axiom}{Rationality Axiom}

% hireability axiom (ha)
% hiring is done by evaluating a hireability statistic
\newtheorem{axiom}{Hireability Axiom}

% occupational essence corollary (oec)
% all of an occupation's job posts are essentially the same
% because all job posts require sufficient similarity,
% and because employers will not hire anyone who is not hireable,
% in accordance with the hireability statistic,
% therefore, all of an occupation's job posts, despite their accidents,
% are sufficiently recognized as a true instance of that occupation
\newtheorem{axiom}{Occupational Essence Corollary}

% labor market completeness axiom (lmca)
% there is sufficient talent in the labor market to outsource difficult tasks
% this is tautological, because the labor market is already producing its current output
\newtheorem{axiom}{Labor Market Completeness Axiom}

% task duration axiom (tda)
% define a time to completion function: task difficulty => task duration
\newtheorem{axiom}{Task Duration Axiom}

% potential employability theorem (pet)
% the potential employability of a person in a particular occupation
% is the percentage of that occupation's operation total time duration
% that their skill set allows them to accomplish
\newtheorem{theorem}{Potential Employability Theorem}

% endregion

% region: definitions 

% endregion


% equations
% equivalence function
\newcommand{\eq}{\text{eq}}

% skill set interchangeability
\newcommand{\esset}{\textit{\ss}}

% skill set interchangeability vector
\newcommand{\essetvec}{\textbf{\textit{\ss}}}

% skill set interchangeability matrix
\newcommand{\Esset}{\textbf{\ss}}

% attribute equivalence
% \newcommand{\aeq}{\textit{ä}}
\newcommand{\aeq}{{\"a}}

% linear-logistic trigonometrically-scaled equivalence estimator
\newcommand{\EquivalenceEstimator}{
% \begin{gather}
%     \label{eqn:equivalence_estimator}
%     \eq(x,M) = 
%         x\left\{
%             1 + M (1 - x) \exp[-
%                 \tan\bigg(
%                     \frac{\pi}{2}
%                     \cos
%                         \bigg(
%                             \frac{\pi}{2} 
%                             % M (1 - M)
%                             x (1 - M)
%                         \bigg)
%                 \bigg)
%                 (x - M)
%             ]
%         \right\} ^ {
%             -\frac{M}{x}
%     },\\
%     x, M \in [0,1].
% \end{gather}
% 
\begin{gather}
    \label{eqn:equivalence_estimator}
    \eq(x,M) = 
        x\left\{
            1 + M (1 - x) \exp[-
                b(x - M)
            ]
        \right\} ^ {
            -\frac{M}{x}
    },\\
    b = 
    \tan\bigg[
        \frac{\pi}{2}
        \cos^{M(1 - M)}
        % \cos^{x(1 - M)}
            \bigg(
                \frac{\pi}{2} 
                % M (1 - M)
                x (1 - M)
            \bigg)
    \bigg],\\
    x, M \in [0,1].
\end{gather}
% 
% \begin{gather}
%     \label{eqn:equivalence_estimator}
%     \eq(x,M) = 
%         x\left\{
%             1 + M (1 - x) \exp[-
%                 \tan\bigg(
%                     \frac{\pi}{2}
%                         \cos^{x(1 - M)}
%                         \bigg(
%                             \frac{\pi}{2} 
%                             x (1 - M)
%                         \bigg)
%                     \bigg)
%                 (x - M)
%             ]
%         \right\} ^ {
%             -\frac{M}{x}
%     },\\
%     x, M \in [0,1].
% \end{gather}
}

% skill set interchangeability
\newcommand{\InterchangeabilityFunction}{
    \begin{gather}
        \esset_{k,q} = 
        \esset(s(\boldsymbol{a_k}, \boldsymbol{a_q}), M) = 
        \eq(s(\boldsymbol{a_k}, \boldsymbol{a_q}), M)
    \end{gather}
}

% interchangeability matrix
\newcommand{\InterchangeabilityMatrix}{
    \begin{gather}
        \Esset = 
        \begin{bmatrix}
            \textit{\ss}_{1,1} & \hdots & \textit{\ss}_{n,1}\\
            \vdots & \ddots & \vdots\\
            \textit{\ss}_{1,n} & \hdots & \textit{\ss}_{n,n}\\
        \end{bmatrix} = 
        \begin{bmatrix}
            1 & \hdots & \textit{\ss}_{k,1} & \hdots & \textit{\ss}_{n,1}\\
            \vdots & \ddots & \vdots & \vdots & \vdots\\
            \textit{\ss}_{1,k} & \hdots & 1 & \hdots & \textit{\ss}_{n,k}\\
            \vdots & \vdots & \vdots & \ddots & \vdots\\
            \textit{\ss}_{1,n} & \hdots & \textit{\ss}_{k,n} & \hdots & 1\\
        \end{bmatrix}\\
    \end{gather}
}

% binary interchangeability (hireability)
\newcommand{\SignalFunction}{
    \begin{gather}
        h_{k,q} = 
        h(\esset_{k,q}) = 
        \begin{cases}
            1, \text{if } \esset_{k,q} \geq 0.5.\\
            0, \text{otherwise.}
        \end{cases}
    \end{gather}
}

% attribute equivalence matrix
\newcommand{\AttributeEquivalence}{
    \begin{gather}
        \aeq_{i}^{k} = 
        \aeq(\boldsymbol{a_k}, M) = 
        \eq\left(
            \frac{
                a_{i}^{k}
            }{
                \max{a_{j}^{k}}
            }
            , M
        \right)
    \end{gather}
}

% figures
\input{figures.tex}

% tables
% General Occupational Statistics
\newcommand{\GeneralOccupationalStatistics}{
    \begin{longtblr}[
        caption = {General Occupational Statistics},
        label = {tab:onet_careers},
        note{1} = {Number of SOC occupations plus variants.},
        note{2} = {Total employment levels in the United States in 2022.},
        note{3} = {Employment-weighted mean wages in the United States in 2022.},
        note{4} = {Total cluster wages as a percentage of total wages in the United States in 2022.},
        remark{Source} = {\cites{bls2022data}{onet2023data}.}
    ]{
        colspec = {X[0.5, l, m]X[0, c, m]X[0.25, c, m]X[0.25, c, m]X[0, c, m]},
        width = \linewidth,
        rowhead = 1,
        rowfoot = 0
    }
        \toprule
            Cluster & N\TblrNote{1} & Employment\TblrNote{2} & Wage\TblrNote{3} & Market Share\TblrNote{4} \\ 
        \midrule
            Business Management \& Administration &  65 & 26,639,371 & \$59,924.00 & 21.05\% \\ 
            Health Science &  99 & 16,856,404 & \$62,753.00 & 13.95\% \\ 
            Marketing &  26 & 13,961,830 & \$45,512.00 & 8.38\% \\ 
            Manufacturing & 138 & 13,644,785 & \$44,169.00 & 7.95\% \\ 
            Hospitality &  42 & 17,416,290 & \$28,965.00 & 6.65\% \\ 
            Transportation, Distribution \& Logistics &  62 & 11,018,200 & \$45,511.00 & 6.61\% \\ 
            Architecture \& Construction &  86 & 9,005,327 & \$53,142.00 & 6.31\% \\ 
            Education \& Training &  64 & 7,266,480 & \$62,621.00 & 6.00\% \\ 
            Finance &  22 & 5,255,514 & \$70,325.00 & 4.87\% \\ 
            Law, Public Safety, Corrections \& Security &  35 & 4,790,620 & \$61,985.00 & 3.92\% \\ 
            Information Technology &  22 & 3,588,065 & \$82,061.00 & 3.88\% \\ 
            Human Services &  38 & 7,823,620 & \$36,809.00 & 3.80\% \\ 
            Science, Technology, Engineering \& Mathematics &  70 & 2,207,815 & \$100,683.00 & 2.93\% \\ 
            Agriculture, Food \& Natural Resources &  40 & 2,140,244 & \$51,134.00 & 1.44\% \\ 
            Arts, Audio/Video Technology \& Communications &  40 & 1,692,270 & \$57,836.00 & 1.29\% \\ 
            Government \& Public Administration &  24 & 1,166,588 & \$62,636.00 & 0.96\% \\
        \bottomrule
    \end{longtblr}
}

% Summary of Occupations
\newcommand{\SummaryOccupations}{
    \begin{table}[H]
        \centering
            \begin{talltblr}[
            caption = {Summary of Sample Occupations},
            label = {tab:summary_occupations},
            note{1} = {``STEM" stands for ``Science, Technology, Engineering, and Mathematics".},
            note{2} = {Total employment levels in the United States in 2022.},
            note{3} = {Median wages in the United States in 2022.},
            remark{Source} = {\cites{bls2022data}{onet2023data}.}
            ]{
            colspec = {ccccc},
            % colspec = {X[c, m]Q[c, m]X[c, m]X[c, m]X[c, m]},
            width = \linewidth
            }
                \toprule
                    SOC & Occupation & Cluster & Employment\TblrNote{2} & Wage\TblrNote{3} \\ 
                \midrule
                    19-2012 & Physicists & STEM\TblrNote{1} & 18,840 & \$152,430.00 \\
                    17-2141 & Mechanical Engineers & STEM\TblrNote{1} & 277,560 & \$95,300.00 \\ 
                    13-2041 & Credit Analysts & Finance & 71,960 & \$77,440.00 \\ 
                    35-9021 & Dishwashers & Hospitality & 431,840 & \$28,130.00 \\ 
                \bottomrule
            \end{talltblr}
    \end{table}
} 

% Mechanical Engineers
\newcommand{\MatchesEngineers}{
    \begin{longtblr}[
        caption = {Best Career Matches -- Mechanical Engineers},
        label = {tab:similarity_engineers},
        remark{Note} = {Similarity scores estimated with Euclidean matching and linear weights.},
        remark{Source} = {Author's calculation based on \cite{onet2023data}.}
    ]{
        colspec = {X[l, m]X[0, c, m]},
        width = \linewidth,
        rowhead = 1,
        rowfoot = 0
    }
        \toprule
            Comparison Occupation & Similarity \\ 
        \midrule
            Mechanical Engineers & 1.00 \\ 
            Aerospace Engineers & 0.90 \\ 
            Marine Engineers and Naval Architects & 0.88 \\ 
            Nuclear Engineers & 0.88 \\ 
            Mechatronics Engineers & 0.87 \\ 
            Microsystems Engineers & 0.86 \\ 
            Photonics Engineers & 0.86 \\ 
            Fuel Cell Engineers & 0.86 \\ 
            Automotive Engineers & 0.86 \\ 
            Materials Engineers & 0.86 \\ 
            Mining and Geological Engineers, Including Mining Safety Engineers & 0.86 \\
        \bottomrule
    \end{longtblr}
}

% Physicists
\newcommand{\MatchesPhysicists}{
    \begin{longtblr}[
        caption = {Best Career Matches -- Physicists},
        label = {tab:similarity_physicists},
        remark{Note} = {Similarity scores estimated with Euclidean matching and linear weights.},
        remark{Source} = {Author's calculation based on \cite{onet2023data}.}
    ]{
        colspec = {X[l, m]X[0, c, m]},
        width = \linewidth,
        rowhead = 1,
        rowfoot = 0
    }
        \toprule
            Comparison Occupation & Similarity \\ 
        \midrule
            Physicists & 1.00 \\ 
            Astronomers & 0.85 \\ 
            Mathematicians & 0.80 \\ 
            Physics Teachers, Postsecondary & 0.80 \\ 
            Materials Scientists & 0.79 \\ 
            Engineering Teachers, Postsecondary & 0.79 \\ 
            Nanosystems Engineers & 0.78 \\ 
            Aerospace Engineers & 0.77 \\ 
            Biochemists and Biophysicists & 0.77 \\ 
            Nuclear Engineers & 0.77 \\ 
            Computer and Information Research Scientists & 0.77 \\ 
        \bottomrule
    \end{longtblr}
}

% Credit Analysts
\newcommand{\MatchesCreditAnalysts}{
    \begin{longtblr}[
        caption = {Best Career Matches -- Credit Analysts},
        label = {tab:similarity_credit_analysts},
        remark{Note} = {Similarity scores estimated with Euclidean matching and linear weights.},
        remark{Source} = {Author's calculation based on \cite{onet2023data}.}
    ]{
        colspec = {X[l, m]X[0, c, m]},
        width = \linewidth,
        rowhead = 1,
        rowfoot = 0
    }
        \toprule
            Comparison Occupation & Similarity \\ 
        \midrule
            Credit Analysts & 1.00 \\ 
            Accountants and Auditors & 0.87 \\ 
            Loan Interviewers and Clerks & 0.86 \\ 
            Budget Analysts & 0.85 \\ 
            Securities, Commodities, and Financial Services Sales Agents & 0.85 \\ 
            Financial Examiners & 0.85 \\ 
            Business Intelligence Analysts & 0.85 \\ 
            Tax Preparers & 0.85 \\ 
            Insurance Underwriters & 0.84 \\ 
            Bookkeeping, Accounting, and Auditing Clerks & 0.84 \\ 
            Tax Examiners and Collectors, and Revenue Agents & 0.83 \\ 
        \bottomrule
    \end{longtblr}
}

% Dishwashers
\newcommand{\MatchesDishwashers}{
    \begin{longtblr}[
        caption = {Best Career Matches -- Dishwashers},
        label = {tab:similarity_dishwashers},
        remark{Note} = {Similarity scores estimated with Euclidean matching and linear weights.},
        remark{Source} = {Author's calculation based on \cite{onet2023data}.}
    ]{
        colspec = {X[l, m]X[0, c, m]},
        width = \linewidth,
        rowhead = 1,
        rowfoot = 0
    }
        \toprule
            Comparison Occupation & Similarity \\ 
        \midrule
            Dishwashers & 1.00 \\ 
            Janitors and Cleaners, Except Maids and Housekeeping Cleaners & 0.88 \\ 
            Maids and Housekeeping Cleaners & 0.88 \\ 
            Pressers, Textile, Garment, and Related Materials & 0.87 \\ 
            Models & 0.87 \\ 
            Postal Service Mail Sorters, Processors, and Processing Machine Operators & 0.87 \\ 
            Sewing Machine Operators & 0.86 \\ 
            Laundry and Dry-Cleaning Workers & 0.86 \\ 
            Orderlies & 0.86 \\ 
            Cutters and Trimmers, Hand & 0.86 \\ 
            Shoe and Leather Workers and Repairers & 0.86 \\
        \bottomrule
    \end{longtblr}
}

% Similarity matrix
\newcommand{\SimilarityMatrix}{
    \begin{table}[H]
        \centering
        \begin{talltblr}[
            caption = {Similarity Matrix},
            label = {tab:similarity_matrix},
            remark{Note 1} = {Similarity scores estimated with Euclidean matching and linear weights.},
            remark{Note 2} = {Comparison occupations on the horizontal axis.},
            remark{Source} = {Author's calculation based on \cite{onet2023data}.}
        ]{
            colspec = {X[c, m]X[c, m]X[c, m]X[c, m]X[c, m]},
            width = \linewidth
        }
            \toprule
                & Mechanical Engineers & Physicists & Credit Analysts & Dishwashers \\ 
            \midrule
                Mechanical Engineers & 1.00 & 0.75 & 0.54 & 0.38 \\ 
                Physicists & 0.80 & 1.00 & 0.57 & 0.34 \\ 
                Credit Analysts & 0.75 & 0.66 & 1.00 & 0.54 \\ 
                Dishwashers & 0.62 & 0.53 & 0.66 & 1.00 \\ 
            \bottomrule
        \end{talltblr}
    \end{table}
}

% Detailed skill sets
\newcommand{\DetailedSkillSets}{
    \begin{longtblr}[
        caption = {Detailed Skill Sets},
        label = {tab:detailed_skill_sets},
        remark{Source} = {\cite{onet2023data}.}
    ]{
        % colspec = {X[1.5, c, m]X[1, c, m]X[1, c, m]X[1, c, m]X[1, c, m]},
        colspec = {X[0.4, l, m]X[0.2, c, m]X[0.2, c, m]X[0.2, c, m]X[0.2, c, m]},
        width = \linewidth,
        rowhead = 1,
        rowfoot = 0
    }
        \toprule
            Competency & Mechanical Engineers & Physicists & Credit Analysts & Dishwashers \\
        \midrule
            Engineering and Technology & 87 & 86 & 13 & 0 \\ 
            Industrial Design & 84 & 45 & 0 & 0 \\ 
            Applied Mathematics & 79 & 94 & 68 & 0 \\ 
            Mechanical & 78 & 28 & 0 & 12 \\ 
            Physics & 73 & 97 & 0 & 0 \\ 
            Oral Comprehension & 70 & 84 & 59 & 32 \\ 
            Reading Comprehension & 70 & 84 & 57 & 30 \\ 
            Oral Expression & 70 & 82 & 59 & 30 \\ 
            Deductive Reasoning & 70 & 79 & 64 & 29 \\ 
            Pure Mathematics & 70 & 79 & 59 & 14 \\ 
            Mathematical Reasoning & 68 & 84 & 59 & 16 \\ 
            Written Comprehension & 68 & 79 & 59 & 29 \\ 
            Complex Problem-Solving & 68 & 70 & 43 & 30 \\ 
            Computers and Electronics & 67 & 83 & 43 & 0 \\ 
            Natural Science & 66 & 82 & 5 & 0 \\ 
            Operations Analysis & 66 & 52 & 34 & 0 \\ 
            Active Learning & 63 & 79 & 52 & 29 \\ 
            Information Ordering & 63 & 71 & 52 & 30 \\ 
            Number Facility & 61 & 79 & 63 & 16 \\ 
            Critical Thinking & 61 & 70 & 57 & 30 \\ 
            Monitoring & 61 & 57 & 43 & 30 \\ 
            Written Expression & 59 & 80 & 57 & 23 \\ 
            Inductive Reasoning & 59 & 79 & 59 & 29 \\ 
            Category Flexibility & 59 & 73 & 50 & 29 \\ 
            Problem Sensitivity & 59 & 70 & 57 & 29 \\ 
            Judgment and Decision & 59 & 64 & 54 & 30 \\ 
            Originality & 57 & 75 & 37 & 25 \\ 
            Writing & 57 & 71 & 55 & 25 \\ 
            Native Language & 57 & 70 & 60 & 38 \\ 
            Active Listening & 57 & 68 & 57 & 32 \\ 
            Near Vision & 57 & 57 & 61 & 34 \\ 
            Administration and Management & 57 & 54 & 42 & 37 \\ 
            Quality Control Analysis & 57 & 46 & 11 & 29 \\ 
            Chemistry & 56 & 61 & 0 & 21 \\ 
            Fluency of Ideas & 55 & 73 & 41 & 27 \\ 
            Speaking & 55 & 68 & 57 & 30 \\ 
            Visualization & 55 & 61 & 25 & 29 \\ 
            Systems Analysis & 55 & 57 & 43 & 16 \\ 
            Systems Evaluation & 55 & 55 & 45 & 16 \\ 
            Speech Recognition & 54 & 57 & 50 & 30 \\ 
            Technology Design & 54 & 48 & 0 & 0 \\ 
            Operations Monitoring & 54 & 34 & 14 & 30 \\ 
            Instructing & 52 & 66 & 34 & 20 \\ 
            Time Management & 52 & 52 & 43 & 30 \\ 
            Troubleshooting & 52 & 29 & 0 & 29 \\ 
            Education and Training & 51 & 72 & 29 & 41 \\ 
            Administrative & 51 & 40 & 59 & 0 \\ 
            Learning Strategies & 50 & 71 & 36 & 18 \\ 
            Selective Attention & 50 & 57 & 41 & 30 \\ 
            Persuasion & 50 & 50 & 36 & 27 \\ 
            Coordination & 50 & 48 & 36 & 30 \\ 
            Production and Processing & 49 & 34 & 29 & 0 \\ 
            Speech Clarity & 48 & 71 & 45 & 27 \\ 
            Flexibility of Closure & 48 & 55 & 45 & 29 \\ 
            Perceptual Speed & 48 & 46 & 45 & 30 \\ 
            Customer and Personal Service & 48 & 37 & 39 & 38 \\ 
            Far Vision & 46 & 52 & 30 & 32 \\ 
            Management of Personnel Resources & 45 & 45 & 27 & 29 \\ 
            Speed of Closure & 43 & 55 & 41 & 29 \\ 
            Negotiation & 43 & 39 & 39 & 29 \\ 
            Social Perceptiveness & 41 & 45 & 41 & 29 \\ 
            Service Orientation & 41 & 43 & 43 & 29 \\ 
            Visual Color Discrimination & 41 & 41 & 21 & 29 \\ 
            Management of Material Resources & 41 & 32 & 13 & 13 \\ 
            Multitasking & 39 & 41 & 29 & 29 \\ 
            Management of Financial Resources & 39 & 30 & 27 & 11 \\ 
            Hearing Sensitivity & 39 & 30 & 16 & 29 \\ 
            Memorization & 37 & 48 & 34 & 21 \\ 
            Auditory Attention & 37 & 32 & 21 & 29 \\ 
            Telecommunications & 37 & 23 & 7 & 0 \\ 
            Equipment Selection & 37 & 21 & 0 & 29 \\ 
            Installation & 37 & 0 & 0 & 0 \\ 
            Building and Construction & 36 & 15 & 16 & 0 \\ 
            Personnel and Human Resources & 35 & 38 & 23 & 0 \\ 
            Public Safety and Security & 35 & 26 & 15 & 26 \\ 
            Programming & 34 & 55 & 14 & 0 \\ 
            Depth Perception & 34 & 30 & 5 & 29 \\ 
            Operation and Control & 34 & 9 & 0 & 30 \\ 
            Sales and Marketing & 33 & 20 & 29 & 0 \\ 
            Finger Dexterity & 32 & 29 & 25 & 34 \\ 
            Law and Government & 32 & 28 & 54 & 28 \\ 
            Reaction Time & 30 & 0 & 0 & 29 \\ 
            Transportation & 28 & 12 & 0 & 33 \\ 
            Communications and Media & 25 & 46 & 23 & 19 \\ 
            Economics and Accounting & 25 & 26 & 74 & 0 \\ 
            Arm-Hand Steadiness & 25 & 0 & 11 & 41 \\ 
            Repairing & 25 & 0 & 0 & 29 \\ 
            Equipment Maintenance & 23 & 0 & 0 & 29 \\ 
            Medicine and Dentistry & 23 & 0 & 0 & 0 \\ 
            Geography & 21 & 15 & 25 & 0 \\ 
            Biology & 20 & 16 & 0 & 0 \\ 
            Trunk Strength & 14 & 14 & 14 & 43 \\ 
            Manual Dexterity & 14 & 0 & 9 & 52 \\ 
            Wrist-Finger Speed & 9 & 0 & 0 & 21 \\ 
            Rate Control & 5 & 0 & 0 & 32 \\ 
            Spatial Orientation & 5 & 0 & 0 & 14 \\ 
            Psychology & 0 & 28 & 10 & 0 \\ 
            Therapy and Counseling & 0 & 21 & 0 & 0 \\ 
            Sociology and Anthropology & 0 & 0 & 16 & 0 \\ 
            History and Archeology & 0 & 0 & 13 & 0 \\ 
            Control Precision & 0 & 0 & 7 & 32 \\ 
            Extent Flexibility & 0 & 0 & 0 & 43 \\ 
            Static Strength & 0 & 0 & 0 & 39 \\ 
            Multilimb Coordination & 0 & 0 & 0 & 37 \\ 
            Stamina & 0 & 0 & 0 & 32 \\ 
            Gross Body Coordination & 0 & 0 & 0 & 30 \\ 
            Speed of Limb Movement & 0 & 0 & 0 & 30 \\ 
            Response Orientation & 0 & 0 & 0 & 29 \\ 
            Dynamic Strength & 0 & 0 & 0 & 25 \\ 
            Gross Body Equilibrium & 0 & 0 & 0 & 21 \\ 
            Glare Sensitivity & 0 & 0 & 0 & 21 \\ 
            Sound Localization & 0 & 0 & 0 & 14 \\ 
            Night Vision & 0 & 0 & 0 & 14 \\ 
            Peripheral Vision & 0 & 0 & 0 & 14 \\ 
            Explosive Strength & 0 & 0 & 0 & 11 \\ 
            Dynamic Flexibility & 0 & 0 & 0 & 0 \\ 
            Foreign Language & 0 & 0 & 0 & 0 \\ 
            Philosophy and Theology & 0 & 0 & 0 & 0 \\ 
            Fine Arts & 0 & 0 & 0 & 0 \\ 
            Food Production & 0 & 0 & 0 & 0 \\
        \bottomrule
    \end{longtblr}
}

% references
\addbibresource{references.bib}

% endregion

% region: display breaks 
\allowdisplaybreaks

% endregion

% region: title 
\title{
    Employability and Competitiveness\\
    in Efficient Labor Markets
}
\author{Cao Bittencourt}
\affiliation{{B. Sc. in Economics from EPGE (FGV), RJ, Brazil.}}
\affiliation{{Statistician at Atlas Career Guide Inc., FL, USA.}}
\date{\today}

% endregion

%SECTION: PAPER 
\begin{document}

% region: abstract 
\begin{abstract}
    In this document, the Employability Theorem is demonstrated from a set of fairly tautological axioms, which are presupposed in quantitative career choice and career development methods.
\end{abstract}

\begin{keyword}
    Employability theorem; 
    Career choice; 
    Career development; 
    Vocational choice; 
    Occupational Information Network; 
    O*NET.
\end{keyword}

% endregion

% region: title 
\maketitle

% endregion

% region: contents 
\tableofcontents

% endregion

% region: introduction 
\newpage
\section{Introduction}

% endregion

% region: presuppositions 
\newpage
\section{The Issue of Occupational Complexity}
% region: attribute definition 1 
\begin{definition}[Skill]
    \label{skl_tasks}
    % statement
    A professional attribute, competency, or skill, of a person $k$ can be conceptualized as a cummulative sum of successes on
    binary outcome variables representing tasks of progressive difficulty which require only that skill:
    \begin{gather}
        a_{i}^{k} = \sum_{l=0}^{l_i}
        T_{i_{l}}^{k}
        ,
    \end{gather}
    where
    \begin{gather}
        T_{i_{l}}^{k} =
        \begin{cases}
            1, \ \text{if $k$ succedes in a task $T_{i}^{l}$ of difficulty level $l$;} \\
            0, \ \text{otherwise.}
        \end{cases}
    \end{gather}
    
    Or, more rigorously,
    \begin{gather}
        a_{i}^{k} = \sum_{l=0}^{l_i}
        T(l, l_{i}^{k})
        ,
    \end{gather}
    where
    \begin{gather}
        T(l, l_{i}^{k}) =
        T_{i_{l}}^{k} =
        [l \leq l_{i}^{k}] =
        \begin{cases}
            1, \ l \leq l_{i}^{k}; \\
            % 0, \ \text{otherwise.}
            0, \ l > l_{i}^{k}.
        \end{cases}
    \end{gather}
    and $l_{i}^{k} \in [0, l_i]$ is the maximum difficulty level on which $k$ still succeedes. Thus, we can define a person $k$'s skill level in an attribute $i$ as the sum
    of their successful trials on a $T_{i} = \{T_{i}^{0}, \dots, T_{i}^{l_i}\}$ set of tasks of increasing
    difficulty.

    Furthermore, as we assume scales are truncated (i.e. there is a maximum
    difficulty level $l_i$, and a trivial difficulty level, which has to be zero),
    we can also interpret $a_{i}^{k}$ as the \textit{portion} of tasks one is able
    to accomplish out of all difficulty levels for that skill. By normalizing $l_i$
    to $100$, for example, we have:
    
        $a_{i}^{k} = 0 \iff k$ cannot perform even the most basic of attribute $i$'s tasks;
        $a_{i}^{k} = 10 \iff k$ can perform only the bottom 10\% of attribute $i$'s tasks, but nothing more;
        $a_{i}^{k} = 50 \iff k$ can perform the easiest half of attribute $i$'s tasks, but not the most difficult half;
        $a_{i}^{k} = 100 \iff k$ can perform all of attribute $i$'s tasks.
    

    Finally, we can define $a_{i}^{k}$ for a continuum of task difficulty $l \in [0,1]$:
    \begin{gather}
        a_{i}^{k} =
        \int_{0}^{1}{
        T(l, l_{i}^{k})
        dl
        }
        ,
    \end{gather}
    where $T(l, l_{i}^{k})$ is defined as before.
\end{definition}
% endregion
% region: attribute definition 2 
\begin{definition}[Skill]
\label{skl_difficulty}
% statement
    Alternatively, we can think of a person $k$'s professional attribute, competency, or skill, as the difficulty of the most difficult task they can accomplish, normalized by the difficulty of the most objectively difficult task of that particular skill:
    \begin{gather}
        a_{i}^{k} =
        \frac{
            l_{i}^{k}
        }{
            l_{i}
        }
        ,
    \end{gather}
    which we normalize by setting $l_{i} = 1$, so that
    \begin{gather}
        a_{i}^{k} =
        \frac{
            l_{i}^{k}
        }{
            1
        }
        =
        l_{i}^{k}
        ,
    \end{gather}
    and $l_{i}^{k} \in [0,1]$. With this normalization, example interpretations of $a_{i}^{k}$ are:
    
        $a_{i}^{k} = 0 \iff k$ cannot perform even the most basic of attribute $i$'s tasks;
        $a_{i}^{k} = 0.10 \iff k$ can only perform tasks of up to 10\% the difficulty of attribute $i$'s most difficult task, but nothing more;
        $a_{i}^{k} = 0.50 \iff k$ can perform tasks of up to half the difficulty of attribute $i$'s most difficult task, but nothing more;
        $a_{i}^{k} = 1 \iff k$ can perform all of attribute $i$'s tasks.
    

    This is, perhaps, the most natural conceptual model for understanding
    competencies, as, generally, it is more intuitive to think of skill as the
    maximum of one's capacity, rather than the portion of tasks one could
    ly accomplish.

    But, again, [because we assume scales to be truncated], this latter interpretation
    actually implies and is implied by the former. For if a task is of the same
    difficulty as another, then they are just as difficult in relation to that
    skill's most difficult task (i.e. they require the same percentage of the
    scale's upper limit to be performed), and, likewise, are also included in the
    same difficulty ``bracket" (i.e. they are equivalent to the same skill test in
    the aggregate binary outcome interpretation), and, therefore, pressupose the
    same $a_{i}^{k}$ skill level.

    Of course, this equivalence is quite trivial, given that
    \begin{gather}
        \int_{0}^{1}{
        T(l, l_{i}^{k})
        dl
        }
        =
        1 \times \int_{0}^{l_{i}^{k}}{dl} +
        0 \times \int_{l_{i}^{k}}^{1}{dl}
        =
        l_{i}^{k} - 0
        =
        \frac{l_{i}^{k}}{1}
        =
        a_{i}^{k}
        .
    \end{gather}

    This means the percentage of a skill's tasks one can accomplish is also
    the difficulty of the most difficult task one can accomplish relative to that
    skill's most difficult task.

    So, however one decides to interpret skill levels, the conclusion remains the
    same: to be skilled in an attribute is to be able to perform the activities
    associated with that attribute. Put simply, the capacity to act follows virtue,
    for virtue is, itself, the capacity to act.
\end{definition}
% endregion
% region: skill sufficiency lemma (ssl) 
% skills are sufficient to accomplish tasks
P.S.: SSL
\begin{align}
    a_{i}^{k} := \frac{l_{i}^{k}}{l_{i}} \in [0,1]
    \\
    \tilde{T}_{i}^{k} 
    &= 
    \int_{0}^{l_i}{
        T(l,l_{i}^{k})
        dl
    }
    \left(
        \int_{0}^{l_i}{
            T(l,l_{i})
            dl
        }
    \right) ^ {-1}
    \\
    &= 
    \left(
        \int_{0}^{l_{i}^{k}}{
            1
            \times
            dl
        }
        +
        \int_{l_{i}^{k}}^{l_i}{
            0
            \times
            dl
        }
    \right)
    \times
    \left(
        \int_{0}^{l_i}{
            1
            \times
            dl
        }
    \right) ^ {-1}
    \\
    &= 
    \frac{l_{i}^{k} - 0}{l_{i} - 0}
    \\
    &= 
    \frac{l_{i}^{k}}{l_{i}}
    \\
    &\therefore
    a_{i}^{k} = \tilde{T}_{i}^{k}
\end{align}

P.S.: SCL
\begin{align}
    \boldsymbol{a_k}
    &:= (a_{1}^{k}, \dots, a_{m}^{k})
    ,
    \boldsymbol{a_q}
    := (a_{1}^{q}, \dots, a_{m}^{q})
    \in [0,1]^{m}
    \
    \forall
    \
    k,q \in \{1, \dots, n\}
    \\
    l_{q}^{k}
    &\leq 
    l_{q}^{q}
    \
    \forall
    \
    k,q \in \{1, \dots, n\}
    \\
    \tilde{T}_{q}^{k}
    &= 
    \int_{0}^{l_{q}^{q}}{
        T(l,l_{q}^{k})
        dl
    }
    \left(
        \int_{0}^{l_{q}^{q}}{
            T(l,l_{q}^{q})
            dl
        }
    \right) ^ {-1}
    \\
    &= 
    \left(
        \int_{0}^{l_{q}^{k}}{
            1
            \times
            dl
        }
        +
        \int_{l_{q}^{k}}^{l_{q}^{q}}{
            0
            \times
            dl
        }
    \right)
    \times
    \left(
        \int_{0}^{l_{q}^{q}}{
            1
            \times
            dl
        }
    \right) ^ {-1}
    \\
    &= 
    \frac{l_{q}^{k} - 0}{l_{q}^{q} - 0}
    \\
    &= 
    \frac{l_{q}^{k}}{l_{q}^{q}}
    \in [0,1]
    \\
    &\because
    \neg
    \
    l_{q}^{k}
    >
    l_{q}^{q}
    \\
    &\because
    l_{q}^{k} = 
    f(
        \boldsymbol{l_k},
        \boldsymbol{l_q}
    ) 
    =
    f(
        \boldsymbol{a_k},
        \boldsymbol{a_q}
    )?
    \in [0,1]
    \
    \forall
    \
    k,q \in \{1, \dots, n\}
    \\
    &\therefore
    \tilde{T}_{q}^{k}
    =
    f(
        \boldsymbol{a_k},
        \boldsymbol{a_q}
    )
\end{align}

% preamble
Now, even though these results are basically tautological, they are still important to guide our intuition. In fact, our first insight towards the Employability Theorem, namely the Skill Sufficiency Lemma (SSL), follows directly from the definitions above.
% statement
\begin{lemma}[Skill Sufficiency Lemma]
    \label{ssl}
    According to the SSL, skills are necessary and sufficient to accomplish tasks.
    In particular, to have a skill level of $a_{i}^{k} \in [0,1]$ in attribute $i$ is a necessary and sufficient condition for one to be capable of accomplishing the easier $a_{i}^{k}$ portion of that attribute's tasks.
\end{lemma}

% proof
\begin{proof}
    By definition,
    \begin{gather}
        T(l, l_{i}^{k}) =
        \begin{cases}
            1, \ l \leq l_{i}^{k}; \\
            0, \ \text{otherwise.}
        \end{cases}
    \end{gather}
    is a binary indicator of person $k$'s ability to accomplish a task of difficulty $l \in [0,1]$ which requires only attribute $i$.
    
    With this,
    \begin{gather}
        \tilde{T}_{i}^{k}
        =
        \int_{0}^{1}{
        T(l, l_{i}^{k})
        dl
        }
    \end{gather}
    is the percentage of tasks requiring only attribute $i$ that $k$ can accomplish.

    But both equivalent definitions of $k$'s skill level in attribute $i$, are
    \begin{gather}
        a_{i}^{k} = 
        \int_{0}^{1}{
        T(l, l_{i}^{k})
        dl
        }
        =
        l_{i}^{k}
        ,
    \end{gather}
    which is precisely the $\tilde{T}_{i}^{k}$ aggregation of $T(l, l_{i}^{k})$ in the $[0,1]$ interval.

    Therefore, having a skill level of $a_{i}^{k}$ is a necessary and sufficient condition to be capable of accomplishing the easier $a_{i}^{k}$ portion of attribute $i$'s tasks:
    \begin{gather}
        a_{i}^{k} =
        \tilde{T}_{i}^{k}
        \iff
        \int_{0}^{1}{
        T(l, l_{i}^{k})
        dl
        } =
        \int_{0}^{1}{
        T(l, l_{i}^{k})
        dl
        }
        .
    \end{gather}
\end{proof}

% endregion
% region: complex task definition 
% statement
\begin{definition}[Complex Task]
    \label{complex_task}
    A task is said to be complex if it relies on more than one attribute to be accomplished. More precisely, $T_{ij}^{l}$ is a complex task of attributes $i$ and $j$, if its binary outcome indicator is of the form
    \begin{gather}
        T(l, l_{ij}^{k}) =
        [l \leq l_{ij}^{k}],
        % \begin{cases}
        %     1, \ l \leq l_{ij}^{k}; \\
        %     0, \ \text{otherwise.}
        % \end{cases}
    \end{gather}
    where
    \begin{gather}
        l_{ij}^{k} = f(l_{i}^{k}, l_{j}^{k})
    \end{gather}
    is a strictly increasing aggregation function that returns the maximum difficulty level of the complex task $T_{ij}^{l}$ a person $k$ can accomplish based on each attribute $T_{ij}^{l}$ requires. Or, generalizing for any complex task $T_{q}^{l}$ of $m$ attributes, requiring an entire skill set $\boldsymbol{a_q} = (a_{1}^{q}, \dots, a_{m}^{q})$ to be accomplished, 
    \begin{gather}
        T(l, l_{q}^{k}) =
        [l \leq l_{q}^{k}],
        % \begin{cases}
        %     1, \ l \leq l_{q}^{k}; \\
        %     0, \ \text{otherwise.}
        % \end{cases}
    \end{gather}
    where
    \begin{gather}
        l_{q}^{k} = 
        f(\boldsymbol{l_{q}^{k}}) =
        f(l_{1}^{k}, \dots, l_{m}^{k})
    \end{gather}
    and 
    \begin{gather}
        \frac{
            \partial
            f(\boldsymbol{l_{q}^{k}})
        }{
            \partial 
            l_{i}^{k}
        } > 0
        \
        \forall 
        \
        i \in \{1, \dots, m\}
        .
    \end{gather}
        
    This means none of the attributes required by the complex task are completely disposable (i.e. they are all helpful in some way). For instance, the task $T_{i}^{l}$, previously defined, with binary outcome $T(l, l_{i}^{k})$ is not complex, because
    \begin{gather}
        \frac{
            \partial
            l_{i}^{k}
        }{
            \partial 
            l_{i}^{k}
        } = 1
        ,
    \end{gather}
    but
    \begin{gather}
        \frac{
            \partial
            l_{i}^{k}
        }{
            \partial 
            l_{j}^{k}
        } = 0
        , 
    \end{gather}
    where $i \neq j$ and $i, j \in \{1, \dots, m\}$. Or, say the aggregation function is given by
    \begin{gather}
        f(l_{i}^{k}, l_{j}^{k}) = 
        l_{i}^{k} - l_{j}^{k}
        ,
    \end{gather}
    so that attribute $j$ actually hinders productivity:
    \begin{gather}
        \frac{
            \partial
            l_{i}^{k}
        }{
            \partial 
            l_{j}^{k}
        } = -1
        .
    \end{gather}
        
    None of these are complex tasks, for they do not coherently mobilize multiple attributes towards a unified goal.
    
    \begin{subdefinition}[Weak Complexity]
        Now, beyond these most basic rules, we can define stricter versions of ``task complexity'' with additional assumptions. The first version, of weak complexity, requires that
        \begin{gather}
            \frac{
                \partial ^ 2 {f(\boldsymbol{l_{q}^{k}})}
            }{
                \partial 
                l_{i}^{k}
                \partial
                l_{j}^{k}
            }
            > 0
            \
            \forall
            \
            % i, j \in \{1, \dots, m\}
            % ,
            % i \neq j
            i \neq j \in \{1, \dots, m\}
            ,
        \end{gather}
        meaning attributes are complementary.
    \end{subdefinition}
    
    \begin{subdefinition}[Moderate Complexity]
        A task is of moderate complexity if its aggregation function also meets the following criteria:
        \begin{gather}
            \lim_{
                l_{i}^{k} \rightarrow 0
            }{
                f(\boldsymbol{l_{q}^{k}})
            } = 0
            \
            \forall 
            \
            i \in \{1, \dots, m\}
            ,
        \end{gather}
        so that a person $k$'s capacity to perform the complex task is weakly increasing on their capacity to perform the simple tasks of its required attributes, and goes to zero when they are unskilled in at least one of these. Thus, a moderately complex task is not reducible to any proper subset of its attributes.
        
        For instance, a task of the form
    \begin{gather}
        T(l, l_{ij}^{k})
        = [l \leq (1 + l_{i}^{k}) \times (1 + l_{j}^{k}) - 1]
    \end{gather}
    is not moderately complex, as person $k$ does not need every attribute to accomplish the task. Indeed, if $k$ has precisely zero capacity in either skill $i$ or $j$, then $T_{ij}^{l}$ collapses to unidimensional, or simple, tasks $T_{i}^{l}$ when
    \begin{align}
        T(l, l_{ij}^{k})
        &= [l \leq (1 + l_{i}^{k}) \times (1 + 0) - 1]
        \\
        &= [l \leq l_{i}^{k}]
        \\
        &= T(l, l_{i}^{k})
        ,
    \end{align}
    or $T_{j}^{l}$ when
    \begin{align}
        T(l, l_{ij}^{k})
        &= [l \leq (1 + 0) \times (1 + l_{j}^{k}) - 1]
        \\
        &= [l \leq l_{j}^{k}]
        \\
        &= T(l, l_{j}^{k})
        ,
    \end{align}
    in which case $T_{ij}^{l}$ is not \textit{really} (moderately) complex, but rather a convolution of simple tasks. Notice, however, this does not imply there cannot be a degree of substitution between attributes. That is, moderate task complexity only means a task must require all of its attributes in \textit{some} level, even if its functional form allows for substitution.
    
    \end{subdefinition}
    
    \begin{subdefinition}[Strong Complexity]
        The strictest definition of task complexity adds the constraint that skills are aggregated by the Leontief function:
        \begin{gather}
            f(\boldsymbol{l_{q}^{k}}) = 
            \min(\boldsymbol{l_{q}^{k}})
            .
        \end{gather}

        Here, attributes are assumed to be perfect complements, which need to be combined in exactly the same quantities for maximum efficacy. In other words, having additional skills does not help to accomplish the task, but being unskilled in even a single attribute can undermine the whole effort. Hence, productivity is limited by the lowest competency.
    \end{subdefinition}

\end{definition}
% endregion
% region: skill composition lemma (scl) 
% skills are composable to accomplish complex tasks
% statement
\begin{lemma}[Skill Composition Lemma]
    \label{scl}
    The Skill Composition Lemma (SCL) is a generalization of the SSL and states that skills are composable to accomplish complex tasks. More precisely, let $T_{q}^{l}$ be an activity of difficulty level $l$ that requires the $\boldsymbol{a_q} = (a_{1}^{q}, \dots, a_{m}^{q})$ skill set (i.e. $T_{q}^{l}$ is a complex task). With this, we demonstrate that any rational and sufficiently qualified economic agent can naturally ``piece together'', that is \textit{compose}, attributes $\{1, \dots, m\}$ to accomplish the $T_{q}^{l}$ complex task.
    
    % proof
    \begin{proof}
        Given 
        \begin{gather}
            \tilde{T}_{q}^{k} = 
            \int_{0}^{1}{
                T(l, l_{q}^{k})
                dl
            }
        \end{gather}
    \end{proof}
\end{lemma}
% endregion
% region: occupational reducibility axiom (ora) 
% occupations can be reduced to their tasks
% statement
\label{ora}
\begin{axiom}[Occupational Reducibility Axiom]
    Occupations can be reduced to their tasks.
\end{axiom}
% endregion
% region: (weak) occupational complexity axiom (woca) 
% all of an occupation's tasks can be thought of as only
% one, indivisible, holistic task, requiring their entire skill set
% we call this an occupation's "operation"
% all an occupation does is its "operation"
% define an aggregation function to compose skills an an "operation"
% operation production: evaluated aggregation function
% max production: occupation's evaluated aggregation function
% productivity: production truncated and normalized by max production
% statement
\begin{axiom}[Occupational Complexity Axiom]
    \label{woca}
    All of an occupation's tasks can be thought of as one indivisible task, which mobilizes their entire skill set. We call this ``holistic task'' an occupation's \textit{operation}.

    Mathematically, an occupational operation is just a series of complex tasks on a continuum of difficulty levels normalized to the unit interval, all of which are indispensable for the whole operation to be accomplished.
    
    \begin{subaxiom}[Strong Occupational Complexity Axiom, SOCA]
        
        Let us denote, then, ``operational output'' (o.o.) with the standard IPA (cite ipa specification for $\mho$) symbol for the near-close near-back rounded vowel (i.e. the ``double o'' sound in words such as ``boot''):
        \begin{gather}
            \mho_{q}^{k} = 
            \left\lfloor
                \int_{0}^{1}{
                    T(l,l_{q}^{k})
                    dl
                }
            \right\rfloor
            .
        \end{gather}
    
        By this formulation, the amount of an occupations's operation a person $k$ can output is the floor of what they accomplish of the operation's complex tasks. Of course, this value is zero if they are not perfectly qualified, which at first can seem too strong of a pressuposition, but is, in fact, an effective strategy to ``side-step'' the issue of occupational complexity and is, also, quite reasonable, as we shall demonstrate below.
    \end{subaxiom}

    \begin{subaxiom}[Moderate Occupational Complexity Axiom, MOCA]
        dsds
    \end{subaxiom}

    \begin{subaxiom}[Weak Occupational Complexity Axiom, WOCA]
        \begin{gather}
            \mho_q =
            \min\left(
                \boldsymbol{w_q}
                \times
                \boldsymbol{\mho_q}(
                    \boldsymbol{\ell_q}
                )
            \right)
            ,
        \end{gather}
        assuming
        \begin{gather}
            \left[
                \boldsymbol{\tilde{T}_{q}}
                \geq
                \boldsymbol{\ell_q}
            \right]
            ,
        \end{gather}
        if production is split into job subtypes; or
        \begin{gather}
            \mho_q
            :=
            \sum_{v=1}^{w_q}{
                \frac{
                    \left[
                        \tilde{T}_{q}^{v}
                        =
                        1
                    \right]
                }{
                    \int_{0}^{1}{
                        \text{ta}(l)
                        dl
                    }
                }
            }
            :=
            \sum_{v=1}^{w_q}{
                \left[
                    \tilde{T}_{q}^{v}
                    =
                    1
                \right]
                \times
                1
            }
            =
            \sum_{v=1}^{w_q}{
                \left[
                    \tilde{T}_{q}^{v}
                    =
                    1
                \right]
            }
            ,
        \end{gather}
        if production is organized independently.
    \end{subaxiom}

    We could specifiy even weaker versions of the axiom, which would result in the same conclusions we shall demonstrate in this paper. For mathematical convenience, though, we assume weak occupational complexity (WOCA).

    [intuition for operational output formulation]
    
    [clarify the indivisibility of the ``holistic task'' does not imply tasks cannot be outsourced, but that the whole operation has to be accomplished in its entirety]

    [examples in defense of operational output formulation]
    
    [how this operational output helps to side-step the issue of occupational complexity]

    So, in a way, a ``holistic task'', or operation, is a ``doubly complex'' task, as, in addition to being complex as already defined, it is also strictly complex in its difficulty levels, for failing to accomplish even the most basic of difficulty levels nullifies the entire operation. It is, in other words, a strongly complex task itself made up of various complex tasks that perfectly complement one another.

    Aggregate occupational operation output:
    \begin{align}
        \mho_{q} 
        &= 
        \left\lfloor
        \sum_{k=1}^{n}
            % [k \in {\Lambda}^{-1}(q)]
            [k \in {\Lambda}_q]
            ?
            % \tilde{Y}_{q}^{k}
        \right\rfloor
        % \\
        % &= 
        % \left\lfloor
        % \sum_{k=1}^{n}
        % \left(
        %     [k \in {\Lambda}^{-1}(q)]
        %     \times
        %     \left(
        %         \frac{
        %         \prod_{i=1}^{m}{
        %             {
        %                 \max(
        %                     1 + a_{i}^{k},
        %                     1 + a_{i}^{q}
        %                 )
        %             } ^ {
        %                 \sigma_{i}^{q}
        %             }
        %         }
        %     }{
        %         \prod_{i=1}^{m}{
        %             {
        %                 (
        %                     1 + a_{i}^{q}
        %                 )
        %             } ^ {
        %                 \sigma_{i}^{q}
        %             }
        %         }
        %     }
        %     \right)
        % \right)
        % \right\rfloor
        ,
    \end{align}
    where ${\Lambda}_q$ is the set of people working in occupation $q$.

    Labor market taxa ($\Lambda$):
    \begin{gather}
        \Lambda_{1}^{1}
        = \Lambda(1,1)
        = \{
            1, \dots, n
            \}
        \iff k, q \in 
        \Lambda_{1}^{1}
        \\
        \Lambda = \{
            \Lambda_{1}^{1},
            \dots,
            \Lambda_{n}^{\bar{L}}
            \}
        \\
        \Lambda^{-1}(k) = \Lambda_{k}^{\bar{L}}
    \end{gather}

\end{axiom}
% endregion
% region: occupational composition lemma (ocl) 
% sca, ora, oca => ocl
% skill sets are composable to accomplish occupations' "operations"
% statement
\label{ocl}
\begin{lemma}[Occupational Composition Lemma]
    Skill sets are composable to accomplish occupations' operations.
\end{lemma}

% proof
\begin{proof}

\end{proof}
% endregion

\section{Employer Behavior}
% region: employer rationality axiom (era) 
% employers are rational
% employers will only pay for employees to work on tasks they can accomplish
% if required, employers will outsource parts of an occupation's operation
% statement
\begin{axiom}[Rationality Axiom]
    \label{era}
    Employers are rational and will only pay for employees to work on tasks they can accomplish. Additionaly, employers will outsource parts of an occupation's operation if their employees cannot accomplish the entire operation.
\end{axiom}
% endregion
% region: hireability axiom (ha) 
% hiring is done by evaluating a hireability statistic
% statement
\begin{axiom}[Hireability Axiom]
    \label{ha}
    % Any rational employer hires employees by evaluating a hireability statistic, which quantifies potential employees' skill set similarity with an occupation, their educational attainment, and years of experience.
    Any rational employer hires employees by evaluating a hireability statistic, which quantifies potential employees' expected productivity, their educational attainment, and years of experience.

    \begin{subaxiom}[Weak Hireability Axiom]
        \begin{gather}
            \mathbb{E}
                \left| 
                    h_{q}^{k} - 
                    % \mathbb{E}h_{q}^{k}
                    \mathbb{E}(h_{q}^{k})
                \right| 
            \in [0,1]
        \end{gather}
    \end{subaxiom}
    
    \begin{subaxiom}[Moderate Hireability Axiom]
        \begin{gather}
            % \text{E}(
            %     \left| 
            %         h_{q}^{k} - 
            %         \text{E}(h_{q}^{k}) 
            %     \right| 
            % ) = 0
            \mathbb{E}
                \left| 
                    h_{q}^{k} - 
                    % \mathbb{E}h_{q}^{k}
                    \mathbb{E}(h_{q}^{k})
                \right| 
            = 0
        \end{gather}
    \end{subaxiom}
    
    \begin{subaxiom}[Strong Hireability Axiom]
        \begin{gather}
            % E(h_{q}^{k}) = h_{q}^{k}
            % \\
            % \mathbb{E}h_{q}^{k} = h_{q}^{k}
            % \\
            \mathbb{E}(h_{q}^{k}) = h_{q}^{k}
        \end{gather}
    \end{subaxiom}
\end{axiom}
% endregion

\section{Task Difficulty and Time Allocation}
% region: time allocation definition 
\begin{definition}[Time Allocation]
    \label{ta}
    % statement
\end{definition} 

% endregion

basic presuppositions
basic lemmas
complex tasks
occupations are but tasks
occupations' tasks are complex
occupations' tasks are holistic (operation)

more difficult tasks presuppose the easier tasks have been accomplished
i.e. $l \in [0,1]$ is a ``progress bar'' of an occupation's operation
strongly holistic: each task $l \geq \bar{l}$ requires all the previous $l \in [0, \bar{l}], \bar{l} \in [0,1]$ difficulty levels to be accomplished. in addition, if all $l \in [0,1]$ levels are not all accomplished, the whole effort is vain and the operation is not completed (i.e. round down $\mho_q$ when calculating operational output). furthermore, each and every $l \in [0,1]$ difficulty level cannot be outsourced (i.e. only a perfectly qualified worker can output a unit of the occupation's operation).

individual's time constraint is spent entirely on trying to accomplish the complex holistic task by themselves. therefore, there is no optimization to be done.

\begin{align*}
    \mho_q 
    = 
    \sum_{k=1}^{n}{
        \left\lfloor
            [k \in \Lambda_q]
            \times
            \mho_{q}^{k}
        \right\rfloor
    }
    = 
    \sum_{k=1}^{n}{
        \left\lfloor
            [k \in \Lambda_q]
            \times
            \int_{0}^{1}{
                T_{q}(l,l_{q}^{k})
                dl
            }
        \right\rfloor
    }
\end{align*}

moderately holistic: each task $l \geq \bar{l}$ requires all the previous $l \in [0, \bar{l}], \bar{l} \in [0,1]$ difficulty levels to be accomplished. in addition, if all $l \in [0,1]$ levels are not all accomplished, the whole effort is vain and the operation is not completed (i.e. round down $\mho_q$ when calculating operational output). however, each and every $l \in [0,1]$ difficulty level can be outsourced (i.e. workers can output partial units of the occupation's operation, which contribute to the operation's completion).
    
because of outsourcing, individual's time constraint is spent working from where another worker ``left off'', so that even if a worker cannot accomplish the entire operation by themselves, they can still contribute to the operation's completion by reducing the time highly skilled workers will have to spend on relatively more trivial tasks.

The first worker spends their entire unitary time allowance trying their hardest to accomplish the highest amount of tasks they can. When they hit their skill cap, they restart their efforts, so as to spend their entire time allowance helping out the next worker:
\begin{gather*}
    \int_{0}^{\tilde{T}_{q}^{k}}{
        T_{q}(l,l_{q}^{k})
        \times
        \text{TA}(l)
        dl
    }
    +
    \int_{0}^{\bar{l}}{
        T_{q}(l,l_{q}^{k})
        \times
        \text{TA}(l)
        dl
    }
    = 1
    \\
    \int_{0}^{\bar{l}}{
        1
        \times
        \text{TA}(l)
        dl
    }
    = 1 -
    \int_{0}^{\tilde{T}_{q}^{k}}{
        1
        \times
        \text{TA}(l)
        dl
    }
    \\
    \int_{0}^{\bar{l}}{
        \text{TA}(l)
        dl
    }
    = 
    \int_{\tilde{T}_{q}^{k}}^{1}{
        \text{TA}(l)
        dl
    }
    \\
    \text{TA}(\bar{l}) -
    \text{TA}(0)
    = 
    \text{TA}(1) -
    \text{TA}(\tilde{T}_{q}^{k})
    \\
    \text{TA}(\bar{l})
    = 
    \text{TA}(1) -
    \text{TA}(\tilde{T}_{q}^{k})
    \\
    \bar{l}
    = 
    \text{TA}^{-1}
    \left(
        \text{TA}(1) -
        \text{TA}(\tilde{T}_{q}^{k})
    \right)
    ,
\end{gather*}
so that $k$ accomplishes tasks of difficulty levels $0$ through $\tilde{T}_{q}^{k}$ on their ``first run'', and restarts their effort to provide additional $l \in \left[0,\text{TA}^{-1}
\left(
    \text{TA}(1) -
    \text{TA}(\tilde{T}_{q}^{k})
\right)
\right]$ levels worth of complex tasks. Thus, the next worker does not need to start from zero, but rather from where $k$ ``left off'': either $\tilde{T}_{q}^{k}$, $\bar{l}$, or some $l \in [0, \tilde{T}_{q}^{k}]$.
        
\begin{align*}
    \mho_q 
    &= 
    \left\lfloor
    \sum_{k=1}^{n}{
        [k \in \Lambda_q]
        \times
        \mho_{q}^{k}
    }
    \right\rfloor
    \\
    &= 
    \left\lfloor
    \sum_{k=1}^{n}{
        [k \in \Lambda_q]
        \times
        \int_{0}^{1}{
            T_{q}(l,l_{q}^{k})
            dl
        }
    }
    \right\rfloor
\end{align*}

weakly holistic: each task $l \geq \bar{l}$ requires all the previous $l \in [0, \bar{l}], \bar{l} \in [0,1]$ difficulty levels to be accomplished. however, if not all $l \in [0,1]$ levels are accomplished, the whole effort is not vain and the operation is partially completed (i.e. do not round $\mho_q$ when calculating operational output). furthermore, each and every $l \in [0,1]$ difficulty level can be outsourced (i.e. workers can output partial units of the occupation's operation, which contribute to the operation's completion).

% region: weak skill differences axiom, wsda 
\begin{axiom}[Weak Skill Differences Axiom, WSDA]
    There are, or there could be, skill differences among people in the workforce (i.e. workers are not all ``clones'' of one another or equally competent). So, the expected value of productivity is:
    \begin{gather}
        \mathbb{E}[
            \tilde{T}_{q}^{k}
        ]
        \in
        [0,1]
        ,
    \end{gather}
    instead of 
    \begin{gather}
        \mathbb{E}[
            \tilde{T}_{q}^{k}
        ]
        =
        \tilde{T}_{q}^{k}
        = 1
        ,
    \end{gather}
    for all $k,q \in \{1, \dots, n\}$. This means employers do not expected every worker to be perfectly qualified and will adjust their hiring and production strategies accordingly.
\end{axiom}
% endregion

% endregion

% region: demonstration 
\section{The Employability Theorem}
\subsection{What is Employability?}
% region: employability definition 
\begin{definition}[Employability]
    \label{def_employability}
    In any labor market, employability is the percentage of available jobs in which one could be hired:
    \begin{gather}
    \tilde{W}_{q}^{k}
    :=
    \left[
        h_{q}^{k}
        \geq
        \frac{1}{2}
    \right]
    \sum_{v=1}^{p}
    \left[
        \tilde{T}_{q}^{k}
        \geq
        \tilde{T}_{q}^{v}
    \right]
    \tilde{w}_{q}^{v}
    \in
    [0,1]
    ,
    \\
    \sum_{v=1}^{p}
    \tilde{w}_{q}^{v}
    :=
    \left(
        \frac{1}{w_q}
    \right)
    \sum_{v=1}^{p}
    w_{q}^{v}
    % \frac{w_{q}^{v}}{w_q}
    := 1
    ,
    \end{gather}
    where $\tilde{T}_{q}^{v} \in [0,1]$ is the minimum productivity required to be hired in one of $p$ types of positions in a labor market with $w_q$ job posts; while $h_{q}^{k} \in [0,1]$ is a hireability statistic accounting for other selection criteria, such as years of education, experience, etc.
    
    And we can further aggregate employability for $n$ occupations to assess how many of all $W$ jobs in the economy are suitable for one's skill set:
    \begin{gather}
    \tilde{W}_{k}
    :=
    \sum_{q=1}^{n}
    \tilde{W}_{q}^{k}
    :=
    \sum_{q=1}^{n}
    \left[
        h_{q}^{k}
        \geq
        \frac{1}{2}
    \right]
    \sum_{v=1}^{p}
    \left[
        \tilde{T}_{q}^{k}
        \geq
        \tilde{T}_{q}^{v}
    \right]
    \tilde{w}_{q}^{v}
    \in
    [0,1]
    ,
    \\
    \sum_{q=1}^{n}
    \tilde{w}_{q}
    :=
    \left(
        \frac{1}{W}
    \right)
    \sum_{q=1}^{n}
    {w}_{q}
    :=
    1
    ,
    \\
    W
    :=
    \sum_{q=1}^{n}
    {w}_{q}
    .
    \end{gather}
\end{definition}

% endregion
\subsection{Introductory Example: Employability with Two Types of Workers}
With these basic axioms in place, we can attempt to derive an employability coefficient as presented in Definition \ref{def_employability} above. For ease of understanding, though, let us begin with a simple example and, then, proceed with a more complete, and robust, theorem.

In this subsection, we shall estimate employability in an occupation $q$'s labor market where there are two types of workers with varying productivity. The first type -- call them ``juniors" -- have lower skill and cannot accomplish tasks with difficulty levels $l \in [0,1]$ higher than $\tilde{T}_{q}^{\text{Jr}} \in [0,1)$. And the other type of employee are perfectly qualified ``seniors'', with $\tilde{T}_{q}^{\text{Sr}} = 1$ productive capacity.

Now, because of weak occupational complexity (WOCA), employers will maximize operational output by producing the entire $l \in [0,1]$ spectrum of occupation $q$'s complex tasks, subject to each task's duration. This can be done either by having only perfectly qualified employees work on these independently, from beginning to end, or by splitting responsabilities into two, or more, types of jobs, thus allowing for less qualified, ``junior'' employees, to work alongside ``seniors'' towards the common goal of accomplishing the whole operation.

Additionally, because we assume there to be skill differences among workers in the labor market, any rational employer will always, and rightly, expect potential employees to be of varying skill levels, rather than all perfectly qualified, so that splitting responsabilities into separate positions will not only be an alternative mode of hiring and producing, but in fact the optimal one.

Therefore, employers will stratify job offers based on required competence, providing ``junior'' and ``senior'' positions, both dedicated to their own subset of complex tasks with difficulty levels appropriate for employees' capacity.

Notice this does not mean those working on ``junior'' positions will, necessarily, be ``juniors'' themselves, that is, less qualified. Indeed, if talent is abundant in the labor market, these positions will have to be filled by more qualified, or even perfectly qualified, ``senior'' employees. For if there were only one type of job, spanning the entire responsability spectrum, these highly qualified workers would already have to accomplish ``junior'' tasks themselves, in order to maximize operational output. However, by having two, or more, types of jobs, requiring more, or less, productivity, they may specialize to the measure there are sufficient employees allocated to easier tasks.

Either way, if the available talent is enough to output occupation $q$'s operation, employability in such a market will be determined by the ratio of junior and senior job posts, as we demonstrate below.

% region: binary employability theorem (bet)
\begin{theorem}[Binary Employability Theorem, BET]
    \label{bet}
    In a labor market with two types of workers with varying productivity, each worker's employability is the inverse of their operational output.
    \begin{proof}
        In the binary case, ``junior'' productive output will be given by:
        \begin{gather}
            \mho_{q}^{\text{Jr}} 
            :=
            \frac{
                1
            }{
                \int_{0}^{\tilde{T}_{q}^{\text{Jr}}}
                \text{ta}(l)
                dl
            }
            = 
            \left(
                \int_{0}^{\tilde{T}_{q}^{\text{Jr}}}
                    \text{ta}(l)
                    dl
            \right) ^ {-1}
            ,
        \end{gather}
        where $\text{ta}(l)$ is the time allocation function of occupation $q$'s complex tasks, and time allowance (the numerator) is set to one. Analogously, ``senior'' output is:
        \begin{gather}
            \mho_{q}^{\text{Sr}}
            :=
            \frac{
                1
            }{
                \int_{\tilde{T}_{q}^{\text{Jr}}}^{1}
                \text{ta}(l)
                dl
            }
            = 
            \left(
                \int_{\tilde{T}_{q}^{\text{Jr}}}^{1}
                    \text{ta}(l)
                    dl
            \right) ^ {-1}
            .
        \end{gather}

        Furthermore, as a mismatch in operational output due to time allocation differences between ``junior'' and ``senior'' tasks would result in wasted production, a rational employer will optimally ``orchestrate'' the productive effort by offering just enough ``senior'' job posts in the labor market to meet ``junior'' productivity. So, by setting ``junior'' job posts to $w_{q}^{\text{Jr}} > 0$ and ``senior'' job posts to $w_{q}^{\text{Sr}} > 0$, we get the ratio between ``junior'' and ``senior'' positions required to output any level of occupation $q$'s operation:
        \begin{align}
            w_{q}^{\text{Sr}} 
            \times 
            \mho_{q}^{\text{Sr}} 
            &= 
            w_{q}^{\text{Jr}} 
            \times 
            \mho_{q}^{\text{Jr}}
            \\
            \therefore
            w_{q}^{\text{Sr}}
            \times 
            \left(
                \int_{\tilde{T}_{q}^{\text{Jr}}}^{1}
                    \text{ta}(l)
                    dl
            \right) ^ {-1}
            &= 
            w_{q}^{\text{Jr}}
            \times 
            \left(
                \int_{0}^{\tilde{T}_{q}^{\text{Jr}}}
                    \text{ta}(l)
                    dl
            \right) ^ {-1}
            \\
            \therefore
            w_{q}^{\text{Sr}}
            &= 
            w_{q}^{\text{Jr}}
            \times
            \left(
                \frac{
                    \int_{\tilde{T}_{q}^{\text{Jr}}}^{1}
                        \text{ta}(l)
                        dl
                }{
                    \int_{0}^{\tilde{T}_{q}^{\text{Jr}}}
                        \text{ta}(l)
                        dl
                }
            \right)
            .
        \end{align}
        
        With this, ``senior'' employability (i.e. the percentage of job posts for which they could be hired) is
        \begin{gather}
            \tilde{w}_{q}^{\text{Sr}} =
            \frac{
                w_{q}^{\text{Jr}} + 
                w_{q}^{\text{Sr}}
            }{
                w_{q}^{\text{Jr}} + 
                w_{q}^{\text{Sr}}
            }
            = 1
            ,
        \end{gather}
        while ``junior'' employability is
        \begin{align}
            \tilde{w}_{q}^{\text{Jr}} 
            &=
            \frac{
                w_{q}^{\text{Jr}}
            }{
                w_{q}^{\text{Jr}} + 
                w_{q}^{\text{Sr}}
            }
            \\&=
            \frac{
                w_{q}^{\text{Jr}}
            }{
                w_{q}^{\text{Jr}} + 
                w_{q}^{\text{Jr}} \times
                \left(
                    \frac{
                        \int_{\tilde{T}_{q}^{\text{Jr}}}^{1}
                            \text{ta}(l)
                            dl
                    }{
                        \int_{0}^{\tilde{T}_{q}^{\text{Jr}}}
                            \text{ta}(l)
                            dl
                    }
                \right)
            }
            \\&=
            \left(
                1 + 
                \frac{
                    \int_{\tilde{T}_{q}^{\text{Jr}}}^{1}
                        \text{ta}(l)
                        dl
                }{
                    \int_{0}^{\tilde{T}_{q}^{\text{Jr}}}
                        \text{ta}(l)
                        dl
                }
            \right) ^ {-1}
            \\&=
            \left(
                1 + 
                \frac{
                    \int_{0}^{1}
                        \text{ta}(l)
                        dl    
                        -
                    \int_{0}^{\tilde{T}_{q}^{\text{Jr}}}
                        \text{ta}(l)
                        dl
                }{
                    \int_{0}^{\tilde{T}_{q}^{\text{Jr}}}
                        \text{ta}(l)
                        dl
                }
            \right) ^ {-1}
            \\&=
            \left(
                1 + 
                \frac{
                    1 -
                    \int_{0}^{\tilde{T}_{q}^{\text{Jr}}}
                        \text{ta}(l)
                        dl
                }{
                    \int_{0}^{\tilde{T}_{q}^{\text{Jr}}}
                        \text{ta}(l)
                        dl
                }
            \right) ^ {-1}
            \\&=
            \left(
                1 + 
                \frac{
                    1
                }{
                    \int_{0}^{\tilde{T}_{q}^{\text{Jr}}}
                        \text{ta}(l)
                        dl
                }
                -
                \frac{
                    \int_{0}^{\tilde{T}_{q}^{\text{Jr}}}
                        \text{ta}(l)
                        dl
                }{
                    \int_{0}^{\tilde{T}_{q}^{\text{Jr}}}
                        \text{ta}(l)
                        dl
                }
            \right) ^ {-1}
            \\&=
            \left(
                1 + 
                \frac{
                    1
                }{
                    \int_{0}^{\tilde{T}_{q}^{\text{Jr}}}
                        \text{ta}(l)
                        dl
                }
                -
                1
            \right) ^ {-1}
            \\&=
            \left(
                \frac{
                    1
                }{
                    \int_{0}^{\tilde{T}_{q}^{\text{Jr}}}
                        \text{ta}(l)
                        dl
                }
            \right) ^ {-1}
            \\&=
            \int_{0}^{\tilde{T}_{q}^{\text{Jr}}}
                \text{ta}(l)
                dl
            .
        \end{align}
        
        Thus, the employability of a partially qualified worker, that is a ``junior'', is precisely the percentage of an operation's total time duration their skill set allows them to accomplish (i.e. the inverse of their operational output).    
    \end{proof}
\end{theorem}
% endregion
\subsection{General Employability Theorem}
% region: maximum-monotonic labor stratification definition 
Now, to generalize this conclusion, we shall define notation in terms of maximum labor stratification, a productive arrangement where there are several job subtypes, indeed as many as there are jobs themselves, each with a limited spectrum of responsabilities.

\begin{definition}[Maximum Labor Stratification]
    \label{mmls_def}
    Hence, mathematically,
    \begin{gather}
    l \in \left[
    \ell_{v-1}
    ,
    \ell_{v}
    \right]
    ,
    \end{gather}
    with
    \begin{align}
    \ell_{v} \in [0,1]
    \
    \forall
    \
    &v \in 
    \left\{
        1, \dots, w_q
    \right\}
    ,
    \\
    \ell_{w_q} &:= 1
    ,
    \\
    \ell_0 &:= 0
    \end{align}
    is one of $w_q$ responsability spectra in a maximally stratified labor market, in which employment levels are unitary, or given by
    \begin{gather}
    \sum_{v=1}^{w_q}
    1
    =
    w_q
    ,
    \end{gather}
    so that any available position is its own job subtype and covers only a restrictive range of task difficulty, accounting for
    \begin{gather}
    \Omega_{q}^{v}
    := 
    \frac{1}{\mho_{q}^{v}}
    = 
    \int_{
        \ell_{v-1}
    }^{
        \ell_{v}
    }
    \text{ta}(l)dl
    \in [0,1]
    \end{gather}
    of an operation's total time duration,
    \begin{gather}
    \sum_{v=1}^{w_q}
    \Omega_{q}^{v} 
    := 
    \sum_{v=1}^{w_q}
    \int_{
        \ell_{v-1}
    }^{
        \ell_{v}
    }
    \text{ta}(l)dl
    =
    \int_{0}^{1}
    \text{ta}(l)dl
    :=
    1
    .
    \end{gather}
    
    Intuitively speaking, we would say production in a maximally (and monotonically) stratified labor market is not ``independent'', in the sense that employees do not work on an occupation's operation from beginning to end. This means each of them will spend all their time allowance producing a partial operational output, that is a multiple of a difficulty subinterval of complex tasks, which will, in turn, contribute, alonside the partial outputs of other employees, to accomplish[ing?] the occupational operation in its entirety.
    
    However, in a maximum labor stratification setting, these partial operational outputs will not be produced merely via ``senior'' and ``junior'' positions, as previously, but rather within a myriad of levels in a production hierarchy, approximating a continuum of ``seniority'' as the workforce becomes large enough.
    % However, in a maximally stratified labor market, these partial operational outputs will not be produced merely via ``senior'' and ``junior'' positions, as previously, but rather within a myriad of levels in a production hierarchy, approximating a continuum of ``seniority'' as the workforce becomes large enough.
    
    Again, this does not mean employees are, themselves, more or less competent, only that available job posts are preemptively stratified with respect to task difficulty, in order to maximize employers' hiring pool and safeguard production in the case workers are not sufficiently qualified to produce the whole responsability spectrum independently (see ``Maximum Labor Stratification Lemma'').    
\end{definition}
% endregion
% region: infinite stratification lemma (isl) 
Having understood what maximum labor stratification is, one may wonder whether there could be more than $w_q$ job subtypes in a labor market. For though it is intuitive to think of $w_q$, the workforce size, as the upper bound for stratification, if we allow for partial hiring, with ``fractional jobs'',
\begin{gather}
w_{q}^{v} 
\in 
[0, w_q]
\
\forall
\
v \in \{1, \dots, p\}
,
\\
\sum_{v=1}^{p}
w_{q}^{v}
:=
w_q
,
\end{gather}
where $p \in \{1, 2, 3, \dots\}$ is the number of positions in a labor market, then workers can allocate fractions of their time allowance to multiple responsability spectra, and the productive arrangement we have just defined, may not, technically speaking, be ``maximally stratified''. 

Indeed, if it were possible to stratify beyond $w_q$, rational employers would readly do so, for, again, labor stratification reduces the uncertainty around production and serves as an insurance policy to guarantee the available talent is sufficient to output an occupation's operation.

But, because of this, if $p$ can be greater than $w_q$, the optimal production strategy would, logically, be to offer as many types of jobs as possible, even infinitely many.

Hence, infinite labor stratification is defined as an economic configuration where labor markets are subdivided into infinitesimal jobs, each contributing very little to production. In fact, in such a market, ``job posts'' are so small as to be indistinguishable from tasks themselves\footnote{See the Proportional Employment Condition in ``Maximum Operational Output Lemma''.}
\begin{align}
    &
    \because
    \lim_{p \rightarrow \infty}{
        \left(
            \ell_{v}
            -
            \ell_{v-1}
        \right)
    }
    =
    0
    % \\
    % &
    \implies
    \Omega_{q}^{v}
    :=
    \int_{
        \ell_{v-1}
    }^{
        \ell_{v}
    }
    \text{ta}(l)dl
    =
    \text{ta}(l)
    \\
    &
    \land
    \lim_{p \rightarrow \infty}{
        \tilde{w}_{q}^{v}
    }
    := 
    \lim_{p \rightarrow \infty}{
        \left(
            \frac{
                w_{q}^{v}
            }{
                w_q
            }
        \right)
    }
    :=
    \tilde{w}_{q}(l)
    =
    \text{ta}(l)
    \in [0,1]
    \
    \forall
    \
    v \in \{1, \dots, p\}
    \\
    &
    \therefore
    w_{q}(l)
    = 
    w_q \times \tilde{w}_{q}(l)
    =
    w_q \times \text{ta}(l)
    \land
    \int_{0}^{1}{
        w_{q}(l)
        dl
    }
    =
    w_q
    .
    \end{align}
    
Therefore, employers are guaranteed maximum insurance against workers' potential underqualification; and employability is simply
\begin{gather}
\tilde{W}_{q}^{k} 
= 
\left[
    h_{q}^{k}
    \geq
    \frac{1}{2}
\right]
\int_{0}^{1}
T(l, \l_{q}^{k})
\tilde{w}_{q}(l)
dl
= 
\left[
    h_{q}^{k}
    \geq
    \frac{1}{2}
\right]
\int_{0}^{\tilde{T}_{q}^{k}}
\text{ta}(l)
dl
,
\end{gather}
where the hireability statistic $h_{q}^{k} \in [0,1]$ accounts for hiring requirements other than productivity; and $\tilde{w}_{q}(l)$ is the proportion of fractional positions for a particular job subtype, which coincides with its time allocation when there are infinite ``jobs'', each dedicated to a single, infinitely narrow task. We note, as well, this formula is the same as it was in binary labor stratification (with ``junior'' and ``senior'' positions). Thus, again, employability is the percentage of an operation's total duration one can accomplish. 

We may formalize this conclusion as follows.
\begin{lemma}[Infinite Stratification Lemma, ISL]
    \label{isl}
    If fractional job posts are allowed, with
    \begin{gather}
        w_{q}^{v} 
        \in 
        [0, w_q]
        \
        \forall
        \
        v \in \{1, \dots, p\}
        ,
        p \in \{1, 2, 3, \dots\}
        ,
        \\
        \sum_{v=1}^{p}
        w_{q}^{v}
        :=
        w_q
        ,
        \end{gather}
        employers' optimal choice is to infinitely split positions as infinitesimal tasks,
        \begin{gather}
            \lim_{p \rightarrow \infty}{
                \tilde{w}_{q}^{v}
            }
            := 
            \tilde{w}_{q}(l)
            =
            \text{ta}(l)
            ,
            \end{gather}
            so that employability becomes:
            \begin{gather}
                \tilde{W}_{q}^{k} 
                = 
                \left[
                    h_{q}^{k}
                    \geq
                    \frac{1}{2}
                \right]
                \int_{0}^{\tilde{T}_{q}^{k}}
                \text{ta}(l)
                dl
                .
                \end{gather}
    \begin{proof}
        See above.
    \end{proof}
\end{lemma}
% endregion

All this said, infinitely stratified markets are rather abstract, and it is not realistic to think of actual job posts as infinitesimal tasks; for, then, the very concept of a ``job'' itself disappears. Fractional positions do not make much sense in reality, where jobs usually deal with a set of multiple responsabilities. Furthermore, a maximally -- though not infinitely -- stratified labor market with sufficient positions, will, in practice, yield the same results when $w_q$ is large enough, so that we do not even need to consider infinite labor stratification as a production strategy.

% region: maximum stratification axiom (msa) 
\begin{axiom}[Maximum Stratification Axiom, MSA]
    Therefore, let us assume
    \begin{gather}
    p \in \{1, \dots, w_q\}
    ,
    \\
    \sum_{v=1}^{p}
    w_{q}^{v}
    :=
    w_q
    ,
    \end{gather}
    and
    \begin{gather}
    w_{q}^{v}
    \geq
    1
    % \in 
    % [1, w_q]
    \
    \forall
    \
    v \in \{1, \dots, p\}
    ,
    \end{gather}
    as it is somewhat arbitrary setting minimum employment levels to any value other than one; for then it would always be optimal to choose an even smaller value than that, in which case we would converge back to an infinitely stratified labor market. Thus, we define there has to be at least one worker per position.
\end{axiom}
% endregion

With this, we can now demonstrate that, given the above, maximum labor stratification is, in fact, the most efficient production strategy and, so, holds in the labor market. But, to do so, we must first derive an upper limit for aggregate operational output, irrespective of productive arrangement, to serve as our ``benchmark'' and show other strategies cannot yield higher production.

% region: maximum operational output lemma (mool) 
\begin{lemma}[Maximum Operational Output Lemma, MOOL]
    The maximum operational output of any labor market is exactly the number of employees in its workforce:
    \begin{gather}
        \mho_{q}^{*}
        =
        \mho(
            \boldsymbol{w_{q}^{*}},
            \boldsymbol{\mho_q}
        )
        = 
        \min(
            \boldsymbol{w_{q}^{*}}
            \times
            \boldsymbol{\mho_q}
        )
        = w_q
        ,
    \end{gather}
    where $\boldsymbol{w_{q}^{*}}$ is the vector of optimal employment levels in a labor market with $w_q$ employees; and $\boldsymbol{\mho_q}$, the vector of partial operational outputs. Or, assuming maximum labor stratification with unitary employment levels,
    \begin{gather}
        \mho_{q}^{*}
        =
        \mho(
            \boldsymbol{1},
            \boldsymbol{\mho_q}(
                \boldsymbol{\ell_{q}^{*}}
            )
        )
        =
        \min(
            \boldsymbol{1}
            \times
            \boldsymbol{\mho_q}(
                \boldsymbol{\ell_{q}^{*}}
            )
        )
        = w_q
        ,
    \end{gather}
    where $\boldsymbol{\ell_{q}^{*}}$ are optimal stratification bounds for the responsability spectra of occupation $q$'s job posts (see ``Optimal Stratification Lemma'' below).
    
    Moreover, when optimizing employment levels, this maximum production can only be attained when the percentage of each position relative to the entire workforce respects the Proportional Employment Condition (PEC):
    \begin{gather}
        \boldsymbol{\tilde{w}_{q}^{*}}
        :=
        \frac{
            \boldsymbol{w_{q}^{*}}
        }{
            w_q
        }
        =
        \boldsymbol{\Omega_{q}}
        ,
    \end{gather}
    which determines the ratio, or proportion, of a particular job subtype in a labor market is the percentage of an operation's total time duration,
    \begin{gather}
        \boldsymbol{1} ^ {\top}
        \cdot
        \boldsymbol{\Omega_{q}}
        := 1
        ,
    \end{gather}
    accounted by it.
    
    \begin{proof}
        We begin with the most trivial of economic configurations, that of independent production with perfectly qualified workers. In this scenario, each employee devotes their unitary time allowance, which coincides with the total time duration of occupation $q$'s operation,
        \begin{gather}
            \int_{0}^{1}{
                \text{ta}(l)
                dl
            }
            :=
            1
            ,
        \end{gather}
        to output exactly one productive unit:
        \begin{gather}
            1
            \times
            \left(
                \int_{0}^{1}{
                    \text{ta}(l)dl
                }
            \right) ^ {-1}
            =
            1
            ;
            % .
        \end{gather}
        while $w_q$ of such employees working in parallel, yield an output of
        % Therefore, $w_q$ of such employees working in parallel, yield an output of
        \begin{align}
            w_q 
            \times 
            \left(
                \int_{0}^{1}{
                    \text{ta}(l)dl
                }
            \right) ^ {-1}
            =
            w_q
            .
        \end{align}
        
        Here, we have taken occupation $q$'s responsability spectrum $l \in [0,1]$ as a whole, or as a single, ``holistic'', task, covering all its activities; and we have found the maximum amount that can be produced of it is one unit per worker, or $w_q$ aggregate units.
        
        However, it can be easier to comprehend this result if we analyze responsability spectra individually, as if a perfectly qualified, independent, employee worked on a series of tasks, which sum to their time allowance,
        \begin{gather}
            \boldsymbol{1} ^ {\top}
            \cdot
            \boldsymbol{\Omega_{q}}
            := 
            1
            .
        \end{gather}
        
        With this, we note that, as each worker's time allowance is the same as operations' total duration, failing to output any single task by overemphasizing another would nullify the whole productive effort. Hence, the optimal choice of hours to allocate to any responsability spectrum has to be the minimum time required to complete it, or
        % With this, we note that, as each worker's time allowance is the same as operations' total time requirements, failing to output any single task by overemphasizing another would nullify the whole operation, when production is independent. Hence, the optimal choice of hours to allocate to any responsability spectrum has to be the minimum time requirement to complete it, or
        \begin{gather}
            \Omega_{q}^{\ell} 
            \in
            [0,1]
            .
        \end{gather}
        
        Furthermore, by the definition of partial operational output (ref) above, one outputs $\mho_{q}^{\ell}$ when spending their unitary time allowance to produce a responsability spectrum. So, the output, with only $\Omega_{q}^{\ell}$ time units, is:
        % Furthermore, by the definition of partial operational output (ref) above, one outputs $\mho_{q}^{\ell}$ when spending their unitary time allowance to produce some responsability spectrum. So, operational output, with only $\Omega_{q}^{\ell}$ time units, is:
        % Furthermore, by the definition of partial operational output (ref) above, one outputs $\mho_{q}^{\ell}$ when spending their unitary time allowance to produce some responsability spectrum. Therefore, with only $\Omega_{q}^{\ell}$ time units,  operational output is:
        \begin{gather}
            \Omega_{q}^{\ell}
            \mho_{q}^{\ell}
            :=
            \left(
                \frac{1}{\mho_{q}^{\ell}}
            \right)
            \times
            \mho_{q}^{\ell}
            =
            1
            .
        \end{gather}
        
        Finally, as weak occupational complexity implies the production function is homothetic, the aggregate operational output of $w_q$ perfectly qualified employees working independently is:
        \begin{gather}
            \mho_{q}^{*}
            =
            \min(
                \boldsymbol{\Omega_q}
                \times
                \boldsymbol{\mho_q}
            )
            \times
            w_q
            =
            \Omega_{q}^{\ell}
            \mho_{q}^{\ell}
            \times
            w_q
            =
            1
            \times
            w_q
            =
            w_q
            .
        \end{gather}
        
        Therefore, a perfectly qualified employee working full-time and independently can output one unit of an occupation's complex tasks with one unit of their time (i.e. their entire time allowance). And, likewise, a workforce with $w_q$ employees identical to this one produces $w_q$ units of operational output. Or, to put it simply, a maximally productive person acchieves maximum production.
        
        We, now, proceed with the binary setting presented above, where employers choose a $\tilde{w}_{q}^{\text{Jr}} \in (0,1)$ percentage of less qualified (i.e. ``junior'') job posts to offer, which determines the remaining $\tilde{w}_{q}^{\text{Sr}} := 1 - \tilde{w}_{q}^{\text{Jr}} \in (0,1)$ percentage of perfectly qualified (or ``senior'') positions.
        
        In this case,
        \begin{align}
            \mho(\tilde{w}_{q}^{\text{Jr}})
            &=
            \min\left(
                \tilde{w}_{q}^{\text{Jr}}
                \times
                \mho_{q}^{\text{Jr}}
                ,
                \tilde{w}_{q}^{\text{Sr}}
                \times
                \mho_{q}^{\text{Sr}}
            \right)
            \\
            &=
            \min\left(
                \frac{
                    \tilde{w}_{q}^{\text{Jr}}
                }{
                    \int_{0}^{
                        \tilde{T}_{q}^{\text{Jr}}
                    }{
                        \text{ta}(l)
                        dl
                    }
                }
                ,
                \frac{
                    1 - \tilde{w}_{q}^{\text{Jr}}
                }{
                    \int_{
                        \tilde{T}_{q}^{\text{Jr}}
                    }^{1}{
                        \text{ta}(l)
                        dl
                    }
                }
            \right)
            \\
            &=
            \min\left(
                \frac{
                    \tilde{w}_{q}^{\text{Jr}}
                }{
                    \int_{0}^{
                        \tilde{T}_{q}^{\text{Jr}}
                    }{
                        \text{ta}(l)
                        dl
                    }
                }
                ,
                \frac{
                    1 - \tilde{w}_{q}^{\text{Jr}}
                }{
                    \int_{0}^{1}{
                        \text{ta}(l)
                        dl
                    }
                    -
                    \int_{0}^{
                        \tilde{T}_{q}^{\text{Jr}}
                    }{
                        \text{ta}(l)
                        dl
                    }
                }
            \right)
            \\
            &=
            \min\left(
                \frac{
                    \tilde{w}_{q}^{\text{Jr}}
                }{
                    \Omega_{q}^{\text{Jr}}
                }
                ,
                \frac{
                    1 - \tilde{w}_{q}^{\text{Jr}}
                }{
                    1 - \Omega_{q}^{\text{Jr}}
                }
            \right)
            ,
        \end{align}
        whereas the operational output of employing $\Omega_{q}^{\text{Jr}} \in (0,1)$ is
        % whereas the operational output of $\Omega_{q}^{\text{Jr}} \in [0,1]$ is
        \begin{align}
            \mho(\Omega_{q}^{\text{Jr}}) 
            =
            \min\left(
                \frac{
                    \Omega_{q}^{\text{Jr}}
                }{
                    \Omega_{q}^{\text{Jr}}
                }
                ,
                \frac{
                    1 - \Omega_{q}^{\text{Jr}}
                }{
                    1 - \Omega_{q}^{\text{Jr}}
                }
            \right)
            =
            \frac{
                \Omega_{q}^{\text{Jr}}
            }{
                \Omega_{q}^{\text{Jr}}
            }
            =
            \frac{
                1 - \Omega_{q}^{\text{Jr}}
            }{
                1 - \Omega_{q}^{\text{Jr}}
            }
            =
            1
            .
        \end{align}
        
        With this, if $\tilde{w}_{q}^{\text{Jr}}$ is set to $\tilde{w}_{q}^{\text{Jr}} > \Omega_{q}^{\text{Jr}}$, then
        \begin{align}
            &
            1 - \tilde{w}_{q}^{\text{Jr}}
            <
            1 - \Omega_{q}^{\text{Jr}}
            \\
            &\therefore
            \frac{
                \tilde{w}_{q}^{\text{Jr}}
            }{
                \Omega_{q}^{\text{Jr}}
            }
            >
            1
            >
            \frac{
                1 - \tilde{w}_{q}^{\text{Jr}}
            }{
                1 - \Omega_{q}^{\text{Jr}}
            }
            \\
            &\therefore
            \mho(\tilde{w}_{q}^{\text{Jr}}) 
            =
            \min\left(
                \frac{
                    \tilde{w}_{q}^{\text{Jr}}
                }{
                    \Omega_{q}^{\text{Jr}}
                }
                ,
                \frac{
                    1 - \tilde{w}_{q}^{\text{Jr}}
                }{
                    1 - \Omega_{q}^{\text{Jr}}
                }
            \right)
            =
            \frac{
                1 - \tilde{w}_{q}^{\text{Jr}}
            }{
                1 - \Omega_{q}^{\text{Jr}}
            }
            <
            1
            \\
            &\implies
            \mho(\tilde{w}_{q}^{\text{Jr}})
            <
            \mho(\Omega_{q}^{\text{Jr}})
            =
            1
            ;
        \end{align}
        and, if $\tilde{w}_{q}^{\text{Jr}} < \Omega_{q}^{\text{Jr}}$,
        \begin{align}
            &
            1 - \tilde{w}_{q}^{\text{Jr}}
            >
            1 - \Omega_{q}^{\text{Jr}}
            \\
            &\therefore
            \frac{
                \tilde{w}_{q}^{\text{Jr}}
            }{
                \Omega_{q}^{\text{Jr}}
            }
            <
            1
            <
            \frac{
                1 - \tilde{w}_{q}^{\text{Jr}}
            }{
                1 - \Omega_{q}^{\text{Jr}}
            }
            \\
            &\therefore
            \mho(\tilde{w}_{q}^{\text{Jr}}) 
            =
            \min\left(
                \frac{
                    \tilde{w}_{q}^{\text{Jr}}
                }{
                    \Omega_{q}^{\text{Jr}}
                }
                ,
                \frac{
                    1 - \tilde{w}_{q}^{\text{Jr}}
                }{
                    1 - \Omega_{q}^{\text{Jr}}
                }
            \right)
            =
            \frac{
                \tilde{w}_{q}^{\text{Jr}}
            }{
                \Omega_{q}^{\text{Jr}}
            }
            <
            1
            \\
            &\implies
            \mho(\tilde{w}_{q}^{\text{Jr}})
            <
            \mho(\Omega_{q}^{\text{Jr}})
            =
            1
            .
        \end{align}
        
        Hence, 
        % Therefore, 
        \begin{gather}
            \mho(\tilde{w}_{q}^{\text{Jr}}) < 
            \mho(\Omega_{q}^{\text{Jr}}) 
            = 1
            \
            \forall
            \
            \tilde{w}_{q}^{\text{Jr}}
            \neq
            \Omega_{q}^{\text{Jr}}
            \in 
            (0,1)
            .
        \end{gather}
        
        Analogously, with multiple job subtypes, optimal operational output is:
        \begin{gather}
            \mho(\boldsymbol{\Omega_q})
            = 
            \min(
                \boldsymbol{\Omega_q}
                \times
                \boldsymbol{\mho_q}
            )
            =
            \frac{
                \Omega_{q}^{\ell}
            }{
                \Omega_{q}^{\ell}
            }
            =
            1
            ,
        \end{gather}
        for, again, since
        \begin{gather}
            1
            =:
            \boldsymbol{1} ^ {\top}
            \cdot
            \boldsymbol{\tilde{w}_{q}}
            =
            \boldsymbol{1} ^ {\top}
            \cdot
            \boldsymbol{\Omega_{q}}
            := 1
            ,
        \end{gather}
        choosing any $\tilde{w}_{q}^{\ell} \neq \Omega_{q}^{\ell}$ implies the proportion of at least one position, say $\tilde{w}_{q}^{r}$, is impacted, and aggregate output along with it, either because
        % choosing any $\tilde{w}_{q}^{\ell} \neq \Omega_{q}^{\ell}$ implies the proportion of at least one position, say $\tilde{w}_{q}^{r}$, is impacted, and thus aggregate output along with it, either because
        % choosing any $\tilde{w}_{q}^{\ell} \neq \Omega_{q}^{\ell} \in [0,1]$ implies the proportion of at least one position, say $\tilde{w}_{q}^{r}$, is impacted, and thus aggregate output along with it, either because
        \begin{align}
            &
            \tilde{w}_{q}^{\ell} > \Omega_{q}^{\ell}
            \\
            &\therefore
            \frac{
                \tilde{w}_{q}^{\ell}
            }{
                \Omega_{q}^{\ell}
            }
            > 
            1
            > 
            \frac{
                \tilde{w}_{q}^{r}
            }{
                \Omega_{q}^{r}
            }
            \\
            &\therefore
            \mho(\boldsymbol{\tilde{w}_q})
            = 
            \min(
                \boldsymbol{\tilde{w}_q}
                \times
                \boldsymbol{\mho_q}
            )
            =
            \frac{
                \tilde{w}_{q}^{r}
            }{
                \Omega_{q}^{r}
            }
            < 
            1
            \\
            &\implies
            \mho(\boldsymbol{\tilde{w}_q})
            <
            \mho(\boldsymbol{\Omega_q})
            =
            1
            ;
        \end{align}
        or, alternatively, because
        \begin{align}
            &
            \tilde{w}_{q}^{\ell} < \Omega_{q}^{\ell}
            \\
            &\therefore
            \frac{
                \tilde{w}_{q}^{\ell}
            }{
                \Omega_{q}^{\ell}
            }
            < 
            1
            < 
            \frac{
                \tilde{w}_{q}^{r}
            }{
                \Omega_{q}^{r}
            }
            \\
            &\therefore
            \mho(\boldsymbol{\tilde{w}_q})
            = 
            \min(
                \boldsymbol{\tilde{w}_q}
                \times
                \boldsymbol{\mho_q}
            )
            =
            \frac{
                \tilde{w}_{q}^{\ell}
            }{
                \Omega_{q}^{\ell}
            }
            < 
            1
            \\
            &\implies
            \mho(\boldsymbol{\tilde{w}_q})
            <
            \mho(\boldsymbol{\Omega_q})
            =
            1
            .
        \end{align}
        
        Thus, 
        % \begin{align}
        %     % &
        %     \mho(
        %         \boldsymbol{\tilde{w}_{q}},
        %         \boldsymbol{\mho_q}
        %     ) 
        %     &< 
        %     \mho(
        %         \boldsymbol{\Omega_{q}},
        %         \boldsymbol{\mho_q}
        %     ) 
        %     = 1
        %     \\
        %     \therefore
        %     % &\therefore
        %     \mho(
        %         \boldsymbol{w_{q}},
        %         \boldsymbol{\mho_q}
        %     )
        %     &< 
        %     \mho(
        %         w_{q}
        %         \times
        %         \boldsymbol{\Omega_{q}},
        %         \boldsymbol{\mho_q}
        %     ) 
        %     = w_q
        %     \\
        %     % &
        %     \forall
        %     \
        %     \boldsymbol{\tilde{w}_{q}}
        %     \neq
        %     \boldsymbol{\Omega_{q}}
        %     \in
        %     \
        %     &
        %     [0,1]^{p}
        %     ,
        %     p \in \{1, \dots, w_q\}
        %     ,
        %     \\
        %     1 =:
        %     \boldsymbol{1} ^ {\top}
        %     \cdot
        %     \boldsymbol{\tilde{w}_{q}}
        %     &=
        %     \boldsymbol{1} ^ {\top}
        %     \cdot
        %     \boldsymbol{\Omega_{q}}
        %     := 1
        %     .
        %     % ,
        %     % \\
        %     % &
        %     % \boldsymbol{\tilde{w}_{q}}
        %     % ,
        %     % \boldsymbol{\Omega_{q}}
        %     % > 
        %     % \boldsymbol{0}
        %     % ,
        %     % \\
        %     % &
        %     % \boldsymbol{1} ^ {\top}
        %     % \cdot
        %     % \boldsymbol{\tilde{w}_{q}}
        %     % =
        %     % \boldsymbol{1} ^ {\top}
        %     % \cdot
        %     % \boldsymbol{\Omega_{q}}
        %     % = 1
        %     % .
        % \end{align}
        \begin{gather}
            \mho(
                \boldsymbol{\tilde{w}_{q}},
                \boldsymbol{\mho_q}
            ) 
            < 
            \mho(
                \boldsymbol{\Omega_{q}},
                \boldsymbol{\mho_q}
            ) 
            = 1
            \\
            \therefore
            \mho(
                \boldsymbol{w_{q}},
                \boldsymbol{\mho_q}
            )
            < 
            \mho(
                w_{q}
                \times
                \boldsymbol{\Omega_{q}},
                \boldsymbol{\mho_q}
            ) 
            = w_q
            \\
            \forall
            \
            \boldsymbol{\tilde{w}_{q}}
            \neq
            \boldsymbol{\Omega_{q}}
            \in
            [0,1]^{p}
            ,
            p \in \{1, \dots, w_q\}
            ,
            \\
            1 =:
            \boldsymbol{1} ^ {\top}
            \cdot
            \boldsymbol{\tilde{w}_{q}}
            =
            \boldsymbol{1} ^ {\top}
            \cdot
            \boldsymbol{\Omega_{q}}
            := 1
            .
        \end{gather}
        
        We can derive the same conclusion for a maximally stratified labor market, as well. But here, instead of determining the proportion of job subtypes with a $\boldsymbol{w_{q}^{*}}$ vector of employment levels, employers maximize production selecting optimal $\boldsymbol{\ell_{q}^{*}}$ responsability bounds for $w_q$ unique job posts.
        
        So, let 
        \begin{gather}
            \boldsymbol{\ell_{q}^{*}}
            :=
            (\ell_{0}^{*}, \dots, \ell_{w_q}^{*})
            :=
            (0, \dots, 1) 
            \in [0,1]^{w_q}
            ,
        \end{gather}
        with
        \begin{gather}
        \sum_{v=1}^{w_q}
        \int_{
            \ell_{v-1}^{*}
        }^{
            \ell_{v}^{*}
        }
        \text{ta}(l)dl
        =
        \int_{0}^{1}
        \text{ta}(l)dl
        :=
        1
        \end{gather}
        be the vector of optimal responsability bounds that maximizes operational output, such that
        \begin{gather}
            \mho(\boldsymbol{\ell_{q}^{*}})
            =
            \min(
                \boldsymbol{1}
                \times
                \boldsymbol{\mho_q}(
                    \boldsymbol{\ell_{q}^{*}}
                )
            )
            =
            1
            \times
            \left(
                \int_{
                    \ell_{v-1}^{*}
                }^{
                    \ell_{v}^{*}
                }
                \text{ta}(l)dl
            \right) ^ {-1}
            :=
            w_q
            ,
        \end{gather}
        as with the previous economic configurations.
        
        Note employers could, again, attempt to increase production beyond this level if they, now, reduced the responsabilities of a particular job subtype by setting
        \begin{align}
        \ell_v < \ell_{v}^{*}
        \implies
            \left(
                \int_{
                    \ell_{v-1}^{*}
                }^{
                    \ell_{v}
                }
                \text{ta}(l)dl
            \right) ^ {-1}
            >
            \left(
                \int_{
                    \ell_{v-1}^{*}
                }^{
                    \ell_{v}^{*}
                }
                \text{ta}(l)dl
            \right) ^ {-1}
            :=
            w_q
            .
        \end{align}
        
        Nevertheless, because every worker has the same unitary time allowance, this would also entail the complementary subinterval of complex tasks $l \in [\ell_{v}, \ell_{v}^{*}]$ would either not be produced at all, in which case
        \begin{align}
            \mho(\boldsymbol{\ell_{q}})
            =
            0
            \times
            \left(
                \int_{
                    \ell_{v}
                }^{
                    \ell_{v}^{*}
                }
                \text{ta}(l)dl
            \right) ^ {-1}
            =
            0
            ,
        \end{align}
        or that it would be produced with a $1 - \omega_{q}^{v} \in [0,1]$ fraction of a time unit, yielding some quantity
        \begin{align}
            \mho(
                \boldsymbol{\ell_{q}}
                ,\boldsymbol{\omega_{q}}
            )
            =
            (1 - \omega_{q}^{v})
            \times
            \left(
                \int_{
                    \ell_{v}
                }^{
                    \ell_{v}^{*}
                }
                \text{ta}(l)dl
            \right) ^ {-1}
            ,
        \end{align}
        where $\omega_{q}^{v} \in [0,1]$ is the percentage of worker $v$'s time allowance dedicated to the emphasized $l \in [\ell_{v-1}^{*}, \ell_{v}]$ responsability spectrum.
        
        Furthermore, because aggregate operational output is given by the Leontief production function, 
        \begin{gather}
            \mho_{q}
            :=
            \mho(
                \boldsymbol{1},
                \boldsymbol{\mho_q}(
                    \boldsymbol{\ell_{q}}
                    ,\boldsymbol{\omega_{q}}
                )
            )
            :=
            \min(
                \boldsymbol{1}
                \times
                \boldsymbol{\mho_q}(
                    \boldsymbol{\ell_{q}}
                    ,\boldsymbol{\omega_{q}}
                )
            )
            ,
            \\
            \boldsymbol{\mho_q}(
                \boldsymbol{\ell_{q}}
                ,\boldsymbol{\omega_{q}}
            )
            :=
            (
                \mho_{q}^{1}
                ,
                \dots
                ,
                \mho_{q}^{w_q}
            )
            ,
            \\
            \mho_{q}^{v}
            := 
            \min
            \left(
                    \frac{
                        \omega_{q}^{v}
                    }{
                        \int_{\ell_{v-1}^{*}}^{\ell_v}{
                            \text{ta}(l)dl
                        }
                    }
                    ,
                    \frac{
                        1 - \omega_{q}^{v}
                    }{
                        \int_{\ell_v}^{\ell_{v}^{*}}{
                            \text{ta}(l)dl
                        }
                    }
                \right)
            ,
        \end{gather}
        it would be pointless if only a subset of employees were to increase their operational output by themselves; for an occupation's complex tasks are all complementary: they work together to acchieve its operation. Hence, for $\mho_{q}(
            \boldsymbol{\ell_q}
            ,\boldsymbol{\omega_q}
        )$ to be greater than $
        \mho_{q}(
            \boldsymbol{\ell_{q}^{*}}
        )
        =
        w_q$,
        \begin{gather}
            \mho_{q}^{v} > w_q
            \
            \forall
            \
            v \in \{1, \dots, w_q\}
            ,
        \end{gather}
        which requires all partial operational outputs to surpass the following point of equilibrium:
        \begin{align}
            &\mho(
            % &\mho_{q}(
                \boldsymbol{\ell_q}
                ,\boldsymbol{\omega_q}
            ) 
            =
            \mho(
            % \mho_{q}(
                \boldsymbol{\ell_{q}^{*}}
            ) 
            = 
            \min(
                \boldsymbol{1}
                \times
                \boldsymbol{\mho_q}(
                    \boldsymbol{\ell_{q}^{*}}
                )
            )
            =
            w_q
            \\
            &\iff
            \frac{
                \omega_{q}^{v}
            }{
                \int_{
                    \ell_{v-1}^{*}
                }^{
                    \ell_{v}
                }{
                    \text{ta}(l)dl
                }
            }
            =
            \frac{
                1 - \omega_{q}^{v}
            }{
                \int_{
                    \ell_{v}
                }^{
                    \ell_{v}^{*}
                }{
                    \text{ta}(l)dl
                }
            }
            =
            w_q
            \
            \forall
            \
            v \in \{1, \dots, w_q\}
            \\
            &\iff
            \omega_{q}^{v}
            =
            w_q
            \int_{
                \ell_{v-1}^{*}
            }^{
                \ell_{v}
            }{
                \text{ta}(l)
                dl
            }
            \land
            1 - \omega_{q}^{v}
            =
            w_q
            \int_{
                \ell_{v}
            }^{
                \ell_{v}^{*}
            }{
                \text{ta}(l)
                dl
            }
            .
        \end{align}
        % \begin{align}
        %     &\mho_{q}(
        %         \boldsymbol{\ell_q}
        %         ,\boldsymbol{\omega_q}
        %     ) 
        %     =
        %     \mho_{q}(
        %         \boldsymbol{\ell_{q}^{*}}
        %     ) 
        %     = 
        %     \min(
        %         \boldsymbol{1}
        %         \times
        %         \boldsymbol{\mho_q}(
        %             \boldsymbol{\ell_{q}^{*}}
        %         )
        %     )
        %     =
        %     w_q
        %     \\
        %     &\iff
        %     \omega_{q}^{v}
        %     \mho_{q}^{v}(
        %         \ell_{v-1}^{*}
        %         ,\ell_{v}
        %     )
        %     =
        %     (1-\omega_{q}^{v})
        %     \mho_{q}^{v}(
        %         \ell_{v}
        %         ,\ell_{v}^{*}
        %     )
        %     =
        %     w_q
        %     \
        %     \forall
        %     \
        %     v \in \{1, \dots, w_q\}
        %     \\
        %     &\iff
        %     \omega_{q}^{v}
        %     =
        %     \frac{w_q}{
        %         \mho_{q}^{v}(
        %             \ell_{v-1}^{*}
        %             ,\ell_{v}
        %         )
        %     }
        %     :=
        %     w_q
        %     \Omega_{q}^{v}(
        %         \ell_{v-1}^{*}
        %         ,\ell_{v}
        %     )
        %     :=
        %     w_q
        %     \int_{
        %         \ell_{v-1}^{*}
        %     }^{
        %         \ell_{v}
        %     }{
        %         \text{ta}(l)
        %         dl
        %     }
        %     \\
        %     &\land
        %     1 - \omega_{q}^{v}
        %     =
        %     \frac{w_q}{
        %         \mho_{q}^{v}(
        %             \ell_{v}
        %             ,\ell_{v}^{*}
        %         )
        %     }
        %     :=
        %     w_q
        %     \Omega_{q}^{v}(
        %         \ell_{v}
        %         ,\ell_{v}^{*}
        %     )
        %     :=
        %     w_q
        %     \int_{
        %         \ell_{v}
        %     }^{
        %         \ell_{v}^{*}
        %     }{
        %         \text{ta}(l)
        %         dl
        %     }
        %     ,
        % \end{align}
        % where $\mho_{q}^{v}(\ell_{v},\ell_{r})$, with
        % \begin{gather}
        %     \mho_{q}^{v}
        %     =
        %     \min
        %     \left[
        %         \omega_{q}^{v}
        %         \mho_{q}^{v}(
        %             \ell_{v-1}^{*}
        %             ,\ell_{v}
        %         )
        %         ,
        %         (1 - \omega_{q}^{v})
        %         \mho_{q}^{v}(
        %             \ell_{v}
        %             ,\ell_{v}^{*}
        %         )
        %     \right]
        %     ,
        % \end{gather}
        % and $\Omega_{q}^{v}(\ell_{v},\ell_{r}), v, r \in \{1, \dots, w_q\}$, are short-hand notations for partial operational output and its inverse, total time allocation.
        
        Now, if any single $\omega_{q}^{v} \in [0,1]$ is set to
        \begin{gather}
            \omega_{q}^{v} > 
            w_q
            \int_{
                \ell_{v-1}^{*}
            }^{
                \ell_{v}
            }{
                \text{ta}(l)
                dl
            }
            % w_q
            % \Omega_{q}^{v}(
            %     \ell_{v-1}^{*}
            %     ,\ell_{v}
            % )
            ,
        \end{gather}
        then, indeed,
        \begin{gather}
            \frac{
                \omega_{q}^{v}
            }{
                \int_{
                    \ell_{v-1}^{*}
                }^{
                    \ell_{v}
                }{
                    \text{ta}(l)dl
                }
            }
            % \omega_{q}^{v}
            % \mho_{q}^{v}(
            %     \ell_{v-1}^{*}
            %     ,\ell_{v}
            % )
            >
            w_q
            ,
        \end{gather}
        but also
        \begin{align}
            &
            \frac{
                1 - \omega_{q}^{v}
            }{
                \int_{
                    \ell_{v}
                }^{
                    \ell_{v}^{*}
                }{
                    \text{ta}(l)dl
                }
            }
            % (1-\omega_{q}^{v})
            % \mho_{q}^{v}(
            %     \ell_{v}
            %     ,\ell_{v}^{*}
            % )
            <
            w_q
            \\
            &\implies
            \mho_{q}^{v} = 
            \min\left(
                \frac{
                    \omega_{q}^{v}
                }{
                    \int_{
                        \ell_{v-1}^{*}
                    }^{
                        \ell_{v}
                    }
                    \text{ta}(l)dl
                }
                ,
                \frac{
                    1-\omega_{q}^{v}
                }{
                    \int_{
                        \ell_{v}
                    }^{
                        \ell_{v}^{*}
                    }
                    \text{ta}(l)dl
                }
            \right)
            =
            \frac{
                1 - \omega_{q}^{v}
            }{
                \int_{
                    \ell_{v}
                }^{
                    \ell_{v}^{*}
                }{
                    \text{ta}(l)dl
                }
            }
            % \frac{
            %     1-\omega_{q}^{v}
            % }{
            %     \Omega_{q}^{v}(
            %         \ell_{v}
            %         ,\ell_{v}^{*}
            %     )
            % }
            <
            w_q
            \\
            &\therefore
            % &\implies
            \mho_{q}(
                \boldsymbol{\ell_q}
                ,\boldsymbol{\omega_q}
            ) 
            <
            \mho_{q}(
                \boldsymbol{\ell_{q}^{*}}
            )
            =
            w_q
            ;
        \end{align}
        and, conversely, 
        \begin{align}
            % &
            % \omega_{q}^{v} < 
            % w_q
            % \Omega_{q}^{v}(
            %     \ell_{v-1}^{*}
            %     ,\ell_{v}
            % )
            % \\
            &
            \omega_{q}^{v} < 
            w_q
            \int_{
                \ell_{v-1}^{*}
            }^{
                \ell_{v}
            }{
                \text{ta}(l)dl
            }
            \\
            % &\implies
            % \omega_{q}^{v}
            % \mho_{q}^{v}(
            %     \ell_{v-1}^{*}
            %     ,\ell_{v}
            % )
            % % \frac{
            % %     \omega_{q}^{v}
            % % }{
            % %     \Omega_{q}^{v}(
            % %         \ell_{v-1}^{*}
            % %         ,\ell_{v}
            % %     )
            % % }
            % < 
            % w_q
            % <
            % (1-\omega_{q}^{v})
            % \mho_{q}^{v}(
            %     \ell_{v}
            %     ,\ell_{v}^{*}
            % )
            &\implies
            \frac{
                \omega_{q}^{v}
            }{
                \int_{
                    \ell_{v-1}^{*}
                }^{
                    \ell_{v}
                }
                \text{ta}(l)dl
            }
            % \frac{
            %     \omega_{q}^{v}
            % }{
            %     \Omega_{q}^{v}(
            %         \ell_{v-1}^{*}
            %         ,\ell_{v}
            %     )
            % }
            < 
            w_q
            <
            \frac{
                1-\omega_{q}^{v}
            }{
                \int_{
                    \ell_{v}
                }^{
                    \ell_{v}^{*}
                }
                \text{ta}(l)dl
            }
            % \frac{
            %     1-\omega_{q}^{v}
            % }{
            %     \Omega_{q}^{v}(
            %         \ell_{v}
            %         ,\ell_{v}^{*}
            %     )
            % }
            \\
            &\implies
            \mho_{q}^{v} = 
            \min\left(
                \frac{
                    \omega_{q}^{v}
                }{
                    \int_{
                        \ell_{v-1}^{*}
                    }^{
                        \ell_{v}
                    }
                    \text{ta}(l)dl
                }
                ,
                \frac{
                    1-\omega_{q}^{v}
                }{
                    \int_{
                        \ell_{v}
                    }^{
                        \ell_{v}^{*}
                    }
                    \text{ta}(l)dl
                }
            \right)
            % \min\left[
            %     \frac{
            %         \omega_{q}^{v}
            %     }{
            %         \Omega_{q}^{v}(
            %             \ell_{v-1}^{*}
            %             ,\ell_{v}
            %         )
            %     }
            %     ,
            %     \frac{
            %         1-\omega_{q}^{v}
            %     }{
            %         \Omega_{q}^{v}(
            %             \ell_{v}
            %             ,\ell_{v}^{*}
            %         )
            %     }
            % \right]
            =
            \frac{
                \omega_{q}^{v}
            }{
                \int_{
                    \ell_{v-1}^{*}
                }^{
                    \ell_{v}
                }
                \text{ta}(l)dl
            }
            % \frac{
            %     \omega_{q}^{v}
            % }{
            %     \Omega_{q}^{v}(
            %         \ell_{v-1}^{*}
            %         ,\ell_{v}
            %     )
            % }
            <
            w_q
            \\
            &\therefore
            % &\implies
            \mho_{q}(
                \boldsymbol{\ell_q}
                ,\boldsymbol{\omega_q}
            ) 
            <
            \mho_{q}(
                \boldsymbol{\ell_{q}^{*}}
            )
            =
            w_q
            ;
        \end{align}
        so that 
        \begin{gather}
            \nexists
            \
            \boldsymbol{\ell_q}
            % \neq
            % \boldsymbol{\ell_{q}^{*}}
            ,
            \boldsymbol{\omega_q}
            \in
            [0,1]^{w_q}
            \
            \big|
            \
            \mho_{q}(
                \boldsymbol{\ell_q}
                ,\boldsymbol{\omega_q}
            ) 
            >
            \mho_{q}(
                \boldsymbol{\ell_{q}^{*}}
            )
            =
            % \min(
            %     \boldsymbol{1}
            %     \times
            %     \boldsymbol{\mho_q}(
            %         \boldsymbol{\ell_{q}^{*}}
            %     )
            % )
            % =
            w_q
            ,
            \\
            \sum_{v=1}^{w_q}
            \left(
                \int_{
                    \ell_{v-1}^{*}
                }^{
                    \ell_{v}
                }
                \text{ta}(l)dl
                +
                \int_{
                    \ell_{v}
                }^{
                    \ell_{v}^{*}
                }
                \text{ta}(l)dl
            \right)
            =
            \int_{0}^{1}
            \text{ta}(l)dl
            :=
            1
            .
        \end{gather}
        
        Finally, even with an $\boldsymbol{\omega_{q}^{v}}$ vector of partial time allocations for each worker, at least one difficulty subinterval would have to be neglected to emphasize another,
        \begin{gather}
            \because
            1 =:
            \boldsymbol{1} ^ \top
            \cdot
            \boldsymbol{\omega_{q}^{v}}
            =
            \boldsymbol{1} ^ \top
            \cdot
            \boldsymbol{\Omega_{q}^{v}}
            := 1
            \implies
            \min\left(
                \boldsymbol{\omega_{q}^{v}}
                \times
                \boldsymbol{\mho_q}(
                    \boldsymbol{\ell_{q}}
                )
            \right)
            <
            w_q
            \\
            \therefore
            \mho_q(
                \boldsymbol{\ell_{q}^{v}},
                \boldsymbol{\omega_{q}^{v}}
            )
            <
            \mho_q(
                \boldsymbol{\ell_{q}^{*}}
            )
            =
            w_q
            ,
        \end{gather}
        as before.
        
        Thus, we have demonstrated there cannot be, in any productive arrangement, a higher aggregate operational output than $w_q$, that is the number of employees in the workforce, as all attempts to increase production, actually, end up hindering it. 
        
        The intuition for this is quite simple. Production strategies can merely distribute the available talent across an occupation's responsability spectrum: they are but ways of splitting and organizing tasks conveniently (via independent production, or any level of labor stratification); they do not, however, change activities' time requirements, nor the time allowances of employees, both of which are, by definition, equivalent. So, these economic configurations only serve to ``safeguard'' operational output against worker's potential underqualification. The main limiting factors to production, then, are workers' capacity and time itself. Hence, we may say, somewhat tautologically, the most one can produce in a day is a ``day's work''.
    \end{proof}
\end{lemma}
% endregion

In the lemma above, we have assumed there to be an optimal $\boldsymbol{\ell_{q}^{*}}$ vector of responsability bounds maximizing operational output in stratified production. We shall, now, devote out attention to describing what such a vector would have to be like and, thus, how production is optimally stratified.

% region: optimal stratificaion lemma (osl) 
\begin{lemma}[Optimal Stratification Lemma, OSL]
    Because in a maximally and monotonically stratified labor market every position is its own job subtype (for, again, employment levels are unitary), optimal production is, then, obtained not by choosing how many workers to allocate to tasks of varying difficulty levels, but instead by setting appropriate responsability ranges for each position (i.e. which tasks to allocate \textit{to} workers). The bounds for these ranges are:
    \begin{gather}
    \ell_v
    =
    \text{TA}^{-1}\left(
        \frac{v}{w_q}
        +
        \text{TA}(0)
    \right)
    \
    \forall
    \
    v \in 
    \left\{
        1, \dots, w_q
    \right\}
    ,
    \end{gather}
    where $\text{TA}(l)$ is the anti-derivative of the time allocation function $\text{ta}(l)$, and $\text{TA}^{-1}(l)$, its inverse.
    
    \begin{proof}
        
        We have just demonstrated that the maximum operational output in any labor market, with or without unique, unitary, positions, is exactly 
        % We have just demonstrated that the maximum operational output in any labor market is exactly 
        \begin{gather}
        \mho_{q}^{*}
            =
            \min(
                \boldsymbol{w_{q}^{*}}
                \times
                \boldsymbol{\mho_q}
            )
            = 
            \min(
                \boldsymbol{1}
                \times
                \boldsymbol{\mho_q}(
                    \boldsymbol{\ell_{q}^{*}}
                )
            )
            =
            w_q
        % \mho_{q}^{*} =
        % \min\left(
        %     \boldsymbol{w_q}
        %     \times
        %     \boldsymbol{\mho_{q}}
        % \right)
        % =
        % w_q
        ,
        \end{gather}
        or the number of employees in its workforce.
        
        Therefore, optimal bounds for responsability spectra can be calculated by equating partial operational outputs with maximum production; for if maximum-monotonic labor stratification is to be optimal, it must yield the same partial outputs as any efficient production strategy.
        
        So, for the first job subtype,
        \begin{gather}
        1 \times \left(
            \int_{\ell_0}^{\ell_1}
            \text{ta}(l)
            dl
        \right) ^ {-1}
        =
        1 \times \left(
            \int_{0}^{\ell_1}
            \text{ta}(l)
            dl
        \right) ^ {-1}
        =
        w_q
        ,
        \end{gather}
        which means the partial operational output of the first worker, whose tasks range from $\ell_0 = 0$ to $\ell_1 \in [0,1]$ exclusively, should produce the same amount of the $l \in [0,\ell_1]$ responsability spectrum as would be produced in an economic configuration with maximum operational output (e.g. with $w_q$ perfectly qualified employees working independently).
        
        Thus, solving for $\ell_1$, we get:
        \begin{align}
        1 
        &\times 
        \left(
            \int_{0}^{\ell_1}
            \text{ta}(l)
            dl
        \right) ^ {-1}
        =
        w_q
        \\
        &\therefore
        \int_{0}^{\ell_1}
        \text{ta}(l)
        dl
        =
        \frac{1}{w_q}
        \\
        &\therefore
        \text{TA}(l)
        \big|_{0}^{\ell_1}
        =
        \text{TA}(\ell_1)
        -
        \text{TA}(0)
        =
        \frac{1}{w_q}
        \\
        &\therefore
        \text{TA}^{-1}(
            \text{TA}(\ell_1)
        )
        =
        \text{TA}^{-1}\left(
            \frac{1}{w_q}
            +
            \text{TA}(0)
        \right)
        \\
        &\therefore
        \ell_1
        =
        \text{TA}^{-1}\left(
            \frac{1}{w_q}
            +
            \text{TA}(0)
        \right)
        .
        \end{align}
        
        Likewise, for the second worker,
        \begin{align}
        1 
        &\times 
        \left(
            \int_{\ell_1}^{\ell_2}
            \text{ta}(l)
            dl
        \right) ^ {-1}
        =
        w_q
        \\
        &\therefore
        \int_{\ell_1}^{\ell_2}
        \text{ta}(l)
        dl
        =
        \frac{1}{w_q}
        \\
        &\therefore
        \text{TA}(l)
        \big|_{\ell_1}^{\ell_2}
        =
        \text{TA}(\ell_2)
        -
        \text{TA}(\ell_1)
        =
        \frac{1}{w_q}
        \\
        &\therefore
        \text{TA}^{-1}(
            \text{TA}(\ell_2)
        )
        =
        \text{TA}^{-1}\left(
            \frac{1}{w_q}
            +
            \text{TA}(\ell_1)
        \right)
        \\
        &\therefore
        \text{TA}^{-1}(
            \text{TA}(\ell_2)
        )
        =
        \text{TA}^{-1}\left(
            \frac{1}{w_q}
            +
            \frac{1}{w_q}
            +
            \text{TA}(0)
        \right)
        \\
        &\therefore
        \ell_2
        =
        \text{TA}^{-1}\left(
            \frac{2}{w_q}
            +
            \text{TA}(0)
        \right)
        .
        \end{align}
        
        For the third worker,
        \begin{align}
        1 
        &\times 
        \left(
            \int_{\ell_2}^{\ell_3}
            \text{ta}(l)
            dl
        \right) ^ {-1}
        =
        w_q
        \\
        &\therefore
        \int_{\ell_2}^{\ell_3}
        \text{ta}(l)
        dl
        =
        \frac{1}{w_q}
        \\
        &\therefore
        \text{TA}(l)
        \big|_{\ell_2}^{\ell_3}
        =
        \text{TA}(\ell_3)
        -
        \text{TA}(\ell_2)
        =
        \frac{1}{w_q}
        \\
        &\therefore
        \text{TA}^{-1}(
            \text{TA}(\ell_3)
        )
        =
        \text{TA}^{-1}\left(
            \frac{1}{w_q}
            +
            \text{TA}(\ell_2)
        \right)
        \\
        &\therefore
        \text{TA}^{-1}(
            \text{TA}(\ell_3)
        )
        =
        \text{TA}^{-1}\left(
            \frac{1}{w_q}
            +
            \frac{1}{w_q}
            +
            \frac{1}{w_q}
            +
            \text{TA}(0)
        \right)
        \\
        &\therefore
        \ell_3
        =
        \text{TA}^{-1}\left(
            \frac{3}{w_q}
            +
            \text{TA}(0)
        \right)
        .
        \end{align}
        
        And so on and so forth, up to the very last worker:
        \begin{align}
        1 
        &\times 
        \left(
            \int_{\ell_{w_q - 1}}^{\ell_{w_q}}
            \text{ta}(l)
            dl
        \right) ^ {-1}
        =
        w_q
        \\
        &\therefore
        \int_{\ell_{w_q - 1}}^{\ell_{w_q}}
        \text{ta}(l)
        dl
        =
        \frac{1}{w_q}
        \\
        &\therefore
        \text{TA}(l)
        \big|_{\ell_{w_q - 1}}^{\ell_{w_q}}
        =
        \text{TA}(\ell_{w_q})
        -
        \text{TA}(\ell_{w_q - 1})
        =
        \frac{1}{w_q}
        \\
        &\therefore
        \text{TA}^{-1}(
            \text{TA}(\ell_{w_q})
        )
        =
        \text{TA}^{-1}\left(
            \frac{1}{w_q}
            +
            \text{TA}(\ell_{w_q - 1})
        \right)
        \\
        &\therefore
        \text{TA}^{-1}(
            \text{TA}(\ell_{w_q})
        )
        =
        \text{TA}^{-1}\left(
            \frac{1}{w_q}
            +
            \dots
            +
            \frac{1}{w_q}
            +
            \text{TA}(0)
        \right)
        \\
        &\therefore
        \ell_{w_q}
        =
        \text{TA}^{-1}\left(
            \frac{w_q}{w_q}
            +
            \text{TA}(0)
        \right)
        := 1
        \\
        &\iff
        \text{TA}^{-1}\left(
            \frac{w_q}{w_q}
            +
            \text{TA}(0)
        \right)
        =
        \text{TA}^{-1}\left(
            1
            +
            \text{TA}(0)
        \right)
        =
        1
        \\
        &\iff
        \text{TA}\left(
            \text{TA}^{-1}\left(
            1
            +
            \text{TA}(0)
        \right)
        \right)
        =
        \text{TA}(1)
        \\
        &\iff
        \text{TA}(1)
        -
        \text{TA}(0)
        =
        \int_{0}^{1}{
            \text{ta}(l)
            dl
        }
        =
        1
        ,
        \end{align}
        which is true, by definition,
        \begin{align}
        &\because
        \text{ta}(l) 
        :=
        \text{ttc}(l)
        \times
        \left(
            \int_{0}^{1}{
                \text{ttc}(l)
                dl
            }
        \right) ^ {-1}
        \\
        % &\implies
        &\therefore
        \int_{0}^{1}{
            \text{ta}(l)
            dl
        }
        =
        \left(
            \int_{0}^{1}
            \text{ttc}(l)
            dl
        \right) ^ {-1}
        \times
        \int_{0}^{1}{
            \text{ttc}(l)
            dl
        }
        =
        1
        .
        \end{align}
        
        % And, with this condition met, we can finally arrive, by the induction above, to a general form of optimal responsability bounds:
        And, with this condition met, we can finally arrive, by the induction above, to a general form of optimal responsability ranges:
        \begin{gather}
        \ell_v
        =
        \text{TA}^{-1}\left(
            \frac{v}{w_q}
            +
            \text{TA}(0)
        \right)
        \
        \forall
        \
        v \in 
        \left\{
            1, \dots, w_q
        \right\}
        .
        \end{gather}    
    \end{proof}
\end{lemma}
% endregion

Having derived the optimal responsability ranges for a maximally stratified labor market with unitary employment levels, we shall demonstrate other types of labor stratification cannot attain in an efficient economy because: 1) no other strategy has lower productivity requirements (MPL); 2) and, given MSA, there is only one set of optimal responsability bounds and employment levels (ESL).

% region: minimum productivity lemma (mpl) 
\begin{lemma}[Minimum Productivity Lemma, MPL]
    Maximally stratified markets have the lowest barrier of entry out of all valid productive arrangements.
    
    \begin{proof}
        To show maximally stratified markets pose the lowest barrier of entry, let us consider the minimum required productivity for each economic configuration.
        
        If production is organized independently,
        \begin{gather}
        \mho_q = w_q
        \iff
        \tilde{T}_{q}^{v}
        =
        1
        \
        \forall
        \
        v \in \{1, \dots, w_q\}
        ,
        \end{gather}
        that is, either all $w_q$ employees are perfectly qualified, or maximum operational output (see MOOL) is not acchieved. Moreover, in a binary setting,
        \begin{gather}
        \mho_q = w_q
        \iff
        \tilde{T}_{q}^{v}
        \geq
        \tilde{T}_{q}^{\text{Jr}}
        \in
        [0,1)
        \
        \forall
        \
        v \in \{1, \dots, w_{q}^{\text{Jr}}\}
        \land
        \tilde{T}_{q}^{v}
        =
        1
        ,
        \end{gather}
        for the rest of the workforce (i.e. all junior employess have at least junior productivity, and all senior employees are perfectly qualified), which means productivity requirements are lower in this economic configuration, with a weighted productivity of, at least,
        \begin{gather}
        \tilde{w}_{q}^{\text{Jr}}
        \times
        \tilde{T}_{q}^{\text{Jr}}
        +
        (
            1 - \tilde{w}_{q}^{\text{Jr}}
        )
        \times
        1
        <
        1
        \because
        \tilde{T}_{q}^{\text{Jr}}
        \in
        [0,1)
        \land
        \tilde{w}_{q}^{\text{Jr}}
        \in
        (0,1)
        ,
        \end{gather}
        rather than $1 \times 1 = 1,$ with independent production. And, if there were three levels of seniority, with $w_{q}^{\text{Ir}}$ interns (less qualified than juniors and seniors), productivity requirements would be even lower:
        \begin{align}
        \tilde{w}_{q}^{\text{Ir}}
        \times
        \tilde{T}_{q}^{\text{Ir}}
        +
        \tilde{w}_{q}^{\text{Jr}}
        \times
        \tilde{T}_{q}^{\text{Jr}}
        +
        (
            1 - \tilde{w}_{q}^{\text{Jr}} - \tilde{w}_{q}^{\text{Ir}}
        )
        \times
        1
        \\
        <
        \tilde{w}_{q}^{\text{Jr}}
        \times
        \tilde{T}_{q}^{\text{Jr}}
        +
        (
            1 - \tilde{w}_{q}^{\text{Jr}}
        )
        \times
        1
        \\
        <
        1
        \\
        \because
        \tilde{T}_{q}^{\text{Ir}}
        <
        \tilde{T}_{q}^{\text{Jr}}
        \in
        [0,1)
        \land
        \tilde{w}_{q}^{\text{Ir}},
        \tilde{w}_{q}^{\text{Jr}}
        \in
        (0,1)
        \land
        \tilde{w}_{q}^{\text{Ir}}
        +
        \tilde{w}_{q}^{\text{Jr}}
        \in
        (0,1)
        .
        \end{align}
        
        And the pattern continues for all levels of labor stratification, up to the limit of $w_q$ unique positions in a maximally and monotonically stratified labor market, where minimum required productivity is the lowest it can be when infinite stratification is ruled out (MSA):
        \begin{gather}
        \sum_{v=1}^{w_q}{
            \tilde{w}_{q}^{v}
            \times
            \tilde{T}_{q}^{v}
        }
        =
        \left(
            \frac{1}{w_q}
        \right)
        \sum_{v=1}^{w_q}{
            \ell_{q}^{v}
        }
        =
        \left(
            \frac{1}{w_q}
        \right)
        \sum_{v=1}^{w_q}{
            \text{TA}^{-1}\left(
            \frac{v}{w_q}
            +
            \text{TA}(0)
        \right)
        }
        .
        \end{gather}
        
        And we could summarize this as follows:
        \begin{gather}
            \underline{\tilde{T}_{q}^{\text{IP}}}
            = 
            1
            ,
            \\
            \underline{\tilde{T}_{q}^{\text{BS}}}
            = 
            \tilde{T}_{q}^{\text{Jr}}
            <
            1
            \therefore
            \underline{\tilde{T}_{q}^{\text{BS}}}
            <
            \underline{\tilde{T}_{q}^{\text{IP}}}
            ,
            \\
            \underline{\tilde{T}_{q}^{\text{3S}}}
            = 
            \tilde{T}_{q}^{\text{Ir}}
            <
            \tilde{T}_{q}^{\text{Jr}}
            <
            1
            \therefore
            \underline{\tilde{T}_{q}^{\text{3S}}}
            <
            \underline{\tilde{T}_{q}^{\text{BS}}}
            <
            \underline{\tilde{T}_{q}^{\text{IP}}}
            ,
            \\
            \vdots
            \\
            \underline{\tilde{T}_{q}^{\text{MS}}}
            = 
            \ell_1
            <
            \dots
            <
            \tilde{T}_{q}^{\text{3S}}
            <
            \dots
            <
            1
            \therefore
            \underline{\tilde{T}_{q}^{\text{MS}}}
            <
            \dots
            <
            \underline{\tilde{T}_{q}^{\text{3S}}}
            <
            \underline{\tilde{T}_{q}^{\text{BS}}}
            <
            \underline{\tilde{T}_{q}^{\text{IP}}}
            ,
        \end{gather}
        where the underline denotes the minimum required productivity to implement each strategy, while the superscripts stand for ``independent production'' (IP), ``binary stratification'' (BS), ``three-fold stratification'' (3S), and ``maximum stratification'' (MS), respectively.
    \end{proof}
\end{lemma}
% endregion

Now that we have proven maximum labor stratification with unitary employment is the production strategy with the lowest barrier of entry, let us also show it is the only feasible mode of maximum stratification.

% region: equivalent/efficient stratification lemma (esl) 
\begin{lemma}[Efficient Stratification Lemma, ESL]
    Any efficient labor market where employers choose both $\boldsymbol{w_q}$ and $\boldsymbol{\ell_q}$ with $p \in \{1, \dots, w_q\}$ types of job posts converges to maximum labor stratification with
    $
    p = w_q
    $
    unique positions;
    \begin{gather}
    \boldsymbol{\ell_q} = 
    \boldsymbol{\ell_{q}^{*}}
    :=
    (
        \ell_{0}^{*}
        ,
        \dots
        ,
        \ell_{p}^{*}
    )
    :=
    (
        0
        ,
        \dots
        ,
        1
    )
    ,
    \\
    \ell_{v}^{*}
    =
    \text{TA}^{-1}\left(
        \frac{v}{w_q}
        +
        \text{TA}(0)
    \right)
    \
    \forall
    \
    v \in 
    \left\{
        1, \dots, w_q
    \right\}
    \end{gather}
    optimal responsability bounds; and unitary employment levels,
    $
    \boldsymbol{w_q} =
    \boldsymbol{1}
    .
    $
    
    \begin{proof}[Proof 1]
        In the MPL above, we have shown maximum stratification minimizes productivity requirements and, thus, the chance workers may not be sufficiently qualified for their responsabilities (i.e. it is the safest production strategy). So, 
        \begin{gather}
            p := w_q
        \end{gather}
        is the optimal number of positions in a labor market where employers can split workers' activities without gain or loss to production (WOCA), while attaining the same $w_q$ maximum operational output as all economic strategies (MOOL).
        % is the optimal number of positions in a labor market where employers can split workers' activities without gain or loss to production (WOCA), yielding the same $w_q$ maximum operational output as all economic strategies (MOOL).
        
        In addition, by the Maximum Stratification Axiom,
        \begin{align}
            &
            \sum_{v=1}^{p}{
                w_{q}^{v}
            }
            := 
            w_q
            \land
            w_{q}^{v} 
            \geq 
            1
            \
            \forall
            \
            v \in \{1, \dots, p\},
            p \in \{1, \dots, w_q\}
            \\
            &
            \therefore
            p := w_q
            \land
            \sum_{v=1}^{w_q}
            1
            =
            w_q
            \implies
            \sum_{v=1}^{p}{
                w_{q}^{v}
            }
            =
            \sum_{v=1}^{w_q}{
                w_{q}^{v}
            }
            =
            w_q
            \\
            &
            \iff
            w_{q}^{v}
            =
            1
            \
            \forall
            \
            v \in \{1, \dots, w_q\}
            \\
            &
            \iff
            \boldsymbol{w_q} =
            \boldsymbol{1}
            .
            \end{align}
            
        Therefore, the above implies, given the PEC and OSL,
        \begin{align}
        &
        \int_{\ell_{v-1}}^{\ell_{v}}{
            \text{ta}(l)
            dl
        }
        =
        \tilde{w}_{q}^{v} =
        \frac{1}{w_q}
        \land
        \left(
            \int_{
                \ell_{v-1}^{*}
            }^{
                \ell_{v}^{*}    
            }{
                \text{ta}(l)
                dl
            }
        \right) ^ {-1}
        :=
        w_q
        \\
        &
        \implies
        \int_{\ell_{v-1}}^{\ell_{v}}{
            \text{ta}(l)
            dl
        }
        =
        \frac{1}{
            \left(
                \int_{
                    \ell_{v-1}^{*}
                }^{
                    \ell_{v}^{*}    
                }{
                    \text{ta}(l)
                    dl
                }
            \right) ^ {-1}
        }
        =
        \int_{
            \ell_{v-1}^{*}
        }^{
            \ell_{v}^{*}    
        }{
            \text{ta}(l)
            dl
        }
        \\
        &
        \iff
        \ell_{v}
        =
        \ell_{v}^{*}
        =
        \text{TA}^{-1}\left(
            \frac{v}{w_q}
            +
            \text{TA}(0)
        \right)
        \
        \forall
        \
        v \in \{1, \dots, w_q\}
        .
        \end{align}
    \end{proof}
    
    \begin{proof}[Proof 2]
        Though simple in itself, it is easier to understand this proof iteratively. Hence, as before, we can define the number of job subtypes to be
        \begin{gather}
            p := w_q
            ,
        \end{gather}
        as this is the upper limit for labor stratification (MSA), which maximally safeguards production (MPL, MOOL); then, proceed by setting
        \begin{gather}
        w_{q}^{p} 
        := 
        w_{q}^{w_q} 
        = 
        1
        ,
        \end{gather}
        as MSA requires $w_{q}^{v} \geq 1 \ \forall \ v \in \{1, \dots, p\}$. Thus, by the Proportional Employment Condition, 
        \begin{gather}
        \int_{
            \ell_{p-1}
        }^{
            \ell_{p}
        }{
            \text{ta}(l)dl
        }
        =
        \int_{
            \ell_{w_q-1}
        }^{
            \ell_{w_q}
        }{
            \text{ta}(l)dl
        }
        :=
        \int_{
            \ell_{w_q-1}
        }^{1}{
            \text{ta}(l)dl
        }
        =
        \frac{1}{w_q}
        ,
        \end{gather}
        we estimate
        \begin{align}
            \ell_{p - 1}
            &=
            \text{TA}^{-1}\left(
                \text{TA}(1)
                -
                \frac{1}{w_q}
                \right)
            \\
            &=
            \text{TA}^{-1}\left(
                1
                +
                \text{TA}(0)
                -
                \frac{1}{w_q}
            \right)
            \\
            &=
            \text{TA}^{-1}\left(
                \frac{w_q}{w_q}
                -
                \frac{1}{w_q}
                +
                \text{TA}(0)
            \right)
            \\
            &=
            \text{TA}^{-1}\left(
                \frac{w_q - 1}{w_q}
                +
                \text{TA}(0)
            \right)
            \\
            &=
            \ell_{w_q - 1}^{*}
            \\
            &\because
            \text{TA}(1)
            -
            \text{TA}(0)
            =
            \int_{0}^{1}{
                \text{ta}(l)
                dl
            }
            =
            1
        \end{align}
        by minimizing the amount of perfectly qualified workers for efficient production. And, similarly, for the $
        l \in [
            \ell_{p - 2},
            \ell_{p - 1}
        ]
        $ responsability spectrum, if $w_{p-1} \geq 1$ is
        \begin{gather}
        w_{p-1} := w_{w_q - 1} = 1
        ,
        \end{gather}
        we minimize productivity requirements:
        \begin{align}
            &
            \int_{
                \ell_{p-2}
            }^{
                \ell_{p-1}
            }{
                \text{ta}(l)dl
            }
            =
            \int_{
                \ell_{w_q-2}
            }^{
                \ell_{w_q-1}
            }{
                \text{ta}(l)dl
            }
            =
            \frac{1}{w_q}
            \\
            &
            \implies
            \ell_{p-2}
            =
            \ell_{w_q-2}^{*}
            =
            \text{TA}^{-1}\left(
                \frac{w_q - 2}{w_q}
                +
                \text{TA}(0)
            \right)
            .
        \end{align}
        Again, $\ell_{p - 3}$ is derived by minimizing the number of highly productive workers:
        \begin{align}
            &
            \int_{
                \ell_{p-3}
            }^{
                \ell_{p-2}
            }{
                \text{ta}(l)dl
            }
            =
            \int_{
                \ell_{w_q-3}
            }^{
                \ell_{w_q-2}
            }{
                \text{ta}(l)dl
            }
            =
            \frac{1}{w_q}
            \\
            &
            \implies
            \ell_{p-3}
            =
            \ell_{w_q-3}^{*}
            =
            \text{TA}^{-1}\left(
                \frac{w_q - 3}{w_q}
                +
                \text{TA}(0)
            \right)
            ;
        \end{align}
        and, so on and so forth, always minimizing productivity requirements with $w_{q}^{v} := 1 \ \forall \ v \in \{1, \dots, w_q\}$. Notice the pattern here is the same as in the Optimal Stratification Lemma. Therefore, responsability bounds, and also employment levels, are the same as well. Hence,
        \begin{gather}
        p = w_q,
        \\
        \boldsymbol{\ell_q} = 
        \boldsymbol{\ell_{q}^{*}}
        :=
        (
            \ell_{0}^{*}
            ,
            \dots
            ,
            \ell_{p}^{*}
        )
        :=
        (
            0
            ,
            \dots
            ,
            1
        )
        ,
        \\
        \ell_{v}^{*}
        =
        \text{TA}^{-1}\left(
            \frac{v}{w_q}
            +
            \text{TA}(0)
        \right)
        \
        \forall
        \
        v \in 
        \left\{
            1, \dots, w_q
        \right\}
        ,
        \\
        \boldsymbol{w_q} =
        \boldsymbol{1}
        .
        \end{gather}    
    \end{proof}
    % [explain intution].
\end{lemma}
% endregion

Finally, let us show the economy's productivity has to be capable of supporting maximum-monotonic labor stratification.
% region: productivity sufficiency lemma (psl) 
\begin{lemma}[Productivity Sufficiency Lemma, PSL]
    The available talent in a labor market is, at least, sufficient to allow for maximally stratified production.
    
    \begin{proof}
        If talent were not sufficient to produce occupation $q$'s entire $l \in [0,1]$ responsability spectrum, then, as aggregate operation output is given by the Leontief function (see WOCA, MOOL),
        \begin{gather}
        \mho_q 
        := 
        \mho(
            \boldsymbol{w_q}
            ,
            \boldsymbol{\mho_q}(
                \boldsymbol{\ell_q}
            )
        )
        := 
        \min\left(
            \boldsymbol{w_q}
            \times
            \boldsymbol{\mho_q}(
                \boldsymbol{\ell_q}
            )
        \right)
        ,
        \end{gather}
        employers' optimal choice would be to save their resources and completely shutdown the productive effort. Therefore,
        \begin{gather}
        \lnot \
        \mho_q > 0
        \implies
        \boldsymbol{w_q} =
        \boldsymbol{0}
        \\
        \therefore
        w_q > 0 
        \implies 
        \mho_q > 0
        \iff
        \boldsymbol{w_q}
        \geq
        \boldsymbol{1}
        % \in
        % [1,w_q] ^ {p}
        ,
        \\
        \boldsymbol{1} ^ \top
        \cdot
        \boldsymbol{w_q}
        :=
        w_q
        .
        \end{gather}
        
        In other words, simply because this occupation's labor market exists we know the talent employed is sufficient to output all its responsability spectra.
        
        Furthermore, as rational employers will not overhire (ERA), for this would reduce their profit, we also know not a single position in the labor market violates the Proportional Employment Condition (see MOOL).
        
        Otherwise, employers would lay off excess workers to downscale the workforce from a suboptimal $w_q > 0$ to some $w_{q}^{*} \leq w_q$, again, to save resources. Thus, the current workforce, necessarily, has to be of the optimal
        \begin{gather}
        \sum_{v=1}^{p}
        w_{q}^{v}
        =
        w_q
        =
        w_{q}^{*}
        \end{gather}
        size and respect the PEC,
        \begin{gather}
        \tilde{w}_{q}^{v}
        =
        \Omega_{q}^{v}
        \in
        [0,1]
        ,
        \\
        \sum_{v=1}^{p}
        \Omega_{q}^{v}
        :=
        1
        \end{gather}
        at every level. Hence, we must have precisely $w_{q}^{v} = w_q \times \Omega_{q}^{v} \geq 1 \ \forall \ v \in \{1,\dots, p\}, p \in \{1,\dots, w_q\}$ employees in each position.
        
        In addition, we have ruled out infinite labor stratification (see MSA), and demonstrated any efficiently stratified labor market is characterized by the very same responsability spectra, with $w_q$ unique positions, and unitary employment (OSL, ESL). So, the labor market cannot be more than maximally stratified in accordance with Definition \ref{def_employability}. 
        
        At last, from all valid production strategies we have considered, maximum labor stratification is that which has the lowest barrier of entry, minimizing productivity requirements (MPL).
        
        Therefore, if a labor market has any employees at all, the available talent in it has to be, at least, sufficient for maximally stratified production:
        \begin{gather}
            \tilde{T}_{q}^{v}
            \geq
            % \ell_{qv}^{*}
            % =
            \text{TA}^{-1}\left(
                \frac{v}{w_q}
                +
                \text{TA}(0)
            \right)
            \
            \forall
            \
            v \in 
            \left\{
                1, \dots, w_q
            \right\}
            .
        \end{gather}
    \end{proof}
\end{lemma}
% endregion

From the above, it follows logically that maximum-monotonic labor stratification is the optimal production strategy and, so, holds in the labor market.
% region: maximum-monotonic labor stratification lemma (mlsl) 
\begin{lemma}[Maximum Labor Stratification Lemma, MLSL]
    \label{mlsl}
    The Maximum Labor Stratification Lemma (MLSL) states that a perfectly rational employer (ERA), which expects there could be skill differences in the workforce (WSDA), and can split operational output without gain or loss to production (WOCA), will, therefore, strategically stratify their job offers monotonically, and even maximally, so that, if indeed there happens to be skill differences in the labor market, they can, then, allocate less competent workers to easier roles, and avoid wasting talent, thus ``saving their best'' for the most demanding tasks.

    \begin{proof}
        We have demonstrated any productive arrangement can only yield, at most, $w_q$ units of an occupation $q$'s operation, and this if the talent employed is sufficiently qualified (MOOL).

        We have also demonstrated that, simply because a labor market exists at all, its workers' productivity has to be, at least, sufficient for maximally stratified production (PSL) when infinite stratification is ruled out (ISL, MSA).

        Furthermore, Definition \ref{complex_task} implies there is no upside to employing underqualified workers, as 
        \begin{gather}
            \tilde{T}_{q}^{k}
            <
            \tilde{T}_{q}^{v}
            \implies
            \mho_{q}^{k}
            =
            \left[
                \tilde{T}_{q}^{k}
                \geq
                \tilde{T}_{q}^{v}
            \right]
            \mho_{q}^{v}
            =
            0
            ,
        \end{gather}
        that is, if an employee cannot fully output a responsability spectrum, their contribution to production is void, limiting aggregate operational output. This means choosing any production strategy other than maximum-monotonic labor stratification is a risk with no upside, when workers may have varying productivity (WSDA). For, as we have shown, this arrangement by itself guarantees minimum productivity requirements (PSL) and maximum operational output (MOOL). So, it would be irrational of employers to organize production in another manner. Or, more succinctly, we can write for all $v \in \{1, \dots, w_q\}$:
        \begin{gather}
            \mathbb{E}[
                \mho_{q}^{\text{IP}}
                \
                |
                \
                \mathbb{E}[
                    \tilde{T}_{q}^{v}
                ]
                \in [0,1]
            ]
            \\
            \leq
            \mathbb{E}[
                \mho_{q}^{\text{BS}}
                \
                |
                \
                \mathbb{E}[
                    \tilde{T}_{q}^{v}
                ]
                \in [0,1]
            ]
            \\
            \leq
            \mathbb{E}[
                \mho_{q}^{\text{3S}}
                \
                |
                \
                \mathbb{E}[
                    \tilde{T}_{q}^{v}
                ]
                \in [0,1]
            ]
            \\
            \vdots
            \\
            \leq
            \mathbb{E}[
                \mho_{q}^{\text{MS}}
                \
                |
                \
                \mathbb{E}[
                    \tilde{T}_{q}^{v}
                ]
                \in [0,1]
            ]
            \\
            =
            \mathbb{E}[
                \mho_{q}^{\text{IP}}
                \
                |
                \
                \tilde{T}_{q}^{v}
                = 1
            ]
            =
            w_q
            ,
        \end{gather}
        where each of the terms above represents the expected value of aggregate operational output in production strategies other than infinite stratification, given the workforce's expected productivity.

        In other words, splitting responsabilities in accordance with competence is always as productive as the maximum operational output (viz. that which is obtained when employing perfectly qualified workers independently), provided employees are sufficiently qualified for their responsabilities. But, again, this is, by definition, guaranteed by employers' rationality, as well as the simple fact the economy is already producing its current operational output (see PSL). 
        
        Therefore, employing potentially underqualified workers to output the entire responsability spectrum $l \in [0,1]$ independently can only be as productive as the labor stratification strategy, but never more than it. Indepedent production, then, is a suboptimal strategy when employers expect there to be skill differences in the workforce. And the same logic also applies to less than maximally-stratified arrangements.
        
        Thus, maximum labor stratification follows as an insurance policy against worker's potential underqualification: for if talent is lacking in the labor market, there is nothing to gain by employing individuals which are not sufficiently qualified for a difficult job, whereas if talent is abundant, there is nothing to lose when employing overqualified individuals to a job below their skill level.

        Hence, given the same $w_q$ workforce, operational output in a maximally stratified labor market is always greater or equal to the output of any other economic configuration. It is, therefore, always optimal to monotonically and maximally stratify responsabilities across $w_q$ unique positions, each focused on increasingly demanding tasks.
    \end{proof}
\end{lemma}
% endregion

With this, we have shown maximally stratified production is the only efficient arrangement that holds in reality. So, we can, finally, derive a general employability coefficient by estimating employability in such markets.
% region: general employability theorem (get) 
\begin{get_theorem}[GET]
    Because maximum labor stratification is the safest and most efficient production strategy, rational employers will always choose to implement it. Therefore, an individual's employability in a maximally stratified economy,
    \begin{gather}
        \tilde{W}_k 
        =
        \left(
            \frac{1}{W}
        \right)
        \sum_{q=1}^{n}
        \left[
            h_{q}^{k}
            \geq
            \frac{1}{2}
        \right]
        \sum_{v=1}^{w_q}
        \left[
            \tilde{T}_{q}^{k}
            \geq
            \text{TA}^{-1}
            \left(
                \frac{v}{w_q}
                +
                \text{TA}(0)
            \right)
        \right]
    \end{gather}
    is their actual employability in reality.
    \begin{proof}
        We have just demonstrated (MLSL) that maximum labor stratification is the only optimal productive arrangement and, given our assumptions, attains in reality. So, one's employability in this market is their actual employability.

        Moreover, in accordance with Definition \ref{def_employability}, employability is
        \begin{gather}
            \tilde{W}_{k}
            :=
            \sum_{q=1}^{n}
            \tilde{W}_{q}^{k}
            :=
            \sum_{q=1}^{n}
            \left[
                h_{q}^{k}
                \geq
                \frac{1}{2}
            \right]
            \sum_{v=1}^{p}
            \left[
                \tilde{T}_{q}^{k}
                \geq
                \tilde{T}_{q}^{v}
            \right]
            \tilde{w}_{q}^{v}
            ,
        \end{gather}
        which in a maximally stratified market becomes (see OSL, ESL)
        \begin{align}
            \tilde{W}_{k}
            &=
            \sum_{q=1}^{n}
            \left[
                h_{q}^{k}
                \geq
                \frac{1}{2}
            \right]
            \sum_{v=1}^{w_q}
            \left[
                \tilde{T}_{q}^{k}
                \geq
                \ell_{q}^{v}
            \right]
            \left(
                \frac{1}{w_q}
            \right)
            \\
            &=
            \left(
                \frac{1}{W}
            \right)
            \sum_{q=1}^{n}
            \left[
                h_{q}^{k}
                \geq
                \frac{1}{2}
            \right]
            \sum_{v=1}^{w_q}
            \left[
                \tilde{T}_{q}^{k}
                \geq
                \text{TA}^{-1}
                \left(
                    \frac{v}{w_q}
                    +
                    \text{TA}(0)
                \right)
            \right]
            .
        \end{align}
    \end{proof}
\end{get_theorem}
% endregion
% endregion

\subsection{Corollaries}
The General Employability Theorem has a few interesting collolaries, the first, and simplest, of which is a more compact, and intuitive, version of it.
% region: simplified employability corollary 
\begin{corollary}[Simplified Employability Corollary, SEC]
    We want to show that, as with the BET and ISL, so too in a maximally and monotonically stratified labor market, employability is the percentage of an operation's total time duration one is capable of producing. Or, mathematically,
    \begin{gather}
        \tilde{W}_{q}^{k}
        =
        \int_{0}^{\tilde{T}_{q}^{k}}
        \text{ta}(l)
        dl
        :=
        \Omega_{q}^{k}
        \in 
        [0,1]
        \
        \forall
        \
        k,q \in \{1, \dots, n\}
        .
    \end{gather}

    \begin{proof}
        To prove this result, let us, then, first consider what would be the employability of person $k$ if they had exactly the minimum required productivity for every job subtype. So, for instance, when $v = 1$,
        \begin{align}
            \tilde{T}_{q}^{k}
            :=
            \ell_1
            =
            \text{TA}^{-1}
            \left(
                \frac{1}{w_q}
                +
                \text{TA}(0)
            \right)
            \implies
            \tilde{W}_{q}^{k}
            =
            \frac{1}{w_q}
            ,
        \end{align}
        as a productivity coefficient of $\tilde{T}_{q}^{k} = \ell_1$ is just enough to be hireable on the easiest job in occupation $q$'s labor market, but not on the second, much less on the remaining, more difficult, positions.
        
        Likewise, for other values of $v$, we have
        \begin{gather}
            \tilde{T}_{q}^{k}
            :=
            \ell_2
            =
            \text{TA}^{-1}
            \left(
                \frac{2}{w_q}
                +
                \text{TA}(0)
            \right)
            \implies
            \tilde{W}_{q}^{k}
            =
            \frac{2}{w_q}
            ,
            \\
            \tilde{T}_{q}^{k}
            :=
            \ell_3
            =
            \text{TA}^{-1}
            \left(
                \frac{3}{w_q}
                +
                \text{TA}(0)
            \right)
            \implies
            \tilde{W}_{q}^{k}
            =
            \frac{3}{w_q}
            ,
            \\
            \vdots
            \\
            \tilde{T}_{q}^{k}
            :=
            \ell_{v}
            =
            \text{TA}^{-1}
            \left(
                \frac{v}{w_q}
                +
                \text{TA}(0)
            \right)
            \implies
            \tilde{W}_{q}^{k}
            =
            \frac{v}{w_q}
            ,
        \end{gather}
        so that we may derive the following pattern for any $v \in \{1, \dots, w_q\}$:
        \begin{align}
            &\tilde{T}_{q}^{k}
            =
            \text{TA}^{-1}
            \left(
                \tilde{W}_{q}^{k}
                +
                \text{TA}(0)
            \right)
            \\
            &
            \therefore
            \text{TA}
            (
            % \left(
                \tilde{T}_{q}^{k}
            )
            % \right)
            =
            \text{TA}
            \left(
                \text{TA}^{-1}
                \left(
                    \tilde{W}_{q}^{k}
                    +
                    \text{TA}(0)
                \right)
            \right)
            \\
            &
            \therefore
            \text{TA}
            (
            % \left(
                \tilde{T}_{q}^{k}
            )
            % \right)
            =
            \tilde{W}_{q}^{k}
            +
            \text{TA}(0)
            \\
            &
            \therefore
            \tilde{W}_{q}^{k}
            =
            \text{TA}
            (
            % \left(
                \tilde{T}_{q}^{k}
            )
            % \right)
            -
            \text{TA}(0)
            =
            \int_{0}^{\tilde{T}_{q}^{k}}
            \text{ta}(l)
            dl
            :=
            \Omega_{q}^{k}
            \in 
            [0,1]
            ,
        \end{align}
        as we wanted to show.
        
        However, because $\tilde{T}_{q}^{k} \in [0,1]$ is not as discretized as responsability ranges $l \in [\ell_{v-1},\ell_{v}], v \in \{1, \dots, w_q\}$ are, and because rational employers do not hire insufficiently qualified employees, we must approximate $\tilde{T}_{q}^{k}$ with the closest 
        \begin{gather}
            \tilde{T}_{q}^{\kappa}
            :=
            \left(
                \frac{1}{w_q}
            \right)
            \sum_{v=1}^{w_q}
            \left[
                \tilde{T}_{q}^{k}
                \geq
                \ell_v
            \right]
        \end{gather}
        productivity estimate, such that $\tilde{T}_{q}^{k} \geq \tilde{T}_{q}^{\kappa}$, but $\tilde{T}_{q}^{k} \approx \tilde{T}_{q}^{\kappa}$, where $
        \tilde{T}_{q}^{\kappa} = \ell_{\kappa} \in \{\ell_0, \dots, \ell_{w_q}\}$ determines the most demanding task for which $k$ is still productive. Therefore, the adjusted coefficient is:
        \begin{gather}
            \tilde{W}_{q}^{k}
            =
            \int_{0}^{\tilde{T}_{q}^{\kappa}}
            \text{ta}(l)
            dl
            :=
            \Omega_{q}^{\kappa}
            \approx
            \int_{0}^{\tilde{T}_{q}^{k}}
            \text{ta}(l)
            dl
            \in 
            [0,1]
            ,
        \end{gather}
        when $w_q$ is large enough.
        
        Of course, this assumes candidate $k$ is evaluated as ``employable'' in accordance with the hireability statistic
        \begin{gather}
            \left[
                h_{q}^{k}
                \geq
                \frac{1}{2}
            \right]
            ,
        \end{gather}
        which accounts for selection criteria besides minimum required productivity. Hence, a more complete formulation would be:
        \begin{gather}
            \tilde{W}_{q}^{k}
            =
            \left[
                h_{q}^{k}
                \geq
                \frac{1}{2}
            \right]
            \int_{0}^{\tilde{T}_{q}^{\kappa}}
            \text{ta}(l)
            dl
            % \in 
            % [0,1]
            ;
        \end{gather}
        or, in the aggregate form,
        \begin{gather}
            \tilde{W}_k 
            = 
            \left(
                \frac{1}{W}
            \right)
            \sum_{q=1}^{n}
            \left[
                h_{q}^{k}
                \geq
                \frac{1}{2}
            \right]
            \int_{0}^{\tilde{T}_{q}^{\kappa}}
            \text{ta}(l)
            dl
            % \\
            :=
            \left(
                \frac{1}{W}
            \right)
            \sum_{q=1}^{n}
            \left[
                h_{q}^{k}
                \geq
                \frac{1}{2}
            \right]
            \Omega_{q}^{\kappa}
            \\
            % &
            \
            \forall
            \
            \tilde{T}_{q}^{\kappa} \in \{
                \ell_{0}, 
                \dots, 
                \ell_{w_q}
            \}
            ;
            k,q \in \{1, \dots, n\}
            .
        \end{gather}
    \end{proof}        
\end{corollary}

% endregion

In addition, the General Employability Theorem can be used to prove the General Competitiveness Corollary (GCC) with the following definition.
% region: competitiveness definition 
\begin{definition}[Competitiveness]
    \label{def_competitiveness}
    Labor market competitiveness can be defined in a variety of ways. The typical is to think of competitiveness as a ratio of job seekers to the number of available positions. Thus, we say an occupation's labor market is ``competitive'' if there are too many incumbents per job post.
    
    However, this definition has two main flaws: 1) it can be somewhat cumbersome, if not impossible, to gather all necessary data, for every labor market, under shifting conditions, to accurately assess competitiveness; 2) and even if such data are available and trustworthy, incumbents per job posts, in and of itself, is not that much of an interpretable, or at least complete, statistic.
    
    So, we may propose an additional, alternative, definition of competitiveness as the percentage of the workforce which would be \textit{willing} and \textit{able} to compete for jobs in a particular labor market:
    \begin{align}
        \tilde{\text{vs}}_{k}
        &:=
        \sum_{q=1}^{n}
        \left[
            \ddot{u}_{k}^{q}
            \geq
            \ddot{u}_{q}^{q}
        \right]
        \tilde{W}_{k}^{q}
        \\
        &:=
        \sum_{q=1}^{n}
        \left[
            \ddot{u}_{k}^{q}
            \geq
            \ddot{u}_{q}^{q}
        \right]
        \left[
            h_{k}^{q}
            \geq
            \frac{1}{2}
        \right]
        \sum_{v=1}^{p}
        \left[
            \tilde{T}_{k}^{q}
            \geq
            \tilde{T}_{k}^{v}
        \right]
        \tilde{w}_{k}^{v}
        \in
        [0,1]
        ,
        \end{align}
    where
    \begin{gather}
        \left[
            \ddot{u}_{k}^{q}
            \geq
            \ddot{u}_{q}^{q}
        \right]
        ,
        \\
        \ddot{u}_{k}^{q} 
        :=
        \text{eq}(
            \hat{u}_{k}^{q}
        )
        :=
        \text{eq}\left(
            u(
            \boldsymbol{y},
            \boldsymbol{\upsilon_q},
            \textbf{A},
            \dots
        )
        \right)
        \in
        [0,1]
        ,
    \end{gather}
    or equivalent, scaled, utility\footnote{The \textit{umlauts} are not derivatives, but a shorthand notation for the equivalence operator defined in another paper.} indicates if the relative utility of person $q$, with preferences for attributes $\boldsymbol{\upsilon_q} \in [0,1] ^ m$, working at a job with a skill set $\boldsymbol{a_q} \in [0,1] ^ m$, and earning a wage of $y_q$, would find it sufficiently more rewarding to switch to an occupation $k$, requiring a skill set $\boldsymbol{a_k} \in [0,1] ^ m$, and paying $y_k$; while the other terms are defined as previously.
    
    Thus, what the $\tilde{\text{vs}}_{k}$ (``versus'') competitiveness statistic tells us is that if workers of type $q$ are willing to compete for an occupation $k$'s job posts, they are evaluated in terms of their employability (cf. Definition \ref{def_employability}) and, if found sufficiently qualified, counted as viable incumbents. Or, in other words, competitiveness is the employability of willing and able workers from other labor markets (i.e. it is a complement of employability).
\end{definition}
% endregion

And, with this, we may derive the General Competitiveness Corollary.
% region: general competitiveness corollary (gcc)
\begin{corollary}[General Competitiveness Corollary, GCC]
    \label{gcc}
    The competitiveness of an occupation's labor market is the percentage of the aggregate workforce $W$ that is willing and able to compete for its job posts.

    \begin{proof}
        By Definition \ref{def_competitiveness} and the General Employability Theorem,
        \begin{align}
            \tilde{\text{vs}}_k 
            &:=
            \sum_{q=1}^{n}
            \left[
                \ddot{u}_{k}^{q}
                \geq
                \ddot{u}_{q}^{q}
            \right]
            \left[
                h_{k}^{q}
                \geq
                \frac{1}{2}
            \right]
            \sum_{v=1}^{p}
            \left[
                \tilde{T}_{k}^{q}
                \geq
                \tilde{T}_{k}^{v}
            \right]
            \tilde{w}_{k}^{v}
            \\
            &=
            \left(
                \frac{1}{W}
            \right)
            \sum_{q=1}^{n}
            \left[
                \ddot{u}_{k}^{q}
                \geq
                \ddot{u}_{q}^{q}
            \right]
            \left[
                h_{k}^{q}
                \geq
                \frac{1}{2}
            \right]
            \sum_{v=1}^{w_k}
            \left[
                \tilde{T}_{k}^{q}
                \geq
                \ell_v
            \right]
            \\
            &=
            \left(
                \frac{1}{W}
            \right)
            \sum_{q=1}^{n}
            \left[
                \ddot{u}_{k}^{q}
                \geq
                \ddot{u}_{q}^{q}
            \right]
            \left[
                h_{k}^{q}
                \geq
                \frac{1}{2}
            \right]
            \sum_{v=1}^{w_k}
            \left[
                \tilde{T}_{k}^{q}
                \geq
                \text{TA}^{-1}
                \left(
                    \frac{v}{w_k}
                    +
                    \text{TA}(0)
                \right)
            \right]
            .
        \end{align}
    \end{proof}
\end{corollary}
% endregion
% endregion

% region: implementation 
\section{Example Implementation}
\subsection{Functional Specifications}
\subsection{Occupational Information Network Data}
\subsection{Results}

% endregion

% region: discussion 
\section{Discussion}

% endregion

% region: conclusion 
\section{Conclusion}

% endregion

% region: references 
\newpage
\printbibliography[
    heading=bibintoc,
    title={References}
]

% endregion

% region: appendix 
\newpage
\section*{Appendix A -- Basic Definitions}
\addcontentsline{toc}{section}{Appendix A -- Basic Definitions}

\section*{Appendix B -- Employability and Competitiveness Statistics}
\addcontentsline{toc}{section}{Appendix B -- Employability and Competitiveness Statistics}

\section*{Appendix C -- Proof Layout}
\addcontentsline{toc}{section}{Appendix C -- Proof Layout}

% endregion

\end{document}