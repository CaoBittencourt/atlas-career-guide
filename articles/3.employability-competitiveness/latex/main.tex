%SECTION: SETUP 
% region: document class 
\documentclass[hidelinks, nonatbib]{elsarticle}
\makeatletter
\let\c@author\relax
\makeatother

% endregion

% region: packages 
\usepackage{mathtools}
\usepackage{amssymb}
\usepackage{amsthm}
\usepackage{float}
\usepackage{tabularray}
\usepackage{enumitem}
\setlist[enumerate]{label*=\arabic*.}
\usepackage[
    backend=biber,
    style=apa
    % ,maxalphanames=3
]{biblatex}
\usepackage{hyperref}
\usepackage{accents}

% endregion

% region: source files 
% theorems
% region: environments 
% skill sufficiency axiom (ssa)
% skills are sufficient to accomplish tasks
\newtheorem{axiom}{Skill Sufficiency Axiom}

% skill composition axiom (sca)
% skills are composable to accomplish complex tasks
\newtheorem{axiom}{Skill Composition Axiom}

% occupational reducibility axiom (ora)
% occupations can be reduced to their tasks
\newtheorem{axiom}{Occupational Reducibility Axiom}

% occupational complexity axiom (oca)
% all of an occupation's tasks can be thought of as only
% one, indivisible, holistic task, requiring their entire skill set
% we call this an occupation's "operation"
% all an occupation does is its "operation"
% define an aggregation function to compose skills an an "operation"
% operation production: evaluated aggregation function
% max production: occupation's evaluated aggregation function
% productivity: production truncated and normalized by max production
\newtheorem{axiom}{Occupational Complexity Axiom}

% occupational composition corollary (occ)
% sca, ora, oca => occ
% skill sets are composable to accomplish occupations' "operations"
\newtheorem{corollary}{Occupational Composition Corollary}

% occupational divisibility remark 1 (odr 1) on occ
% one can only accomplish the part of an occupation's operation
% that one is sufficiently qualified for
% one can accomplish parts of an operation even if productivity < 1

% occupational divisibility remark 2 (odr 2) on occ
% if one cannot accomplish parts of an occupation's operation,
% then these harder levels of the operation can be outsourced
% though a holistic operation is not divisible, difficulty levels are

% rationality axiom (ra)
% employers are rational
% employers will only pay for employees to work on tasks they can accomplish
% if required, employers will outsource parts of an occupation's operation
\newtheorem{axiom}{Rationality Axiom}

% hireability axiom (ha)
% hiring is done by evaluating a hireability statistic
\newtheorem{axiom}{Hireability Axiom}

% occupational essence corollary (oec)
% all of an occupation's job posts are essentially the same
% because all job posts require sufficient similarity,
% and because employers will not hire anyone who is not hireable,
% in accordance with the hireability statistic,
% therefore, all of an occupation's job posts, despite their accidents,
% are sufficiently recognized as a true instance of that occupation
\newtheorem{axiom}{Occupational Essence Corollary}

% labor market completeness axiom (lmca)
% there is sufficient talent in the labor market to outsource difficult tasks
% this is tautological, because the labor market is already producing its current output
\newtheorem{axiom}{Labor Market Completeness Axiom}

% task duration axiom (tda)
% define a time to completion function: task difficulty => task duration
\newtheorem{axiom}{Task Duration Axiom}

% potential employability theorem (pet)
% the potential employability of a person in a particular occupation
% is the percentage of that occupation's operation total time duration
% that their skill set allows them to accomplish
\newtheorem{theorem}{Potential Employability Theorem}

% endregion

% region: definitions 

% endregion


% equations
% equivalence function
\newcommand{\eq}{\text{eq}}

% skill set interchangeability
\newcommand{\esset}{\textit{\ss}}

% skill set interchangeability vector
\newcommand{\essetvec}{\textbf{\textit{\ss}}}

% skill set interchangeability matrix
\newcommand{\Esset}{\textbf{\ss}}

% attribute equivalence
% \newcommand{\aeq}{\textit{ä}}
\newcommand{\aeq}{{\"a}}

% linear-logistic trigonometrically-scaled equivalence estimator
\newcommand{\EquivalenceEstimator}{
% \begin{gather}
%     \label{eqn:equivalence_estimator}
%     \eq(x,M) = 
%         x\left\{
%             1 + M (1 - x) \exp[-
%                 \tan\bigg(
%                     \frac{\pi}{2}
%                     \cos
%                         \bigg(
%                             \frac{\pi}{2} 
%                             % M (1 - M)
%                             x (1 - M)
%                         \bigg)
%                 \bigg)
%                 (x - M)
%             ]
%         \right\} ^ {
%             -\frac{M}{x}
%     },\\
%     x, M \in [0,1].
% \end{gather}
% 
\begin{gather}
    \label{eqn:equivalence_estimator}
    \eq(x,M) = 
        x\left\{
            1 + M (1 - x) \exp[-
                b(x - M)
            ]
        \right\} ^ {
            -\frac{M}{x}
    },\\
    b = 
    \tan\bigg[
        \frac{\pi}{2}
        \cos^{M(1 - M)}
        % \cos^{x(1 - M)}
            \bigg(
                \frac{\pi}{2} 
                % M (1 - M)
                x (1 - M)
            \bigg)
    \bigg],\\
    x, M \in [0,1].
\end{gather}
% 
% \begin{gather}
%     \label{eqn:equivalence_estimator}
%     \eq(x,M) = 
%         x\left\{
%             1 + M (1 - x) \exp[-
%                 \tan\bigg(
%                     \frac{\pi}{2}
%                         \cos^{x(1 - M)}
%                         \bigg(
%                             \frac{\pi}{2} 
%                             x (1 - M)
%                         \bigg)
%                     \bigg)
%                 (x - M)
%             ]
%         \right\} ^ {
%             -\frac{M}{x}
%     },\\
%     x, M \in [0,1].
% \end{gather}
}

% skill set interchangeability
\newcommand{\InterchangeabilityFunction}{
    \begin{gather}
        \esset_{k,q} = 
        \esset(s(\boldsymbol{a_k}, \boldsymbol{a_q}), M) = 
        \eq(s(\boldsymbol{a_k}, \boldsymbol{a_q}), M)
    \end{gather}
}

% interchangeability matrix
\newcommand{\InterchangeabilityMatrix}{
    \begin{gather}
        \Esset = 
        \begin{bmatrix}
            \textit{\ss}_{1,1} & \hdots & \textit{\ss}_{n,1}\\
            \vdots & \ddots & \vdots\\
            \textit{\ss}_{1,n} & \hdots & \textit{\ss}_{n,n}\\
        \end{bmatrix} = 
        \begin{bmatrix}
            1 & \hdots & \textit{\ss}_{k,1} & \hdots & \textit{\ss}_{n,1}\\
            \vdots & \ddots & \vdots & \vdots & \vdots\\
            \textit{\ss}_{1,k} & \hdots & 1 & \hdots & \textit{\ss}_{n,k}\\
            \vdots & \vdots & \vdots & \ddots & \vdots\\
            \textit{\ss}_{1,n} & \hdots & \textit{\ss}_{k,n} & \hdots & 1\\
        \end{bmatrix}\\
    \end{gather}
}

% binary interchangeability (hireability)
\newcommand{\SignalFunction}{
    \begin{gather}
        h_{k,q} = 
        h(\esset_{k,q}) = 
        \begin{cases}
            1, \text{if } \esset_{k,q} \geq 0.5.\\
            0, \text{otherwise.}
        \end{cases}
    \end{gather}
}

% attribute equivalence matrix
\newcommand{\AttributeEquivalence}{
    \begin{gather}
        \aeq_{i}^{k} = 
        \aeq(\boldsymbol{a_k}, M) = 
        \eq\left(
            \frac{
                a_{i}^{k}
            }{
                \max{a_{j}^{k}}
            }
            , M
        \right)
    \end{gather}
}

% figures
\input{figures.tex}

% tables
% General Occupational Statistics
\newcommand{\GeneralOccupationalStatistics}{
    \begin{longtblr}[
        caption = {General Occupational Statistics},
        label = {tab:onet_careers},
        note{1} = {Number of SOC occupations plus variants.},
        note{2} = {Total employment levels in the United States in 2022.},
        note{3} = {Employment-weighted mean wages in the United States in 2022.},
        note{4} = {Total cluster wages as a percentage of total wages in the United States in 2022.},
        remark{Source} = {\cites{bls2022data}{onet2023data}.}
    ]{
        colspec = {X[0.5, l, m]X[0, c, m]X[0.25, c, m]X[0.25, c, m]X[0, c, m]},
        width = \linewidth,
        rowhead = 1,
        rowfoot = 0
    }
        \toprule
            Cluster & N\TblrNote{1} & Employment\TblrNote{2} & Wage\TblrNote{3} & Market Share\TblrNote{4} \\ 
        \midrule
            Business Management \& Administration &  65 & 26,639,371 & \$59,924.00 & 21.05\% \\ 
            Health Science &  99 & 16,856,404 & \$62,753.00 & 13.95\% \\ 
            Marketing &  26 & 13,961,830 & \$45,512.00 & 8.38\% \\ 
            Manufacturing & 138 & 13,644,785 & \$44,169.00 & 7.95\% \\ 
            Hospitality &  42 & 17,416,290 & \$28,965.00 & 6.65\% \\ 
            Transportation, Distribution \& Logistics &  62 & 11,018,200 & \$45,511.00 & 6.61\% \\ 
            Architecture \& Construction &  86 & 9,005,327 & \$53,142.00 & 6.31\% \\ 
            Education \& Training &  64 & 7,266,480 & \$62,621.00 & 6.00\% \\ 
            Finance &  22 & 5,255,514 & \$70,325.00 & 4.87\% \\ 
            Law, Public Safety, Corrections \& Security &  35 & 4,790,620 & \$61,985.00 & 3.92\% \\ 
            Information Technology &  22 & 3,588,065 & \$82,061.00 & 3.88\% \\ 
            Human Services &  38 & 7,823,620 & \$36,809.00 & 3.80\% \\ 
            Science, Technology, Engineering \& Mathematics &  70 & 2,207,815 & \$100,683.00 & 2.93\% \\ 
            Agriculture, Food \& Natural Resources &  40 & 2,140,244 & \$51,134.00 & 1.44\% \\ 
            Arts, Audio/Video Technology \& Communications &  40 & 1,692,270 & \$57,836.00 & 1.29\% \\ 
            Government \& Public Administration &  24 & 1,166,588 & \$62,636.00 & 0.96\% \\
        \bottomrule
    \end{longtblr}
}

% Summary of Occupations
\newcommand{\SummaryOccupations}{
    \begin{table}[H]
        \centering
            \begin{talltblr}[
            caption = {Summary of Sample Occupations},
            label = {tab:summary_occupations},
            note{1} = {``STEM" stands for ``Science, Technology, Engineering, and Mathematics".},
            note{2} = {Total employment levels in the United States in 2022.},
            note{3} = {Median wages in the United States in 2022.},
            remark{Source} = {\cites{bls2022data}{onet2023data}.}
            ]{
            colspec = {ccccc},
            % colspec = {X[c, m]Q[c, m]X[c, m]X[c, m]X[c, m]},
            width = \linewidth
            }
                \toprule
                    SOC & Occupation & Cluster & Employment\TblrNote{2} & Wage\TblrNote{3} \\ 
                \midrule
                    19-2012 & Physicists & STEM\TblrNote{1} & 18,840 & \$152,430.00 \\
                    17-2141 & Mechanical Engineers & STEM\TblrNote{1} & 277,560 & \$95,300.00 \\ 
                    13-2041 & Credit Analysts & Finance & 71,960 & \$77,440.00 \\ 
                    35-9021 & Dishwashers & Hospitality & 431,840 & \$28,130.00 \\ 
                \bottomrule
            \end{talltblr}
    \end{table}
} 

% Mechanical Engineers
\newcommand{\MatchesEngineers}{
    \begin{longtblr}[
        caption = {Best Career Matches -- Mechanical Engineers},
        label = {tab:similarity_engineers},
        remark{Note} = {Similarity scores estimated with Euclidean matching and linear weights.},
        remark{Source} = {Author's calculation based on \cite{onet2023data}.}
    ]{
        colspec = {X[l, m]X[0, c, m]},
        width = \linewidth,
        rowhead = 1,
        rowfoot = 0
    }
        \toprule
            Comparison Occupation & Similarity \\ 
        \midrule
            Mechanical Engineers & 1.00 \\ 
            Aerospace Engineers & 0.90 \\ 
            Marine Engineers and Naval Architects & 0.88 \\ 
            Nuclear Engineers & 0.88 \\ 
            Mechatronics Engineers & 0.87 \\ 
            Microsystems Engineers & 0.86 \\ 
            Photonics Engineers & 0.86 \\ 
            Fuel Cell Engineers & 0.86 \\ 
            Automotive Engineers & 0.86 \\ 
            Materials Engineers & 0.86 \\ 
            Mining and Geological Engineers, Including Mining Safety Engineers & 0.86 \\
        \bottomrule
    \end{longtblr}
}

% Physicists
\newcommand{\MatchesPhysicists}{
    \begin{longtblr}[
        caption = {Best Career Matches -- Physicists},
        label = {tab:similarity_physicists},
        remark{Note} = {Similarity scores estimated with Euclidean matching and linear weights.},
        remark{Source} = {Author's calculation based on \cite{onet2023data}.}
    ]{
        colspec = {X[l, m]X[0, c, m]},
        width = \linewidth,
        rowhead = 1,
        rowfoot = 0
    }
        \toprule
            Comparison Occupation & Similarity \\ 
        \midrule
            Physicists & 1.00 \\ 
            Astronomers & 0.85 \\ 
            Mathematicians & 0.80 \\ 
            Physics Teachers, Postsecondary & 0.80 \\ 
            Materials Scientists & 0.79 \\ 
            Engineering Teachers, Postsecondary & 0.79 \\ 
            Nanosystems Engineers & 0.78 \\ 
            Aerospace Engineers & 0.77 \\ 
            Biochemists and Biophysicists & 0.77 \\ 
            Nuclear Engineers & 0.77 \\ 
            Computer and Information Research Scientists & 0.77 \\ 
        \bottomrule
    \end{longtblr}
}

% Credit Analysts
\newcommand{\MatchesCreditAnalysts}{
    \begin{longtblr}[
        caption = {Best Career Matches -- Credit Analysts},
        label = {tab:similarity_credit_analysts},
        remark{Note} = {Similarity scores estimated with Euclidean matching and linear weights.},
        remark{Source} = {Author's calculation based on \cite{onet2023data}.}
    ]{
        colspec = {X[l, m]X[0, c, m]},
        width = \linewidth,
        rowhead = 1,
        rowfoot = 0
    }
        \toprule
            Comparison Occupation & Similarity \\ 
        \midrule
            Credit Analysts & 1.00 \\ 
            Accountants and Auditors & 0.87 \\ 
            Loan Interviewers and Clerks & 0.86 \\ 
            Budget Analysts & 0.85 \\ 
            Securities, Commodities, and Financial Services Sales Agents & 0.85 \\ 
            Financial Examiners & 0.85 \\ 
            Business Intelligence Analysts & 0.85 \\ 
            Tax Preparers & 0.85 \\ 
            Insurance Underwriters & 0.84 \\ 
            Bookkeeping, Accounting, and Auditing Clerks & 0.84 \\ 
            Tax Examiners and Collectors, and Revenue Agents & 0.83 \\ 
        \bottomrule
    \end{longtblr}
}

% Dishwashers
\newcommand{\MatchesDishwashers}{
    \begin{longtblr}[
        caption = {Best Career Matches -- Dishwashers},
        label = {tab:similarity_dishwashers},
        remark{Note} = {Similarity scores estimated with Euclidean matching and linear weights.},
        remark{Source} = {Author's calculation based on \cite{onet2023data}.}
    ]{
        colspec = {X[l, m]X[0, c, m]},
        width = \linewidth,
        rowhead = 1,
        rowfoot = 0
    }
        \toprule
            Comparison Occupation & Similarity \\ 
        \midrule
            Dishwashers & 1.00 \\ 
            Janitors and Cleaners, Except Maids and Housekeeping Cleaners & 0.88 \\ 
            Maids and Housekeeping Cleaners & 0.88 \\ 
            Pressers, Textile, Garment, and Related Materials & 0.87 \\ 
            Models & 0.87 \\ 
            Postal Service Mail Sorters, Processors, and Processing Machine Operators & 0.87 \\ 
            Sewing Machine Operators & 0.86 \\ 
            Laundry and Dry-Cleaning Workers & 0.86 \\ 
            Orderlies & 0.86 \\ 
            Cutters and Trimmers, Hand & 0.86 \\ 
            Shoe and Leather Workers and Repairers & 0.86 \\
        \bottomrule
    \end{longtblr}
}

% Similarity matrix
\newcommand{\SimilarityMatrix}{
    \begin{table}[H]
        \centering
        \begin{talltblr}[
            caption = {Similarity Matrix},
            label = {tab:similarity_matrix},
            remark{Note 1} = {Similarity scores estimated with Euclidean matching and linear weights.},
            remark{Note 2} = {Comparison occupations on the horizontal axis.},
            remark{Source} = {Author's calculation based on \cite{onet2023data}.}
        ]{
            colspec = {X[c, m]X[c, m]X[c, m]X[c, m]X[c, m]},
            width = \linewidth
        }
            \toprule
                & Mechanical Engineers & Physicists & Credit Analysts & Dishwashers \\ 
            \midrule
                Mechanical Engineers & 1.00 & 0.75 & 0.54 & 0.38 \\ 
                Physicists & 0.80 & 1.00 & 0.57 & 0.34 \\ 
                Credit Analysts & 0.75 & 0.66 & 1.00 & 0.54 \\ 
                Dishwashers & 0.62 & 0.53 & 0.66 & 1.00 \\ 
            \bottomrule
        \end{talltblr}
    \end{table}
}

% Detailed skill sets
\newcommand{\DetailedSkillSets}{
    \begin{longtblr}[
        caption = {Detailed Skill Sets},
        label = {tab:detailed_skill_sets},
        remark{Source} = {\cite{onet2023data}.}
    ]{
        % colspec = {X[1.5, c, m]X[1, c, m]X[1, c, m]X[1, c, m]X[1, c, m]},
        colspec = {X[0.4, l, m]X[0.2, c, m]X[0.2, c, m]X[0.2, c, m]X[0.2, c, m]},
        width = \linewidth,
        rowhead = 1,
        rowfoot = 0
    }
        \toprule
            Competency & Mechanical Engineers & Physicists & Credit Analysts & Dishwashers \\
        \midrule
            Engineering and Technology & 87 & 86 & 13 & 0 \\ 
            Industrial Design & 84 & 45 & 0 & 0 \\ 
            Applied Mathematics & 79 & 94 & 68 & 0 \\ 
            Mechanical & 78 & 28 & 0 & 12 \\ 
            Physics & 73 & 97 & 0 & 0 \\ 
            Oral Comprehension & 70 & 84 & 59 & 32 \\ 
            Reading Comprehension & 70 & 84 & 57 & 30 \\ 
            Oral Expression & 70 & 82 & 59 & 30 \\ 
            Deductive Reasoning & 70 & 79 & 64 & 29 \\ 
            Pure Mathematics & 70 & 79 & 59 & 14 \\ 
            Mathematical Reasoning & 68 & 84 & 59 & 16 \\ 
            Written Comprehension & 68 & 79 & 59 & 29 \\ 
            Complex Problem-Solving & 68 & 70 & 43 & 30 \\ 
            Computers and Electronics & 67 & 83 & 43 & 0 \\ 
            Natural Science & 66 & 82 & 5 & 0 \\ 
            Operations Analysis & 66 & 52 & 34 & 0 \\ 
            Active Learning & 63 & 79 & 52 & 29 \\ 
            Information Ordering & 63 & 71 & 52 & 30 \\ 
            Number Facility & 61 & 79 & 63 & 16 \\ 
            Critical Thinking & 61 & 70 & 57 & 30 \\ 
            Monitoring & 61 & 57 & 43 & 30 \\ 
            Written Expression & 59 & 80 & 57 & 23 \\ 
            Inductive Reasoning & 59 & 79 & 59 & 29 \\ 
            Category Flexibility & 59 & 73 & 50 & 29 \\ 
            Problem Sensitivity & 59 & 70 & 57 & 29 \\ 
            Judgment and Decision & 59 & 64 & 54 & 30 \\ 
            Originality & 57 & 75 & 37 & 25 \\ 
            Writing & 57 & 71 & 55 & 25 \\ 
            Native Language & 57 & 70 & 60 & 38 \\ 
            Active Listening & 57 & 68 & 57 & 32 \\ 
            Near Vision & 57 & 57 & 61 & 34 \\ 
            Administration and Management & 57 & 54 & 42 & 37 \\ 
            Quality Control Analysis & 57 & 46 & 11 & 29 \\ 
            Chemistry & 56 & 61 & 0 & 21 \\ 
            Fluency of Ideas & 55 & 73 & 41 & 27 \\ 
            Speaking & 55 & 68 & 57 & 30 \\ 
            Visualization & 55 & 61 & 25 & 29 \\ 
            Systems Analysis & 55 & 57 & 43 & 16 \\ 
            Systems Evaluation & 55 & 55 & 45 & 16 \\ 
            Speech Recognition & 54 & 57 & 50 & 30 \\ 
            Technology Design & 54 & 48 & 0 & 0 \\ 
            Operations Monitoring & 54 & 34 & 14 & 30 \\ 
            Instructing & 52 & 66 & 34 & 20 \\ 
            Time Management & 52 & 52 & 43 & 30 \\ 
            Troubleshooting & 52 & 29 & 0 & 29 \\ 
            Education and Training & 51 & 72 & 29 & 41 \\ 
            Administrative & 51 & 40 & 59 & 0 \\ 
            Learning Strategies & 50 & 71 & 36 & 18 \\ 
            Selective Attention & 50 & 57 & 41 & 30 \\ 
            Persuasion & 50 & 50 & 36 & 27 \\ 
            Coordination & 50 & 48 & 36 & 30 \\ 
            Production and Processing & 49 & 34 & 29 & 0 \\ 
            Speech Clarity & 48 & 71 & 45 & 27 \\ 
            Flexibility of Closure & 48 & 55 & 45 & 29 \\ 
            Perceptual Speed & 48 & 46 & 45 & 30 \\ 
            Customer and Personal Service & 48 & 37 & 39 & 38 \\ 
            Far Vision & 46 & 52 & 30 & 32 \\ 
            Management of Personnel Resources & 45 & 45 & 27 & 29 \\ 
            Speed of Closure & 43 & 55 & 41 & 29 \\ 
            Negotiation & 43 & 39 & 39 & 29 \\ 
            Social Perceptiveness & 41 & 45 & 41 & 29 \\ 
            Service Orientation & 41 & 43 & 43 & 29 \\ 
            Visual Color Discrimination & 41 & 41 & 21 & 29 \\ 
            Management of Material Resources & 41 & 32 & 13 & 13 \\ 
            Multitasking & 39 & 41 & 29 & 29 \\ 
            Management of Financial Resources & 39 & 30 & 27 & 11 \\ 
            Hearing Sensitivity & 39 & 30 & 16 & 29 \\ 
            Memorization & 37 & 48 & 34 & 21 \\ 
            Auditory Attention & 37 & 32 & 21 & 29 \\ 
            Telecommunications & 37 & 23 & 7 & 0 \\ 
            Equipment Selection & 37 & 21 & 0 & 29 \\ 
            Installation & 37 & 0 & 0 & 0 \\ 
            Building and Construction & 36 & 15 & 16 & 0 \\ 
            Personnel and Human Resources & 35 & 38 & 23 & 0 \\ 
            Public Safety and Security & 35 & 26 & 15 & 26 \\ 
            Programming & 34 & 55 & 14 & 0 \\ 
            Depth Perception & 34 & 30 & 5 & 29 \\ 
            Operation and Control & 34 & 9 & 0 & 30 \\ 
            Sales and Marketing & 33 & 20 & 29 & 0 \\ 
            Finger Dexterity & 32 & 29 & 25 & 34 \\ 
            Law and Government & 32 & 28 & 54 & 28 \\ 
            Reaction Time & 30 & 0 & 0 & 29 \\ 
            Transportation & 28 & 12 & 0 & 33 \\ 
            Communications and Media & 25 & 46 & 23 & 19 \\ 
            Economics and Accounting & 25 & 26 & 74 & 0 \\ 
            Arm-Hand Steadiness & 25 & 0 & 11 & 41 \\ 
            Repairing & 25 & 0 & 0 & 29 \\ 
            Equipment Maintenance & 23 & 0 & 0 & 29 \\ 
            Medicine and Dentistry & 23 & 0 & 0 & 0 \\ 
            Geography & 21 & 15 & 25 & 0 \\ 
            Biology & 20 & 16 & 0 & 0 \\ 
            Trunk Strength & 14 & 14 & 14 & 43 \\ 
            Manual Dexterity & 14 & 0 & 9 & 52 \\ 
            Wrist-Finger Speed & 9 & 0 & 0 & 21 \\ 
            Rate Control & 5 & 0 & 0 & 32 \\ 
            Spatial Orientation & 5 & 0 & 0 & 14 \\ 
            Psychology & 0 & 28 & 10 & 0 \\ 
            Therapy and Counseling & 0 & 21 & 0 & 0 \\ 
            Sociology and Anthropology & 0 & 0 & 16 & 0 \\ 
            History and Archeology & 0 & 0 & 13 & 0 \\ 
            Control Precision & 0 & 0 & 7 & 32 \\ 
            Extent Flexibility & 0 & 0 & 0 & 43 \\ 
            Static Strength & 0 & 0 & 0 & 39 \\ 
            Multilimb Coordination & 0 & 0 & 0 & 37 \\ 
            Stamina & 0 & 0 & 0 & 32 \\ 
            Gross Body Coordination & 0 & 0 & 0 & 30 \\ 
            Speed of Limb Movement & 0 & 0 & 0 & 30 \\ 
            Response Orientation & 0 & 0 & 0 & 29 \\ 
            Dynamic Strength & 0 & 0 & 0 & 25 \\ 
            Gross Body Equilibrium & 0 & 0 & 0 & 21 \\ 
            Glare Sensitivity & 0 & 0 & 0 & 21 \\ 
            Sound Localization & 0 & 0 & 0 & 14 \\ 
            Night Vision & 0 & 0 & 0 & 14 \\ 
            Peripheral Vision & 0 & 0 & 0 & 14 \\ 
            Explosive Strength & 0 & 0 & 0 & 11 \\ 
            Dynamic Flexibility & 0 & 0 & 0 & 0 \\ 
            Foreign Language & 0 & 0 & 0 & 0 \\ 
            Philosophy and Theology & 0 & 0 & 0 & 0 \\ 
            Fine Arts & 0 & 0 & 0 & 0 \\ 
            Food Production & 0 & 0 & 0 & 0 \\
        \bottomrule
    \end{longtblr}
}

% references
\addbibresource{references.bib}

% endregion

% region: display breaks 
\allowdisplaybreaks

% endregion

% region: title 
\title{
    Employability and Competitiveness\\
    in Efficient Labor Markets
}
\author{Cao Bittencourt}
\affiliation{{B. Sc. in Economics from EPGE (FGV), RJ, Brazil.}}
\affiliation{{Statistician at Atlas Career Guide Inc., FL, USA.}}
\date{\today}

% endregion

%SECTION: PAPER 
\begin{document}

% region: abstract 
\begin{abstract}
    In this article, we propose a psychometrically-based microeconomic framework to assess individuals' employability in efficiently organized labor markets as a function of their capacity on measureable professional attributes, such as those mapped by the Occupational Information Network (O*NET) in their databases. We demonstrate this value coincides precisely with the normalized duration of occupations' tasks their skill set allows them to accomplish, or what we term ``inverse operational output''. We further demonstrate labor market competitiveness to be a complement of this employability metric, given by the percentage of an occupation's job posts subject to competition with incumbent workers (i.e. competitiveness is the employability of job seekers from other fields). In our next paper, we intend to implement these two ``psychometric labor economic" constructs with O*NET occupational data to illustrate their application.
    % In this article, we propose a microeconomic framework to evaluate individuals' employability in efficiently organized labor markets as a function of their capacity on measureable professional attributes. We demonstrate employability coincides precisely with the normalized total duration of occupations' tasks their skill set allows them to accomplish, or what we term ``inverse operational output''. We further demonstrate labor market competitiveness can be defined as a complement of this metric, given by the percentage of an occupation's job posts subject to competition with incumbent workers (i.e. competitiveness is the employability of job seekers from other fields).
    % In this article, we propose a microeconomic framework to evaluate individuals' employability in efficiently organized labor markets as a function of their capacity on measureable professional attributes. We demonstrate this value coincides precisely with the normalized total duration of occupations' tasks their skill set allows them to accomplish, or what we term ``inverse operational output''. We further demonstrate labor market competitiveness can be defined as a complement of this employability metric, given by the percentage of an occupation's job posts subject to competition with incumbent workers (i.e. competitiveness is the employability of job seekers from other fields).
    % [We, then, provide an example implementation of these methods with data from the United States Occupational Information Network  (O*NET) BLS etc.]
\end{abstract}

\begin{keyword}
    employability;
    competitiveness;
    career choice;
    career development;
    vocational choice;
    Occupational Information Network;
    BLS.
    % O*NET.
\end{keyword}

% endregion

% region: title 
\maketitle

% endregion

% region: contents 
\tableofcontents

% endregion

% region: introduction 
\newpage
\section{Introduction}
This is the third in a series of articles on applied statistical methods for data-driven decision making in the context of vocational choice and career development. In our first paper \parencite{cao_intro_match}, we introduced the most basic, baseline, model for quantitative career matching \parencite[][]{repo_intro_career_matching, atlas.match} and calculated career compatibility coefficients for SOC occupations \parencite[see][]{bls2018soc} with data from the United States' primary source of occupational information, the Occupational Information Network \parencite{onet2024about}.

In the second  paper \parencite{cao_gene_comp}, we analyzed the differences between generalist and specialist workers in terms of their skill sets, and compared their overall competence as a weighted mean using an attribute equivalence statistical estimator \parencite[see][]{atlas.aeq,atlas.gene,atlas.comp, repo_gene_comp}.

We now continue this series with two new constructs to measure: 1) how hireable individuals are (again, given their skill sets); 2) and how competitive are the labor markets in which they search for jobs. However, as immediately noticeable, the content of these constructs is somewhat besides the scope of the methodology we developed. Indeed, though very much useful, the psychometric models we have so far cannot, by themselves, even begin to answer the questions proposed here; for such are the object of Economic Science, not Psychometrics.

Psychometricians tend to apply themselves to the theoretical and practical aspects of measuring psychological phenomena -- from personality \parencite[e.g.][]{epi,bigfive}, to intelligence \parencite[e.g.][]{stanford_binet5,wisc5}, and everything in between \parencite[e.g.][]{mmpi3,holland1997making,schein1993careeranchors} --, and are generally more concerned with deriving reliable and interpretable scales to describe human behavior for its own sake. That is to say, they focus on the foundational problems of measurement per se, often in a much more ``concrete'' fashion than Economists.

In a sense, then, Psychometricians are actually more ``micro-oriented'' than Microeconomists, who are among the ``least abstract'' types in Economics \parencite[cf.][]{jehle_reny_micro,romer_macro}. For even these (Microeconomists), begin their investigation of human activity on a higher, and much more abstracted, level than Psychometricians. Likewise, in the field of Labor Economics, analyses are usually restricted to macroeconomic ``heights'' \parencite[e.g.][]{romer_human_capital, barro_1991, nelson_phelps_1966, becker1994, becker1962, lucas_1988, uhr_et_al_2020, mulligan_sala_1995}, drastically simplifying social and psychological variables, which are only taken into account insofar as they allow for a better understanding of market dynamics. A complex concept like \textit{human capital}, for instance, is reduced to educational attainment \parencite[refer to][for a list of human capital proxies]{schütt_human_capital}, instead of quantified on a more detailed -- let us say, ``micro-microeconomic'' -- scale, as they would be in Psychometrics.

Of course, these procedures are both adequate for each of their respective disciplines. For sure, it makes sense to heavily abstract when dealing with macroeconomic events, just as it also makes sense to be precise when it comes to measuring, say, human intelligence. In Macroeconomics, it \textit{is} correct to assume complexity ``evens itself out'' somewhere within a lower layer of specificity. On the other hand, Psychometricians, appropriately, devote most of their time to these very realities Economists ``take for granted'', but rarely ``venture'' beyond the sphere of psychological theory into more abstract, mathematical, topics. And, similarly, the language of Economics, in turn, does not lend itself well to interdisciplinary dialogue, limiting the amount it benefits (from) other fields.

We think the isolation of these sciences keeps them from realizing some of their own applications, two of which we exemplify with our goal of deriving employability and competitiveness statistics from O*NET's psychometric data. In fact, as other researchers before us \parencite[see][]{tom_behavioral_labor_econ}, we find there to be an important, yet mostly uncharted, intersection between Psychometrics, Labor Economics, and Applied Mathematics in general, which has been termed ``Behavioral Labor Economics'' -- meaning, in this case, the ``rooting'' of labor economic models in reliable psychometric scales as an attempt to construct them ``from the ground up'', so to speak.

The presuppositions for such a theory have been scattered for decades across several academic and empirical studies \parencite[e.g.][]{winter_behavioral_labor_econ, berg_behavioral_labor_econ, onet1998data, gippity2024}. But, unfortunately, none of the currently available literature \parencite[see][]{onet_publications} concerns itself with our subject of interest or makes use of the same methods.

Thus, we find ourselves in need of developing our own ``psychometrically-grounded'' economic models, in order to accurately measure employability and competitiveness in efficient labor markets. Finally, we hope that, in the process, we may help bridge the gap between Psychometrics and Labor Economics.
% endregion

% region: presuppositions 
\newpage
\section{The Issue of Occupational Complexity}
\label{issue_of_occupational_complexity}
The basis for our analyses, in the previous two papers, has been a robust database of professional attributes, derived from O*NET's systematic surveys and psychometric evaluations of occupations' skill requirements. This means our data for estimating employability and competitiveness metrics have the form
\begin{gather}
    \textbf{A}
    :=
    \begin{bmatrix}
        a_{1}^{1} & \hdots & a_{1}^{n} \\
        \vdots    & \ddots & \vdots    \\
        a_{m}^{1} & \hdots & a_{m}^{n} \\
    \end{bmatrix}
    .
\end{gather}

We call this matrix \textbf{A} the skill set, or career profile, matrix. It is a collection of all $n$ skill sets in the economy, each of which can be described as
\begin{gather}
    \boldsymbol{a_k}
    :=
    (a_{1}^{k}, \dots, a_{m}^{k})
    \in
    [0,1] ^ m
    ,
\end{gather}
where $a_{i}^{k}$ is $i$-th professional attribute of a person $k$, while $\boldsymbol{a_k}$ is their skill set. The full list of these attributes can be found in \textcite{onet2023data} and includes skills such as: ``writing'', ``memorization'', ``programming'', ``management of financial resources'', ``stamina'', ``trunk strength'', ``food production'', ``Fine Arts'', etc.

Now, though the O*NET data are quite trustworthy, it is plain to see we cannot directly infer from them what individuals' employability and labor markets' competitiveness might be. And the main reason for this \textit{occupational complexity}.

The issue of ``occupational complexity'' is the term we use to say occupations are composed of a multitude of activities, some simple, requiring only one of their attributes, and some which depend on more than a single skill. The problem is we do not know exactly which of their tasks are the most prevalent and how much of their time is spent on each of them.

Therefore, we cannot assess our labor economic metrics from the data alone. So, the only option for calculating employability and competitiveness is to solve occupational complexity by means of a logical and sound theorical framework.

\subsection{Professional Attributes and Tasks}
% region: simple task definition 
\begin{definition}[Simple Task]
    \label{def_task}
    To begin our investigation, we define $T_i(l)$ as a task of difficulty level $l \in [0,1]$ which requires only competence in one professional attribute $i \in \{1, \dots, m\}$ and
    \begin{gather}
        T(l,\ell_{i}^{k})
        :=
        [\ell_{i}^{k} \geq l]
        :=
        \begin{cases}
            1, \text{ if } \ell_{i}^{k} \geq l,
            \\
            0, \text{ if } \ell_{i}^{k} < l,
        \end{cases}
    \end{gather}
    as its binary outcome indicator, given a person $k$'s capacity to execute attribute $i$-related tasks, so that $T(l,\ell_{i}^{k})$ yields one if $k$ is sufficiently qualified at difficulty level $l$; or zero, otherwise.
\end{definition}
% endregion
% region: scale uniformity  
\begin{axiom}[Scale Uniformity Axiom, SUA]
    \label{sua}
    Let us also assume
    \begin{gather}
        L \sim U(0,1)
        ,
    \end{gather}
    in order for difficulty levels to be equally represented throughout the scale.
    
    Note, however, scale uniformity is not so much an axiom, but more of a convenient notation. In fact, if task difficulty is not uniformly distributed for a particular attribute, then, given such uniform scales, its underlying distribution will affect our estimation of productivity statistics (see Definition \ref{def_prod}) and, therefore, the assessed workforce's talent overall.
    
    That is, if an attribute's tasks concentrate on the lower end of the $l \in [0,1]$ scale, then most people will be fairly competent in it; and vice-versa if tasks concentrate on the higher end of the difficulty scale.
    
    But, mathematically, it is far simpler to work with uniform scales and shift difficulty levels' probabilistic complexity, if any, to the \textit{talent} (i.e. productivity) distribution; hence, we pressupose scale uniformity.
\end{axiom}
% endregion
With this, we can define productivity in a more convenient way.
% region: productivity distribution 
\begin{definition}[Productivity]
    \label{def_prod}
    A person $k$'s productivity in any set of tasks is the percentage of tasks for which they are sufficiently qualified,
    \begin{gather}
        \tilde{T}_{i}^{k}
        :=
        \int_{0}^{1}
        \text{pdf}(l)
        \times
        T(l, \ell_{i}^{k})
        dl
        \in 
        [0,1]
        ,
    \end{gather}
    which we can simplify with scale uniformity,
    \begin{align}
        &
        \because
        L \sim U(0,1)
        \implies
        \text{pdf}(l)
        =
        \frac{1}{1-0}
        =
        1
        \
        \forall
        \
        l \in [0,1]
        \\
        &
        \therefore
        \tilde{T}_{i}^{k}
        =
        \int_{0}^{1}
        T(l, \ell_{i}^{k})
        dl
        .
    \end{align}
\end{definition}

% endregion

We proceed, now, to reframe the concepts of skills and skill sets we have just spoken about in terms of tasks.
% region: attribute reducibility observation (aro) 
% attributes can be reduced to their tasks
\begin{observation}[Attribute Reducibility]
    \label{aro}
    % statement
    A professional attribute, or skill, is reducible to its activities, for it is only a quantitatively meaningful category insofar as it describes individuals' success in accomplishing tasks.
\end{observation}
% endregion

% region: attribute transitivity observation (ato) 
% attributes can be reduced to their tasks
\begin{observation}[Attribute Transitivity]
    \label{ato}
    % statement
    Furthermore,
    \begin{gather}
        \ell_{i}^{k}
        \geq
        \ell_{i}^{q}
        \in
        [0,1]
        \implies 
        T(l,\ell_{i}^{k}) = 1
        \
        \forall
        \
        l
        \leq
        \ell_{i}^{q}
        \because
        T(l,\ell_{i}^{k})
        :=
        [\ell_{i}^{k} \geq l]
        ;
    \end{gather}
    that is, capacity to execute tasks has the transitive property. So, it would make no sense to measure one's professional attributes with anything other than the maximum difficulty of tasks one can accomplish.
\end{observation}
% endregion

% region: skill definition 
\begin{definition}[Skill]
    \label{def_skill}
    Hence, an individual $k$'s skill, or competence, in some professional attribute $i \in \{1, \dots, m\}$ is given by the difficulty of its most difficult task they can accomplish,
    \begin{gather}
        a_{i}^{k}
        :=
        \ell_{i}^{k}
        \in
        [0,1]
        ,
    \end{gather}
    while their skill set is a vector of $m$ attributes,
    \begin{gather}
        \boldsymbol{a_k}
        :=
        (
            a_{1}^{k}
            ,
            \dots
            ,
            a_{m}^{k}
        )
        :=
        (
            \ell_{1}^{k}
            ,
            \dots
            ,
            \ell_{m}^{k}
        )
        \in
        [0,1] ^ m
        .
    \end{gather}
\end{definition}
% endregion

Notice, though, we have above implicitly assumed each attribute measures only one thing. Let us, then, make this assumption explicit.
% region: attribute atomicity axiom (aaa)
\begin{axiom}[Attribute Atomicity Axiom, AAA]
    \label{aaa}
    % statement
    Professional attributes and their tasks are well-defined. Thus, a skill is an ``atomic'' collection of tasks, which may or may not be correlated with other skills, but which, in practice, cannot be further reduced to any elementary ``particle''.
    
    Indeed, if skills were not well-defined in this sense, each and every one of them would entail a regress into microbiological, chemical and, quite literally \textit{atomic}, phenomena so as to be properly defined by indivisible, bedrock ``subskills''. For instance, when we say an individual is good at ``deductive reasoning'' (an attribute in \cite{onet2023data}), we could also describe their intelligence in terms of underlying neurological structures in the brain and, similarly, in the cellular level, within the neurons themselves, etc. But this would not help our analysis.

    We assume, then, attributes are ``atomic'' or, if not so, that all their ``subattributes'' are perfectly correlated to the main, ``high-level'', abstraction which we call a ``skill''. Or, mathematically,
    \begin{gather}
        a_{\iota_i}^{k}
        =
        a_{i}^{k}
        \
        \forall
        \
        \iota_i
        \in 
        \{1, 2, 3, \dots\},
        i \in \{1, \dots, m\}
        ,
    \end{gather}
    where subattributes' difficulty scales follow the uniform distribution (SUA).
\end{axiom}
% endregion

Given the above, we have our first stepping stone towards a general theorem of employability and competitiveness in efficient labor markets. It is a rather tautological lemma, but shall guide our intuition to more complicated results.
% region: skill sufficiency lemma (ssl) 
% skills are sufficient to accomplish tasks
% statement
\begin{lemma}[Skill Sufficiency Lemma, SSL]
    \label{ssl}
    According to the SSL, skills are necessary and sufficient to accomplish tasks.
    In particular, to have a skill level of $a_{i}^{k} \in [0,1]$ in attribute $i$ implies one is capable of accomplishing the easier
    \begin{gather}
        % \tilde{T}_{k}^{i}
        % \tilde{T}_{ki}
        % \tilde{T}_{ik}
        \tilde{T}_{i}^{k}
        :=
        \int_{0}^{1}
        T(l, \ell_{i}^{k})
        % T(l, \ell_{k}^{i})
        dl
        =
        a_{i}^{k}
        \in
        [0,1]
    \end{gather}
    portion of that attribute's tasks.
\end{lemma}

% proof
\begin{proof}
    By Definitions \ref{def_task}, \ref{def_prod}, \ref{def_skill}, as well as Axioms \ref{sua} and \ref{aaa},
    \begin{align}
        &
        \tilde{T}_{i}^{k}
        :=
        \int_{0}^{1}
        T(l, \ell_{i}^{k})
        dl
        :=
        \int_{0}^{1}
        [
            \ell_{i}^{k}
            \geq
            l
        ]
        dl
        =
        \int_{0}^{\ell_{i}^{k}}
        1
        \times
        dl
        +
        \int_{\ell_{i}^{k}}^{1}
        0
        \times
        dl
        =
        l\big|_{0}^{\ell_{i}^{k}}
        + 0
        \\
        &
        \therefore
        a_{i}^{k}
        :=
        \ell_{i}^{k}
        \implies
        \tilde{T}_{i}^{k}
        =
        l\big|_{0}^{\ell_{i}^{k}}
        =
        a_{i}^{k}
        - 0
        =
        a_{i}^{k}
        \in
        [0,1]
        .
    \end{align}

    Therefore, because competencies are atomic and their difficulty, uniformly distributed, the percentage of tasks one can accomplish coincides precisely with their skill level. That is, to be skilled in an attribute is, really, the same thing as being able to perform its activities.
    
    Indeed, we can even conceptualize a person $k$'s $a_{i}^{k}$ skill not only as the level of the hardest task they accomplish of attribute $i$, but also as a sum of successful Bernoulli trials on a $T_i(l) \ \forall \ l \in [0,1]$ series of infinitely small tasks with increasing difficulty.
    
    Or, intuitively speaking, we could say: the capacity to act follows virtue, for virtue is, itself, the capacity to act.
\end{proof}
% endregion

Our next step is generalizing the SSL for tasks that require multiple skills.
% region: complex task definition 
% statement
\begin{definition}[Complex Task]
    \label{complex_task}
    A task is said to be complex if it relies on more than one attribute to be accomplished. More precisely, $T_{ij}(l)$ is a complex task of attributes $i$ and $j$ if its binary outcome indicator is of the form
    \begin{gather}
        T(l, \ell_{ij}^{k})
        :=
        [\ell_{ij}^{k} \geq l],
    \end{gather}
    where
    \begin{gather}
        \ell_{ij}^{k}
        :=
        f(\ell_{i}^{k}, \ell_{j}^{k})
        ,
        f
        :
        [0,1]^2
        \rightarrow
        [0,1]
        ,
    \end{gather}
    is a strictly increasing aggregation function that returns the maximum difficulty level of the complex task $T_{ij}(l)$ a person $k$ can accomplish based on each attribute $T_{ij}(l)$ depends upon. And, for any complex task $T_{q}(l)$ of $m$ attributes, requiring an entire skill set $\boldsymbol{a_q} := (a_{1}^{q}, \dots, a_{m}^{q})$ to be accomplished, 
    \begin{gather}
        T(l, \ell_{q}^{k})
        :=
        [\ell_{q}^{k} \geq l],
    \end{gather}
    where
    \begin{gather}
        \ell_{q}^{k}
        := 
        f(\boldsymbol{\ell_k}, \boldsymbol{\ell_q})
        :=
        f\big(
        % f(
            (\ell_{1}^{k}, \dots, \ell_{m}^{k})
            ,
            (\ell_{1}^{q}, \dots, \ell_{m}^{q})
            % \ell_{1}^{k}, \dots, \ell_{m}^{k}
            % ;
            % \ell_{1}^{q}, \dots, \ell_{m}^{q}
        \big)
        % )
        ,
        f
        :
        [0,1]^{2m}
        \rightarrow
        [0,1]
        ,
    \end{gather}
    and 
    \begin{gather}
        \frac{
            \partial
            f(\boldsymbol{\ell_k}, \boldsymbol{\ell_q})
        }{
            \partial 
            \ell_{i}^{k}
        } > 0
        \land
        \frac{
            \partial
            f(\boldsymbol{\ell_k}, \boldsymbol{\ell_q})
        }{
            \partial 
            \ell_{i}^{q}
        } < 0
        \
        \forall 
        \
        i \in \{1, \dots, m\}
        .
    \end{gather}
        
    This means none of the attributes required by the complex task are completely disposable (i.e. they are all helpful in some way).
    
    \begin{subdefinition}[Weak Complexity]
        Now, beyond these most basic rules, we can define stricter versions of ``task complexity'' with additional assumptions. The first version, of weak complexity, requires that
        \begin{gather}
            \frac{
                \partial ^ 2 {
                    f(\boldsymbol{\ell_k}, \boldsymbol{\ell_q})
                }
            }{
                \partial 
                \ell_{i}^{k}
                \partial
                \ell_{j}^{k}
            }
            > 0
            \
            \forall
            \
            i \neq j \in \{1, \dots, m\}
            ,
        \end{gather}
        meaning attributes are all complementary.
    \end{subdefinition}
    
    \begin{subdefinition}[Moderate Complexity]
        A task is of moderate complexity if its aggregation function also meets the following criteria:
        \begin{gather}
            \lim_{
                \ell_{i}^{k}
                \rightarrow 0
            }{
                f(\boldsymbol{\ell_k}, \boldsymbol{\ell_q})
            }
            = 0
            \
            \forall 
            \
            i \in \{1, \dots, m\}
            ,
        \end{gather}
        so that a person $k$'s capacity to perform the complex task is weakly increasing on their capacity to perform the simple tasks of its required attributes, and goes to zero when they are unskilled in at least one of them. Thus, a moderately complex task is not reducible to any proper subset of its attributes.
        
        Notice this does not imply there cannot be a degree of substitution between attributes. That is, moderate complexity only means a task must require all its attributes in \textit{some} level, even if its functional form allows for skill substitution.
    \end{subdefinition}
    
    \begin{subdefinition}[Strong Complexity]
        The strictest definition of task complexity adds the constraint that skills are aggregated by the Leontief function,
        \begin{gather}
            f(\boldsymbol{\ell_k}, \boldsymbol{\ell_q})
            :=
            \min(\boldsymbol{\ell_k})
            ,
        \end{gather}
        or a variant of it. Hence, having additional skills does not help to accomplish a task, but being unskilled in a single attribute can undermine the whole effort.
    \end{subdefinition}

\end{definition}
% endregion
All this said, our employability theorem and its corollaries do not require any specific level of task complexity. So we assume no more than the basic definition (\ref{complex_task}). 
% region: skill composition lemma (scl) 
% skills are composable to accomplish complex tasks
% statement
\begin{lemma}[Skill Composition Lemma, SCL]
    \label{scl}
    The Skill Composition Lemma is a generalization of the SSL and states that skills are composable to accomplish complex tasks. More precisely,
    \begin{gather}
        \tilde{T}_{k}^{q}
        :=
        \int_{0}^{1}
        T(l, \ell_{k}^{q})
        dl
        =
        f(\boldsymbol{a_k}, \boldsymbol{a_q})
        \in
        [0,1]
    \end{gather}
    is the percentage of complex tasks that depend on skill set $\boldsymbol{a_q}$ a person with skill set 
    $\boldsymbol{a_k}$ can accomplish.
    % proof
    \begin{proof}
        Let $T_{q}(l)$ denote an activity of difficulty level $l \in [0,1]$ associated with an $\boldsymbol{a_q} := (a_{1}^{q}, \dots, a_{m}^{q}) \in [0,1]^m$ skill set -- rather than a mere subset of it --, such that $T_{q}(l)$ is a complex task; while $\boldsymbol{a_k} := (a_{1}^{k}, \dots, a_{m}^{k}) \in [0,1]^m$ is a person $k$'s skill set. With this,
        \begin{align}
            &
            \tilde{T}_{k}^{q}
            :=
            \int_{0}^{1}
            T(l, \ell_{k}^{q})
            dl
            :=
            \int_{0}^{1}
            [
                \ell_{k}^{q}
                \geq
                l
            ]
            dl
            =
            \int_{0}^{\ell_{k}^{q}}
            1
            \times
            dl
            +
            \int_{\ell_{k}^{q}}^{1}
            0
            \times
            dl
            =
            l\big|_{0}^{\ell_{k}^{q}}
            + 0
            \\
            &
            \therefore
            \ell_{k}^{q}
            :=
            f(\boldsymbol{\ell_k}, \boldsymbol{\ell_q})
            \land
            \boldsymbol{a_k}
            :=
            \boldsymbol{\ell_k}
            \
            \forall
            \
            k,q \in \{1, \dots, n\}
            \\
            &
            \implies
            \tilde{T}_{k}^{q}
            =
            l\big|_{0}^{\ell_{k}^{q}}
            =
            f(\boldsymbol{a_k}, \boldsymbol{a_q})
            - 0
            =
            f(\boldsymbol{a_k}, \boldsymbol{a_q})
            \in
            [0,1]
            .
        \end{align}
        
        Thus, any economic agent, if sufficiently qualified, can naturally \textit{compose} attributes $\{1, \dots, m\}$ to accomplish a complex task.
    \end{proof}
\end{lemma}
% endregion

% region: time allocation definition 
In addition to the above, we shall also specifiy tasks' duration by means of a time allocation function.
\begin{definition}[Time Allocation]
    \label{ta}
    % statement
    Time allocation is a continuous function associating tasks' difficulty with their duration, both normalized to the unit interval:
    \begin{gather}
        \ta(l)
        :=
        \ttc(l)
        \times
        \left(
            \int_{0}^{1}
            \ttc(l)
            dl
        \right) ^ {-1}
        ,
        \\
        \ttc
        :
        [0,1]
        \rightarrow
        \mathbb{R}^{+}        
        \land
        \int_{0}^{1}
            \ttc(l)
            dl
            >
            0
        ,
    \end{gather}
    where $\ttc(l)$ is the number of hours to complete a task given its difficulty level.
    
    Notice, as well, we termed normalized task duration ``time allocation''. This is on purpose, as time constraints are assumed to coincide with aggregate normalized duration (Axiom \ref{taa}). Or, in other words, all employees have the same unitary time allowance, while tasks' normalized duration, likewise integrates to a dimensionless time unit, so that any worker, if sufficiently qualified, can output every task by themselves. Therefore, an independent employee producing the entire $l \in [0,1]$ responsibility spectrum must spend their time allowance in accordance with normalized duration, or the \textit{percentage} of a single worker's time that has to be \textit{allocated} to complete a task; hence the name ``time allocation''.
    % Notice, as well, we termed normalized task duration ``time allocation''. This is on purpose, as employees' ``time allowances'', or time constraints, are assumed to coincide with aggregate normalized duration. Or, in other words, workers all have the same unitary time allowance, while tasks' normalized duration, likewise integrates to a dimensionless time unit, so that each worker, if sufficiently qualified, can produce the entire $l \in [0,1]$ responsibility spectrum by themselves. Therefore, any independent employee has to spend their time allowance in accordance with normalized duration, or the \textit{percentage} of a single worker's time that has to be \textit{allocated} to complete a task; hence the name ``time allocation''.
\end{definition}
% endregion

Finally, we formalize our assumption in Definition \ref{ta}, with Axiom \ref{taa}.
% region: time allowance axiom (taa) 
\begin{axiom}[Time Allowance Axiom, TAA]
    \label{taa}
    All workers' time allowances are unitary and perfectly coincide with the
    \begin{gather}
        \int_{0}^{1}
        \ta(l)
        dl
        =
        \left(
            \int_{0}^{1}
            \ttc(l)
            dl
        \right) ^ {-1}
        \times
        \int_{0}^{1}{
            \ttc(l)
            dl
        }
        =
        1
    \end{gather}
    aggregate normalized duration of occupations' complex tasks.
\end{axiom}
% endregion

\subsection{Addressing Occupational Complexity}
% region: occupational reducibility observation (oro) 
% occupations can be reduced to their tasks
Having understood what ``tasks'' are mathematically, we can proceed with our goal of quantifying employability. To this end, we note the following.

\begin{observation}[Occupational Reducibility]
    \label{oro}
    % statement
    From a practical standpoint, occupations are reducible to their activities: a ``job'' is nothing but a collection of tasks which have to be executed in a particular time frame.
\end{observation}

Thus, we conclude one is employable to the measure of one's productivity: if a person can perform an occupation's tasks, they can, thereby, be employed in its labor market.

However, how exactly should we assess productivity when speaking of whole occupations? For, though a job is just a collection of tasks, we do not know \textit{which} tasks -- whether simple, or complex, and if so, how complex -- constitute ``an occupation''. Moreover, these activities of varying complexity and productivity requirements usually vary from position to position. So, it appears our situation here is somewhat nebulous. In fact, we could even go so far as to replace our initial observation and say: an occupation is a ``black box of tasks''.

Again, this is what we refer to as the issue of ``occupational complexity'' and it is our main obstacle towards an economicly precise concept of employability and labor market competitiveness. But we have a good solution for this problem.
% endregion

% region: occupational atomicity observation (oao)
\begin{observation}[Attribute Complementary and Occupational Atomicity]
    \label{oao}
    % statement
    Professional attributes serve different purposes in different occupations, and each occupation's attributes complement one another to output a mostly homogenous ``product'', which is the essence of its activity and makes it unique among all occupations. Hence, what a particular job produces is an indivisible, ``atomic'', set of tasks, essentially other than what is produced elsewhere in the labor market.
\end{observation}

For example, an artist, an airline pilot and a surgeon have ``manual dexterity'' as part of their skill set \parencite[][]{onet_art,onet_pilot,onet_surgeon}. Nevertheless, one's dexterity is combined with their creativity and applied to producing art, while the others' are combined with specific, technical, knowledge and applied, respectively, to maneuvering airbuses and performing surgery. And, for this reason, each of these's skills are not necessarily transferable to the other's activities: thus, indeed, to be talented with the scalpel does not mean one is any good with a paint brush; for the painter's ability is not really ``manual dexterity'' itself, but ``artistic-manual-dexterity'', whereas a surgeon's is ``surgical-manual-dexterity'' (the hyphens emphasize occupational atomicity, or indivisibility, as the same competency, combined to another set of skills yields quite different results).

And we could further state all occupations' simple tasks (i.e. those requiring only one attribute) amount to very little, and perhaps nothing, in terms of time allocation, so that employees in each labor market spend their time allowances on occupations' complex, and essential, tasks (i.e. those that differentiate them). For, if this was not the case, it would not imply an occupation is not complex, only that it was incorrectly categorized: it would not be an issue with the theory here proposed, but an empirical, classification error. Therefore, in such cases, the occupation should be split into however many suboccupations are needed until each of them is indivisible (i.e. ``atomic'').

Finally, as with attribute atomicity (see Axiom \ref{aaa}), occupational atomicity too does not rule out some level of specialization and skill substitution within a job: we do not assume every single position is identical, or even deals with exactly the same subject matter; what we do suppose is that an occupation's job posts are sufficiently and essentially equivalent, despite irrelevant differences in difficulty and specialization, to the point where a person's productivity and employability remains constant across them (i.e. occupations are ``well-defined'').
% endregion

% region: (weak) occupational complexity axiom (woca) 
% all of an occupation's tasks can be thought of as only
% one, indivisible, holistic task, requiring their entire skill set
% we call this an occupation's "operation"
% all an occupation does is its "operation"
% define an aggregation function to compose skills an an "operation"
% operation production: evaluated aggregation function
% max production: occupation's evaluated aggregation function
% productivity: production truncated and normalized by max production
% statement
And, because of these two insightful, yet fairly uncontroversial observations (viz. of reducibility and atomicity), it seems we have good logical grounds to ``sidestep'' the issue of occupational complexity by an axiom.
\begin{axiom}[Occupational Complexity Axiom]
    \label{oca}
    Any occupation can be thought of as one indivisible activity that mobilizes workers' entire skill set. We call this ``holistic task'' an occupation's \textit{operation}.

    Mathematically, an occupational operation is just a series of complex tasks on a continuum of difficulty levels normalized to the unit interval, all of which are indispensable for the whole operation to be accomplished.
    
    \begin{subaxiom}[Strong Occupational Complexity Axiom, SOCA]
        Let us denote, then, ``operational output'' (abbrev. ``o.o.'') with the standard IPA \parencite[][]{ipa} symbol for the near-close near-back rounded vowel (i.e. the ``double o'' sound in words such as ``boot''):
        \begin{gather}
            \oo_q
            :=
            \oo_{q}^{\text{IP}}
            :=
            \sum_{v=1}^{w_q}{
                \left[
                    \tilde{T}_{v}^{q}
                    =
                    1
                \right]
                \times
                \left(
                    \int_{0}^{1}{
                        \ta(l)
                        dl
                    }
                \right) ^ {-1}
            }
            =
            \sum_{v=1}^{w_q}{
                \left[
                    \tilde{T}_{v}^{q}
                    =
                    1
                \right]
            }
            ,
        \end{gather}
        where the upperscript ``IP'' indicates production is organized independently, as each worker outputs the $l \in [0,1]$ responsibility spectrum by themselves, so that the market's aggregate output is precisely the number of perfectly qualified employees in it (i.e. those capable of producing all tasks without outsourcing).
        
        Intuitively, this initial formulation entails a scenario of several individuals working in parallel entirely disconnected from one another; which, of course, is hardly the case in a real economy. Thus, we should define occupational complexity in weaker terms, allowing for at least a degree of outsourcing.
    \end{subaxiom}

    \begin{subaxiom}[Moderate Occupational Complexity Axiom, MOCA] With moderate occupational complexity, we assume production can be split into $p_q \in \left\{1, 2, 3, \dots\right\}$ positions, or job subtypes, each responsible for their own subset of tasks with increasing difficulty levels.
    
    Therefore, though the operation, in itself, remains ``indivisible'', employers may split it apart and outsource it, so long as every subtask is completed (i.e. if this ``stratified'' operation is, then, ``pieced back together'' with all its parts).

    More precisely,
    \begin{gather}
        \oo_q
        :=
        \min\left(
            \boldsymbol{\ddot{w}_{q}}(
                \boldsymbol{w_{q}},
                \boldsymbol{\ell_{q}},
                \boldsymbol{\tilde{T}_{q}}
            )
            \times
            \boldsymbol{\oo_{q}}(
                \boldsymbol{\ell_{q}}
            )
        \right)
        +
        [p_q > 1]
        \times
        \boldsymbol{w_{q}} ^ {\top}
        \cdot 
        \boldsymbol{\varsigma_{q}}(
            \boldsymbol{\ell_{q}}
        )
        ,
    \end{gather}
    where
    \begin{gather}
        \boldsymbol{\oo_{q}}(
            \boldsymbol{\ell_{q}}
        )
        :=
        \left(
            \oo_{1}^{q}
            ,
            \dots
            ,
            \oo_{p_q}^{q}
        \right)
        :=
        \left(
            \frac{1}{
                \int_{0}^{\ell_{1}^{q}}
                \ta(l)dl
            }
            ,
            \dots
            ,
            \frac{1}{
                \int_{\ell_{p_q}^{q}}^{1}
                \ta(l)dl
            }
        \right)
    \end{gather}
    is the vector of partial operational outputs, as a function of
    \begin{gather}
        \boldsymbol{\ell_q}
        :=
        \left(
            \ell_{0}^{q}
            ,
            \dots
            ,
            \ell_{p_q}^{q}
        \right)
        :=
        \left(
            0
            ,
            \dots
            ,
            1
        \right)
        \in
        [0,1] ^ {p_q}
        ,
        \\
        \sum_{v=1}^{p_q}{
            \int_{
                \ell_{v-1}^{q}
            }^{
                \ell_{v}^{q}
            }
            \ta(l)dl
        }
        :=
        1
    \end{gather}
    responsibility bounds; while
    \begin{gather}
        \boldsymbol{\ddot{w}_{q}}(
            \boldsymbol{w_{q}},
            \boldsymbol{\ell_{q}},
            \boldsymbol{\tilde{T}_{q}}
        )
        :=
        \left(
            \sum_{r=1}^{w_{1}^{q}}
            \left[
                \tilde{T}_{1}^{r}
                \geq
                \ell_{1}^{q}
            \right]
            ,
            \dots
            ,
            \sum_{r=1}^{w_{p_q}^{q}}
            \left[
                \tilde{T}_{p_q}^{r}
                \geq
                \ell_{p_q}^{q}
            \right]
        \right)
        \geq
        \boldsymbol{0}
        ,
        \\
        0
        \leq
        \boldsymbol{1} ^ {\top}
        \cdot
        \boldsymbol{\ddot{w}_{q}}(
            \boldsymbol{w_{q}},
            \boldsymbol{\ell_{q}},
            \boldsymbol{\tilde{T}_{q}}
        )
        \leq
        \boldsymbol{1} ^ {\top}
        \cdot
        \boldsymbol{\ddot{w}_{q}}(
            \boldsymbol{w_{q}},
            \boldsymbol{\ell_{q}},
            \boldsymbol{1}
        )
        =
        w_q
    \end{gather}
    are effective\footnote{
        The \textit{umlaut} operator was defined in a previous paper 
        \parencite{cao_gene_comp} and is not to be mistaken with the Newtonian derivative notation. Here, as before, it only indicates an equivalence function, in this case a quite simple one, is applied to its operand vector.
    } (i.e. sufficiently qualified) workers per position, from a pool of
    \begin{gather}
        w_q
        :=
        \boldsymbol{1} ^ {\top}
        \cdot
        \boldsymbol{w_q}
        ,
        \\
        \boldsymbol{w_{q}} := 
        (
            w_{1}^{q}
            ,
            \dots
            ,
            w_{p_q}^{q}
        )
        \geq
        \boldsymbol{0}
        ,
    \end{gather}
    individuals; and
    \begin{gather}
        \boldsymbol{\varsigma_{q}}(
            \boldsymbol{\ell_{q}}
        )
        :=
        \left(
            \varsigma_{1}^{q}
            ,
            \dots
            ,
            \varsigma_{p_q}^{q}
        \right)
        \in
        \mathbb{R} ^ {p_q}
        ,
        \\
        \varsigma_{v}^{q}
        :=
        \int_{\ell_{v-1}^{q}}^{\ell_{v}^{q}}
        \text{sg}(l)dl
        -
        \int_{\ell_{v-1}^{q}}^{\ell_{v}^{q}}
        \text{sc}(l)dl
    \end{gather}
    is the net stratification effect, which measures whether the gains in efficiency due to splitting job posts into separate positions,
    \begin{gather}
        \sum_{v=1}^{p_q}
        \int_{\ell_{v-1}^{q}}^{\ell_{v}^{q}}
        \text{sg}(l)dl
        \geq 0
        ,
        \text{sg}
        :
        [0,1]
        \rightarrow
        \mathbb{R}^{+}
    \end{gather}
    outweigh its cost,
    \begin{gather}
        \sum_{v=1}^{p_q}
        \int_{\ell_{v-1}^{q}}^{\ell_{v}^{q}}
        \text{sc}(l)dl
        \geq 0
        ,
        \text{sc}
        :
        [0,1]
        \rightarrow
        \mathbb{R}^{+}
        .
    \end{gather}

    Note the Leontief function here signifies, again, moderately complex operations are ``indivisible'', so that aggregate production is set to the lowest partial operational output.
    \end{subaxiom}

    \begin{subaxiom}[Weak Occupational Complexity Axiom, WOCA]
        \label{woca}
        Weakly complex operations are just the same as in MOCA, with an additional assumption about the net effects of labor stratification:
        \begin{gather}
            \boldsymbol{w_{q}} ^ {\top}
            \cdot 
            \boldsymbol{\varsigma_{q}}(
                \boldsymbol{\ell_{q}}
            )
            =
            0
            \implies
            \oo_q
            =
            \min\left(
                \boldsymbol{\ddot{w}_{q}}(
                    \boldsymbol{w_{q}},
                    \boldsymbol{\ell_{q}},
                    \boldsymbol{\tilde{T}_{q}}
                )
                \times
                \boldsymbol{\oo_{q}}(
                    \boldsymbol{\ell_{q}}
                )
            \right)
            .
    \end{gather}

    In other words, weak occupational complexity asserts the gains and costs of splitting job posts are both negligeable or cancel each other out. This means employers may stratify positions without either gain or loss to production.
        % \begin{gather}
        %     \oo_q
        %     :=
        %     \min\left(
        %         \boldsymbol{w_{q}^{S}}
        %         \times
        %         \boldsymbol{\oo_q}(
        %             \boldsymbol{\ell_q}
        %         )
        %     \right)
        %     ,
        % \end{gather}
        % where
        % \begin{gather}
        %     \boldsymbol{w_{q}^{S}}
        %     :=
        %     \left(
        %         \sum_{r=1}^{w_{q}^{1}}
        %         \left[
        %             \tilde{T}_{1}^{r}
        %             \geq
        %             \tilde{T}_{1}^{q}
        %         \right]
        %         ,
        %         \dots
        %         ,
        %         \sum_{r=1}^{w_{q}^{p_q}}
        %         \left[
        %             \tilde{T}_{p_q}^{r}
        %             \geq
        %             \tilde{T}_{p_q}^{q}
        %         \right]
        %     \right)
        % \end{gather}
        % is the vector of sufficiently qualified workers in a labor market; or, with
        % \begin{gather}
        %     \left[
        %         \tilde{T}_{v}^{r}
        %         \geq
        %         \tilde{T}_{v}^{q}
        %     \right]
        %     \
        %     \forall
        %     \
        %     r \in \{1, \dots, w_{v}^{q}\}
        %     ,
        %     v \in \{1, \dots, p_q\}
        %     ,
        %     \\
        %     \implies
        %     \sum_{r=1}^{w_{v}^{q}}
        %     \left[
        %         \tilde{T}_{v}^{r}
        %         \geq
        %         \tilde{T}_{v}^{q}
        %     \right]
        %     =
        %     \sum_{r=1}^{w_{v}^{q}}
        %     1
        %     =
        %     w_{v}^{q}
        %     \\
        %     \therefore
        %     \boldsymbol{w_{q}^{S}}
        %     =
        %     \boldsymbol{w_q}
        %     :=
        %     (
        %         w_{q}^{1}
        %         ,
        %         \dots
        %         ,
        %         w_{q}^{p_q}
        %     )
        %     ,
        % \end{gather}
        % so that workers in each position are all sufficiently qualified for their responsibilities\footnote{
        %     See ``Employer Rationality Axiom'' and ``Productivity Sufficiency Lemma".
        % },
        % \begin{gather}
        %     \oo_q
        %     :=
        %     \min\left(
        %         \boldsymbol{w_q}
        %         \times
        %         \boldsymbol{\oo_q}(
        %             \boldsymbol{\ell_q}
        %         )
        %     \right)
        %     ;
        % \end{gather}
        % and, alternatively,
        % \begin{gather}
        %     \oo_q
        %     :=
        %     \oo_{q}^{\text{IP}}
        %     :=
        %     \sum_{v=1}^{w_q}{
        %         \left[
        %             \tilde{T}_{v}^{q}
        %             =
        %             1
        %         \right]
        %     }
        %     ,
        % \end{gather}
        % if employers opt for an independent production strategy.
    \end{subaxiom}

    And we could further specifiy even weaker versions of the axiom, with any
    \begin{gather}
        \oo_q
        :=
        \oo\left(
            \boldsymbol{\ddot{w}_{q}}(
                \boldsymbol{w_{q}},
                \boldsymbol{\ell_{q}},
                \boldsymbol{\tilde{T}_{q}}
            )
            \times
            \boldsymbol{\oo_{q}}(
                \boldsymbol{\ell_{q}}
            )
        \right)
        +
        [p_q > 1]
        \times
        \boldsymbol{w_{q}} ^ {\top}
        \cdot 
        \boldsymbol{\varsigma_{q}}(
            \boldsymbol{\ell_{q}}
        )
    \end{gather}
    aggregation function, yielding the very same conclusions we demonstrate in this paper, if it is ``well-behaved'' enough (e.g. if it satisfies Inada conditions). But, for simplicity's sake and mathematical convenience, we shall assume weak occupational complexity going forward.

    % [artist, airline pilot, surgeon: operation must be completed or else the output is null; fazer referência implícita a Nossa Senhora de Aparecida]
\end{axiom}
% endregion

% region: occupational composition lemma (ocl) 
% sca, ora, oca => ocl
% skill sets are composable to accomplish occupations' "operations"
% statement
\label{ocl}
\begin{lemma}[Occupational Composition Lemma, OCL]
    Skill sets are composable to accomplish occupations' operations.
\end{lemma}

% proof
\begin{proof}
    We have just concluded an occupation is a collection of complex tasks, which in turn depend on an $\boldsymbol{a_q} := (a_{1}^{q},\dots, a_{m}^{q})$ vector of professional attributes to be accomplished (see Definition \ref{complex_task}). By the Skill Composition Lemma, then,
    \begin{gather}
        \tilde{T}_{k}^{q}
        :=
        \int_{0}^{1}
        T(l, \ell_{k}^{q})
        dl
        =
        f(\boldsymbol{a_k}, \boldsymbol{a_q})
        \in
        [0,1]
    \end{gather}
    is one's productivity when employed at occupation $q$'s job posts.
\end{proof}
% endregion

% region: section summary 
To conclude this section, let us summarize what we have so far defined. With the aim of estimating individuals' employability in efficiently organized labor markets, as well as these's competitiveness, we analyzed occupations in terms of their tasks and the required competence to complete them. We started from a fairly tautological notion of ``skill'' as one's capacity to accomplish tasks in a given domain and generalized, from this, another notion, that of complex tasks. Finally, we observed occupations are, in practice, reducible to their tasks and, though very much complex and hard to quantify in this regard, with an important axiom (\ref{woca}), we derived a theoretically sound method to measure overall productivity in any occupation's labor market. Thus, we have effectively ``sidestepped'' the issue of occupational complexity by assuming complexity.
% endregion

\section{Market Conditions and Employer Behavior}
In this section, we lay out some presuppositions concerning market dynamics. To put them briefly, we make three very general assertions: 1) employers act cohesively as wage takers, choosing only the quantities of job subtypes to offer and each position's responsibilities (i.e. the model is static with respect to wages); 2) workers may, or may not, have varying skill levels, and employers take this into account when choosing which hiring and production strategy to implement; 3) employers can evaluate workers more or less accurately and do not hire insufficiently qualified individuals.
% region: employer rationality axiom (era) 
% employers are rational
% employers will only pay for employees to work on tasks they can accomplish
% if required, employers will outsource parts of an occupation's operation
% statement
\begin{axiom}[Employer Rationality Axiom, ERA]
    \label{era}
    Employers are rational and only hire individuals to work on tasks for which they are qualified. Furthermore, if labor stratification is allowed, they will split job posts and outsource activities if it is expected employees cannot accomplish the whole operation.

    Mathematically, a rational employer's optimization problem,
    \begin{align}
        \max_{
            p_q,
            \boldsymbol{w_q},
            \boldsymbol{\ell_q}
        }
            &
            \mathbb{E}\left[
                \oo\left(
                    \boldsymbol{\ddot{w}_{q}}(
                        \boldsymbol{w_{q}},
                        \boldsymbol{\ell_{q}},
                        \boldsymbol{\tilde{T}_{q}}
                    )
                    \times
                    \boldsymbol{
                        \oo_q
                    }(
                        \boldsymbol{\ell_q}
                    )
                \right)
                \
                |
                \
                \mathbb{E}[
                    \boldsymbol{\tilde{T}_q}
                ]
            \right]
            \\
            \text{s.t.}
            \quad
            &
            p_q \in \left\{1, 2, 3, \dots\right\}
            ,
            \\
            &
            \boldsymbol{\ell_q} \in [0,1] ^ {p_q}
            ,
            \\
            &
            \sum_{v=1}^{p_q}{
                \int_{
                    \ell_{v-1}^{q}
                }^{
                    \ell_{v}^{q}
                }
                \ta(l)dl
            }
            =
            1
            ,
            \\
            &
            \boldsymbol{w_{q}} \geq \boldsymbol{0}
            ,
            \\
            &
            \boldsymbol{1} ^ {\top}
            \cdot
            \boldsymbol{w_q}
            \leq
            w_q
            ,
            \\
            &
            \sum_{v=1}^{w_q}
            w_{v}^{q}
            y_{v}^{q}
            =
            w_q
            y_q
            ,
    \end{align}
    is to choose vectors of employment levels $\boldsymbol{w_q}$ and responsibility bounds $\boldsymbol{\ell_q}$ for each of $p_q$ positions and implement a production strategy that maximizes operational output, given $\mathbb{E}[\boldsymbol{\tilde{T}_q}]$, or the expected productivity in the workforce.
\end{axiom}
% endregion

Note we have, here, assumed production is profitable to begin with, such that revenue is at least equal to the sum of wages paid,
    \begin{gather}
        Y_q
        :=
        \sum_{v=1}^{w_q}
        w_{v}^{q}
        y_{v}^{q}
        =
        w_q
        y_q
        ,
    \end{gather}
plus any additional operational costs; otherwise, rational employers' would not attempt to maximize operational output (as in Axiom \ref{era}) and, indeed, the labor market would not be in equilibrium, or even exist at all (cf. ``Productivity Sufficiency Lemma'' below).

% region: constant compensation lemma (ccl)
\begin{lemma}[Constant Compensation Lemma, CCL]
    \label{ccl}
    The CCL states wages are
    \begin{gather}
        y_{v}^{q}
        =
        y_q
        ,
    \end{gather}
    for all $v \in \{1, \dots, p_q\}, p_q \in \{1, 2, 3, \dots\}$ job subtypes, where $y_q > 0 $ is the mean wage in occupation $q$'s labor market.

    \begin{proof}
        In independent production, there are $w_q$ jobs, all of which are the same and, thus, pay the same. In a stratified market, however, the responsibilities of these positions are split into $p_q \in \{1, 2, 3, \dots\}$ job subtypes.
        
        So, let us assume, for a moment, each one of these jobs have varying durations, as their respective tasks, indeed, do so
        \begin{gather}
            \because
            \int_{\ell_{v-1}^{q}}^{\ell_{v}^{q}}
            \ta(l)
            dl
            \leq
            \int_{0}^{1}
            \ta(l)
            dl
            =
            1
            .
        \end{gather}
        
        With this, for operational output to be same (Axiom \ref{woca}), if positions' tasks have shorter duration, we would have to multiply the number of job posts by
        \begin{gather}
            p_q
            \times
            \int_{\ell_{v-1}^{q}}^{\ell_{v}^{q}}
            \ta(l)
            dl
            \times
            \left(
                \int_{\ell_{v-1}^{q}}^{\ell_{v}^{q}}
                \ta(l)
                dl
            \right) ^ {-1}
            =
            p_q
            ,
        \end{gather}
        and also adjust wages proportionally, because rational employers do not pay for tasks employees do not  accomplish (Axiom \ref{era}); or, in other words, all $w_q$ jobs with equivalent $l \in [0,1]$ responsibilities,
        \begin{gather}
            \int_{0}^{1}
            \ta(l)
            dl
            =
            1
        \end{gather}
        unitary duration, and $y_{v}^{q} = y_q \ \forall \ v \in \{1, \dots, w_q\}$ compensation would be broken down into $w_q \times p_q$ jobs with $l \in [\ell_{v-1}^{q},\ell_{v}^{q}]$ responsibilities,
        \begin{gather}
            \int_{\ell_{v-1}^{q}}^{\ell_{v}^{q}}
            \ta(l)
            dl
        \end{gather}
        less than unitary duration, and
        \begin{gather}
            y_{v}^{q}
            =
            y_q
            \times
            \int_{\ell_{v-1}^{q}}^{\ell_{v}^{q}}
            \ta(l)
            dl
            \ 
            \forall 
            \ v \in \{1, \dots, p_q\}
            ,
            \ p_q \in \{1, 2, 3, \dots\}
        \end{gather}
        proportional compensation.

        Therefore, any worker that wished to specialize in a particular task range would take the maximum amount of these ``partial jobs'' their unitary time allowance (see Axiom \ref{taa}) would permit and be compensated an aggregate wage
        \begin{gather}
            y_{v}^{q}
            \times
            \oo_{v}^{q}
            =
            \left(
                y_q
                \times
                \int_{\ell_{v-1}^{q}}^{\ell_{v}^{q}}
                \ta(l)
                dl
            \right)
            \times
            \left(
                \int_{\ell_{v-1}^{q}}^{\ell_{v}^{q}}
                \ta(l)
                dl
            \right) ^ {-1}
            =
            y_q
            ,
        \end{gather}
        for all $v \in \{1, \dots, p_q\}, p_q \in \{1, 2, 3, \dots\}$. But this is just the same as saying that wages in a stratified market are constant across occupations' positions.
    \end{proof}
\end{lemma} 
% endregion

This, in turn, greatly simplifies rational employers' optimization problem.
% region: wage-independent optimization lemma (wiol) 
\begin{lemma}[Wage-Independent Optimization Lemma, WIOL]
    \label{wiol}
    The optimal number of workers of each type to hire is estimated by
    \begin{gather}
        \ddot{w}_{v}^{q}
        \times
        \oo_{v}^{q}
        =
        \ddot{w}_{r}^{q}
        \times
        \oo_{r}^{q}
        \
        \forall
        \
        v,r \in \{1, \dots, p_q\}
        ,
        p_q \in \{1, 2, 3, \dots\}
        .
    \end{gather}

    \begin{proof}
        Because weak cccupational complexity (Axiom \ref{woca}) implies optimization is done via the ``no waste condition'',
        \begin{gather}
            \min\left(
                \frac{
                    \boldsymbol{\ddot{w}_{q}}
                    \times
                    \boldsymbol{\oo_q}
                }{
                    \boldsymbol{y_q}
                }
            \right)
            =
            \frac{
                \ddot{w}_{v}^{q}
                \times
                \oo_{v}^{q}
            }{
                y_{v}^{q}
            }
            =
            \frac{
                \ddot{w}_{r}^{q}
                \times
                \oo_{r}^{q}
            }{
                y_{r}^{q}
            }
            ,
        \end{gather} 
        for all $v,r \in \{1, \dots, p_q\}, p_q \in \{1, 2, 3, \dots\}$, with the CCL above we have
        \begin{gather}
            \frac{
                \ddot{w}_{v}^{q}
                \times
                \oo_{v}^{q}
            }{
                y_q
            }
            =
            \frac{
                \ddot{w}_{r}^{q}
                \times
                \oo_{r}^{q}
            }{
                y_q
            }
            \therefore
            \frac{
                \ddot{w}_{v}^{q}
                \times
                \oo_{v}^{q}
            }{
                1
            }
            =
            \frac{
                \ddot{w}_{r}^{q}
                \times
                \oo_{r}^{q}
            }{
                1
            }
            \therefore
            \ddot{w}_{v}^{q}
            \times
            \oo_{v}^{q}
            =
            \ddot{w}_{r}^{q}
            \times
            \oo_{r}^{q}
            .
        \end{gather}
    \end{proof}
\end{lemma}
% endregion

% region: productivity differentia axiom, pda 
\begin{axiom}[Productivity Differentia Axiom, PDA]
    There are, or there could be, skill differences in the workforce (i.e. employees are, likely, not all equally competent ``clones'' of one another). So, the expected value of productivity is:
    \begin{gather}
        \mathbb{E}[
            \tilde{T}_{v}^{q}
        ]
        \in
        [0,1]
        ,
    \end{gather}
    instead of 
    \begin{gather}
        \mathbb{E}[
            \tilde{T}_{v}^{q}
        ]
        =
        \tilde{T}_{v}^{q}
        = 1
        ,
    \end{gather}
    for all $v \in \{1, \dots, w_q\}$. This means employers do not expected every worker to be perfectly qualified and will adjust their hiring and production strategies accordingly.
\end{axiom}
% endregion
Note we did not assign any specific probability distribution to workers' productivity. Hence, this axiom is as general as it can be.
% region: hireability axiom (ha) 
% hiring is done by evaluating a hireability statistic
% statement
\begin{axiom}[Hireability Axiom, HA]
    \label{ha}
    Furthermore, we assume hiring is done by evaluating a statistic that quantifies potential employees' expected productivity, their educational attainment, years of experience, and so on and so forth. Again, as we do not want to overcomplicate this initial model, we shall not introduce issues of uncertainty around the hiring process. Let us define, then,
    \begin{gather}
        h_{k}^{q}
        \in
        [0,1]
        \
        \forall
        \
        v \in \{1, \dots, w_q\}
    \end{gather}
    as employers' expectations of individual's hireability and determine them to be always correct (i.e. there is no information asymmetry in hiring).
\end{axiom}
% endregion
% endregion

% region: demonstration 
\section{The Employability Theorem}
\subsection{What is Employability?}
When we speak of ``employability'' what we generally mean is rather trivial: the capacity to find employment. Thus, we say someone is ``employable'' if they are easily \textit{hireable} and could get a job in a large portion of the labor market.

% region: employability definition 
\begin{definition}[Employability]
    \label{def_employability}
    Mathematically, then, the employability of a person $k$ is the percentage of job posts for which they are sufficiently qualified:
    \begin{gather}
    \tilde{W}_{k}^{q}
    :=
    \left[
        h_{k}^{q}
        \geq
        \frac{1}{2}
    \right]
    \sum_{v=1}^{p_q}
    \left[
        \tilde{T}_{k}^{q}
        \geq
        \ell_{v}^{q}
    \right]
    \tilde{w}_{v}^{q}
    \in
    [0,1]
    ,
    \\
    \sum_{v=1}^{p_q}
    \tilde{w}_{v}^{q}
    :=
    \left(
        \frac{1}{w_q}
    \right)
    \sum_{v=1}^{p_q}
    w_{v}^{q}
    := 1
    ,
    \end{gather}
    where $\ell_{v}^{q} \in [0,1]$ is the minimum productivity required to be hired in one of $p_q$ types of positions in a labor market with a $w_q$ workforce size; while $h_{k}^{q} \in [0,1]$ is the hireability statistic accounting for other selection criteria, such as years of education, experience, etc.
    
    And we can further aggregate employability for $n$ occupations to assess how many of all available $w$ jobs in the economy are suitable for one's skill set:
    \begin{gather}
    \tilde{W}_{k}
    :=
    \left(
        \frac{1}{w}
    \right)
    \sum_{q=1}^{n}
    W_{k}^{q}
    :=
    \left(
        \frac{1}{w}
    \right)
    \sum_{q=1}^{n}
    \left[
        h_{k}^{q}
        \geq
        \frac{1}{2}
    \right]
    \sum_{v=1}^{p_q}
    \left[
        \tilde{T}_{k}^{q}
        \geq
        \ell_{v}^{q}
    \right]
    w_{v}^{q}
    \in
    [0,1]
    ,
    \\
    w
    :=
    \sum_{q=1}^{n}
    w_{q}
    :=
    \sum_{q=1}^{n}
    \sum_{v=1}^{p_q}
    w_{v}^{q}
    .
    \end{gather}
\end{definition}

% endregion
\subsection{Introductory Example: Employability with Two Types of Workers}
With these basic axioms in place, we can attempt to derive an employability coefficient as presented in Definition \ref{def_employability} above. For ease of understanding, though, let us begin with a simple example and, then, proceed with a more complete, and robust, theorem.

In this subsection, we shall estimate employability in an occupation $q$'s labor market where there are two types of workers with varying productivity. The first type -- call them ``juniors" -- have lower skill and cannot accomplish tasks with difficulty levels $l > \ell_{\text{Jr}}^{q} \in [0,1)$. And the other type of employee are perfectly qualified ``seniors'', with $\tilde{T}_{\text{Sr}}^{q} = 1$ productive capacity.

Now, because of weak occupational complexity (WOCA), employers will maximize operational output by producing the entire $l \in [0,1]$ spectrum of occupation $q$'s complex tasks, subject to each task's duration. This can be done either by having only perfectly qualified employees work on these independently, from beginning to end, or by splitting responsibilities into two, or more, types of jobs, thus allowing for less qualified, ``junior'' employees, to work alongside ``seniors'' towards the common goal of accomplishing the whole operation.

Additionally, because we assume there to be skill differences among workers in the labor market, any rational employer will always, and rightly, expect potential employees to be of varying skill levels, rather than all perfectly qualified, so that splitting responsibilities into separate positions will not only be an alternative mode of hiring and producing, but in fact the optimal one.

Therefore, employers will stratify job offers based on required competence, providing ``junior'' and ``senior'' positions, both dedicated to their own subset of complex tasks with difficulty levels appropriate for each employee.

Notice this does not mean those working on ``junior'' positions will, necessarily, be ``juniors'' themselves, that is, less qualified. Indeed, if talent is abundant in the labor market, these positions will have to be filled by more qualified, or even perfectly qualified, ``senior'' employees. For if there were just one type of job, spanning the entire responsibility spectrum, these highly qualified workers would already have to accomplish ``junior'' tasks, in order to maximize operational output. However, by having two, or more, types of jobs, requiring more, or less, productivity, they may specialize to the measure there are sufficient employees allocated to easier tasks.

Either way, if the available talent is enough to output occupation $q$'s operation, employability in such a market will be determined by the ratio of junior and senior job posts, as we demonstrate below.

% region: binary employability theorem (bet)
\begin{theorem}[Binary Employability Theorem, BET]
    \label{bet}
    In a labor market with two types of workers with varying productivity, each type of worker's employability is the inverse of their maximum operational output.
    \begin{proof}
        In the binary case, ``junior'' productive output will be given by
        \begin{gather}
            \oo_{\text{Jr}}^{q}
            :=
            \frac{
                1
            }{
                \int_{0}^{\ell_{\text{Jr}}^{q}}
                \ta(l)
                dl
            }
            = 
            \left(
                \int_{0}^{\ell_{\text{Jr}}^{q}}
                    \ta(l)
                    dl
            \right) ^ {-1}
            ,
        \end{gather}
        where $\ta(l)$ is the time allocation function of occupation $q$'s complex tasks, and time allowance (the numerator) is set to one. Analogously, ``senior'' output is
        \begin{gather}
            \oo_{\text{Sr}}^{q}
            :=
            \frac{
                1
            }{
                \int_{\ell_{\text{Jr}}^{q}}^{1}
                \ta(l)
                dl
            }
            = 
            \left(
                \int_{\ell_{\text{Jr}}^{q}}^{1}
                    \ta(l)
                    dl
            \right) ^ {-1}
            .
        \end{gather}

        Furthermore, as a mismatch in operational output due to time allocation differences between ``junior'' and ``senior'' tasks would result in wasted production, a rational employer will optimally ``orchestrate'' the productive effort by offering just enough ``senior'' job posts in the labor market to meet ``junior'' productivity\footnote{See the ``no waste condition" in Lemma \ref{wiol}.}. So, by setting ``junior'' job posts to $w_{\text{Jr}}^{q} > 0$ and ``senior'' job posts to $w_{\text{Sr}}^{q} > 0$, we get the ratio between ``junior'' and ``senior'' positions required to output any level of occupation $q$'s operation:
        \begin{align}
            w_{\text{Sr}}^{q}
            \times 
            \oo_{\text{Sr}}^{q}
            &= 
            w_{\text{Jr}}^{q}
            \times 
            \oo_{\text{Jr}}^{q}
            \\
            \therefore
            w_{\text{Sr}}^{q}
            \times 
            \left(
                \int_{\ell_{\text{Jr}}^{q}}^{1}
                \ta(l)
                dl
            \right) ^ {-1}
            &= 
            w_{\text{Jr}}^{q}
            \times
            \left(
                \int_{0}^{\ell_{\text{Jr}}^{q}}
                    \ta(l)
                    dl
            \right) ^ {-1}
            \\
            \therefore
            w_{\text{Sr}}^{q}
            &= 
            w_{\text{Jr}}^{q}
            \times
            \left(
                \frac{
                    \int_{\ell_{\text{Jr}}^{q}}^{1}
                    \ta(l)
                    dl
                }{
                    \int_{0}^{\ell_{\text{Jr}}^{q}}
                    \ta(l)
                    dl
                }
            \right)
            .
        \end{align}
        
        With this, ``senior'' employability (i.e. the percentage of job posts for which they could be hired) is
        \begin{gather}
            \tilde{W}_{\text{Sr}}^{q}
            =
            \frac{
                w_{\text{Jr}}^{q} + 
                w_{\text{Sr}}^{q}
            }{
                w_{\text{Jr}}^{q} + 
                w_{\text{Sr}}^{q}
            }
            = 1
            ,
        \end{gather}
        while ``junior'' employability is
        \begin{align}
            \tilde{W}_{\text{Jr}}^{q} 
            &=
            \frac{
                w_{\text{Jr}}^{q}
            }{
                w_{\text{Jr}}^{q} + 
                w_{\text{Sr}}^{q}
            }
            \\&=
            \frac{
                w_{\text{Jr}}^{q}
            }{
                w_{\text{Jr}}^{q} + 
                w_{\text{Jr}}^{q} \times
                \left(
                    \frac{
                        \int_{\ell_{\text{Jr}}^{q}}^{1}
                            \ta(l)
                            dl
                    }{
                        \int_{0}^{\ell_{\text{Jr}}^{q}}
                            \ta(l)
                            dl
                    }
                \right)
            }
            \\&=
            \left(
                1 + 
                \frac{
                    \int_{\ell_{\text{Jr}}^{q}}^{1}
                        \ta(l)
                        dl
                }{
                    \int_{0}^{\ell_{\text{Jr}}^{q}}
                        \ta(l)
                        dl
                }
            \right) ^ {-1}
            \\&=
            \left(
                1 + 
                \frac{
                    \int_{0}^{1}
                        \ta(l)
                        dl    
                        -
                    \int_{0}^{\ell_{\text{Jr}}^{q}}
                        \ta(l)
                        dl
                }{
                    \int_{0}^{\ell_{\text{Jr}}^{q}}
                        \ta(l)
                        dl
                }
            \right) ^ {-1}
            \\&=
            \left(
                1 + 
                \frac{
                    1 -
                    \int_{0}^{\ell_{\text{Jr}}^{q}}
                        \ta(l)
                        dl
                }{
                    \int_{0}^{\ell_{\text{Jr}}^{q}}
                        \ta(l)
                        dl
                }
            \right) ^ {-1}
            \\&=
            \left(
                1 + 
                \frac{
                    1
                }{
                    \int_{0}^{\ell_{\text{Jr}}^{q}}
                        \ta(l)
                        dl
                }
                -
                \frac{
                    \int_{0}^{\ell_{\text{Jr}}^{q}}
                        \ta(l)
                        dl
                }{
                    \int_{0}^{\ell_{\text{Jr}}^{q}}
                        \ta(l)
                        dl
                }
            \right) ^ {-1}
            \\&=
            \left(
                1 + 
                \frac{
                    1
                }{
                    \int_{0}^{\ell_{\text{Jr}}^{q}}
                        \ta(l)
                        dl
                }
                -
                1
            \right) ^ {-1}
            \\&=
            \left(
                \frac{
                    1
                }{
                    \int_{0}^{\ell_{\text{Jr}}^{q}}
                        \ta(l)
                        dl
                }
            \right) ^ {-1}
            \\&=
            \int_{0}^{\ell_{\text{Jr}}^{q}}
                \ta(l)
                dl
            .
        \end{align}
        
        Thus, the employability of a partially qualified worker, that is a ``junior'', is precisely the percentage of an operation's total time duration their skill set allows them to accomplish (i.e. the inverse of their operational output).    
    \end{proof}
\end{theorem}
% endregion
\subsection{Employability With Multiple Types of Workers}
\subsubsection{Maximum Labor Stratification}
% region: maximum-monotonic labor stratification definition 
Now, to generalize this conclusion, we shall define notation in terms of maximum labor stratification, a productive arrangement where there are several job subtypes, indeed as many as there are jobs themselves, each with a limited spectrum of responsibilities.

\begin{definition}[Maximum Labor Stratification]
    \label{mmls_def}
    Hence, mathematically,
    \begin{gather}
    l \in \left[
    \ell_{v-1}^{q}
    ,
    \ell_{v}^{q}
    \right]
    ,
    \end{gather}
    with
    \begin{align}
    \ell_{v}^{q} \in [0,1]
    \
    \forall
    \
    &v \in 
    \left\{
        1, \dots, w_q
    \right\}
    ,
    \\
    \ell_{w_q}^{q}&:= 1
    ,
    \\
    \ell_{0}^{q} &:= 0
    \end{align}
    is one of $w_q$ responsibility spectra in a maximally stratified labor market, in which employment levels are unitary, or given by
    \begin{gather}
    \sum_{v=1}^{w_q}
    1
    =
    w_q
    ,
    \end{gather}
    so that any available position is its own job subtype and covers only a restrictive range of task difficulty, accounting for
    \begin{gather}
    \Omega_{v}^{q}
    := 
    \frac{1}{\oo_{v}^{q}}
    = 
    \int_{
        \ell_{v-1}^{q}
    }^{
        \ell_{v}^{q}
    }
    \ta(l)dl
    \in [0,1]
    \end{gather}
    of an operation's total duration,
    \begin{gather}
    \sum_{v=1}^{w_q}
    \Omega_{v}^{q} 
    := 
    \sum_{v=1}^{w_q}
    \int_{
        \ell_{v-1}^{q}
    }^{
        \ell_{v}^{q}
    }
    \ta(l)dl
    =
    \int_{0}^{1}
    \ta(l)dl
    =
    1
    .
    \end{gather}
    
    Intuitively speaking, we would say production in a maximally (and monotonically) stratified labor market is not ``independent'' (cf. SOCA), in the sense that employees do not work on an occupation's operation from beginning to end. This means each of them will spend all their time allowance producing a partial operational output, that is a multiple of a difficulty subinterval of complex tasks, which will, in turn, contribute, alonside the partial outputs of other employees, to accomplishing the occupational operation in its entirety.
    
    However, in a maximum labor stratification setting, these partial operational outputs will not be produced merely via ``senior'' and ``junior'' positions, as previously, but rather within a myriad of levels in a production hierarchy, approximating a ``continuum of seniority'' as the workforce becomes large enough.
    % However, in a maximally stratified labor market, these partial operational outputs will not be produced merely via ``senior'' and ``junior'' positions, as previously, but rather within a myriad of levels in a production hierarchy, approximating a continuum of ``seniority'' as the workforce becomes large enough.
    
    Again, this does not mean employees are, themselves, more or less competent, only that available job posts are preemptively stratified with respect to task difficulty, in order to maximize employers' hiring pool and safeguard production in the case workers are not sufficiently qualified to produce the whole responsibility spectrum independently (see ``Maximum Labor Stratification Lemma'').    
\end{definition}
% endregion
% region: infinite stratification lemma (isl) 
Having understood what maximum labor stratification is, one may wonder whether there could be more than $w_q$ job subtypes in a labor market. For though it is intuitive to think of $w_q$, the workforce size, as the upper bound for stratification, if we allow for partial hiring, with ``fractional jobs'',
\begin{gather}
w_{v}^{q} 
> 0
% \in 
% [0, w_q]
\
\forall
\
v \in \{1, \dots, p_q\}
,
\\
\sum_{v=1}^{p_q}
w_{v}^{q}
:=
w_q
,
\end{gather}
where $p_q \in \{1, 2, 3, \dots\}$ is the number of positions in a labor market, then workers can allocate fractions of their time allowance to multiple responsibility spectra, and the productive arrangement we have just defined, may not, technically speaking, be ``maximally stratified''. 

Indeed, if it were possible to stratify beyond $w_q$, rational employers would readly do so, for, again, labor stratification reduces the uncertainty around production and serves as an insurance policy to guarantee the available talent is sufficient to output an occupation's operation.

But, because of this, if $p_q$ can be greater than $w_q$, the optimal production strategy would, logically, be to offer as many types of jobs as possible, even infinitely many.

% region: infinite labor stratification definition
\begin{definition}[Infinite Labor Stratification]
    Thus, infinite labor stratification is defined as an economic configuration where labor markets are subdivided into infinitesimal jobs, each contributing very little to production. In fact, in such a market, ``job posts'' are so small as to be indistinguishable from tasks themselves\footnote{See the Proportional Employment Condition in ``Maximum Operational Output Lemma''.}
\begin{align}
    &
    \because
    \lim_{p_q \rightarrow \infty}{
        \left(
            \ell_{v}^{q}
            -
            \ell_{v-1}^{q}
        \right)
    }
    =
    0
    % \\
    % &
    \implies
    \Omega_{v}^{q}
    :=
    \int_{
        \ell_{v-1}^{q}
    }^{
        \ell_{v}^{q}
    }
    \ta(l)dl
    =
    \ta(l)
    \\
    &
    \land
    \lim_{p_q \rightarrow \infty}{
        \tilde{w}_{v}^{q}
    }
    := 
    \lim_{p_q \rightarrow \infty}{
        \left(
            \frac{
                w_{v}^{q}
            }{
                w_q
            }
        \right)
    }
    % :=
    =:
    \tilde{w}_{q}(l)
    =
    \ta(l)
    \in [0,1]
    \
    \forall
    \
    v \in \{1, \dots, p_q\}
    \\
    &
    \therefore
    w_{q}(l)
    = 
    w_q \times \tilde{w}_{q}(l)
    =
    w_q \times \ta(l)
    \land
    \int_{0}^{1}{
        w_{q}(l)
        dl
    }
    =
    w_q
    .
    \end{align}
\end{definition} 
% endregion
Therefore, employers are guaranteed maximum insurance against workers' potential underqualification; and employability is simply
\begin{gather}
\tilde{W}_{k}^{q} 
= 
\left[
    h_{k}^{q}
    \geq
    \frac{1}{2}
\right]
\int_{0}^{1}
T(l, \ell_{k}^{q})
\tilde{w}_{q}(l)
dl
= 
\left[
    h_{k}^{q}
    \geq
    \frac{1}{2}
\right]
\int_{0}^{\tilde{T}_{k}^{q}}
\ta(l)
dl
,
\end{gather}
where the hireability statistic $h_{k}^{q} \in [0,1]$ accounts for hiring requirements other than productivity; and $\tilde{w}_{q}(l)$ is the proportion of fractional positions for a particular job subtype, which coincides with its time allocation when there are infinite ``jobs'', each dedicated to a single, infinitely narrow task. We note, as well, this formula is the same as in binary labor stratification (with ``junior'' and ``senior'' positions). So, again, employability is the percentage of an operation's duration one can accomplish. And we may formalize this conclusion as follows.
\begin{lemma}[Infinite Stratification Lemma, ISL]
    \label{isl}
    If fractional job posts are allowed, with
    \begin{gather}
        w_{v}^{q} 
        > 0
        % \in 
        % [0, w_q]
        \
        \forall
        \
        v \in \{1, \dots, p_q\}
        ,
        p_q \in \{1, 2, 3, \dots\}
        ,
        \\
        \sum_{v=1}^{p_q}
        w_{v}^{q}
        :=
        w_q
        ,
        \end{gather}
        employers' optimal choice is to infinitely split positions as infinitesimal tasks,
        \begin{gather}
            \lim_{p_q \rightarrow \infty}{
                \tilde{w}_{v}^{q}
            }
            % :=
            =:
            \tilde{w}_{q}(l)
            =
            \ta(l)
            ,
            \end{gather}
            so that employability becomes:
            \begin{gather}
                \tilde{W}_{k}^{q} 
                = 
                \left[
                    h_{k}^{q}
                    \geq
                    \frac{1}{2}
                \right]
                \int_{0}^{\tilde{T}_{k}^{q}}
                \ta(l)
                dl
                .
                \end{gather}
    \begin{proof}
        See above.
    \end{proof}
\end{lemma}
% endregion

This being said, infinitely stratified markets are rather abstract, and it is not realistic to think of actual job posts as infinitesimal tasks; for, then, the very concept of a ``job'' itself disappears. Fractional positions do not make much sense in reality, where jobs usually deal with a set of multiple responsibilities. Furthermore, a maximally -- though not infinitely -- stratified labor market with sufficient positions, will, in practice, yield the same results when $w_q$ is large enough, so that we do not even need to consider infinite labor stratification as a production strategy.

% region: maximum stratification axiom (msa) 
\begin{axiom}[Maximum Stratification Axiom, MSA]
    Therefore, let us assume
    \begin{gather}
    p_q \in \{1, \dots, w_q\}
    ,
    \\
    \sum_{v=1}^{p_q}
    w_{v}^{q}
    :=
    w_q
    ,
    \end{gather}
    and
    \begin{gather}
    w_{v}^{q}
    \geq
    1
    % \in 
    % [1, w_q]
    \
    \forall
    \
    v \in \{1, \dots, p_q\}
    ,
    \end{gather}
    as it is somewhat arbitrary setting minimum employment levels to any value other than one; for then it would always be optimal to choose an even smaller value than that, in which case we would converge back to an infinitely stratified labor market. Thus, we define there has to be at least one worker per position.
\end{axiom}
% endregion

\subsubsection{Efficiently Stratified Markets}
With this, we can now demonstrate that, given the above, maximum labor stratification is, in fact, the most efficient production strategy and, so, holds in the labor market. But, to do so, we must first derive an upper limit for aggregate operational output, irrespective of productive arrangement, to serve as our ``benchmark'' and show other strategies cannot yield higher production.

% region: maximum operational output lemma (mool) 
\begin{lemma}[Maximum Operational Output Lemma, MOOL]
    The maximum operational output of any labor market is exactly the number of employees in its workforce,
    \begin{gather}
        \bar{\oo}_{q}
        =
        \min(
            \boldsymbol{w_{q}^{*}}
            \times
            \boldsymbol{\oo_q}
        )
        = w_q
        ;
    \end{gather}
    where $\boldsymbol{w_{q}^{*}}$ is the vector of optimal employment levels in a labor market with $w_q$ employees; and $\boldsymbol{\oo_q}$, the vector of partial operational outputs. Or, assuming maximum labor stratification with unitary employment,
    \begin{gather}
        \bar{\oo}_{q}
        =
        \min(
            \boldsymbol{1}
            \times
            \boldsymbol{\oo_q}(
                \boldsymbol{\ell_{q}^{*}}
            )
        )
        = w_q
        ,
    \end{gather}
    where $\boldsymbol{\ell_{q}^{*}}$ are optimal stratification bounds for the responsibility spectra of occupation $q$'s job posts (see ``Optimal Stratification Lemma'' below).
    
    Moreover, this maximum production is only attainable when the portion of each position relative to the entire workforce respects the Proportional Employment Condition (PEC),
    \begin{gather}
        \boldsymbol{\tilde{w}_{q}^{*}}
        :=
        \frac{
            \boldsymbol{w_{q}^{*}}
        }{
            w_q
        }
        =
        \boldsymbol{\Omega_{q}}
        ,
    \end{gather}
    which determines the ratio, or proportion, of a particular job subtype in a labor market is the percentage of an operation's total time duration,
    \begin{gather}
        \boldsymbol{1} ^ {\top}
        \cdot
        \boldsymbol{\Omega_{q}}
        := 1
        % = 1
        ,
    \end{gather}
    accounted by it.
    
    \begin{proof}
        As we want to derive maximum operational output, throughout this proof we assume
        \begin{gather}
            \boldsymbol{\ddot{w}_{q}}
            :=
            \boldsymbol{w_q}
            \iff
            \tilde{T}_{v}^{r}
            \geq
            \ell_{v}^{q}
            ,
        \end{gather}
        for all $
        r \in \{1, \dots, w_{v}^{q}\}
        ,
        v \in \{1, \dots, p_q\}
        ,
        p_q \in \{1, \dots, w_q\}
        $; that is, all employees are sufficiently qualified for their responsibilities.
        
        Given this, we begin with the most trivial of economic configurations, that of independent production with perfectly qualified workers. In this scenario, each employee devotes their unitary time allowance,
        \begin{gather}
            \int_{0}^{1}{
                \ta(l)
                dl
            }
            =
            1
            ,
        \end{gather}
        to output exactly one productive unit,
        \begin{gather}
            1
            \times
            \left(
                \int_{0}^{1}{
                    \ta(l)dl
                }
            \right) ^ {-1}
            =
            1
            ,
        \end{gather}
        while $w_q$ of such employees working in parallel, yield an output of
        \begin{gather}
            \oo_{q}^{\text{IP}}
            :=
            \sum_{v=1}^{w_q}{
                \left[
                    \tilde{T}_{v}^{q}
                    =
                    1
                \right]
                \times
                \left(
                    \int_{0}^{1}{
                        \ta(l)
                        dl
                    }
                \right) ^ {-1}
            }
            =
            \sum_{v=1}^{w_q}{
                \left[
                    \tilde{T}_{v}^{q}
                    =
                    1
                \right]
            }
            =
            w_q
            .
        \end{gather}

        Therefore, a perfectly qualified employee working full-time and independently can output one unit of an occupation's complex tasks with one unit of their time (i.e. their entire time allowance). And, likewise, a workforce with $w_q$ employees identical to this one produces $w_q$ units of operational output. Or, to put it simply, a maximally productive person acchieves maximum production.
        
        We, now, proceed with the binary setting presented above, where employers choose a $\tilde{w}_{\text{Jr}}^{q} \in (0,1)$ percentage of less qualified (i.e. ``junior'') job posts to offer, which determines the remaining $\tilde{w}_{\text{Sr}}^{q} := 1 - \tilde{w}_{\text{Jr}}^{q} \in (0,1)$ percentage of perfectly qualified (or ``senior'') positions. Here, we have
        \begin{align}
            \oo(\tilde{w}_{\text{Jr}}^{q})
            &=
            \min\left(
                \tilde{w}_{\text{Jr}}^{q}
                \times
                \oo_{\text{Jr}}^{q}
                ,
                \tilde{w}_{\text{Sr}}^{q}
                \times
                \oo_{\text{Sr}}^{q}
            \right)
            \\
            &=
            \min\left(
                \frac{
                    \tilde{w}_{\text{Jr}}^{q}
                }{
                    \int_{0}^{
                        \ell_{\text{Jr}}^{q}
                    }{
                        \ta(l)
                        dl
                    }
                }
                ,
                \frac{
                    1 - \tilde{w}_{\text{Jr}}^{q}
                }{
                    \int_{
                        \ell_{\text{Jr}}^{q}
                    }^{1}{
                        \ta(l)
                        dl
                    }
                }
            \right)
            \\
            &=
            \min\left(
                \frac{
                    \tilde{w}_{\text{Jr}}^{q}
                }{
                    \int_{0}^{
                        \ell_{\text{Jr}}^{q}
                    }{
                        \ta(l)
                        dl
                    }
                }
                ,
                \frac{
                    1 - \tilde{w}_{\text{Jr}}^{q}
                }{
                    \int_{0}^{1}{
                        \ta(l)
                        dl
                    }
                    -
                    \int_{0}^{
                        \ell_{\text{Jr}}^{q}
                    }{
                        \ta(l)
                        dl
                    }
                }
            \right)
            \\
            &=
            \min\left(
                \frac{
                    \tilde{w}_{\text{Jr}}^{q}
                }{
                    \Omega_{\text{Jr}}^{q}
                }
                ,
                \frac{
                    1 - \tilde{w}_{\text{Jr}}^{q}
                }{
                    1 - \Omega_{\text{Jr}}^{q}
                }
            \right)
            ,
        \end{align}
        whereas the operational output of employing $\Omega_{\text{Jr}}^{q} \in (0,1)$ is
        % whereas the operational output of $\Omega_{\text{Jr}}^{q} \in [0,1]$ is
        \begin{align}
            \oo(\Omega_{\text{Jr}}^{q}) 
            =
            \min\left(
                \frac{
                    \Omega_{\text{Jr}}^{q}
                }{
                    \Omega_{\text{Jr}}^{q}
                }
                ,
                \frac{
                    1 - \Omega_{\text{Jr}}^{q}
                }{
                    1 - \Omega_{\text{Jr}}^{q}
                }
            \right)
            =
            \frac{
                \Omega_{\text{Jr}}^{q}
            }{
                \Omega_{\text{Jr}}^{q}
            }
            =
            \frac{
                1 - \Omega_{\text{Jr}}^{q}
            }{
                1 - \Omega_{\text{Jr}}^{q}
            }
            =
            1
            .
        \end{align}
        
        With this, if $\tilde{w}_{\text{Jr}}^{q}$ is set to $\tilde{w}_{\text{Jr}}^{q} > \Omega_{\text{Jr}}^{q}$, then
        \begin{align}
            &
            1 - \tilde{w}_{\text{Jr}}^{q}
            <
            1 - \Omega_{\text{Jr}}^{q}
            \therefore
            \frac{
                \tilde{w}_{\text{Jr}}^{q}
            }{
                \Omega_{\text{Jr}}^{q}
            }
            >
            1
            >
            \frac{
                1 - \tilde{w}_{\text{Jr}}^{q}
            }{
                1 - \Omega_{\text{Jr}}^{q}
            }
            \\
            &\therefore
            \oo(\tilde{w}_{\text{Jr}}^{q}) 
            =
            \min\left(
                \frac{
                    \tilde{w}_{\text{Jr}}^{q}
                }{
                    \Omega_{\text{Jr}}^{q}
                }
                ,
                \frac{
                    1 - \tilde{w}_{\text{Jr}}^{q}
                }{
                    1 - \Omega_{\text{Jr}}^{q}
                }
            \right)
            =
            \frac{
                1 - \tilde{w}_{\text{Jr}}^{q}
            }{
                1 - \Omega_{\text{Jr}}^{q}
            }
            <
            1
            \\
            &
            \implies
            \oo(\tilde{w}_{\text{Jr}}^{q})
            <
            \oo(\Omega_{\text{Jr}}^{q})
            =
            1
            ;
        \end{align}
        and, if $\tilde{w}_{\text{Jr}}^{q} < \Omega_{\text{Jr}}^{q}$,
        \begin{align}
            &
            1 - \tilde{w}_{\text{Jr}}^{q}
            >
            1 - \Omega_{\text{Jr}}^{q}
            \therefore
            \frac{
                \tilde{w}_{\text{Jr}}^{q}
            }{
                \Omega_{\text{Jr}}^{q}
            }
            <
            1
            <
            \frac{
                1 - \tilde{w}_{\text{Jr}}^{q}
            }{
                1 - \Omega_{\text{Jr}}^{q}
            }
            \\
            &\therefore
            \oo(\tilde{w}_{\text{Jr}}^{q}) 
            =
            \min\left(
                \frac{
                    \tilde{w}_{\text{Jr}}^{q}
                }{
                    \Omega_{\text{Jr}}^{q}
                }
                ,
                \frac{
                    1 - \tilde{w}_{\text{Jr}}^{q}
                }{
                    1 - \Omega_{\text{Jr}}^{q}
                }
            \right)
            =
            \frac{
                \tilde{w}_{\text{Jr}}^{q}
            }{
                \Omega_{\text{Jr}}^{q}
            }
            <
            1
            \\
            &\implies
            \oo(\tilde{w}_{\text{Jr}}^{q})
            <
            \oo(\Omega_{\text{Jr}}^{q})
            =
            1
            .
        \end{align}
        
        Hence, 
        % Therefore, 
        \begin{gather}
            \oo(\tilde{w}_{\text{Jr}}^{q}) < 
            \oo(\Omega_{\text{Jr}}^{q}) 
            = 1
            \
            \forall
            \
            \tilde{w}_{\text{Jr}}^{q}
            \neq
            \Omega_{\text{Jr}}^{q}
            \in 
            (0,1)
            .
        \end{gather}
        
        Analogously, with multiple job subtypes, optimal operational output is
        \begin{gather}
            \oo(\boldsymbol{\Omega_q})
            = 
            \min(
                \boldsymbol{\Omega_q}
                \times
                \boldsymbol{\oo_q}
            )
            =
            \frac{
                \Omega_{v}^{q}
            }{
                \Omega_{v}^{q}
            }
            =
            1
            ;
        \end{gather}
        for, again, since
        \begin{gather}
            1
            =:
            \boldsymbol{1} ^ {\top}
            \cdot
            \boldsymbol{\tilde{w}_{q}}
            =
            \boldsymbol{1} ^ {\top}
            \cdot
            \boldsymbol{\Omega_{q}}
            := 1
            ,
        \end{gather}
        choosing any $\tilde{w}_{v}^{q} \neq \Omega_{v}^{q}$ implies the proportion of at least one position, say $\tilde{w}_{r}^{q}$, is impacted, and aggregate output along with it, either because
        \begin{align}
            &
            \tilde{w}_{v}^{q} > \Omega_{v}^{q}
            \therefore
            \frac{
                \tilde{w}_{v}^{q}
            }{
                \Omega_{v}^{q}
            }
            > 
            1
            > 
            \frac{
                \tilde{w}_{r}^{q}
            }{
                \Omega_{r}^{q}
            }
            \\
            &\therefore
            \oo(\boldsymbol{\tilde{w}_q})
            = 
            \min(
                \boldsymbol{\tilde{w}_q}
                \times
                \boldsymbol{\oo_q}
            )
            =
            \frac{
                \tilde{w}_{r}^{q}
            }{
                \Omega_{r}^{q}
            }
            < 
            1
            \\
            &
            \implies
            \oo(\boldsymbol{\tilde{w}_q})
            <
            \oo(\boldsymbol{\Omega_q})
            =
            1
            ;
        \end{align}
        or, alternatively, because
        \begin{align}
            &
            \tilde{w}_{v}^{q} < \Omega_{v}^{q}
            \therefore
            \frac{
                \tilde{w}_{v}^{q}
            }{
                \Omega_{v}^{q}
            }
            < 
            1
            < 
            \frac{
                \tilde{w}_{r}^{q}
            }{
                \Omega_{r}^{q}
            }
            \\
            &\therefore
            \oo(\boldsymbol{\tilde{w}_q})
            = 
            \min(
                \boldsymbol{\tilde{w}_q}
                \times
                \boldsymbol{\oo_q}
            )
            =
            \frac{
                \tilde{w}_{v}^{q}
            }{
                \Omega_{v}^{q}
            }
            < 
            1
            \\
            &
            \implies
            \oo(\boldsymbol{\tilde{w}_q})
            <
            \oo(\boldsymbol{\Omega_q})
            =
            1
            .
        \end{align}
        
        Thus, 
        % \begin{gather}
        \begin{align}
            &
            \oo(
                \boldsymbol{\tilde{w}_{q}},
                \boldsymbol{\oo_q}
            ) 
            < 
            \oo(
                \boldsymbol{\Omega_{q}},
                \boldsymbol{\oo_q}
            ) 
            = 1
            \therefore
            \oo(
                \boldsymbol{w_{q}},
                \boldsymbol{\oo_q}
            )
            < 
            \oo(
                w_{q}
                \times
                \boldsymbol{\Omega_{q}},
                \boldsymbol{\oo_q}
            ) 
            = w_q
            \\
            &
            \forall
            \
            \boldsymbol{\tilde{w}_{q}}
            \neq
            \boldsymbol{\Omega_{q}}
            \in
            [0,1]^{p_q}
            ,
            p_q \in \{1, \dots, w_q\}
            ,
            1 =:
            \boldsymbol{1} ^ {\top}
            \cdot
            \boldsymbol{\tilde{w}_{q}}
            =
            \boldsymbol{1} ^ {\top}
            \cdot
            \boldsymbol{\Omega_{q}}
            := 1
            .
        \end{align}
        % \end{gather}
        % \begin{gather}
        %     \oo(
        %         \boldsymbol{\tilde{w}_{q}},
        %         \boldsymbol{\oo_q}
        %     ) 
        %     < 
        %     \oo(
        %         \boldsymbol{\Omega_{q}},
        %         \boldsymbol{\oo_q}
        %     ) 
        %     = 1
        %     \\
        %     \therefore
        %     \oo(
        %         \boldsymbol{w_{q}},
        %         \boldsymbol{\oo_q}
        %     )
        %     < 
        %     \oo(
        %         w_{q}
        %         \times
        %         \boldsymbol{\Omega_{q}},
        %         \boldsymbol{\oo_q}
        %     ) 
        %     = w_q
        %     \\
        %     \forall
        %     \
        %     \boldsymbol{\tilde{w}_{q}}
        %     \neq
        %     \boldsymbol{\Omega_{q}}
        %     \in
        %     [0,1]^{p_q}
        %     ,
        %     p_q \in \{1, \dots, w_q\}
        %     ,
        %     \\
        %     1 =:
        %     \boldsymbol{1} ^ {\top}
        %     \cdot
        %     \boldsymbol{\tilde{w}_{q}}
        %     =
        %     \boldsymbol{1} ^ {\top}
        %     \cdot
        %     \boldsymbol{\Omega_{q}}
        %     := 1
        %     .
        % \end{gather}
        
        We can derive the same conclusion for a maximally stratified labor market, as well. But here, instead of determining the proportion of job subtypes with a $\boldsymbol{w_{q}^{*}}$ vector of employment levels, employers maximize production selecting optimal $\boldsymbol{\ell_{q}^{*}}$ responsibility bounds for $w_q$ unique job posts. So, let 
        \begin{gather}
            \boldsymbol{\ell_{q}^{*}}
            :=
            (\ell_{0}^{q*}, \dots, \ell_{w_q}^{q*})
            :=
            (0, \dots, 1) 
            \in [0,1]^{w_q}
            ,
        \end{gather}
        with
        \begin{gather}
        \sum_{v=1}^{w_q}
        \int_{
            \ell_{v-1}^{q*}
        }^{
            \ell_{v}^{q*}
        }
        \ta(l)dl
        =
        \int_{0}^{1}
        \ta(l)dl
        =
        1
        \end{gather}
        be the vector of optimal responsibility bounds that maximizes operational output, such that
        \begin{gather}
            \oo(\boldsymbol{\ell_{q}^{*}})
            =
            \min(
                \boldsymbol{1}
                \times
                \boldsymbol{\oo_q}(
                    \boldsymbol{\ell_{q}^{*}}
                )
            )
            =
            1
            \times
            \left(
                \int_{
                    \ell_{v-1}^{q*}
                }^{
                    \ell_{v}^{q*}
                }
                \ta(l)dl
            \right) ^ {-1}
            :=
            w_q
            ,
        \end{gather}
        as with the previous economic configurations.
        
        Note employers could, again, attempt to increase production beyond this level if they, now, reduced the responsibilities of a particular job subtype by setting
        \begin{align}
        \ell_{v}^{q} < \ell_{v}^{q*}
        \implies
            \left(
                \int_{
                    \ell_{v-1}^{q*}
                }^{
                    \ell_{v}^{q}
                }
                \ta(l)dl
            \right) ^ {-1}
            >
            \left(
                \int_{
                    \ell_{v-1}^{q*}
                }^{
                    \ell_{v}^{q*}
                }
                \ta(l)dl
            \right) ^ {-1}
            :=
            w_q
            .
        \end{align}
        
        Nevertheless, because every worker has the same unitary time allowance, this would also entail the complementary subinterval of complex tasks $l \in [\ell_{v}^{q}, \ell_{v}^{q*}]$ would either not be produced at all, in which case
        \begin{align}
            \oo(\boldsymbol{\ell_{q}})
            =
            0
            \times
            \left(
                \int_{
                    \ell_{v}^{q}
                }^{
                    \ell_{v}^{q*}
                }
                \ta(l)dl
            \right) ^ {-1}
            =
            0
            ,
        \end{align}
        or that it would be produced with a $1 - \omega_{v}^{q} \in [0,1]$ fraction of a time unit, yielding some quantity
        \begin{align}
            \oo(
                \boldsymbol{\ell_{q}}
                ,\boldsymbol{\omega_{q}}
            )
            =
            (1 - \omega_{v}^{q})
            \times
            \left(
                \int_{
                    \ell_{v}^{q}
                }^{
                    \ell_{v}^{q*}
                }
                \ta(l)dl
            \right) ^ {-1}
            ,
        \end{align}
        where $\omega_{v}^{q} \in [0,1]$ is the percentage of worker $v$'s time allowance dedicated to the emphasized $l \in [\ell_{v-1}^{q*}, \ell_{v}^{q}]$ responsibility spectrum.
        
        Furthermore, because aggregate operational output is given by the Leontief production function, 
        \begin{gather}
            \oo_{q}(
                \boldsymbol{\ell_q}
                ,\boldsymbol{\omega_q}
            )
            :=
            \min(
                \boldsymbol{1}
                \times
                (
                    \oo_{1}^{q}
                    ,
                    \dots
                    ,
                    \oo_{w_q}^{q}
                )
            )
            ,
            \\
            \oo_{v}^{q}
            := 
            \min
            \left(
                    \frac{
                        \omega_{v}^{q}
                    }{
                        \int_{\ell_{v-1}^{q*}}^{\ell_{v}^{q}}{
                            \ta(l)dl
                        }
                    }
                    ,
                    \frac{
                        1 - \omega_{v}^{q}
                    }{
                        \int_{\ell_{v}^{q}}^{\ell_{v}^{q*}}{
                            \ta(l)dl
                        }
                    }
                \right)
            ,
        \end{gather}
        it would be pointless if only a subset of employees were to increase their operational output by themselves; for an occupation's complex tasks are all complementary: they work together to acchieve its operation. Hence, for $\oo_{q}(
            \boldsymbol{\ell_q}
            ,\boldsymbol{\omega_q}
        )$ to be greater than $
        \oo_{q}(
            \boldsymbol{\ell_{q}^{*}}
        )
        =
        w_q$,
        \begin{gather}
            \oo_{v}^{q} > w_q
            \
            \forall
            \
            v \in \{1, \dots, w_q\}
            ,
        \end{gather}
        which requires all partial operational outputs to surpass the following point of equilibrium:
        \begin{align}
            &\oo_{q}(
                \boldsymbol{\ell_q}
                ,\boldsymbol{\omega_q}
            )
            =
            \oo_{q}(
                \boldsymbol{\ell_{q}^{*}}
            ) 
            =
            \min(
                \boldsymbol{1}
                \times
                \boldsymbol{\oo_q}(
                    \boldsymbol{\ell_{q}^{*}}
                )
            )
            =
            w_q
            \\
            &\iff
            \frac{
                \omega_{v}^{q}
            }{
                \int_{
                    \ell_{v-1}^{q*}
                }^{
                    \ell_{v}^{q}
                }{
                    \ta(l)dl
                }
            }
            =
            \frac{
                1 - \omega_{v}^{q}
            }{
                \int_{
                    \ell_{v}^{q}
                }^{
                    \ell_{v}^{q*}
                }{
                    \ta(l)dl
                }
            }
            =
            w_q
            \
            \forall
            \
            v \in \{1, \dots, w_q\}
            \\
            &\iff
            \omega_{v}^{q}
            =
            w_q
            \int_{
                \ell_{v-1}^{q*}
            }^{
                \ell_{v}^{q}
            }{
                \ta(l)
                dl
            }
            \land
            1 - \omega_{v}^{q}
            =
            w_q
            \int_{
                \ell_{v}^{q}
            }^{
                \ell_{v}^{q*}
            }{
                \ta(l)
                dl
            }
            .
        \end{align}
        % \begin{align}
        %     &\oo_{q}(
        %         \boldsymbol{\ell_q}
        %         ,\boldsymbol{\omega_q}
        %     ) 
        %     =
        %     \oo_{q}(
        %         \boldsymbol{\ell_{q}^{*}}
        %     ) 
        %     = 
        %     \min(
        %         \boldsymbol{1}
        %         \times
        %         \boldsymbol{\oo_q}(
        %             \boldsymbol{\ell_{q}^{*}}
        %         )
        %     )
        %     =
        %     w_q
        %     \\
        %     &\iff
        %     \omega_{v}^{q}
        %     \oo_{v}^{q}(
        %         \ell_{v-1}^{q*}
        %         ,\ell_{v}^{q}
        %     )
        %     =
        %     (1-\omega_{v}^{q})
        %     \oo_{v}^{q}(
        %         \ell_{v}^{q}
        %         ,\ell_{v}^{q*}
        %     )
        %     =
        %     w_q
        %     \
        %     \forall
        %     \
        %     v \in \{1, \dots, w_q\}
        %     \\
        %     &\iff
        %     \omega_{v}^{q}
        %     =
        %     \frac{w_q}{
        %         \oo_{v}^{q}(
        %             \ell_{v-1}^{q*}
        %             ,\ell_{v}^{q}
        %         )
        %     }
        %     :=
        %     w_q
        %     \Omega_{v}^{q}(
        %         \ell_{v-1}^{q*}
        %         ,\ell_{v}^{q}
        %     )
        %     :=
        %     w_q
        %     \int_{
        %         \ell_{v-1}^{q*}
        %     }^{
        %         \ell_{v}^{q}
        %     }{
        %         \ta(l)
        %         dl
        %     }
        %     \\
        %     &\land
        %     1 - \omega_{v}^{q}
        %     =
        %     \frac{w_q}{
        %         \oo_{v}^{q}(
        %             \ell_{v}^{q}
        %             ,\ell_{v}^{q*}
        %         )
        %     }
        %     :=
        %     w_q
        %     \Omega_{v}^{q}(
        %         \ell_{v}^{q}
        %         ,\ell_{v}^{q*}
        %     )
        %     :=
        %     w_q
        %     \int_{
        %         \ell_{v}^{q}
        %     }^{
        %         \ell_{v}^{q*}
        %     }{
        %         \ta(l)
        %         dl
        %     }
        %     ,
        % \end{align}
        % where $\oo_{v}^{q}(\ell_{v}^{q},\ell_{r})$, with
        % \begin{gather}
        %     \oo_{v}^{q}
        %     =
        %     \min
        %     \left[
        %         \omega_{v}^{q}
        %         \oo_{v}^{q}(
        %             \ell_{v-1}^{q*}
        %             ,\ell_{v}^{q}
        %         )
        %         ,
        %         (1 - \omega_{v}^{q})
        %         \oo_{v}^{q}(
        %             \ell_{v}^{q}
        %             ,\ell_{v}^{q*}
        %         )
        %     \right]
        %     ,
        % \end{gather}
        % and $\Omega_{v}^{q}(\ell_{v}^{q},\ell_{r}), v, r \in \{1, \dots, w_q\}$, are short-hand notations for partial operational output and its inverse, total time allocation.
        
        Now, if any single $\omega_{v}^{q} \in [0,1]$ is set to
        \begin{gather}
            \omega_{v}^{q} > 
            w_q
            \int_{
                \ell_{v-1}^{q*}
            }^{
                \ell_{v}^{q}
            }{
                \ta(l)
                dl
            }
            % w_q
            % \Omega_{v}^{q}(
            %     \ell_{v-1}^{q*}
            %     ,\ell_{v}^{q}
            % )
            ,
        \end{gather}
        then, indeed,
        \begin{gather}
            \frac{
                \omega_{v}^{q}
            }{
                \int_{
                    \ell_{v-1}^{q*}
                }^{
                    \ell_{v}^{q}
                }{
                    \ta(l)dl
                }
            }
            % \omega_{v}^{q}
            % \oo_{v}^{q}(
            %     \ell_{v-1}^{q*}
            %     ,\ell_{v}^{q}
            % )
            >
            w_q
            ,
        \end{gather}
        but also
        \begin{align}
            &
            \frac{
                1 - \omega_{v}^{q}
            }{
                \int_{
                    \ell_{v}^{q}
                }^{
                    \ell_{v}^{q*}
                }{
                    \ta(l)dl
                }
            }
            <
            w_q
            \\
            &\therefore
            \oo_{v}^{q} = 
            \min\left(
                \frac{
                    \omega_{v}^{q}
                }{
                    \int_{
                        \ell_{v-1}^{q*}
                    }^{
                        \ell_{v}^{q}
                    }
                    \ta(l)dl
                }
                ,
                \frac{
                    1-\omega_{v}^{q}
                }{
                    \int_{
                        \ell_{v}^{q}
                    }^{
                        \ell_{v}^{q*}
                    }
                    \ta(l)dl
                }
            \right)
            =
            \frac{
                1 - \omega_{v}^{q}
            }{
                \int_{
                    \ell_{v}^{q}
                }^{
                    \ell_{v}^{q*}
                }{
                    \ta(l)dl
                }
            }
            <
            w_q
            \\
            &\implies
            % &\implies
            \oo_{q}(
                \boldsymbol{\ell_q}
                ,\boldsymbol{\omega_q}
            ) 
            <
            \oo_{q}(
                \boldsymbol{\ell_{q}^{*}}
            )
            =
            w_q
            ;
        \end{align}
        and, conversely, 
        \begin{align}
            &
            \omega_{v}^{q} < 
            w_q
            \int_{
                \ell_{v-1}^{q*}
            }^{
                \ell_{v}^{q}
            }{
                \ta(l)dl
            }
            \therefore
            \frac{
                \omega_{v}^{q}
            }{
                \int_{
                    \ell_{v-1}^{q*}
                }^{
                    \ell_{v}^{q}
                }
                \ta(l)dl
            }
            < 
            w_q
            <
            \frac{
                1-\omega_{v}^{q}
            }{
                \int_{
                    \ell_{v}^{q}
                }^{
                    \ell_{v}^{q*}
                }
                \ta(l)dl
            }
            \\
            &\therefore
            \oo_{v}^{q} = 
            \min\left(
                \frac{
                    \omega_{v}^{q}
                }{
                    \int_{
                        \ell_{v-1}^{q*}
                    }^{
                        \ell_{v}^{q}
                    }
                    \ta(l)dl
                }
                ,
                \frac{
                    1-\omega_{v}^{q}
                }{
                    \int_{
                        \ell_{v}^{q}
                    }^{
                        \ell_{v}^{q*}
                    }
                    \ta(l)dl
                }
            \right)
            =
            \frac{
                \omega_{v}^{q}
            }{
                \int_{
                    \ell_{v-1}^{q*}
                }^{
                    \ell_{v}^{q}
                }
                \ta(l)dl
            }
            <
            w_q
            \\
            &\implies
            \oo_{q}(
                \boldsymbol{\ell_q}
                ,\boldsymbol{\omega_q}
            ) 
            <
            \oo_{q}(
                \boldsymbol{\ell_{q}^{*}}
            )
            =
            w_q
            ;
        \end{align}
        so that 
        \begin{gather}
            \nexists \
            \boldsymbol{\ell_q}
            % \neq
            % \boldsymbol{\ell_{q}^{*}}
            ,
            \boldsymbol{\omega_q}
            \in
            [0,1]^{w_q}
            \
            \big|
            \
            \oo_{q}(
                \boldsymbol{\ell_q}
                ,\boldsymbol{\omega_q}
            ) 
            >
            \oo_{q}(
                \boldsymbol{\ell_{q}^{*}}
            )
            =
            % \min(
            %     \boldsymbol{1}
            %     \times
            %     \boldsymbol{\oo_q}(
            %         \boldsymbol{\ell_{q}^{*}}
            %     )
            % )
            % =
            w_q
            ,
            \\
            \sum_{v=1}^{w_q}
            \left(
                \int_{
                    \ell_{v-1}^{q*}
                }^{
                    \ell_{v}^{q}
                }
                \ta(l)dl
                +
                \int_{
                    \ell_{v}^{q}
                }^{
                    \ell_{v}^{q*}
                }
                \ta(l)dl
            \right)
            =
            \int_{0}^{1}
            \ta(l)dl
            =
            1
            .
        \end{gather}
        
        Finally, even with an $\boldsymbol{\omega_{v}^{q}}$ vector of partial time allocations for each worker, at least one difficulty subinterval would have to be neglected to emphasize another,
        \begin{align}
            &
            \because
            1 =:
            \boldsymbol{1} ^ \top
            \cdot
            \boldsymbol{\omega_{v}^{q}}
            =
            \boldsymbol{1} ^ \top
            \cdot
            \boldsymbol{\Omega_{v}^{q}}
            := 1
            \therefore
            \min\left(
                \boldsymbol{\omega_{v}^{q}}
                \times
                \boldsymbol{\oo_q}(
                    \boldsymbol{\ell_{q}}
                )
            \right)
            <
            w_q
            \\
            &
            \implies
            \oo_q(
                \boldsymbol{\ell_{v}^{q}},
                \boldsymbol{\omega_{v}^{q}}
            )
            <
            \oo_q(
                \boldsymbol{\ell_{q}^{*}}
            )
            =
            w_q
            ,
        \end{align}
        as before.
        
        Thus, we have demonstrated there cannot be, in any productive arrangement, a higher aggregate operational output than $w_q$, that is the number of employees in the workforce, as all attempts to increase production, actually, end up hindering it. 
        
        The intuition for this is quite simple. Production strategies can merely distribute the available talent across an occupation's responsibility spectrum: they are but ways of splitting and organizing tasks conveniently (via independent production, or any level of labor stratification); they do not, however, change activities' time requirements, nor the time allowances of employees, both of which are, by definition, equivalent. So, these economic configurations only serve to ``safeguard'' operational output against worker's potential underqualification. The main limiting factors to production, then, are workers' capacity and time itself. Hence, we may say, somewhat tautologically, the most one can produce in a day is a ``day's work''.
    \end{proof}
\end{lemma}
% endregion

In the lemma above, we have assumed there to be an optimal $\boldsymbol{\ell_{q}^{*}}$ vector of responsibility bounds maximizing operational output in stratified production. We shall, now, devote out attention to describing what such a vector would have to be like and, thus, how production is optimally stratified.

% region: optimal stratification lemma (osl) 
\begin{lemma}[Optimal Stratification Lemma, OSL]
    Because in a maximally and monotonically stratified labor market every position is its own job subtype (for, again, employment levels are unitary), optimal production is, then, obtained not by choosing how many workers to allocate to tasks of varying difficulty levels, but instead by setting appropriate responsibility ranges for each position (i.e. which tasks to allocate \textit{to} workers). The bounds for these ranges are:
    \begin{gather}
    \ell_{v}^{q*}
    =
    \invTA\left(
        \frac{v}{w_q}
        +
        \TA(0)
    \right)
    \
    \forall
    \
    v \in 
    \left\{
        1, \dots, w_q
    \right\}
    ,
    \end{gather}
    where $\TA(l)$ is the anti-derivative of the time allocation function $\ta(l)$, and $\invTA(l)$, its inverse.
    
    \begin{proof}
        
        We have just demonstrated the ceiling for operational output in any labor market, with or without unique, unitary, positions, is exactly 
        \begin{gather}
        \bar{\oo}_{q}
            =
            \min(
                \boldsymbol{w_{q}^{*}}
                \times
                \boldsymbol{\oo_q}
            )
            = 
            \min(
                \boldsymbol{1}
                \times
                \boldsymbol{\oo_q}(
                    \boldsymbol{\ell_{q}^{*}}
                )
            )
            =
            w_q
        ,
        \end{gather}
        or the number of employees in its workforce.
        
        Therefore, optimal bounds for responsibility spectra can be calculated by equating partial operational outputs with maximum production; for if labor stratification is to be optimal, it must yield the same partial outputs as any efficient production strategy.
        
        So, for the first job subtype,
        \begin{gather}
        1 \times \left(
            \int_{\ell_{0}^{q*}}^{\ell_{1}^{q*}}
            \ta(l)
            dl
        \right) ^ {-1}
        :=
        1 \times \left(
            \int_{0}^{\ell_{1}^{q*}}
            \ta(l)
            dl
        \right) ^ {-1}
        :=
        w_q
        ,
        \end{gather}
        which means the partial operational output of the first worker, whose tasks range from $\ell_{0}^{q*} = 0$ to $\ell_{1}^{q*} \in [0,1]$ exclusively, should produce the same amount of the $l \in [0,\ell_{1}^{q*}]$ responsibility spectrum as would be produced in an economic configuration with maximum operational output (e.g. with $w_q$ perfectly qualified employees working independently).
        
        Thus, solving for $\ell_{1}^{q*}$, we get:
        \begin{align}
        1 
        &\times 
        \left(
            \int_{0}^{\ell_{1}^{q*}}
            \ta(l)
            dl
        \right) ^ {-1}
        =
        w_q
        \\
        &\therefore
        \int_{0}^{\ell_{1}^{q*}}
        \ta(l)
        dl
        =
        \frac{1}{w_q}
        \\
        &\therefore
        \TA(l)
        \big|_{0}^{\ell_{1}^{q*}}
        =
        \TA(\ell_{1}^{q*})
        -
        \TA(0)
        =
        \frac{1}{w_q}
        \\
        &\therefore
        \invTA(
            \TA(\ell_{1}^{q*})
        )
        =
        \invTA\left(
            \frac{1}{w_q}
            +
            \TA(0)
        \right)
        \\
        &\therefore
        \ell_{1}^{q*}
        =
        \invTA\left(
            \frac{1}{w_q}
            +
            \TA(0)
        \right)
        .
        \end{align}
        
        Similarly, for the second worker,
        \begin{align}
        1 
        &\times 
        \left(
            \int_{\ell_{1}^{q*}}^{\ell_{2}^{q*}}
            \ta(l)
            dl
        \right) ^ {-1}
        =
        w_q
        \\
        &\therefore
        \int_{\ell_{1}^{q*}}^{\ell_{2}^{q*}}
        \ta(l)
        dl
        =
        \frac{1}{w_q}
        \\
        &\therefore
        \TA(l)
        \big|_{\ell_{1}^{q*}}^{\ell_{2}^{q*}}
        =
        \TA(\ell_{2}^{q*})
        -
        \TA(\ell_{1}^{q*})
        =
        \frac{1}{w_q}
        \\
        &\therefore
        \invTA(
            \TA(\ell_{2}^{q*})
        )
        =
        \invTA\left(
            \frac{1}{w_q}
            +
            \TA(\ell_{1}^{q*})
        \right)
        \\
        &\therefore
        \invTA(
            \TA(\ell_{2}^{q*})
        )
        =
        \invTA\left(
            \frac{1}{w_q}
            +
            \frac{1}{w_q}
            +
            \TA(0)
        \right)
        \\
        &\therefore
        \ell_{2}^{q*}
        =
        \invTA\left(
            \frac{2}{w_q}
            +
            \TA(0)
        \right)
        .
        \end{align}
        
        For the third worker,
        \begin{align}
        1 
        &\times 
        \left(
            \int_{\ell_{2}^{q*}}^{\ell_{3}^{q*}}
            \ta(l)
            dl
        \right) ^ {-1}
        =
        w_q
        \\
        &\therefore
        \int_{\ell_{2}^{q*}}^{\ell_{3}^{q*}}
        \ta(l)
        dl
        =
        \frac{1}{w_q}
        \\
        &\therefore
        \TA(l)
        \big|_{\ell_{2}^{q*}}^{\ell_{3}^{q*}}
        =
        \TA(\ell_{3}^{q*})
        -
        \TA(\ell_{2}^{q*})
        =
        \frac{1}{w_q}
        \\
        &\therefore
        \invTA(
            \TA(\ell_{3}^{q*})
        )
        =
        \invTA\left(
            \frac{1}{w_q}
            +
            \TA(\ell_{2}^{q*})
        \right)
        \\
        &\therefore
        \invTA(
            \TA(\ell_{3}^{q*})
        )
        =
        \invTA\left(
            \frac{1}{w_q}
            +
            \frac{1}{w_q}
            +
            \frac{1}{w_q}
            +
            \TA(0)
        \right)
        \\
        &\therefore
        \ell_{3}^{q*}
        =
        \invTA\left(
            \frac{3}{w_q}
            +
            \TA(0)
        \right)
        .
        \end{align}
        
        And so on and so forth, up to the very last worker:
        \begin{align}
        1 
        &\times 
        \left(
            \int_{\ell_{w_q - 1}^{q*}}^{\ell_{w_q}^{q*}}
            \ta(l)
            dl
        \right) ^ {-1}
        =
        w_q
        \\
        &\therefore
        \int_{\ell_{w_q - 1}^{q*}}^{\ell_{w_q}^{q*}}
        \ta(l)
        dl
        =
        \frac{1}{w_q}
        \\
        &\therefore
        \TA(l)
        \big|_{\ell_{w_q - 1}^{q*}}^{\ell_{w_q}^{q*}}
        =
        \TA(\ell_{w_q}^{q*})
        -
        \TA(\ell_{w_q - 1}^{q*})
        =
        \frac{1}{w_q}
        \\
        &\therefore
        \invTA(
            \TA(\ell_{w_q}^{q*})
        )
        =
        \invTA\left(
            \frac{1}{w_q}
            +
            \TA(\ell_{w_q - 1}^{q*})
        \right)
        \\
        &\therefore
        \invTA(
            \TA(\ell_{w_q}^{q*})
        )
        =
        \invTA\left(
            \frac{1}{w_q}
            +
            \dots
            +
            \frac{1}{w_q}
            +
            \TA(0)
        \right)
        \\
        &\therefore
        \ell_{w_q}^{q*}
        =
        \invTA\left(
            \frac{w_q}{w_q}
            +
            \TA(0)
        \right)
        =
        \invTA\left(
            1
            +
            \TA(0)
        \right)
        =
        1
        \\
        &
        \because
        \int_{0}^{1}{
            \ta(l)
            dl
        }
        =
        1
        \implies
        \TA(1)
        =
        1 + \TA(0)
        \\
        &
        \therefore
        \ell_{w_q}^{q*}
        =
        \invTA\left(
            1
            +
            \TA(0)
        \right)
        = 
        \invTA\left(
            \TA(1)
        \right)
        = 
        1
        .
        \end{align}
        % And so on and so forth, up to the very last worker:
        % \begin{align}
        % 1 
        % &\times 
        % \left(
        %     \int_{\ell_{w_q - 1}^{q*}}^{\ell_{w_q}^{q*}}
        %     \ta(l)
        %     dl
        % \right) ^ {-1}
        % =
        % w_q
        % \\
        % &\therefore
        % \int_{\ell_{w_q - 1}^{q*}}^{\ell_{w_q}^{q*}}
        % \ta(l)
        % dl
        % =
        % \frac{1}{w_q}
        % \\
        % &\therefore
        % \TA(l)
        % \big|_{\ell_{w_q - 1}^{q*}}^{\ell_{w_q}^{q*}}
        % =
        % \TA(\ell_{w_q}^{q*})
        % -
        % \TA(\ell_{w_q - 1}^{q*})
        % =
        % \frac{1}{w_q}
        % \\
        % &\therefore
        % \invTA(
        %     \TA(\ell_{w_q}^{q*})
        % )
        % =
        % \invTA\left(
        %     \frac{1}{w_q}
        %     +
        %     \TA(\ell_{w_q - 1}^{q*})
        % \right)
        % \\
        % &\therefore
        % \invTA(
        %     \TA(\ell_{w_q}^{q*})
        % )
        % =
        % \invTA\left(
        %     \frac{1}{w_q}
        %     +
        %     \dots
        %     +
        %     \frac{1}{w_q}
        %     +
        %     \TA(0)
        % \right)
        % \\
        % &\therefore
        % \ell_{w_q}^{q*}
        % =
        % \invTA\left(
        %     \frac{w_q}{w_q}
        %     +
        %     \TA(0)
        % \right)
        % := 1
        % \because
        % \ell_{p_q}^{q}
        % := 1
        % \
        % \forall
        % \
        % p_q
        % \in
        % \{1, 2, 3, \dots\}
        % \\
        % &\iff
        % \invTA\left(
        %     \frac{w_q}{w_q}
        %     +
        %     \TA(0)
        % \right)
        % =
        % \invTA\left(
        %     1
        %     +
        %     \TA(0)
        % \right)
        % =
        % 1
        % \\
        % &\iff
        % \TA\left(
        %     \invTA\left(
        %     1
        %     +
        %     \TA(0)
        % \right)
        % \right)
        % =
        % \TA(1)
        % \\
        % &\iff
        % \TA(1)
        % -
        % \TA(0)
        % =
        % \int_{0}^{1}{
        %     \ta(l)
        %     dl
        % }
        % =
        % 1
        % ,
        % \end{align}
        % which is true, by definition,
        % \begin{align}
        % &\because
        % \ta(l) 
        % :=
        % \ttc(l)
        % \times
        % \left(
        %     \int_{0}^{1}{
        %         \ttc(l)
        %         dl
        %     }
        % \right) ^ {-1}
        % \\
        % &\implies
        % % &\therefore
        % \int_{0}^{1}{
        %     \ta(l)
        %     dl
        % }
        % =
        % \left(
        %     \int_{0}^{1}
        %     \ttc(l)
        %     dl
        % \right) ^ {-1}
        % \times
        % \int_{0}^{1}{
        %     \ttc(l)
        %     dl
        % }
        % =
        % 1
        % .
        % \end{align}
        
        And, with this, we can finally arrive, by the induction above, to a general form of optimal responsibility ranges:
        % And, with this condition met, we can finally arrive, by the induction above, to a general form of optimal responsibility ranges:
        \begin{gather}
        \ell_{v}^{q*}
        =
        \invTA\left(
            \frac{v}{w_q}
            +
            \TA(0)
        \right)
        \
        \forall
        \
        v \in 
        \left\{
            1, \dots, w_q
        \right\}
        .
        \end{gather}    
    \end{proof}
\end{lemma}
% endregion

Having derived optimal responsibility ranges for maximally stratified markets with unitary employment levels, we shall demonstrate other types of labor stratification cannot attain in an efficient economy because: 1) no other strategy has lower productivity requirements (MPL); 2) and there can be only one set of optimal responsibility bounds and employment levels (ESL).

% region: minimum productivity lemma (mpl) 
\begin{lemma}[Minimum Productivity Lemma, MPL]
    \label{mpl}
    Maximally stratified markets have the lowest barrier of entry out of all valid productive arrangements.
    
    \begin{proof}
        To prove maximum stratification poses the lowest barrier of entry to the labor market, let us consider what is required in other economic configurations.
        
        If production is organized independently,
        \begin{gather}
        \oo_q = w_q
        \iff
        \tilde{T}_{v}^{q}
        =
        1
        \
        \forall
        \
        v \in \{1, \dots, w_q\}
        ,
        \end{gather}
        that is, either all $w_q$ employees are perfectly qualified, or maximum operational output (see MOOL) is not acchieved. Moreover, in the binary case,
        \begin{gather}
        \oo_q = w_q
        \iff
        \tilde{T}_{v}^{q}
        \geq
        \ell_{\text{Jr}}^{q}
        \in
        [0,1)
        \
        \forall
        \
        v \in \{1, \dots, w_{\text{Jr}}^{q}\}
        \land
        \tilde{T}_{v}^{q}
        =
        1
        ,
        \end{gather}
        for the rest of the workforce (i.e. all junior employees have at least junior productivity, and all senior employees are perfectly qualified), which means productivity requirements are lower with binary stratification when compared to independent production
        \begin{gather}
            \because
            \ell_{\text{Jr}}^{q}
            =
            \underaccent{\bar}{\tilde{T}}_{\text{Jr}}^{q}
            <
            \underaccent{\bar}{\tilde{T}}_{\text{Sr}}^{q}
            = 1
            ,
        \end{gather}
        where $\underaccent{\bar}{\tilde{T}}_{\text{Jr}}^{q}, \underaccent{\bar}{\tilde{T}}_{\text{Sr}}^{q} \in [0,1]$ denote the minimum productivity workers must have to accomplish all tasks in their responsibility spectra,
        $
            l \in [0, \ell_{\text{Jr}}^{q}]
            ,
            l \in (\ell_{\text{Jr}}^{q}, 1]
            ,
        $
        and be employable as ``juniors'' or ``seniors'', respectively.

        Likewise, if there were three levels of seniority, with $w_{\text{Ir}}^{q}$ interns (less qualified than juniors and seniors), productivity requirements would be even lower. And the same holds true for the subsequent production strategies.
        
        To show this, though, we have to ``shift gears'' and go back to looking at tasks individually, in a continuous fashion, rather than ``bundled together'' in responsibility spectra. Indeed, notice above tasks with $l \in [0, \ell_{\text{Jr}}^{q})$, that is easier than the upper bound, still demand
        \begin{gather}
            \underaccent{\bar}{\tilde{T}}_{\text{Jr}}^{q}
            =
            \ell_{\text{Jr}}^{q}
        \end{gather}
        productivity, given the labor market is only binarily stratified. Hence, if one is to be hired for either ``junior'' or ``senior'' positions, they should be competent across their entire responsibility spectrum, not just a part of it. And this becomes worse in independent production, where jobs cover every $l \in [0,1]$ difficulty level and perfect qualification is required of all employees. Thus, 
        \begin{align}
            p_q = 1
            &\implies
            \underaccent{\bar}{\tilde{T}}(l, 1)
            = 1
            \
            \forall
            \
            l \in [0,1]
            ,
            \\
            p_q = 2
            &\implies
            \underaccent{\bar}{\tilde{T}}(l, 2)
            = 
            \begin{cases}
                \ell_{\text{Jr}}^{q}, \text{ if } l \in [0, \ell_{\text{Jr}}^{q}],
                \\
                1, \text{ if } l \in (\ell_{\text{Jr}}^{q}, 1],
            \end{cases}
        \end{align}
        which we can write in the following format:
        \begin{align}
            \underaccent{\bar}{\tilde{T}}(l, 2)
            &= 
            [l \in [0, \ell_{\text{Jr}}^{q}]]
            \times
            \max[0, \ell_{\text{Jr}}^{q}]
            +
            [l \in (\ell_{\text{Jr}}^{q}, 1]]
            \times
            \max(\ell_{\text{Jr}}^{q}, 1]
            .
        \end{align}

        And, in general, for any $l \in [\ell_{r-1}^{q} ,\ell_{r}^{q}]$, $\ell_{r-1}^{q}, \ell_{r}^{q} \in [0,1]$,
        \begin{align}
            \underaccent{\bar}{\tilde{T}}(l, p_q)
            &=
            \sum_{v=1}^{p_q}\left[
                l \in (
                    \ell_{v-1}^{q}
                    ,\ell_{v}^{q}
                ]
            \right]
            \underaccent{\bar}{\tilde{T}}_{v}^{q}
            +
            [l = 0]
            \underaccent{\bar}{\tilde{T}}_{1}^{q}
            \\
            &=
            \sum_{v=1}^{p_q}\left[
                l \in (
                    \ell_{v-1}^{q}
                    ,\ell_{v}^{q}
                ]
            \right]
            \max(
                \ell_{v-1}^{q}
                ,\ell_{v}^{q}
            ]
            +
            [l = 0]
            \max(
                0
                ,\ell_{1}^{q}
            ]
            \\
            &=
            \sum_{v=1}^{p_q}\left[
                l \in (
                    \ell_{v-1}^{q}
                    ,\ell_{v}^{q}
                ]
            \right]
            \ell_{v}^{q}
            +
            [l = 0]
            \ell_{1}^{q}
            ;
        \end{align}
        or, to be more precise, as $p_q \in \{1,2,3,\dots \}$ allows for truncated infinite series, resulting in infinite subdivisions within the unit interval,
        \begin{gather}
            \label{mrp_series}
            \underaccent{\bar}{\tilde{T}}(l, p_q)
            =
            1 
            -
            \sum_{v=1}^{p_q}\left[
                l \in (
                    \ell_{v-1}^{q}
                    ,\ell_{v}^{q}
                ]
            \right]
            (1 - \ell_{v}^{q})
            -
            [l = 0]
            (1 - \ell_{1}^{q})
            .
        \end{gather}

        With this, it is trivial to demonstrate the economic configuration with maximum productivity requirements is independent production:
        \begin{align}
            &
            \arg\max_{p_q}{
                \underaccent{\bar}{\tilde{T}}(l, p_q)
            }
            =
            1
            \because
            l \in [0,1]
            \implies
            \nexists \
            \underaccent{\bar}{\tilde{T}}(l, p_q)
            >
            1
            \
            \forall
            \
            p_q \in \{1,2,3,\dots\}
            \\
            &
            \land \
            \underaccent{\bar}{\tilde{T}}(l, p_q)
            =
            1
            \
            \forall
            \
            l \in [0,1]
            \iff
            p_q = 1
            ,
        \end{align}
        so that our ceiling is $\underaccent{\bar}{\tilde{T}}(l, 1)$.

        Furthermore, by the very definition (\ref{complex_task}) of task,
        \begin{gather}
            \label{mrp}
            T(l, \tilde{T}_{k}^{q})
            := 
            \left[
                \tilde{T}_{k}^{q}
                \geq
                l
            \right]
            \implies
            \nexists \
            \underaccent{\bar}{\tilde{T}}(l, p_q)
            <
            l
            =:
            \underaccent{\bar}{\tilde{T}}(l)
            ,
        \end{gather}
        where $\underaccent{\bar}{\tilde{T}}(l)$ is the baseline productivity to complete some small task with $l \in [0,1]$ difficulty, irrespective of economic configuration.
        
        Now, note Equation (\ref{mrp}) coincides with productivity requirements in an infinitely stratified labor market as described by Equation (\ref{mrp_series}),
        \begin{gather}
            \because
            \lim_{p_q \rightarrow \infty}{
                \underaccent{\bar}{\tilde{T}}(l, p_q)
            }
            =
            l
            ,
        \end{gather}
        for, as the number of job subtypes goes to infinity, the gap between a task's difficulty and the upper and lower bounds of the responsibility spectrum in which it is cointained disappears:
        \begin{align}
            &
            \sum_{v=1}^{p_q}
            \int_{\ell_{v-1}^{q}}^{\ell_{v}^{q}}
            \ta(l)dl
            =
            \int_{0}^{1}
            \ta(l)dl
            =
            1,
            \ell_{v}^{q} \in [0,1]
            \
            \forall
            \
            v \in \{1, \dots, p_q\}
            \land
            \ell_{0}^{q}
            := 0
            \\
            &
            \therefore
            \lim_{p_q \rightarrow \infty}{
                \sum_{v=1}^{p_q}
                \int_{\ell_{v-1}^{q}}^{\ell_{v}^{q}}
                \ta(l)dl
            } 
            =
            1
            \iff
            \lim_{p_q \rightarrow \infty}{
                \ell_{v}^{q} - \ell_{v-1}^{q}
            }
            =
            0
            \\
            &
            \implies
            \lim_{p_q \rightarrow \infty}
            \ell_{1}^{q}
            = 0
            \because
            \lim_{p_q \rightarrow \infty}
            \ell_{1}^{q} - 
            \ell_{0}^{q}
            =
            \lim_{p_q \rightarrow \infty}
            \ell_{1}^{q} - 
            0
            = 0
            \\
            &
            \therefore
            \lim_{p_q \rightarrow \infty}{
                \ell_{r}^{q} - \ell_{r-1}^{q}
            }
            =
            \lim_{p_q \rightarrow \infty}{
                \ell_{r}^{q} - l
            }
            =
            \lim_{p_q \rightarrow \infty}{
                l - \ell_{r-1}^{q}
            }
            =
            0
            \\
            &
            \iff
            \lim_{p_q \rightarrow \infty}{
                \underaccent{\bar}{\tilde{T}}(l, p_q)
            }
            =
            l
            \
            \forall
            \
            l \in
            [\ell_{r-1}^{q}, \ell_{r}^{q}],
            \ell_{r-1}^{q}, \ell_{r}^{q}
            \in [0,1]
            ;
        \end{align}
        whence, again, ``jobs'' themselves collapse into tasks (see ISL).

        So, while independent production is the ceiling for productivity requirements, infinite labor stratification is its floor. Therefore,
        \begin{gather}
            1
            =
            \underaccent{\bar}{\tilde{T}}(l, 1)
            \geq
            \underaccent{\bar}{\tilde{T}}(l, 2)
            \geq
            \dots
            \geq
            \underaccent{\bar}{\tilde{T}}(l, w_q)
            \geq
            \dots
            \geq
            \underaccent{\bar}{\tilde{T}}(l)
            =
            l
            ,
        \end{gather}
        for each and every $l \in [0,1]$. 

        But, because the Maximum Stratification Axiom imposes $p_q \in \{1,\dots, w_q \}$,
        \begin{gather}
            \nexists \
            \underaccent{\bar}{\tilde{T}}(l, p_q)
            <
            \underaccent{\bar}{\tilde{T}}(l, w_q)
            \
            \forall
            \
            l \in [0,1]
            \therefore
            \arg\min_{p_q}{
                \underaccent{\bar}{\tilde{T}}(l, p_q)
            }
            = w_q
            .
        \end{gather}

        This means that, given MSA, all an occupation's infinitesimal tasks can only reach minimum productivity requirements in a maximally stratified  market.
    \end{proof}
\end{lemma}
% endregion

Now that we know maximum labor stratification with unitary employment is the production strategy with the lowest barrier of entry, let us also prove it is the only feasible mode of maximum stratification.

% region: equivalent/efficient stratification lemma (esl) 
\begin{lemma}[Efficient Stratification Lemma, ESL]
    Any efficient labor market where employers choose both $\boldsymbol{w_q}$ and $\boldsymbol{\ell_q}$ with $p_q \in \{1, \dots, w_q\}$ types of job posts converges to maximum labor stratification with
    $
    p_{q}^{*} = w_q
    $
    unique positions;
    \begin{gather}
    \ell_{v}^{q}
    =
    \ell_{v}^{q*}
    =
    \invTA\left(
        \frac{v}{w_q}
        +
        \TA(0)
    \right)
    \
    \forall
    \
    v \in 
    \left\{
        1, \dots, w_q
    \right\}
    \end{gather}
    optimal responsibility bounds; and unitary employment levels,
    $
    \boldsymbol{w_{q}^{*}} =
    \boldsymbol{1}
    .
    $
    
    \begin{proof}
        In the MPL above, we have shown maximum stratification minimizes productivity requirements and, thus, the chance workers may not be sufficiently qualified for their responsibilities (i.e. it is the safest production strategy). So, 
        \begin{gather}
            p_{q}^{*} = w_q
        \end{gather}
        is the optimal number of positions in a labor market where employers can split workers' activities without gain or loss to production (WOCA), while guaranteeing the maximum attainable operational output overall (viz. $w_q$, see MOOL).
        
        In addition, by the Maximum Stratification Axiom and Definition \ref{mmls_def},
        \begin{align}
            &
            \sum_{v=1}^{p_q}{
                w_{v}^{q}
            }
            := 
            w_q
            \land
            w_{v}^{q} 
            \geq 
            1
            \
            \forall
            \
            v \in \{1, \dots, p_q\},
            p_q \in \{1, \dots, w_q\}
            \\
            &
            \therefore
            p_{q}^{*} = w_q
            \land
            \sum_{v=1}^{w_q}
            1
            =
            w_q
            \implies
            \sum_{v=1}^{p_{q}^{*}}{
                w_{v}^{q*}
            }
            =
            \sum_{v=1}^{w_q}{
                w_{v}^{q*}
            }
            =
            w_q
            \\
            &
            \iff
            w_{v}^{q*}
            =
            1
            \
            \forall
            \
            v \in \{1, \dots, w_q\}
            \iff
            \boldsymbol{w_{q}^{*}} =
            \boldsymbol{1}
            .
            \end{align}
            
        Therefore, the above implies, given the PEC and OSL,
        \begin{align}
        &
        \int_{\ell_{v-1}}^{\ell_{v}^{q}}{
            \ta(l)
            dl
        }
        =
        \tilde{w}_{v}^{q*} =
        \frac{1}{w_q}
        \land
        \left(
            \int_{
                \ell_{v-1}^{q*}
            }^{
                \ell_{v}^{q*}    
            }{
                \ta(l)
                dl
            }
        \right) ^ {-1}
        :=
        w_q
        \\
        &
        \implies
        \int_{\ell_{v-1}}^{\ell_{v}^{q}}{
            \ta(l)
            dl
        }
        =
        \frac{1}{
            \left(
                \int_{
                    \ell_{v-1}^{q*}
                }^{
                    \ell_{v}^{q*}    
                }{
                    \ta(l)
                    dl
                }
            \right) ^ {-1}
        }
        =
        \int_{
            \ell_{v-1}^{q*}
        }^{
            \ell_{v}^{q*}    
        }{
            \ta(l)
            dl
        }
        \\
        &
        \iff
        \ell_{v}^{q}
        =
        \ell_{v}^{q*}
        =
        \invTA\left(
            \frac{v}{w_q}
            +
            \TA(0)
        \right)
        \
        \forall
        \
        v \in \{1, \dots, w_q\}
        .
        \end{align}
    \end{proof}
\end{lemma}
% endregion

At last, we demonstrate the economy's productivity has to be capable of supporting maximum-monotonic labor stratification.
% region: productivity sufficiency lemma (psl) 
\begin{lemma}[Productivity Sufficiency Lemma, PSL]
    The available talent in a labor market is, at least, sufficient to allow for maximally stratified production.
    
    \begin{proof}[Proof 1]
        If talent were not sufficient to produce occupation $q$'s entire $l \in [0,1]$ responsibility spectrum, then, as aggregate operation output is given by the Leontief function (see WOCA, MOOL),
        \begin{gather}
            \oo_q
            :=
            \min\left(
                \boldsymbol{\ddot{w}_{q}}(
                    \boldsymbol{w_{q}},
                    \boldsymbol{\ell_{q}},
                    \boldsymbol{\tilde{T}_{q}}
                )
                \times
                \boldsymbol{\oo_{q}}(
                    \boldsymbol{\ell_{q}}
                )
            \right)
        ,
        \end{gather}
        employers' optimal choice would be to save their resources and completely shutdown the productive effort. Therefore,
        \begin{gather}
            \lnot \
            \oo_q > 0
            \iff
            \boldsymbol{\ddot{w}_{q}}
            =
            \boldsymbol{0}
            \\
            \implies
            \boldsymbol{w_{q}^{*}}
            =
            \boldsymbol{0}
            \iff
            \boldsymbol{1} ^ {\top}
            \cdot
            \boldsymbol{w_{q}^{*}}
            :=
            w_q
            =
            0
            \\
            \label{mpc}
            \therefore
            w_q \geq 1
            \iff 
            \oo_q > 0
            \iff
            \boldsymbol{\ddot{w}_{q}}
            \geq
            \boldsymbol{1}
            .
        \end{gather}

        In other words, simply because this occupation's labor market exists we know the talent employed is sufficient to output all its responsibility spectra.
        
        Furthermore, as rational employers will not overhire (ERA), for this would reduce their profit, we also know not a single position in the labor market violates the Proportional Employment Condition (MOOL).
        
        Otherwise, employers would lay off excess workers to downscale the workforce from a suboptimal $w_q \geq 1$ to some $w_{q}^{*} \leq w_q$, again, to save resources. Thus, the current workforce, necessarily, has to be of the optimal
        \begin{gather}
        \sum_{v=1}^{p_q}
        w_{v}^{q}
        =
        w_q
        =
        w_{q}^{*}
        \end{gather}
        size and respect the PEC,
        \begin{gather}
        \tilde{w}_{v}^{q}
        =
        \Omega_{v}^{q}
        \in
        [0,1]
        ,
        \\
        \sum_{v=1}^{p_q}
        \Omega_{v}^{q}
        =
        1
        \end{gather}
        at every level.
        
        In addition, we have ruled out infinite labor stratification (see MSA), and demonstrated any efficiently stratified labor market is characterized by the very same responsibility spectra, with $w_q$ unique positions, and unitary employment (OSL, ESL). So, the labor market cannot be more than maximally stratified in accordance with Definition \ref{def_employability}. 
        
        At last, from all valid production strategies we have considered, maximum labor stratification is that which has the lowest barrier of entry, minimizing productivity requirements (MPL).
        
        Therefore, if a labor market has any employees at all, the available talent in it has to be, at least, sufficient for maximally stratified production:
        \begin{gather}
            \tilde{T}_{v}^{q}
            \geq
            % \ell_{qv}^{*}
            % =
            \invTA\left(
                \frac{v}{w_q}
                +
                \TA(0)
            \right)
            \
            \forall
            \
            v \in 
            \left\{
                1, \dots, w_q
            \right\}
            .
        \end{gather}
    \end{proof}

    \begin{proof}[Proof 2]
        Another proof for this lemma is to start with Equation (\ref{mpc}),
        \begin{gather}
            \boldsymbol{\ddot{w}_q}
            \geq
            \boldsymbol{1}
            \iff
            \ddot{w}_{v}^{q}
            :=
            \sum_{r=1}^{w_{v}^{q}}
            [\tilde{T}_{r}^{q} \geq \ell_{v}^{q}]
            \geq
            1
            \
            \forall
            \
            v \in \{1, \dots, p_q\}
            ,
        \end{gather}
        and write, for every production strategy, what it implies. 

        With independent production (IP),
        \begin{align}
            \boldsymbol{\ddot{w}_q}
            \geq
            \boldsymbol{1}
            &\iff
            \sum_{v=1}^{w_{q}}
            [\tilde{T}_{v}^{q} = 1]
            \geq
            1
            \\
            \therefore
            w_q \geq 1
            &\iff
            \oo_{q}^{\text{IP}}
            \in [1, w_q]
            .
        \end{align}

        So, if occupation $q$'s labor market exists at all (i.e. $w_q \geq 1$), it must have at least one perfectly qualified employee, yielding at least one productive unit.

        Now, binary labor stratification (BS) demands a perfectly qualified ``senior'' and a sufficiently qualified ``junior'' for a minimum aggregate output $\oo_{q}^{\text{BS}} > 1$, greater than that of independent production:
        \begin{align}
            \boldsymbol{\ddot{w}_q}
            \geq
            \boldsymbol{1}
            &\iff
            \sum_{r=1}^{w_{\text{Jr}}^{q}}
            [\tilde{T}_{r}^{q} \geq \ell_{\text{Jr}}^{q}]
            \geq
            1
            \land
            \sum_{r=1}^{w_{\text{Sr}}^{q}}
            [\tilde{T}_{r}^{q} = 1]
            \geq
            1
            \\
            \therefore
            w_q \geq 1
            &\iff
            \oo_{q}^{\text{BS}}
            \in
            \left[
                \min\left(
                    \frac{
                        1
                    }{
                        \int_{0}^{\ell_{\text{Jr}}^{q}}
                        \ta(l)dl
                    }
                    ,
                    \frac{
                        1
                    }{
                        \int_{\ell_{\text{Jr}}^{q}}^{1}
                        \ta(l)dl
                    }
                \right)
                ,
                w_q
            \right]
            .
        \end{align}

        And, again, the pattern repeats for all production strategies, up to maximum labor stratification, where
        \begin{gather}
            \boldsymbol{\ddot{w}_q}
            \geq
            \boldsymbol{1}
            \iff
            \sum_{r=1}^{w_{1}^{q}}
            [\tilde{T}_{r}^{q} \geq \ell_{1}^{q}]
            \geq
            1
            \land
            \dots
            \land
            \sum_{r=1}^{w_{w_q}^{q}}
            [\tilde{T}_{r}^{q} = 1]
            \geq
            1
            .
        \end{gather}
        But
        \begin{align}
            &\because
            p_q = w_q
            \iff
            \boldsymbol{w_q} =
            \boldsymbol{1}
            \land
            \boldsymbol{\ell_q} =
            \boldsymbol{\ell_{q}^{*}}
            \\
            &\therefore
            \sum_{r=1}^{1}
            [\tilde{T}_{r}^{q} \geq \ell_{1}^{q*}]
            \geq
            1
            \land
            \dots
            \land
            \sum_{r=1}^{1}
            [\tilde{T}_{r}^{q} = 1]
            \geq
            1
            \\
            &\therefore
            \sum_{r=1}^{1}
            [\tilde{T}_{r}^{q} \geq \ell_{1}^{q*}]
            =
            1
            \land
            \dots
            \land
            \sum_{r=1}^{1}
            [\tilde{T}_{r}^{q} = 1]
            =
            1
            \\
            &\iff
            \tilde{T}_{v}^{q} \geq \ell_{v}^{q*}
            \
            \forall
            \
            v \in \{1, \dots, w_q\}
            ,
        \end{align}
        which means all employees in a maximally stratified labor market necessarily meet the minimum productivity requirements for their responsibilities (i.e. the available talent has to be at least sufficient for maximally stratified production, for no other strategy has a lower barrier of entry). Therefore,
        \begin{align}
            w_q \geq 1
            &
            \iff 
            \oo_q > 0
            \iff
            \boldsymbol{\ddot{w}_{q}}
            \geq
            \boldsymbol{1}
            \\
            % w_q \geq 1
            &\iff
            \oo_{q}^{\text{MS}}
            \in
            \left[
                \min\left(
                    \frac{1}{
                        \int_{0}^{
                            \ell_{1}^{q*}
                        }{
                            \ta(l)dl
                        }
                    }
                    ,
                    \dots
                    ,
                    \frac{1}{
                        \int_{
                            \ell_{w_q - 1}^{q*}
                        }^{1}{
                            \ta(l)dl
                        }
                    }
                \right)
                ,
                w_q
            \right]
            \\
            &\iff
            \oo_{q}^{\text{MS}}
            \in
            \left[
                \left(
                    \int_{
                        \ell_{v-1}^{q*}
                    }^{
                        \ell_{v}^{q*}
                    }{
                        \ta(l)dl
                    }
                \right) ^ {-1}
                ,
                w_q
            \right]
            \\
            &\iff
            \oo_{q}^{\text{MS}}
            \in
            \left[
                w_q
                ,
                w_q
            \right]
            \\
            &\iff
            \oo_{q}^{\text{MS}}
            =
            w_q
            ,
        \end{align}
        which is also the operational output ceiling for any production strategy. Hence, productivity requirements are minimized, while aggregate output is maximized. Or, put another way, in this arrangement the minimum yields the maximum.
    \end{proof}
\end{lemma}
% endregion

From the above, it follows logically that maximum-monotonic labor stratification is the optimal production strategy and, so, holds in the labor market.
% region: maximum-monotonic labor stratification lemma (mlsl) 
\begin{lemma}[Maximum Labor Stratification Lemma, MLSL]
    \label{mlsl}
    The Maximum Labor Stratification Lemma states that a perfectly rational employer (ERA), which expects there could be skill differences in the workforce (PDA), and can split operational output without gain or loss to production (WOCA), will, therefore, strategically stratify their job offers monotonically, and even maximally, so that, if indeed there happens to be skill differences in the labor market, they can, then, allocate less competent workers to easier roles, and avoid wasting talent, thus ``saving their best'' for the most demanding tasks.

    \begin{proof}
        We have demonstrated any productive arrangement can only yield, at most, $w_q$ units of an occupation $q$'s operation, and this if the talent employed is sufficiently qualified (MOOL).

        We have also demonstrated that, simply because a labor market exists at all, its workers' productivity has to be, at least, sufficient for maximally stratified production (PSL) when infinite stratification is ruled out (ISL, MSA).

        Furthermore, Definition \ref{complex_task} implies there is no upside to employing underqualified workers, as 
        \begin{gather}
            \tilde{T}_{k}^{q}
            <
            \underaccent{\bar}{\tilde{T}}_{v}^{q}
            \implies
            \left[
                \tilde{T}_{k}^{q}
                \geq
                \underaccent{\bar}{\tilde{T}}_{v}^{q}
            \right]
            \oo_{v}^{q}
            =
            0
            ,
        \end{gather}
        that is, if an employee cannot fully output a responsibility spectrum, their contribution to production is void, limiting aggregate operational output. This means choosing any production strategy other than maximum-monotonic labor stratification is a risk with no upside, when workers may have varying productivity (PDA). For, as we have shown, this arrangement by itself guarantees minimum productivity requirements (MPL) and maximum operational output (MOOL). It would be, then, irrational of employers to organize production in another manner. So, we can write for all $v \in \{1, \dots, w_q\}$:
        \begin{gather}
            \mathbb{E}[
                \oo_{q}^{\text{IP}}
                \
                |
                \
                \mathbb{E}[
                    \tilde{T}_{v}^{q}
                ]
                \in [0,1]
            ]
            \\
            \leq
            \mathbb{E}[
                \oo_{q}^{\text{BS}}
                \
                |
                \
                \mathbb{E}[
                    \tilde{T}_{v}^{q}
                ]
                \in [0,1]
            ]
            \\
            \vdots
            \\
            \leq
            \mathbb{E}[
                \oo_{q}^{\text{MS}}
                \
                |
                \
                \mathbb{E}[
                    \tilde{T}_{v}^{q}
                ]
                \in [0,1]
            ]
            \\
            =
            \mathbb{E}[
                \oo_{q}^{\text{IP}}
                \
                |
                \
                \tilde{T}_{v}^{q}
                = 1
            ]
            =
            w_q
            ,
        \end{gather}
        where each of the terms above represents the expected value of aggregate operational output in production strategies other than infinite stratification, given the workforce's expected productivity.

        In other words, splitting responsibilities with respect to competence always produces the maximum operational output (viz. that which is obtained when employing perfectly qualified workers independently), provided employees are sufficiently qualified for their responsibilities. But, again, this is, by definition, guaranteed by employers' rationality, as well as the simple fact the economy is already producing its current operational output (see PSL). 
        
        Therefore, employing potentially underqualified workers to output the entire responsibility spectrum $l \in [0,1]$ independently can only be as productive as the labor stratification strategy, but never more than it. Indepedent production, then, is a suboptimal strategy when employers expect there to be skill differences in the workforce. And the same logic also applies to less than maximally stratified arrangements.
        
        Thus, maximum labor stratification follows as an insurance policy against worker's potential underqualification: for if talent is lacking in the labor market, there is nothing to gain by employing individuals which are not sufficiently qualified for a difficult job, whereas if talent is abundant, there is nothing to lose when employing overqualified individuals to a job below their skill level.

        Hence, given the same $w_q \geq 1$ workforce, operational output in a maximally stratified labor market is always greater or equal to the output of any other economic configuration. It is, therefore, always optimal to monotonically and maximally stratify responsibilities across $w_q$ unique positions, each focused on increasingly demanding tasks.
    \end{proof}
\end{lemma}
% endregion

\subsubsection{General Employability Theorem}
With this, we have shown maximally stratified production is the only efficient arrangement that holds in reality. So, we can, finally, derive a general employability coefficient by estimating employability in such markets.
% region: general employability theorem (get) 
\begin{get_theorem}[GET]
    Because maximum labor stratification is the safest and most efficient production strategy, rational employers will always choose to implement it. Therefore, an individual's employability in a maximally stratified economy,
    \begin{gather}
        \tilde{W}_k 
        =
        \left(
            \frac{1}{w}
        \right)
        \sum_{q=1}^{n}
        \left[
            h_{k}^{q}
            \geq
            \frac{1}{2}
        \right]
        \sum_{v=1}^{w_q}
        \left[
            \tilde{T}_{k}^{q}
            \geq
            \invTA
            \left(
                \frac{v}{w_q}
                +
                \TA(0)
            \right)
        \right]
    \end{gather}
    is their actual employability in reality.
    \begin{proof}
        We have just demonstrated (MLSL) maximum labor stratification is the only optimal productive arrangement and, given our assumptions, attains in reality. Thus, one's employability in this market is their actual employability.

        Moreover, in accordance with Definition \ref{def_employability}, employability is
        \begin{gather}
            \tilde{W}_{k}
            :=
            \left(
                \frac{1}{w}
            \right)
            \sum_{q=1}^{n}
            \left[
                h_{k}^{q}
                \geq
                \frac{1}{2}
            \right]
            \sum_{v=1}^{p_q}
            \left[
                \tilde{T}_{k}^{q}
                \geq
                \ell_{v}^{q}
            \right]
            w_{v}^{q}
            ,
        \end{gather}
        which in a maximally stratified market becomes (see OSL, ESL)
        \begin{align}
            \tilde{W}_{k}
            &=
            \left(
                \frac{1}{w}
            \right)
            \sum_{q=1}^{n}
            \left[
                h_{k}^{q}
                \geq
                \frac{1}{2}
            \right]
            \sum_{v=1}^{p_q}
            \left[
                \tilde{T}_{k}^{q}
                \geq
                \ell_{v}^{q}
            \right]
            \times
            1
            \\
            &=
            \left(
                \frac{1}{w}
            \right)
            \sum_{q=1}^{n}
            \left[
                h_{k}^{q}
                \geq
                \frac{1}{2}
            \right]
            \sum_{v=1}^{w_q}
            \left[
                \tilde{T}_{k}^{q}
                \geq
                \invTA
                \left(
                    \frac{v}{w_q}
                    +
                    \TA(0)
                \right)
            \right]
            .
        \end{align}
    \end{proof}
\end{get_theorem}
% endregion
% endregion

% region: corollaries 
\subsection{Corollaries}
The General Employability Theorem has a few interesting collolaries, the first, and simplest, of which is a more compact, and intuitive, version of it.
% region: simplified employability corollary 
\begin{corollary}[Simplified Employability Corollary, SEC]
    We want to show that, as with the BET and ISL, so too in a maximally and monotonically stratified labor market, employability is the percentage of an operation's total time duration one is capable of producing. Or, mathematically,
    \begin{gather}
        \tilde{W}_{k}^{q}
        =
        \left[
                h_{k}^{q}
                \geq
                \frac{1}{2}
        \right]
        \int_{0}^{\tilde{T}_{k}^{q}}
        \ta(l)
        dl
        =:
        \Omega_{k}^{q}
        \in 
        [0,1]
        \
        \forall
        \
        k,q \in \{1, \dots, n\}
        .
    \end{gather}

    \begin{proof}
        To prove this result, we first calculate what would be the employability of person $k$ if they had exactly the minimum required productivity for every job subtype. So, for instance, when $v = 1$,
        \begin{align}
            \tilde{T}_{k}^{q}
            :=
            \ell_{1}^{q*}
            =
            \invTA
            \left(
                \frac{1}{w_q}
                +
                \TA(0)
            \right)
            \implies
            \tilde{W}_{k}^{q}
            =
            \frac{1}{w_q}
            ,
        \end{align}
        as a productivity coefficient of $\tilde{T}_{k}^{q} = \ell_{1}^{q*}$ is just enough to be hireable on the easiest job in occupation $q$'s labor market, but not on the second, much less on the remaining, more difficult, positions. And, for other values of $v$,
        \begin{gather}
            \tilde{T}_{k}^{q}
            :=
            \ell_{2}^{q*}
            =
            \invTA
            \left(
                \frac{2}{w_q}
                +
                \TA(0)
            \right)
            \implies
            \tilde{W}_{k}^{q}
            =
            \frac{2}{w_q}
            ,
            \\
            \tilde{T}_{k}^{q}
            :=
            \ell_{3}^{q*}
            =
            \invTA
            \left(
                \frac{3}{w_q}
                +
                \TA(0)
            \right)
            \implies
            \tilde{W}_{k}^{q}
            =
            \frac{3}{w_q}
            ,
            \\
            \vdots
            \\
            \tilde{T}_{k}^{q}
            :=
            \ell_{v}^{q*}
            =
            \invTA
            \left(
                \frac{v}{w_q}
                +
                \TA(0)
            \right)
            \implies
            \tilde{W}_{k}^{q}
            =
            \frac{v}{w_q}
            ,
        \end{gather}
        so that we may derive the following pattern for any $v \in \{1, \dots, w_q\}$:
        \begin{align}
            &\tilde{T}_{k}^{q}
            =
            \invTA
            \left(
                \tilde{W}_{k}^{q}
                +
                \TA(0)
            \right)
            \\
            &
            \therefore
            \TA
            (
            % \left(
                \tilde{T}_{k}^{q}
            )
            % \right)
            =
            \TA
            \left(
                \invTA
                \left(
                    \tilde{W}_{k}^{q}
                    +
                    \TA(0)
                \right)
            \right)
            \\
            &
            \therefore
            \TA
            (
            % \left(
                \tilde{T}_{k}^{q}
            )
            % \right)
            =
            \tilde{W}_{k}^{q}
            +
            \TA(0)
            \\
            &
            \therefore
            \tilde{W}_{k}^{q}
            =
            \TA
            (
            % \left(
                \tilde{T}_{k}^{q}
            )
            % \right)
            -
            \TA(0)
            =
            \int_{0}^{\tilde{T}_{k}^{q}}
            \ta(l)
            dl
            =:
            \Omega_{k}^{q}
            \in 
            [0,1]
            ,
        \end{align}
        as we wanted to show.
        
        However, because $\tilde{T}_{k}^{q} \in [0,1]$ is not as discretized as responsibility ranges $l \in [\ell_{v-1}^{q*},\ell_{v}^{q*}], v \in \{1, \dots, w_q\}$, and because rational employers do not hire insufficiently qualified employees, we must approximate $\tilde{T}_{k}^{q}$ with the closest 
        \begin{gather}
            \tilde{T}_{\kappa}^{q}
            :=
            \left(
                \frac{1}{w_q}
            \right)
            \sum_{v=1}^{w_q}
            \left[
                \tilde{T}_{k}^{q}
                \geq
                \ell_{v}^{q*}
            \right]
        \end{gather}
        productivity estimate, such that $\tilde{T}_{\kappa}^{q} = \ell_{\kappa}^{q*} \in \{\ell_{0}^{q*}, \dots, \ell_{w_q}^{q*}\}$, with $\tilde{T}_{\kappa}^{q} \lessapprox \tilde{T}_{k}^{q}$, determines the most demanding task for which $k$ is still productive. Therefore, the adjusted coefficient is:
        \begin{gather}
            \tilde{W}_{k}^{q}
            =
            \int_{0}^{\tilde{T}_{\kappa}^{q}}
            \ta(l)
            dl
            =:
            \Omega_{\kappa}^{q}
            \approx
            \int_{0}^{\tilde{T}_{k}^{q}}
            \ta(l)
            dl
            \in 
            [0,1]
            ,
        \end{gather}
        when $w_q$ is large enough.
        
        Of course, this assumes candidate $k$ is evaluated as ``employable'', in a more general sense, by the hireability statistic
        \begin{gather}
            \left[
                h_{k}^{q}
                \geq
                \frac{1}{2}
            \right]
            .
        \end{gather}
        
        Hence, a more complete formulation would be:
        \begin{gather}
            \tilde{W}_{k}^{q}
            =
            \left[
                h_{k}^{q}
                \geq
                \frac{1}{2}
            \right]
            \int_{0}^{\tilde{T}_{\kappa}^{q}}
            \ta(l)
            dl
            ;
        \end{gather}
        or, in the aggregate form,
        \begin{gather}
            \tilde{W}_k 
            = 
            \left(
                \frac{1}{w}
            \right)
            \sum_{q=1}^{n}
            w_q
            \left[
                h_{k}^{q}
                \geq
                \frac{1}{2}
            \right]
            \int_{0}^{\tilde{T}_{\kappa}^{q}}
            \ta(l)
            dl
            .
        \end{gather}
    \end{proof}        
\end{corollary}

% endregion

In addition, the General Employability Theorem can be used to prove the General Competitiveness Corollary (GCC) with the following definition.
% region: competitiveness definition 
\begin{definition}[Competitiveness]
    \label{def_competitiveness}
    Labor market competitiveness can be defined in a variety of ways. The typical is to think of competitiveness as a ratio of job seekers to the number of available positions. Thus, we say an occupation's labor market is ``competitive'' if there are too many incumbents per job post.
    
    Nevertheless, this definition has two main flaws: 1) it can be somewhat cumbersome, if not impossible, to gather all necessary data, for every labor market, under shifting conditions, to accurately assess competitiveness; 2) and even if such data are available and trustworthy, incumbents per job posts, in and of itself, is not that much of an interpretable, or at least complete, statistic.
    
    So, we may propose an alternative, additional, definition of competitiveness as the percentage of the workforce which would be \textit{willing} and \textit{able} to compete for jobs in a particular labor market:
    \begin{gather}
        \tilde{\text{vs}}_{k}
        :=
        \left(
            \frac{1}{w}
        \right)
        \sum_{q=1}^{n}
        \left[
            u_{q}^{k}
            \geq
            u_{q}^{q}
        \right]
        W_{q}^{k}
    \end{gather}
    where $u_{q}^{k}, u_{q}^{q} \in \mathbb{R}$ are point estimates of the utility function of occupation $q$'s workers when employed, respectively, at a job $k$ or their own, current, positions.
    
    Therefore, what the $\tilde{\text{vs}}_{k}$ (abbrev. versus) competitiveness statistic tells us is that if workers of type $q$ are willing to compete for an occupation $k$'s job posts, they are evaluated in terms of their employability (cf. Definition \ref{def_employability}) and, if found sufficiently qualified, counted as viable incumbents. Or, in other words, competitiveness is the employability of willing and able workers from other labor markets (i.e. it is a complement of employability).
\end{definition}
% endregion
And, with this, we may derive the General Competitiveness Corollary.
% region: general competitiveness corollary (gcc)
\begin{corollary}[General Competitiveness Corollary, GCC]
    \label{gcc}
    The competitiveness of an occupation's labor market is the percentage of the aggregate workforce $w$ that is willing and able to compete for its job posts.

    \begin{proof}
        By Definition \ref{def_competitiveness} and the General Employability Theorem,
        \begin{align}
            \tilde{\text{vs}}_{k}
        &:=
        \left(
            \frac{1}{w}
        \right)
        \sum_{q=1}^{n}
        \left[
            u_{q}^{k}
            \geq
            u_{q}^{q}
        \right]
        W_{q}^{k}
        \\
        &=
        \left(
            \frac{1}{w}
        \right)
        \sum_{q=1}^{n}
        \left[
            u_{q}^{k}
            \geq
            u_{q}^{q}
        \right]
        \left[
            h_{q}^{k}
            \geq
            \frac{1}{2}
        \right]
        \sum_{v=1}^{w_k}
        \left[
            \tilde{T}_{q}^{k}
            \geq
            \invTA
            \left(
                \frac{v}{w_k}
                +
                \TA(0)
            \right)
        \right]
        .
        \end{align}
    \end{proof}
\end{corollary}
% endregion
% endregion

% % region: implementation 
% \section{Example Implementation}
% \subsection{Functional Specifications}
% \subsection{Occupational Information Network Data}
% \subsection{Results}

% % endregion

% region: discussion 
\section{Discussion}
Having mathematically proven the General Employability Theorem and its corollaries, we summarize their intuitive meaning. To this end, let us, therefore, recall the steps we took to these results and what we have learned in the process.

\subsection{Overview}
In the first section of this paper, we defined professional attributes' tasks in a clear, quantitative, fashion; and, assuming scale uniformity, proceeded to speak of productivity as the sum of tasks one can accomplish, given their skill set. We, then, demonstrated skills, themselves, are equivalent to task execution capacity, so that attributes are reducible to their tasks. We also generalized this notion for complex tasks (viz. those requiring multiple attributes) and, with this, were able to reduce whole occupations to their respective \textit{operations}, understood as composite, atomic, activities and the essences of all jobs in the labor market. To avoid economicly unrealistic scenarios, though, we supposed these operations are not literally indivisible and can be outsourced if ``pieced back together'' in a coherent product. We termed this mechanism \textit{labor stratification}.

Afterwards, we described market dynamics by pressuposing these very general axioms: 1) employers take wages as given and manage workers by choosing how much to stratify their job offers and which responsibilities are allocated to each employee; 2) the available talent in the workforce is not necessarily homogenous; 3) and employers take this into account in the hiring process and screen potential employees without information asymmetry. Finally, from this, we showed wages in equilibrium remain the same across occupations' positions.

The subsequent section covered various types of labor stratification, from the most basic binary setting, with ``junior'' and ``senior'' employees, to a rather abstract market, where ``jobs'' are not quite ``jobs'', but infinitely small tasks. We, thus, decided to limit the number of unique responsibility ranges to the number of people in the workforce and called this \textit{maximum labor stratification}.

At last, we provided robust, mathematical, proofs that, from all strategies considered, such a market is, indeed, the most efficient and productive one and, so, attains in reality. It, then, followed individuals' employability in an economy where our assumptions hold true can be calculated by their employability in maximally stratified markets.

In the final, corollaries, section, we conceptualized labor market competitiveness as a complement of employability, in particular, as that of workers from other fields \textit{willing} and \textit{able} to compete for job posts. And here also we concluded, as in two other lemmas in the article, that employability is, simply yet precisely, the percentage of occupations' operations one is capable of producing.

\subsection{Specific Considerations}
Throughout this paper we made several references to a time allocation function, associating tasks' difficulty level with the time it takes to complete them. Indeed, our entire demonstration rests upon this notion. So, we should take a moment to comment on it in detail.

The time allocation function is an indispensable statistical device for differentiating tasks explicitly. For, in fact, if one were to ommit its concept from the model, they would be the implicitly assuming tasks to be equivalent in duration.

Furthermore, as alluded to in the Productivity Sufficiency Lemma above, a labor market's existence, in and of itself, implies the available talent is at least sufficient for maximally stratified production. Hence, to choose a time allocation function is, indirectly, to make a statement on the population's productivity distribution. This is more immediately noticeable in infinitely stratified markets,
\begin{gather}
    \because
    \tilde{w}_q(l)
    =
    \ta(l)
    \
    \forall
    \
    l
    \in
    [0,1]
    ,
\end{gather}
but it also applies in other economic configurations. To say time allocation is accurately described by this or that function, then, is just the same as saying the workforce must have a corresponding productivity, which may, or may not, be a reasonable assumption. Therefore, much care should be devoted to determining time allocation. Fortunately, empirical assessments from a large enough, unbiased, sample of the population can serve as a basis for deriving its functional form, thus, again, bridging the gap between Psychometrics and Labor Economics, as we set out to do in the Introduction.

On a related note, we could add time allocation can be defined for each and every occupation individually, although in this article we have written all time allocation functions the same, in order to keep notation as light as possible. However, it would be more precise to define $\ta(l)$ and $\ttc(l)$ as any
\begin{gather}
    \ttc_{q} : [0,1] \rightarrow \mathbb{R}^{+}
    ,
    \ta_{q} : [0,1] \rightarrow [0,1]
    \
    \forall
    \
    q \in \{1, \dots, n\}
    ,
\end{gather}
so that we do not assume occupations' labor markets behave the exact same.

Moreover, time allocation can also be used to ``modify'' difficulty scales with direct assumptions. For instance, if one wished to express mathematically there are no easy, say, Engineering jobs, one might define task duration to be
\begin{gather}
    \ttc(l, c_q)
    :=
    l ^ {
        \frac{
            c_q
        }{
            1 - c_q
        }
    }
    ,
\end{gather}
where $c_q \in [0,1] \ \forall \ q \in \{1, \dots, n\}$ is an occupation's assessed competence statistic as some function of their required skill set. Again, such ``customizations'' ought to be done with care, to minimize model bias, but it is nonetheless a useful feature of time allocation for conveniently adding realism to empirical analyses.

Moving on to labor market competitiveness, we similarly note workers' utility functions can serve this very same purpose as well and even be defined on more basic levels, not only for occupations as a whole, but also for job subtypes and individual employees, allowing for a great deal of flexibility and expressivity, which can lead to quite interesting labor market dynamics (e.g. worker migration and complex influxes between markets based on wage signals, or momentary trends in preferences for professional attributes, or talent availability).

As for the model's flaws, its main limitation is probably the Weak Occupational Complexity Axiom; for it is not written in the most general form possible and asserts operational output is calculated by the Leontief production function. Again, this was a simplification so as to introduce this rather complicated topic in a more straightforward manner. Nevertheless, in the future, we should suppose even weaker versions of occupational complexity and show the Theorem's conclusions still hold. Hence, generalizing for any aggregation function or, at least, well-behaved ones, it will, then, be ``more properly" called the General Employability Theorem.
% endregion

% region: conclusion 
\section{Conclusion}
This article continues from where we left off in previous works, now with the aim of estimating reliable employability and competitiveness statistics for use in the research and development of real world applications and solutions in the field of vocational choice and career guidance.

Given a lack of scientific literature on quantitative approaches similar to ours both in terms of scope and methodology, despite notable contributions being closer than others to our goals, we found ourselves in need of putting labor economic and psychometric concepts together to construct a mostly novel framework for assessing individuals' expected performance in efficient markets.

Our biggest obstacle in this was the issue of occupational complexity, which we addressed in Section \ref{issue_of_occupational_complexity} of this paper by deducing occupations can, in practice, be reduced to essential and indivisible sets of complex tasks, called \textit{operations}.

With additional assumptions about occupations' operations, their aggregation function, market dynamics, as well as employer behavior, we were able to demonstrate labor markets converge to a maximally stratified arrangement, where each position is its own job subtype. We, thus, proved mathematically one's employability in a maximally stratified labor market is their actual employability in an efficient economy and coincides with the portion of occupations' operations they can accomplish. Likewise, we also proved competitiveness is the employability of incumbent workers from other labor markets.

Finally, we noted the time allocation function used in these estimations is of utmost importance to their accuracy and implicitly states a researcher's belief about the underlying productivity distribution in the workforce. It, therefore, should be derived via careful psychometric assessments and statistically validated if it is to be used in serious applications and career counseling, in general.

In the future, we intend to provide example implementations of the method we have established and further generalize it for other aggregation functions.
% endregion

% region: references 
\newpage
\printbibliography[
    heading=bibintoc,
    title={References}
]

% endregion

% % region: appendix 
% \newpage
% \section*{Appendix A -- Basic Definitions}
% \addcontentsline{toc}{section}{Appendix A -- Basic Definitions}

% \section*{Appendix B -- Employability and Competitiveness Statistics}
% \addcontentsline{toc}{section}{Appendix B -- Employability and Competitiveness Statistics}

% \section*{Appendix C -- Proof Layout}
% \addcontentsline{toc}{section}{Appendix C -- Proof Layout}

% % endregion

\end{document}